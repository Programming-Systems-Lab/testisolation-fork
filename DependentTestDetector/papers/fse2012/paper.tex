%File: paper.tex
%Part of: FSE 2012 paper
%Author: Jochen Wuttke
%Date: 2012-2-06


\documentclass[letterpaper]{sig-alternate}
\usepackage{graphicx}
\usepackage[draft]{fixme}
\usepackage{url}
\usepackage{amsfonts}
\usepackage{amssymb}
\usepackage{amsmath}
\usepackage{algorithmic}
\usepackage{booktabs}
\usepackage{listings}
\usepackage[T1]{fontenc}
\usepackage{lmodern}
\usepackage{color}
\usepackage[draft]{hyperref}
\usepackage{xspace}
\usepackage{subfigure}
\usepackage{bold-extra}

\usepackage{color}
\newcommand{\CodeIn}[1]{{\small\texttt{#1}}}

% % Add line between figure and text
%\makeatletter
%\def\topfigrule{\kern3\p@ \hrule \kern -3.4\p@} % the \hrule is .4pt high
%\def\botfigrule{\kern-3\p@ \hrule \kern 2.6\p@} % the \hrule is .4pt high
%\def\dblfigrule{\kern3\p@ \hrule \kern -3.4\p@} % the \hrule is .4pt high
%\makeatother
 % If there is a line, you can get away with reducing the separation between
 % figures and text.  Don't do this without the line, though.
%\addtolength{\textfloatsep}{-.5\textfloatsep}
%\addtolength{\dbltextfloatsep}{-.5\dbltextfloatsep}
%\addtolength{\floatsep}{-.5\floatsep}
%\addtolength{\dblfloatsep}{-.5\dblfloatsep}

\newtheorem{definition}{Definition}
\newtheorem{theorem}{Theorem}
\newtheorem{corollary}{Corollary}
%\newtheorem{proof}{Proof}
\newcommand{\pass}{\ensuremath{\mathit{PASS}}\xspace}
\newcommand{\fail}{\ensuremath{\mathit{FAIL}}\xspace}

% commands for formalization
\newcommand{\suites}[0]{\ensuremath{\mathcal{S}\xspace}}
\newcommand{\environs}[0]{\ensuremath{\mathcal{E}\xspace}}
\newcommand{\manifest}[1]{\ensuremath{\prec_{#1}}}
\newcommand{\suite}[1]{\ensuremath{ \langle  #1 \rangle }}
\newcommand{\env}[0]{\ensuremath{\mathbf{E}}\xspace}
\newcommand{\exec}[2]{\ensuremath{\varepsilon(#1,#2)}}
\newcommand{\result}[2]{\ensuremath{R(#1|#2)}}

%commands must be the last package imported
% This package provides the \todo{Name}{Comment} command
\usepackage{commands}
%remove syntax highlighting from Java code
\lstset{basicstyle=\ttfamily,tabsize=2,keywordstyle=\ttfamily,stringstyle=\ttfamily,commentstyle=\ttfamily,
captionpos=b,numberstyle=\small\ttfamily,numbersep=1ex,
keywordstyle=\color{red}}

%\renewcommand{\todo}[2]{}

\author{
\hfill Jochen Wuttke \hspace{1cm} K\i{}van\c{c} Mu\c{s}lu \hspace{1cm} Sai Zhang
\hspace{1cm} David Notkin \hfill\\ 
\affaddr{Computer Science \& Engineering}\\ 
\affaddr{University of Washington} \\ 
\affaddr{Seattle, WA, USA} \\
\email{\{wuttke|kivanc|szhang|notkin\}@cs.washington.edu}
}
%\conferenceinfo{ASE}{'09 Auckland, New Zealand}

%\title{Reexamining the Test Independence Assumption}
%Does Test Execution Order Matter?
%\title{Test Dependence: Theory and Observations}
\title{Test Dependence: Theory and Manifestation}

\begin{document}
\maketitle
\begin{abstract}
%{\color{red}
%\noindent Dear friendly pre-reviewer: 
%
%\noindent This is a reasonably complete draft that we plan to submit
%to FSE 2012 on Friday March 16.  There are a few specifics we'd love
%for you to look at, but you're of course free to comment on other
%stuff (including typos, grammar, etc.); the conclusion is probably
%the part we've worked on the least so far, if you need to do something
%more lightly.  (OK, the abstract has been worked on even less!)  Among
%the specifics we'd like to hear about are: (a) The related work is
%reasonable extensive, but isn't done.  If you have any suggestions of
%work we've missed, or work that would help others put our work in
%context, or about the ``feel'' of what it says, let us know. (b) The
%core content is even more important; do we have something here?  Are
%you interested?  If not, why not?
%
%\noindent Thanks in advance for anything you point out!
%
%\noindent David, Jochen, Kivanc and Sai
%}
Test dependence arises when executing a test in different environments
causes it to return different results.  In this paper, we
show through a set of substantive real-world examples that test dependence arises in practice. 
We also show that test dependence can have potentially costly
repercussions such as masking program faults, and can be hard to identify
unless explicitly searched for: We found a dependence
that only manifests when a sequence of three tests are run in a specified, non-default order.
%.  For example, we identify several situations where test dependence masks
%program faults: in these situations, running the test suite in the default order does not expose a fault
%but running the suite in a different order does.  
%We also argue that existing
%tools rarely ``surface'' test dependences in a direct way, making it harder for developers
%to observe them.

We formally define test dependence in terms of test suites as ordered
sequences of tests along with explicit environments in which these tests are
executed. We use this formalization to formulate the concrete problem
of detecting dependence in test suites, prove that a useful special
case is NP-complete, and propose an initial algorithm that
approximates solutions to this problem.

%To a lesser degree, we describe how two trends in software testing may interact
%with test dependence: one, downstream testing tools such as selection, prioritization,
%and parallelization are increasingly common, and may assume that the
%suites they take as input have no test dependences; and, two, automated test
%generation tools are becoming more common, and we provide some initial
%evidence that test dependence appears to be orders of magnitude more common
%in automatically-generated test suites than in manually-produced suites.
%

%WE show that, in practice,
%test dependence does occur sometimes and does have grave consequences.
%We further argue, that given the increasing importance of second-order
%testing techniques, such as prioritization and parallelization, which
%are directly affected by test dependence, this topic deserves
%attention from the research community. As a first step, 
%
%
%\todo{KM}{The abstract was only about the ``theory'' side of our paper, so I
%tried to start writing one more paragraph about the ``manifestation'' side. It
%is not perfect (and not complete) but I still think that something like this
%needed in the abstract.} 
%In the second half of the paper, we explain the dependences in the manual
%written and auto-generated test suites of known and popular open source
%software. We present an initial exploration over six software and report 75
%(resp. 1975) dependent tests in manual written (resp. auto-ge\-ne\-rated) test
%suites.

\end{abstract}

%\category{D.2.4.}{Software/Program Verification}{Assertion Checkers}
\category{D.2.5}{Testing and Debugging}%{Monitors}
%
%\terms{Design, Reliability}
%
\keywords{Software testing; test dependence; test selection; test prioritization}

\section{Introduction}

Informally, \emph{dependent tests} produce different test results when
executed in different contexts. %, while \emph{independent tests} produce
%the same test results regardless of execution order.  
It is easy to
construct an example of dependence between two tests \code{A}
and \code{B}, where running \code{A} and then \code{B} leads
to both tests passing, while running \code{B} and then
\code{A} leads to either or both tests failing---the order
of applying the tests, in this case, changes the execution context.

%~\cite{KapfhammerS03}
%Chays:2000:FTD:347324.348954,
%Gray:1994:QGB:191843.191886}, 

Definitions in the testing literature are generally clear that the
conditions under which a test is executed may affect its result.  The
importance of context in testing has been explored in some depth in
some domains including databases~\cite{Gray:1994:QGB:191843.191886,Chays:2000:FTD:347324.348954,
kapfhammeretal:FSE:2003}, with results about test
generation, test adequacy criteria, etc., and mobile
applications~\cite{Wang:2007:AGC}.
For the database domain, Kapfhammer and Soffa formally
define and distinguish independent test suites from those that are
\emph{non-restricted\/} and thus ``can capture more of an application's
interaction with a database while requiring the constant monitoring of
database state and the potentially frequent re-computations of test
adequacy''~\cite[p.~101]{kapfhammeretal:FSE:2003}.

At the same time, there is little focus on the core
issues of
test dependence itself.
Is this because test dependence does
not arise in practice (beyond domains such as databases)?  Is it because, even if-and-when it does arise, there
are few if any repercussions?  Is it because it is difficult to
notice if-and-when it arises?

\subsection{Manifest Test Dependence}

To explore these questions, we consider a narrow characterization
of test dependence that:
\begin{itemize}
\item Adopts the results of the default, usually implicit,
  order of execution of a test suite as the \emph{expected results}. 
\item Asserts \emph{test dependence\/} when there is a possibly
  reordered subsequence of the original test suite that, when
  executed, has at least one test result that differs from the
  expected result for that test.  
\end{itemize}
That is, we focus on a \emph{manifest\/} perspective of test dependence,
requiring a \emph{concrete\/} variant of the test suite that
\emph{dynamically\/} produces different results from the expected.  Our
definition differs from that of Kapfhammer and Soffa by considering
test results rather than program and database states.
As we discuss later, considering only manifest test dependences allows
us to more easily situate this research in the empirical domain.

%: (a) given the lack of
%attention to test dependence, reporting solely on potential but
%unrealized dependence, that is, ``false positives'', might be of little value; and
%(b) computational advantages arise from computing manifest rather than
%potential dependence.

\subsection{Examples and Repercussions}

We have identified a number of substantive examples of test suites
from fielded programs that manifest dependences.
%in fielded programs. 
%with test suites that manifest test dependence.  
We examined
six projects and found in their human-written test suites a total
of 75 dependent tests ($1.4 \%$). For the same set of
programs, we also generated test suites automatically using
Randoop~\cite{PachecoLET2007} and found that on average $14 \%$ of
the generated tests are dependent.

%Further, considering the increasing importance of
%automated test generation techniques, we wanted to get an impression
%of whether test dependence occurs in automatically generated test
%suites. We applied Randoop to the same
%set of programs and 

By analyzing these examples of test dependence, we identified three
categories of problems that can arise due to the presence of dependent tests.
First, test suites that unexpectedly contain dependent tests can
\emph{mask faults in a program}.  We present examples where
executing a test suite in the default order does not expose the fault, whereas
executing the same test suite in a different order does.
Second, test suites that unexpectedly contain dependent tests can \emph{conceal
weaknesses in the test suite} itself.  We present examples where exposing
dependent tests can identify situations where some tests do not perform
proper initialization.
Third, a test suite containing undocumented test dependences can lead
to \emph{spurious bug reports}.  We present an example where it took the developers
more than a month to realize that the test dependences were intentional,
allowing them to close the bug report without a change to the system.

%In practice, test suites that are thought to include only independent
%tests but that manifest dependence can cause problems including:
%\begin{itemize}
%\item masking faults in the program that are not exposed in one execution order but that are
%in another order; 
%\item exhibiting unexpected
%results if reordered (for instance, by downstream testing techniques such as
%test selection or prioritization), a likely indication of poor test construction; and,
%\item reporting of spurious bugs.
%% if the tests are intended to be dependent but the dependence
%%is undocumented. \todo{DN}{I'm thinking of removing this bullet from the intro.
%%It's a bit different, in that the test writers understand the dependence, of course.
%%I'm just afraid that it'll increase confusion instead of clarifying things.}
%\end{itemize}

%As an example of the first category, the JodaTime library
%(Section~\ref{sec:jodatime}) defines a complex caching system.  Its
%test suite includes tests that check the rather complicated function that normalizes
%object states into cache keys.
%However, an unexpected test dependence between two tests masks a bug in this code. The default
%test execution order exercises an unintended path for one test because an object is
%already cached due to a previously executed test; the fault is exposed if that object
%is not initially cached, as would happen
%if the two tests
%are executed in the reverse order.
%Examples of the other categories are
%described in Section~\ref{sec:examples}.

%Test suites with unknown dependent tests may also exhibit unexpected
%results if reordered by downstream testing techniques such as
%test selection or prioritization  In addition, undocumented
%but desired test dependence can
%lead to spurious bug reports.

\subsection{Test Execution Environment}

Our examples highlight varying execution environments as the
unsurprising central cause of test dependence. Specifically, when a
test is executed in different environments---global variables
with different values, differences in the file system, differences in
data obtained from web services, etc.---it has the potential to return
a different result.  
%
%Changing a test suite's execution order can
%increase the potential to change the execution environment for a given test:
%different tests may be executed before that given test, and they may
%produce an environment that may cause the test to have a different result.
Most of the dependences we see in our
examples ultimately stem from incorrect or incomplete initialization
of the program environment.

Why does this happen? Especially given frameworks such as
%
%It is justified to argue that developers 
%should take the utmost care to initialize their tests correctly and completely. In principle,
%such setup code could be part of each individual test case. Frameworks
JUnit that facilitate the process of clean setup by providing means to
automatically execute methods (\code{setUp()} and \code{tearDown()} in JUnit
3.x, and methods annotated with \code{@Before} and \code{@After} in
JUnit 4.x) that should handle all common setup and clean-up between
test cases. 

It appears that the answer is that developers are as likely
to make mistakes when writing tests as when they are writing other code.
And while frameworks make it easier to get environment setup right, 
they cannot ensure that it is done properly. 
%The frameworks provide effective mechanisms for setting test execution
%environments, but they do not ensure that these mechanisms are used properly.
Like with other code, this means that tests in some cases will have
unintended and unexpected behaviors.  And as programs increase in complexity,
so may tests, which may increase the frequency of such problems in tests,
which may in turn increase the frequency of test dependence.
%And the more complex a programs structure
%is, the more likely it is that some initialization of global variables
%will be forgotten. By identifying test dependence
%as a more broadly discussed issue, and by providing algorithms and
%tools for identifying test dependences, we hope to reduce their


%Yet no framework can ensure that these methods are used
%correctly. 
%Since we are most interested in practical issues, we argue that
%developers are as likely to make mistakes when writing tests, as they
%are when writing code. 
%\todo{mark it red to avoid been overlooked}{frequency and their cost.}

%\subsection{Downstream Testing Tools}

Another situation in which the underlying test execution context can
unexpectedly change is when a tool or technique that takes a test
suite as input is used.  Examples of such techniques include
test selection techniques (that identify a subset of
the input test suite that will guarantee some properties during
regression testing)~\cite{harroldetal:OOPSLA:2001}, test prioritization techniques (that reorder the
input to increase the rate of fault detection)~\cite{Elbaum:2000:PTC:347324.348910}, test parallelization
techniques (that schedule the input for execution across multiple
CPUs), test factoring~\cite{Saff:2005} and test carving~\cite{Elbaum:2006} (which take system tests as
input and create unit tests and differential unit tests,
respectively), etc. 

Of these techniques, we are most concerned about
those that modify the organization of test suites, rather than the tests
they contain.
Many such downstream testing techniques implicitly assume that
there are no test dependences in the input test suite.  Our concern is
that this assumption can cause \emph{delayed problems} in the face
of latent test dependence in the input.  As an
example, test selection may report a subsequence of tests that do not
return the same results as they do when executed originally, as part of the full suite.
%In essence, these tools may have an unstated precondition---''the input
%must have no test dependences''---that may not be checked or satisfied in some cases.
%An alternative would be for the tools to detect and eliminate dependences.
%
%If this selection happened for regression testing, developers may be
%led to investigate only modified and newly added code to find the
%fault. But it is possible that the fault lies
%elsewhere and has not been discovered due to the dependence in the
%test suite. \todo{KM}{I don't buy this. Once you have the failing test, it
%should not be very hard for a reasonable developer to find the real reason
%behind the failure.}

%
%The two fundamental operations that such tools can apply to test
%suites are sub-suite selection, and suite reordering. Our
%formalization in Section~\ref{sec:formalism} and the examples in
%Section~\ref{sec:examples} show that both these operations can
%produce test suites that exhibit manifest dependences, because all
%these operations lead to tests being executed in potentially different
%environments. While there are some differences between selection and
%re-ordering, these are not significant for the following discussion. 
%Therefore, in Section~\ref{sec:related} we consider only
%related work in test prioritization as a representative of this
%entire class of techniques. %, especially due to its concern with reordering.
%

%The value of studying manifest dependence lies in the fact that
%it can impact second
%order techniques, such as test prioritization or parallelization.
%One premise of such techniques is usually that executing tests in any
%order or in parallel will produce the same results. Therefore, we
%consider as test dependence, effects that cause the results of tests
%to differ when they are executed in different environments.
%Such test dependence may arise when the test results rely on a particular
%context that may unexpectedly differ from one execution order to
%another (Figure~\ref{fig:downstream}).  
%For example, if test \code{A} assumes that a global variable has been
%initialized by some other test, executing test \code{A} before those
%tests may cause different test results.
%Conversely, tests are independent when
%they either do not rely on context at all, or assure the correct
%context before executing.



% \todo{KM}{I don't think what this paragraph says is true. I recommend cutting
% it (the above paragraph also contains what it tries to say).} Our examples also
% show extensive use of test infrastructure that can reduce test dependence: in
% JUnit\footnote{\url{http://www.junit.org}}, \code{setUp()} and
% \code{tearDown()} as well as \code{@before} and \code{@after}
% annotations help developers significantly but allowing them to more
% easily establish the execution environment.  However, these and
% related features are mechanisms only and are not intended to, and do
% not, enforce any policies to ensure that developers use the
% mechanisms consistently and effectively.
% 
% Why do some tests allow varying execution environments?  Can't this
% be easily avoided?
% We contend
% that this is for the same reasons that developers still create bugs,
% still don't always initialize program variables, still don't always check
% array lengths before indexing, etc.  By identifying test dependence
% as a more broadly discussed issue, and by providing algorithms and
% tools for identifying test dependences, we hope to reduce their
% frequency and their cost.

\subsection{Contributions}

At its heart, this paper addresses and questions conventional wisdom about the
test independence assumption.  It is intended to balance a precise 
characterization of test dependence with substantive empiric examples and concerns. The contributions of the paper include:
\begin{itemize}
%  \item Evidence from the literature that test independence is broadly assumed but rarely addressed (Section~\ref{sec:related}).
  \item A precise formalization of test dependence in terms of test suites as ordered sequences
  of tests and explicit execution environments for test suites, which enables reasoning about test dependence
  as well as a proof that finding \emph{manifest} dependences is an NP-complete
  problem (Section~\ref{sec:formalism}).
  \item Examples from fielded software of test suites where manifest test
  dependences lead to identifiable concerns with the underlying programs or test suites (Section~\ref{sec:examples}).
  \item Motivation for and presentation of our initial approximate algorithms and a supporting tool for identifying test dependences (Section~\ref{sec:algorithm-tool}).
\end{itemize}

%: in practice, it does not always hold and, 
%this can cause problems. 
%We provide evidence
%from fielded software that shows that, while perhaps uncommon, test
%dependence arises in practice. 
Although we provide evidence that test dependence unexpectedly arises in practice,  
a broad
study of how often test dependences arise, and the costs that these may lead to,
is beyond the scope of this paper. 
We conclude the paper with a set of open questions addressing this and
other possible concerns in Section~\ref{sec:questions}.

%While superficially straightforward, reasoning about test dependence
%%and the potential causes and consequences of test dependence is
%is intricate and non-trivial. We introduce a formalism to help
%understand and reason about test dependence.  The two key aspects
%of the formalism are (a) defining
%test
%suites as ordered sequences of tests and (b) making explicit the
%context in which tests are run. The formalism provides a precise
%basis for defining test dependence, for proving the (NP-hard) complexity
%of determining if a suite can manifest test dependences, and for
%algorithms that can efficiently identify important classes of dependences.
%
%
%may arise due to the context 
%in which tests are
%executed (Section~\ref{sec:formalism} formalizes context). Hence, the
%notion and use of context is the second fundamental part of our
%formalism.


%We raise awareness of the problems caused by the test
%  independence assumption but demonstrating through theory and
%  manifestations of test dependence that the consequences can be
%  severe. 
%  \item We define a formalism to reason about test suites as sequences
%  and show how dependences arise in theory and practice.
%  \item We lay a foundation for efficient heuristic algorithms to
%  detect dependences in existing test suites and show with some
%  examples that heuristics rather than exhaustive algorithms already
%  have signigificant benefits.


%\todo{JW}{I couldn't fit this into the rewritten intro. I might like
%to use in in Sec 4 or 5}
%The two ways
%  of altering context that we address here are \emph{isolation} and
%  \emph{ordering}.  By isolation, we mean executing each test in a
%  test suite separately: for example, in a different instance of JUnit
%  or in a different virtual machine.  This isolates, and may provide a
%  different context for a test, by ensuring that the initial context
%  is reinitialized for each test.  In contrast, most conventional
%  approaches execute tests in a sequence in the same context, giving
%  (for example) the second test an execution context that can in
%  principle depend in part on how the first test may have modified the
%  context.  By ordering (which as we show in
%  Section~\ref{sec:formalism} is strictly more general than
%  isolation), we mean that the sequence in which tests in a test suite
%  are executed can be varied.  A different ordering of test execution
%  can cause a given test to execute in a different context and,
%  perhaps, provide a different result.
%

%\begin{itemize}
%  \item Why do we think knowing about dependencies is important?
%
%Give a neat, clear, if constructed example.
%
%\item Are they a real problem?
%
%  \item What do we do about them?
%\end{itemize}

\section{Related Work}
\label{sec:related}

%Much literature on testing omits details about the structure of a
%test suite; this is reasonable and unsurprising because particulars
%about test suites are often immaterial to the results or discussions.
Denoting a group of test cases as a ``suite of test programs'' began around the
mid-1970's~\cite[p.~217]{brown:CSUR:1974}; similar terms include
``testcase dataset''~\cite{milleretal:ICRS:1975} and ``scenario,''
which an IEEE Standard defines as ``groups of test cases;
synonyms are script, set, or suite''~\cite[p.~10]{IEEE:829-1998}.
Treating test suites explicitly as \emph{mathematical sets} of tests dates at least
to Howden~\cite[p.~554]{howden:ToC:1975} and remains common in the literature.
The execution order of tests in a suite is usually addressed implicitly
or informally, suggesting that the potential of executing a given test
in different contexts is immaterial to those results: that is, test
independence is assumed.

%When a test suite is a set, the test suite passes if and only if all
%the test cases in it pass. 
%Combined with the unordered property of sets this suggests that the
%context in which each test is executed must have no
%effect on its result regardless of the ordering of execution.

%example~\cite[\emph{et alia}]{eldredetal:1978,harroldetal:OOPSLA:2001,staatsetal:ICSE:2011}.  

\subsection{Test Dependence}

In addition to the work by Kapfhammer and
Soffa~\cite{kapfhammeretal:FSE:2003},
there are a handful of categorical references that
acknowledge that tests can
be dependent based on context, suggesting 
ways to document or find situations where the independence
assumption fails to hold.  

The IEEE Standard for Software and System Test
Documentation (829-1998) \S 11.2.7, ``Intercase
Dependencies,'' says in its entirety: ``List the identifiers of
test cases that must be executed prior to this test
case. Summarize
the nature of the dependences''~\cite{IEEE:829-1998}.  The succeeding version of this
standard (829-2008) adds a single sentence: ``If
test cases are documented (in a tool or otherwise) in the order in
which they need to be executed, the Intercase Dependencies for most or
all of the cases may not be needed''~\cite{IEEE:829-2008}.

McGregor and Korson discuss interaction tests that
are intended to identify ``two methods that may directly or indirectly
cause each other to produce incorrect results'' and suggest constructing such
interaction tests by identifying the values shared via parameter passing
between methods
 that two or more test cases share~\cite[p~.69]{mcgregoretal:CACM:1994}.

Bergelson and Exman characterize a form of test dependence
explicitly: given two tests that each pass, the composite
execution of these tests may still
fail~\cite[p.~38]{bergelsonetal:EEE:2006}.  That is, if 
\suite{t_1} executed by itself passes and \suite{t_2} executed by itself passes,
executing the sequence \suite{t_1, t_2} in the same context may fail.

Some practitioners acknowledge test dependence as a possible, albeit low probability, event:
\begin{quote}
Unit testing \dots  
requires that we test the unit in isolation. That is, we
want to be able to say, \emph{to a very high degree of confidence\/} [emphasis added], that
any actual results obtained from the execution of test cases are
purely the result of the unit under test. The introduction of
other units may color our results~\cite{unit-test-def}.
\end{quote}
They further note that other tests, as well as stubs and drivers, are
other units that may ``interfere with the straightforward
execution of one or more test cases.''

A few approaches allow developers to annotate dependent tests and
provide supporting mechanisms to ensure that the test execution framework
respects those annotations.  DepUnit\footnote{\url{https://code.google.com/p/depunit/}}
allows developers to define soft and hard dependences. Soft dependences control
test ordering, while hard dependences in addition control whether specific tests are
run at all.  TestNG\footnote{\url{http://testng.org/}} is a testing framework intended to improve upon JUnit,
and allows dependence annotations and supports a variety of execution policies such as sequential execution
in a single thread, execution of a single test class per thread, etc.\
that respect these dependences.
What distinguishes our work from these approaches is that, while they allow dependences
to be made explicit and respected during execution, they do not help developers
\emph{identify} dependences.  A tool that finds dependences (Section~\ref{sec:tool}) could co-exist
with such frameworks by generating annotations for them.



%Testing frameworks
%%, and 
%%IBM's Rational tool family, and Microsoft's MSDN---
%provide mechanisms
%to help developers define the context for tests more effectively.
%JUnit\footnote{\url{http://junit.org}},
%for example, provides means to
%automatically execute methods (\code{setUp()} and \code{tearDown()} in JUnit
%3.x, and methods annotated with \code{@Before} and \code{@After} in
%JUnit 4.x) to help handle common setup and clean-up tasks between
%tests. Ensuring that these mechanisms are used properly, however, is
%beyond the scope of any framework.
%\todo{DN}{I stole some of the above paragraph from the intro.  We will have
%to accommodate that by removing one and forward/backward references or such.}

%\todo{JW}{During my first read, I'm not sure why this is here and how
%it relates to the previous discussion}\todo{DN}{You're right, I'm taking it
% out.  And both
%of these todos.}


%Characterizing test suites as ``collections'' of test cases is
%increasingly common, especially in descriptions of tools.  For example,
%the Javadoc for the JUnit \code{TestSuite} class includes: ``A TestSuite is a Composite
%of Tests. It runs a collection of test cases.''
%Sun's JUnit Primer says: ``A TestSuite contains a collection of
%tests\dots'', and similar
%terms are found in literature from IBM for the Rational tool family, from Microsoft's MSDN, etc.

%\todo{DN}{Removed the footnote and comments on fail/fail composing to pass.  I
%was wrong and I don't think it matters or is worth confusing things to cover
%it at this point in the paper.}
%\footnote{They
%also assert that if $T_{1}$ and $T_{2}$ each fail in isolation, that their composite
%execution will fail.  This is a variant of the test independence assumption,
%which we also believe is/show is false.  DN: Any examples of fail/fail to pass?
%\todo{KM}{David, I don't believe fail/fail can lead to pass with composition.
%The reason is the following: \\
%Let $\Gamma$ be the initial environment. \\
%T1 == fail implies that R(T1, $\Gamma$) = fail \\
%T2 == fail implies that R(T2, $\Gamma$) = fail \\
%So, whatever order you choose to run T1 and T2, you will at least get one
%failure (the first test that is run) because we know that: \\
%R(\{T1, T2\}, $\Gamma$) = \{fail, *\} and \\
%R(\{T2, T1\}, $\Gamma$) = \{fail, *\}.}}

\subsection{Test Prioritization}

%Test prioritization establishes ``an order for executing test cases in
%a prioritized manner that is most likely to detect software defects
%quickly''~\cite[p.~454]{Singh2001453}.  

Test prioritization seeks to reorder a test suite to detect
software defects more quickly, and is the example of downstream
testing tools that we focus on most closely both because it is
characteristic of the other tools (in the dimensions we address)
and also because of its focus on reordering (perhaps the most common
way to change the execution environment of a test).

Early work in test
prioritization~\cite{Wong:1997:SER:851010.856115,Rothermel:1999:TCP:519621.853398}
laid the foundation for the most commonly used problem definition:
consider the set of all permutations of a test suite and find the best
award value for an objective function over that
set~\cite{Elbaum:2000:PTC:347324.348910}.  The most common objective
functions favor permutations where more faults in the underlying
program  are found with running fewer tests.
%determine the number of tests from the permutations that need
%to be run to detect a set of faults in the underlying program.
A number of results carefully study various prioritization algorithms
empirically, most by Rothermel and colleagues,
spanning over a decade~\cite[\emph{et
alia}]{Rothermel:1999:TCP:519621.853398,Do:2010:ETC:1907658.1908088}.  These evaluations are based in part on the assumption that the set of faults in the underlying program is known beforehand; the possibility that test dependence may unmask additional faults in the program is not studied.
% \todo{DN}{I can't get the et alia above to look right -- where does that semicolon come from?  I'd like it to be [2,9, et alia] or such.
% In a pinch, just include the two numbers.}

Kim and Porter proposed a 
technique that uses the history of test cases run in prior
regression tests to prioritize those that have not yet run for
creating new 
regression test suites~\cite{Kim:2002:HTP:581339.581357}.  Whether
tests were executed is essential to the technique; the results
of specific tests are not.

Echelon defines a
heuristic that exploits both a mapping between tests and
executed program paths and also a binary differencing between two
program versions to select a subset of tests intended to quickly
identify program faults~\cite{Srivastava:2002:EPT:566172.566187}.
Test dependence is not considered in the approach.



Test independence is explicitly asserted as a requirement for
prioritization by Rummel et al.:
\begin{quote}
A test suite contains a tuple of tests \suite{T_1 $\ldots$ T_R} that execute in a specified order.  We require that each test is
independent so that there are no test execution ordering dependencies.  This requirement enables our prioritization algorithm to
re-order the tests in any sequence that maximizes the suite's
ability to isolate defects.  The assumption of test dependence
is acceptable because the JUnit test execution framework
provides \code{setUp} and \code{tearDown} methods that execute before
and after a test case and can be used to clear application state~\cite[p.~1500]{Rummel:2005:TPR:1066677.1067016}.
\end{quote}

%
%when re-executing it in the replay environment. that is the same
%related point as the test factoring work like david's note above.
%%also, 
%I suspect that Elbaum may point that out if he is reviewing this
%paper. I experienced that once, even getting reviews requesting to
%cite a much less related paper of his.



%\subsection*{Test Suites as Collections}


%and it is even the
%characterization of test suites used in Wikipedia (which also contradicts the IEEE Standard by stating that
%``[c]ollections of test cases are sometimes incorrectly termed a test plan, a test script, or even a test scenario.'')~\cite{wiki:test-suites}.
%It has been difficult---and seems unnecessary for our results---to identify precisely when ``collections''
%came into use for test suites.  It seems likely that this arose from the now-common
%terminology for containers that group multiple elements together: sets, lists,
%hash tables, arrays, etc.

\subsection{Syntactic and Semantic Test Dependencies}

\emph{Dependences} in testing are most often considered
to be syntactic dependences between program units, for example
methods calling other methods, and classes using other classes~\cite{bergelsonetal:EEE:2006,briandetal:SEKE:2002}. 
\emph{Syntactic} dependence here means that a unit \code{A} cannot be
compiled and executed without unit \code{B} being present. If we test
such a unit \code{A} without convincing ourselves first that \code{B}
is correct, a test failure for \code{A} is harder to interpret,
because it could just as well indicate a fault in \code{B}.

Zhang and Ryder extend this notion to 
\emph{semantic} dependences,
which is closer to our approach~\cite{zhangetal:TR:2006}. 
They use a notion of
``test outcome'' to determine whether or not syntactically dependent
classes or methods can influence each others results, and consider
only those that can to be semantically dependent.
They give an informal definition of what it means for the execution of a
test to influence the outcome of another test.  We define
this precisely, and we also define manifest test dependence in terms
of execution environments
and test execution order rather than in terms of code use.

Santelices et al.\ define a formal model of how changes might interact
at the source code level and present a technique for detecting such
interactions that arise at
run-time~\cite{Santelices:2010:PDR:1828417.1828487}.  In contrast to
our approach, they identify changes that interact rather than tests
that depend upon each other.

%There are results that identify dependences that
%may surface, for example, when there is aggressive testing of parameter
%settings by a single test case; one (of many) approaches uses
%bounded model checking to vary the parameters~\cite{Sullivan:2004:SAB:1013886.1007531}.



%\todo{DN}{Young says: ``If you are referring to the work I am familiar with, I think what has
%been treated in some depth is combinations of parameter settings in a
%single test case.  That's a sort of dependence as well (e.g., the
%parameters might enable a couple of features that interact in nasty
%ways), but it doesn't involve variations in the ordering of operations
%or test cases.''  I can't find anything on this, but I'm probably searching
%wrong.  If something finds a good citation or two, please include it and write
%something here.  If not, delete this entirely.}

Another kind of dependence helps address
the testing of configurable software, which can be combinatorial with respect to the
set of configurable options~\cite{Cohen97theaetg,Cohen:2003}.  
%In
%practice, there are often far fewer
%configurations that are used and thus should be tested.  This
%often structures the configuration space in a way that allows
%potential dependences to be explored more
%efficiently, as a test suite that binds multiple configuration option values
%can test all configurations that share those settings.
The dependences considered in this approach are not between tests, but rather within
the configuration option space.


%Another set of results searches for
%efficient (small) test suites that aggressively exercise potentially unexpected
%interactions---dependences---among the components of the
%program~\cite{Cohen97theaetg,Cohen:2003}.  These approaches are used for configurable software where
%in principle there are a combinatorial set of instances to test based on variations of the
%configuration options.  The objective is to temper the combinatorial blow-up by identifying
%components sharing some configurable option values; for example, 
%if a block of code is included in such system instances, a test that exercises that block
%can be used as a proxy for interactions between the shared options.
%
%These approaches tend to use covering arrays as a way to determine which of
%the dependencies among components are/are not exercised.  In the case of configurable
%software, there may be a combinatorial number of interactions to be considered.
%These approaches focus on generating a set of effective tests based on program interactions,
%rather than our focus on identifying dependent tests based on ordering, environment, and the
%results of the tests.
%
%
%In principle it could be combinatorial, and the idea of covering
%arrays is that most of the interactions that matter involve just a
%small number of choices.  If there are things that break only for a
%single setting of parameters A, B, C, D, E, F, G, H, we're hosed $A!-(B but
%if something breaks whenever B has value 1 and G has value 2, then
%something like covering arrays has a chance.  



%
%The research we propose is basic research that impacts most
%aspects of testing, ranging from (automatic) test generation through
%regression testing, test case selection and ending with
%considerations on the right test granularity. 
%Similar to the distinguished paper of Staats et
%al.~\cite{staatsetal:ICSE:2011}, we propose to give a rigorous
%foundation to an important aspect of software testing that is
%present but rarely examined in detail in current research.


%\section{What's the essence of what causes the dependences?}
\section{Theory}
\label{sec:formalism}

%\todo{DN}{I think we will need to go over the paper later on in the process to make
%really sure we are consistent about test dependence definitions and our writing.
%If I am clear, we define dependence between two tests; we often (informally?)
%talk about a suite with dependence(s).}
%\todo{JW}{Informally, you are correct. I'll make sure to clarify this
%in the text}
%\todo{DN}{We are also inconsistency about ``dependence'' vs. ``dependency'' vs. ``dependencies'' and
%such.  Probably not a big deal, but if we have time...}
%\todo{DN}{General comment about the intuition below -- it's too long now, which makes it less intuitive.}

A standard textbook 
states that ``[a] test case includes not only input data but
also any relevant \emph{execution conditions}
\dots''~\cite[p.~152, emphasis added]{pezze-young:2007}.   
This characterization is
consistent with the example that 
 piqued our interest in test dependence: we seren\-dip\-itously identified a bug in an open-source
system when we found that running individual tests one-by-one---each
in a newly initialized environment---produced different
results from running the entire test suite normally (\emph{i.e.}, with a single initialized environment followed by
the sequential execution of each test in order)~\cite{DBLP:conf/sigsoft/MusluSW11}.  These tests
shared global variables, and the test results varied depending on the
values stored in these variables.  That is, relevant execution conditions---specifically, pertinent parts
of the implicit \emph{environment} comprising global variables, the file system, operating system services, etc.---were neglected.
%As shown in Section~\ref{sec:examples}, most real examples of test dependence we have seen to date relate
%to mistakes in the initialization of test environments.
%\footnote{Test frameworks such as JUnit 
%provide \emph{mechanisms}, such as \code{@Before} and \code{@After} annotations, to help developers more easily
%establish execution conditions.  They are not intended to, and
%do not, \emph{enforce} any policies
%to ensure that developers use these mechanisms consistently and effectively.}

To characterize the relevant execution conditions precisely, 
our formalism below explicitly represents the notions of
(a) the order in which test cases are executed and (b) the environment in which a test suite is executed.  

Consider two examples 
of how test dependences arise in terms of order and environment (Figure~\ref{fig:dep_examples}).
In the leftmost example, \code{test2} checks
the value of a variable that has been assigned elsewhere. If the tests
are executed in the order \suite{\code{test1},\code{test2}}, both tests will pass,
% Kivanc: Cut: There is no \ldots in test1, so I don't think below explanation
% is needed.
% (assuming \code{test1} does not
% change the value of \code{a} after the initial assignment)
while running \code{test2} first will make it fail.    The rightmost  example extends this
principle to multiple tests. While none of the $n-1$ tests prior to
\code{testn} will fail, they all must execute in this particular order
for \code{testn} to pass. 

The global variables involved are usually buried deep in
the program code, and the assertions do not directly check them,
but rather check values that have been computed from
them. In any non-trivial real-world program, this
deep nesting effectively hides potential dependencies from developers,
and they may only become aware of them when a subtle bug leads them
there.  Therefore, we explicitly
distinguish potential test dependences (Definition~\ref{def:dependency})---those that could cause a variation in test suite results 
under \emph{some} environment and order---and manifest test
dependences (Definition~\ref{def:manifest})---those that are guaranteed to cause a
variation in test suite results under a \emph{specific} environment and order.  

%Consider the two examples (Figure~\ref{fig:dep_examples})
%of the basic
%way dependencies between tests arise in practice. \code{test2} checks
%the value of a variable that has been assigned elsewhere. If the tests
%are executed in the order \code{test1, test2}, both tests will pass (assuming \code{test1} does not
%change the value of \code{a} after the initial assignment),
%while running \code{test2} first will make it fail.  To determine
%potential test dependences would require an analysis of the read/write
%behavior of the tests (for example, under what conditions, if any, does \code{test1} change the value of \code{a}?); such analyses are well-known to be
%imprecise and/or incomplete.    


% to cleanly set all execution conditions for each test
%case. Methods marked with the
%annotations \code{@Before} and \code{@After} are executed before and
%after each test case and are intended to initialize and clean the
%execution conditions for each test case.
%}

%When not all state in the environment is cleanly initialized, test
%dependences can arise, because then the actual state when a test case
%executes can change depending on, for example,


%Very informally, dependence between tests, like the one described above, arises when test cases do not include all% relevant execution conditions.
%In particular, when they compute their result based on a shared global data, and this shared global data (we call it \emph{environment}) is initialized external to the test case.

%\todo{DN}{I'd love to see if we can reduce or eliminate the discussion of JUnit here.  The following
%paragraph, for example, seems confusing in the following sense: if junit has this, why is there a
%dependence problem?  This is obvious to us, but I don't want people thinking about that here. Moved it
%to a footnote -- before/after -- and rewrote a little}
%

%In this section we first give an intuition what causes test
%dependence and what its consequence may be. Then we formally define
%our notion of test dependence, and lastly we demonstrate how these
%notions lead to the \emph{test dependence detection} problem, and how
%we can solve it.

%\todo{JW}{Find a better section heading or remove the heading}
%\subsection{Intuition}



%
%rely on executing in a
%particular context, for example some test may assume that global variables have
%been initialized to specific values, without confirming the expected
%context before they execute. 

%\todo{DN}{I'm at present inclined to move the example to here, but the test structure/results
%text.  That is, use the example to describe those rather than vice versa.  Sort of like: (a) here
%is a simple example; (b) note that a complete analysis of test dependence would require a full
%analysis of the code in test1 to determine possible values for a; (c) instead of considering that
%complicated analysis that would describe the potential for dependence, we use the test oracles/results
%to represent the outcome of the execution, in a given environment.  Or possibly move the
%potential/actual text up instead, and describe how we address it, then the example?}




%Consider the two examples in Fig.~\ref{fig:dep_examples}. The example
%in Fig.~\ref{fig:dep_examples:direct} shows the most basic
%way dependencies between tests arise in practice. \code{test2} checks
%the value of a variable that has been assigned elsewhere. If the tests
%are executed in the order \code{test1, test2}, both tests will pass,
%while running \code{test2} first will make it fail.
\begin{figure}
\subfigure[Direct dependence\label{fig:dep_examples:direct}]{
\begin{minipage}{.47\columnwidth}
\code{test1 \{ $\mathtt{v_1}$ = 4 \}}

\code{test2 \{
  assert $\mathtt{v_1}$==4 \}
}

\mbox{ }\\

\mbox{ } 
\vspace{0.5em}
\end{minipage}
}
\subfigure[Chain dependence\label{fig:dep_example:chain}]{
\begin{minipage}{.5\columnwidth}
\code{test1 \{ $\mathtt{v_1}$ = 1 \}}

\code{test2 \{
  $\mathtt{v_2}$ = $\mathtt{v_1}$ + 1 \}}

\code{...}

\code{testn \{ \\
\mbox{}\hspace{1ex} assert $\mathtt{v_{n-1}}$ == n-1 \}}
\vspace{0.5em}
\end{minipage}
}
\caption{Examples for basic causes of test
dependences}\label{fig:dep_examples}
\end{figure}
%

%\todo{SZ}{for example (b), perphas we can make it more clearer as
%follows:  test1 \{a++\}, test2\{a++\}, ..., testn \{assert a == n-1\}}
%
%\todo{JW}{While this is also a dependence, it does \emph{not} enforce
%any particular order on the first n-1 tests.}
%
%\todo{KM}{I second Jochen. We discussed this quite a bit with him and I think
%he came up with the most readable and easiest example that forces the execution
%of only t1-t1-\ldots-tn to pass and all other orders to fail.}

%Test dependences arise when tests rely on state that is generated
%by other tests. %CLI is a case in point. 
%Most examples we found are quite direct dependecies on
%global variables, where one test implicitly relies on a global variable to be in
%a certain state before executing the sequence of methods to be tested. 
%At a more abstract level, we can think of test dependence as
%read/write conflicts between different transactions. Each test reads
%and writes variables. If a test implicitly assumes a variable to be in
%a particular state, but does not ensure this state before it executes,
%the test may fail.
%\todo{JW}{
%Thinking of the causes for test dependences as read/write conflicts
%might go far. Intuitively, I think the cases that cause parallel
%transactions to abort are the \emph{good} cases for us (because each
%test ensures that it has written the values it needs). All other cases
%seem potentially hazardous. I'll dig into this tomorrow}

\subsection{Definitions}
\label{sec:definitions}

We express test dependences through the results of executing
\emph{ordered} sequences of tests in a given \emph{environment}.

%This is in strong contrast to most
%existing work that considers test suites in general as sets, and thus
%ignores the ordering aspect that is important here.
%Informally, a test dependence arises when the results of executing
%test suites in different orders differ. The following formal
%definitions make our notion of test dependence precise.
%\todo{DN}{I took out the ``informally'' sentence or so here, since we've done that
%already a number of times, and now we're going formal!}

\begin{definition}[Environment]
An \emph{environment} \env for the execution of a test
consists of all values of global variables, files,
operating
system services, etc. that
can be accessed by the test and program code exercised by the test
case.
%
%The set of all possible environments is denoted $\environs$.
\end{definition}

\begin{definition}[Test]
%\todo{DN}{I'm not sure what ``fixed, well-defined inputs means'' and I'm not sure we need it.
%Why not just ``a sequence of program statements''?}
%
%\todo{JW}{Abstractly, all programs take inputs and compute something
%based on them (unless they are a constant function or
%non-deterministic). A test case is not complete without well-defined
%inputs, simply because you can't run the program. The should be
%well-defined and fixed, so that you can repeatedly run your test,
%always get the same result, and be sure what that result should be
%according to your spec. While I agree that this is not well
%formulated, I think if we ask serious testing people to read this and
%we ignore the whole input thing (a lot of test generation only deals
%with generating inputs), we might be in trouble.}

A test is a sequence of program statements, executed with fixed,
well-defined inputs, and an oracle that
decides whether a test passes or fails.
\end{definition}

%\todo{DN}{Should we footnote this next oracle discussion?  And shorten it, since some of the
%soundness/completeness issues are with respect to specifications, which we don't mention/address.
%Should we?}

%\todo{JW}{I realized while I wrote this that eventually we will have
%to include oracles. That's why I at least wanted to mention something
%here. But I don't think it will be possible (and sensible) to do fully
%integrate that now. The paragraph below is a summary of why we need to
%talk about this.}

%Generally, oracles are an important aspect of testing. Here, however,
%only two facts are relevant. First, Staats et al.\ discuss that oracles are
%often neither sound nor complete as we
%define them below often imply unsound oracles. Second, 
%In particular, that means that an oracle can decide that a
%test passes, while the program is incorrect. Test dependences
Simplifying from Staats
et al.~\cite{staatsetal:ICSE:2011}, and without loss of generality,
we consider an oracle to be a boolean predicate over tests and environments.

%While oracles in practice, and specifications in theory, play an
%important role in testing, we do not incorporate them in our
%formalism, because explicit specifications often do not exist, and for
%our purposes the oracle judgement, rather than its full definition, is sufficient.

\begin{definition}[Test Suite]
A test suite\/ $T$ is an $n$-tuple (i.e., ordered sequence) of tests
\suite{t_1, t_2, \dots, t_n}.

%When it is clear which test suite we are talking about, or the details
%of the suite are not important, we use $T$ to denote the entire test
%suite $(t_1, \dots, t_n)$.
\end{definition}

\begin{definition}[Test Execution]
Let\/ $T$ be a test suite and\/ \environs\ the set of all possible
environments.
The function\/ $\varepsilon: T \times \environs \rightarrow
\environs$ is called test
execution. $\varepsilon$ maps the execution of a test\/ $ t \in T$ 
in an environment\/ $\env \in \environs$ to the new (potentially updated)
environment\/ $\env'$.

For the execution of test suites\/ $T = \suite{t_1, t_2, \dots, t_n}$
we use the shorthand\/
$\exec{T}{\env}$ for $\exec{t_n}{\exec{t_{n-1}}{\dots \exec{t_1}
{\env} \dots }}$.
\end{definition}

\begin{definition}[Test Result]
The result of a test $t$ executed in an environment\/ $\env$,
denoted\/ \result{t}{\env} (and sometimes referred to 
as an oracle judgment), is defined by the test's oracle
and is either \pass or \fail.

The result of a test suite\/ \suite{t_1,\dots,t_n}, executed in an
environment\/ \env, denoted\/ \result{\suite{t_1,\dots,t_n}}{\env} is a
sequence of results\/ \suite{o_1,\dots,o_n} with $o_i \in \{\pass,\fail\}$.

%For test outcomes of sequences where all individual outcomes are
%either \pass or \fail, we use the notation $(\pass^*)$ and $(\fail^*)$,
%respectively.

For example, $\result{\suite{t_1, t_2}}{\env_1} = \suite{\fail, \pass}$ represents that 
given the environment\/ $\env_1$, $t_1$ fails and\/ $t_2$ passes.
\end{definition}


%\todo{DN}{Do we need the following paragraph?  I think the dot notation is unnecessary,
%since it's always a test suite, even if not the ``original'' one that we put in this
%place.  Also, I would -- again -- like to remove the references to Junit, VMs, etc. here.}
%
%The notation \result{\cdot}{\env} implies that only the explicitly specified
%tests are run in a given VM, execution, and environment \env. 
%In specific frameworks, such as JUnit, a single test would 
%include any automatically executed setup and the actual test method.

%The terminology for \pass and \fail is general:
%a test passes when its outcome is as expected and
%no assertion is violated; a test fails under all other circumstances.\footnote{In
%the JUnit framework, for example, this means that the outcomes \emph{failure} and \emph{error}
%would both be treated as \fail in the formalism.}

%The execution of tests can change the environment if one of the
%variables in it is modified. We write \exec{T}{\env} for the environment
%that derives from executing test suite $T$ in the original environment
%$\env$. $\exec{T}{\env} \neq \env$ is a necessary but not
%sufficient condition for test order dependencies.

\begin{definition}[Potential Test Dependence] \label{def:dependency}
Given a test suite\/ $T$,
a test\/ $t_l \in T$ is \emph{potentially dependent} on test\/ $t_k
\in T$, if and only if\/
$\exists \env : \result{T}{\env} = \suite{o_1,\dots, o_n} \wedge
\result{\suite{t_k,t_l}}{\env} = \suite{o_k, o_l} \wedge
\result{t_l}{\env} = \neg o_l$.
We write\/ $t_k \prec t_l$ when\/ $t_l$ is potentially dependent on\/ $t_k$.
\end{definition}

This definition is \emph{dynamic} because dependence arises only
if there exists an environment in which actual test results would differ.
It is \emph{potential} as it only requires the existence of such an
environment, but does not
guarantee that the test suite will ever be executed in the context
of such an environment.

%One direct implication of this definition is that $t_k \prec t_l
%\rightarrow \exec{t_k}{\env} \neq \env$, that is $t_k$ must modify the
%environment. Additionally, $t_l$ must read values from the
%environment.
%\todo{DN}{I'm not certain if the rightarrow is implication.  I'm not sure
%why this implication is crucial.}

%\begin{theorem}\label{theorem:pairs}
%
%$( t_1, \dots, t_k ) \prec t_n \Rightarrow \exists_{t \in (t_1, \dots,
%t_k)} : t \prec t_n $
%
%\end{theorem}
%
%\begin{proof}
%Proof by contradiction.
%
%Let $s = (t_1, \dots, t_k)$ be a test sequence of length greater than
%one, and let $s$ be the shortest sequence such that $ s \prec t_n$.
%I.e. there is no real prefix or suffix of $s$ that $t_n$ depends on.
%
%That means that there is an environment $\env_x$, such that $R(s,t_n :
%\env_x) \neq R(s:\env_x) \circ R(t_n:\env_x)$ 
%and for all environments $\env$
%$R(t_2, \dots, t_k, t_n:\env) = R(t_2,\dots, t_k:\env) \circ
%R(t_n:\env)$. In particular, this holds also for the environment
%$ \env_1 = \gamma(t_1, \env_x)$. 
%
%$R(s,t_n :\env_x) \neq R(s:\env_x) \circ R(t_n:\env_x)$
%and $\env$
%$R(t_2, \dots, t_k, t_n:\env_1) = R(t_2,\dots, t_k:\env_1) \circ
%R(t_n:\env_1)$ implies that $R(t_1,t_n:\env_x) \neq R(t_1:\env_x) \circ
%R(t_n:\env_x)$, which contradicts the
%hypothesis that there is no prefix of $s$ that $t_n$ depends on.
%\end{proof}
%
%Theorem~\ref{theorem:pairs} is useful for theoretical work because it
%reduces the complexity of the structures we have to study to pairs of
%tests. From a practical point of view, however, things are not quite
%as easy, because the proof only states the existence of environments
%that will expose the dependency, but does not constructively describe
%how to find these environments. Further, the whole theoretical
%framework by its nature cannot relate this to the way environments are
%constructed by actual testing frameworks such as JUnit. 
%
%With the above proof it is easy to see that the following corollary
%also holds:
%
%\begin{corollary}
%$t_k \prec (t_m, t_n) \Rightarrow t_k \prec t_m \vee t_k \prec t_n$
%\end{corollary}
%
%\begin{proof}
%Case 1: If the outcome of $t_m$ differs, it directly implies $t_k \prec t_m$.
%
%Case 2: The outcome of $t_m$ is the same, but $t_n$ differs.
%$ t_k \prec (t_m,t_n) \Rightarrow (t_k, t_m) \prec t_n \Rightarrow t_k
%\prec t_n$ 
%
%\end{proof}
%
%The above definitions and the properties of order dependencies that
%follow from these definitions allow some reasoning about order
%dependencies already. From a practical point of view, however, it
%would be desirable that the relation implied by the order dependence
%definition were a partial order. However, I do not think that that is
%true. The following theorem frames this:
%
%\begin{theorem}[Cyclic dependencies]
%There exist test suites $S$ such that for some $t_m, t_n \in S: t_m
%\prec t_n \wedge t_n \prec t_m$
%\end{theorem}
%
%This is an artifact of the fact that we allow reordering and our
%definition of order dependency hinges on the \emph{outcome} of
%executions.
%
%Intuitively, test dependencies $t \prec r$ between two tests $t$ and $r$ arise
%when $r$ expects some variable $v$ of the environment to have a specific
%value, but in the environment $\gamma(t,\env)$ $v$ has a different
%value (for some arbitrary $\env$).
%This situation can arise either by $r$ assigning an unexpected value
%to $v$, or $r$ failing to assign the expected value.
%
%\subsubsection{Manifest Dependencies}
%
%
%For practical reasons, the pure existence of an environment that
%exposes dependencies is not sufficient. Most of the time we are more
%concerned whether or not dependencies will become apparent in the
%environment we are actually dealing with.
%
%\todo{JW}{ We are aiming at defining enough theory for useful
%algorithms. In practice we derive our environments from some sort of
%default environment provided by the test framework. While we don't
%know the exact shape of that environment, assuming deterministic
%programs and tests, all other environments created through test
%execution derive from this initial environment. This must have some
%sort of impact on the complexity of our problem.}

%\begin{definition}[Subsequence]
%Given an ordered se\-quence $T = \suite{t_1, \dots t_n}$, an ordered
%sequence $S = \suite{t_i,\dots,t_k}$ is a subsequence of $T$, if and
%only if the length of $S$ is less than or equal to the length of $T$,
%the elements of $S$ are also elements of $T$ and 
%preserve the order of the elements in $T$.
%Analogously to sets, we write $S \subseteq T$ if $S$ is a subsequence of
%$T$.
%\end{definition}

We refine this definition of dependence to require a concrete environment guaranteed
to \emph{manifest} a dependence:
\begin{definition}[Manifest Dependence] \label{def:manifest}
Given a test suite\/ $T$, two dependent tests\/ $t_i, t_j \in T$,
the dependence\/ $t_i \prec t_j$ \emph{manifests} in a given
environment\/
$\env$ if\/ $\exists {S \subseteq T}: t_i, t_j \in S \wedge
\result{T}{\env}
= \suite{o_1, \dots, o_n} \wedge \result{S}{\env} =
\suite{\dots,o_i,o_j} \wedge \result{t_j}{\env} = \neg o_j$. We
write\/ $t_i \manifest{\env} t_j$ for manifest dependence.\footnote{$S \subseteq T$ means that $S$ is a subsequence of
$T$.}
\end{definition}

Note that the dependent tests $t_i$\/ and $t_j$ do not have to be
adjacent in the original test suite, but that they must be adjacent in
the shortest test suite that manifests the dependence.

The intuition behind manifest dependences is that in practice we
do not construct arbitrary environments to execute tests in. Rather,
we use the natural environment $\env_0$ provided by frameworks such as JUnit,
and the only modifications of this environment happen through the
tests and the tested code. Hence, potential dependences manifest only
if there is a sequence of tests $S^*$ whose execution
$\exec{S^*}{\env_0}$  produces the
environment $\env'$ that will reveal the dependency.
The algorithm we propose in Section~\ref{sec:algorithm-tool} detects
dependences by running tests and checking for different test results,
hence it can only detect manifest dependences.
%
%Note that this definition defines manifest dependencies with regard to
%a given test suite $T$ and a given environment $\env$. Thus it
%reduces the space of possible dependencies considerably. In practice,
%the given environment $\env$ is what the text execution framework
%provides \emph{before} it executes the first test.
To improve algorithms that are affected by test dependences, we are
interested in the shortest test suite $S^* \subseteq T$ that manifests a
dependence, because these define the partial order of test execution
that such techniques must respect.

%\begin{theorem}[Dependency bound]
%Let $U$ by a true sub-sequence of length $k$ of 
%a given test suite $T$, and let $U$
%be the shortest such sub-sequence that manifests a dependency with
%test t, i.e. $ U \manifest{\env} t$.
%Then there are at least $k$ variables responsible for manifesting the
%dependency.
%\end{theorem}
%
%\todo{JW}{ I'm not sure if this theorem actually holds. I'm working on
%it, but if anyone has ideas and insights, they are certainly welcome.}
%\todo{KM}{ I don't think this is true, consider the following: \\ 
%Initial environment: a = 0 \\ 
%Test1: {assert a == 3} \\
%Test2: {a = a + 1, assert true} \\
%Test3: {a = a + 2, assert true} \\
%In this setup Test1 depends on any order of Test2 and Test3. This is the minimum
%dependence (i.e., running only Test2 or Test3 won't suffice). However, the
%dependency only requires one variable, `a'. }
%
%\begin{proof}
%Proof by induction.
%Let $k=1$. Trivial. By definition there must be at least one variable
%that creates and manifests the dependency.
%
%Let $k=n+1$. Assume there are less than $k$ variables involved. 
%\end{proof}

In later sections we often talk about executing tests in isolation, or
executing all tests in a test suite in isolation. This is an important
approximation to detecting test dependences.

\begin{definition}[Test Isolation]
The result of executing a test\/ $t$ in isolation, given an
environment\/
$\env_0$ is the result\/ \result{t}{\env_0} of executing that test in
the given environment.  

The result of executing all tests in a test suite\/ \suite{t_1, \dots,
t_m} in isolation is the
sequence of results\/ \suite{\result{t_1}{\env_0}, \dots,
\result{t_n}{\env_0}}.
\end{definition}

\subsection{Detecting Dependent Tests}

From a practical perspective, techniques that affect the ordering of
test suites must respect dependences. Otherwise their results cannot
be interpreted correctly in the presence of dependences. Detecting
dependences in existing test suites is thus an interesting problem.
In the following, we first give a precise definition of the problem of
detecting dependent tests, and then prove that in general this problem
is NP-complete. In Section~\ref{sec:algorithm-tool} we outline an
algorithm that approximates solutions efficiently.

\begin{definition}[Dependent Test Detection Problem]
Given a set suite\/ $T = \suite{t_1, \dots, t_n}$ and an environment\/
$\env_0$, for a given test\/ $t_i \in T$, is there a test suite\/ $S
\subseteq T$ that manifests a test dependence involving\/ $t_i$? 
\end{definition}

%To prove that this problem is NP-complete,
We prove that this problem is NP-hard by reducing the NP-complete Exact Cover problem
to the Dependent Test Detection
problem~\cite{karp:NP:1972}. 
Then we provide a linear time algorithm to verify any answer to the
question.
%Then we sketch an exponential
%time algorithm that can solve the problem.
Together these two parts prove the the Dependent Test Detection Problem is NP-complete.

\begin{theorem}
The problem of finding a test suite that manifests a dependence is
NP-hard.
\end{theorem}

\begin{proof}
%We prove this claim by reducing Exact Cover to Dependent Test
%Detection.
In the Exact Cover problem,
we are given a set $X$ = \{$x_1, x_2, x_3, \dots, x_m$\} and a collection $S$ of subsets of $X$.
The goal is to identify a sub-collection $S^*$ of $S$ such that \textit{each}
element in $X$ is contained in \textit{exactly} one subset in $S^*$.  

Assume a set $V = \{v_1, v_2, v_3, \dots, v_m\}$ of variables,
and a set $S = \{S_1, S_2, \dots, S_n\}$ with $S_i \subseteq V$ for $ 1\leq i
\leq n$. 

We now construct a tested program $P$, and a test suite
$T = \suite{t_1, t_2, \dots t_n , t_{n+1}}$ as follows:

\begin{itemize}

\item $P$ consists of $m$ global variables 
$v_1, v_2,\dots, v_m$, each with initial value 1.

\item 
For $1 \le i \le n$, $t_i$ is constructed as follows:
for $1 \le j \le m$, if $x_j \in S_i$, then adding a
single assignment statement \CodeIn{$v_j$ = $v_j$ - 1} to $t_i$.

$t_{n+1}$ consists only of the oracle
\CodeIn{assert($v_1$ != 0 || $v_2$ != 0 \dots || $v_m$ !=0)}.

\end{itemize}

In the above construction, the tests $t_i$ for $1 \le i \le n$ 
will always pass. The only
test that may fail and thus exhibit different behavior is $t_{n+1}$, which 
\emph{only} fails when each variable $v_i$ appears exactly
once in a test case.

For the given test $t_{n+1}$, if we can
find a sequence \suite{t_{i_1}, t_{i_2},\dots, t_{i_j}}
that makes $t_{n+1}$ fail, the subsets $S^*$ corresponding
to each $t_{i_j}$ are an exact cover of $V$.
\end{proof}

In practice, the structure of the proof directly translates to the
structure of test suites. $t_{n+1}$ is the dependent test, $S$ is
defined by the tests that write variables used by $t_{n+1}$, and every
exact cover of $S$ represents an independent shortest test suite that
is a manifest dependency of $t_{n+1}$.

To complete the proof that Dependent Test Detection is NP-complete, we
provide an algorithm to verify solutions to the problem, that is
linear in the size of the test suite.
Given a test suite $T$ and a test suite $S \subseteq T$ that is said
to manifest a dependency on $t_i$, we first execute $T$, then $S$, and
compare the result for $t_i$ in both executions. 
If the results differ the solution is correct, if they do not differ,
the solution is rejected.
Since in the worst case we have to execute $2n$ tests, the complexity
of this algorithm is linear.




\subsection{Discussion}

This formalism has dual intents:
to lay a foundation for reasoning about test dependence
in a precise way; and
to be consistent with and to allow for approximate and practical algorithms and tools~(Section~\ref{sec:algorithm-tool}).

This second intent, of course, requires a balance of theory and
practice.  First, the dynamic nature of our our view on dependences 
allows us to avoid the complexity issues that come with a static
approach. With a static approach, it would be essential 
to decide how to address undecidability. The most
likely and common approach being to choose soundness with respect to all
possible executions and accepting the consequent imprecision of the analysis.
Second, our focus on manifest dependence, when realized in a tool will
only identify true positives, although it may miss some
dependences (false negatives).  It is often easier to have tools
with this kind of property accepted by practitioners than some other
kinds.  Third, the manifest test dependence problem is NP-complete;
although that is daunting (but less so than undecidability),
approximate algorithms can be defined for large classes of NP-complete
problems.  

The examples in the following section and the algorithm and tool following that
give a better flavor for
why we made these decisions.
The degree to which these are the ``right'' (or at least effective) decisions
is itself an empirical
question beyond the scope of this paper.

%.  The following section we will illustrate these factors in action.
%We discuss several examples in detail,
%showing how these different factors contribute to test dependence in
%real-world applications. 


%The potential for test dependence arises from the test structure and
%the oracle:
%%from the test results: 
%what
%global state do the tests read and write, and does that global state contribute
%to the computed result evaluated by the oracle?  
%At the same time, the \emph{potential} for a test dependence
%is realized only if 
%the values derived from the context \emph{actually} affect the test results.
%affect program state that
%is checked by the oracle can a dependence on the environment affect
%the outcome of a test.
%This potential \emph{manifests} when test results differ between
%executions in different environments.

%The fact that manifestation of test dependence depends on both test
%structures and test results means that dependences can silently propagate
%through sequences of tests before they become apparent.

%The abstract examples above, and the concrete examples presented 
%in Section~\ref{sec:examples} share some common features that
%ultimately lead to dependences. 
%%All applications and libraries we studied 
%They rely on global variables to some extent, and the
%tests that check behavior that depends on these variables usually
%assume these variables to be in some state. This state is typically
%defined by the default execution order of the test suite, and rarely
%established explicitly before each test.

%\todo{sz}{does that make sense to put the following text (needs slight revise)
% after Section 4. I feel they are more relevant to concrete examples.}

%To summarize, the features that contribute to the test dependences
%we discuss are:
%\begin{itemize}
%\item Test results depend on global state.
%\item Tests do not check their preconditions explicitly, but rely on
%the test suite to ensure them.
%%\item 
%%\todo{JW}{The following point has lead to a lot of confusion. We have
%to clarify or remove it}
%The strength of test oracles. Stronger oracles are more likely
%to cause dependence than weaker oracles. 
%%in the sense that they check for concrete
%%values rather than conditions. 
%For example, a check for $x = 5$
%rather than $x > 0$, is more likely to fail, while the latter might
%still be sufficient to check
%whether a specific branch of the program was executed. 
%\todo{KM}{I did not like
%this argument much. Though I agree that having weak oracles would reduce the
%dependences, it is still possible (in theory) to write dependent tests with
%weak oracles.}
%\end{itemize}



% vim:wrap:wm=8:bs=2:expandtab:ts=4:tw=70:



\section{Manifestations}
\label{sec:examples}

\newcommand{\unknown}{N/A\xspace}
\newcommand{\infy}{$\infty$\xspace}

\begin{table*}
\centering
\setlength{\tabcolsep}{0.12\tabcolsep}
\begin{tabular}{|l|c|C|C|C|c|c|c|c|c|c|c|c|c|c|c|c|}
%\toprule
\hline
\textbf{Subject} & & \multicolumn{7}{|c|}{\textbf{\#Detected Dependent Tests}} & \multicolumn{7}{|c|}{\textbf{Analysis Cost (second)}}\\
%\midrule
\cline{3-16}
\textbf{Programs} & \textbf{\#Tests} & \multicolumn{3}{|c|}{\textbf{Randomized}} & \multicolumn{2}{|c|}{\textbf{Exhaustive }} & \multicolumn{2}{|c|}{\textbf{Dependence-Aware}} & \multicolumn{3}{|c|}{\textbf{Randomized}} & \multicolumn{2}{|c|}{\textbf{Exhaustive }} & \multicolumn{2}{|c|}{\textbf{Dependence-Aware}} \\
%\cline{3-8}\cline{10-15}
& & \smalltrialnum & \mediumtrialnum & \trialnum& \; $k$=1 & $k$=2 & \quad $k$=1 \;\; \quad & $k$=2 & \smalltrialnum & \mediumtrialnum & \trialnum & \; $k$=1 & $k$=2 &  \quad $k$=1 \quad \quad & $k$=2  \\
\hline
%\bottomrule
\multicolumn{16}{|l|}{ }\\
\multicolumn{16}{|l|}{\textbf{Human-written unit tests} }\\
\hline
JodaTime & \jodatimetests & 1 & 1 & 6 & 2 &\unknown&& &   57 & 528 & 5538 &1265& \infy & &   \\
XML Security& \xmlsecuritytests & 1 & 4 & 4 &4 &4 & 4 & 4  &65 & 594 & 5977 & 106 &  11927 & 93 & 3322  \\
Crystal & \crystaltests & 18 & 18 & 18 &17&18&  & &14& 131 & 1304 & 166 & 7323 &   & \\
Synoptic & \synoptictests & 1 &1  & 1 & 0 &1 & &&  7 & 67 & 760& 25 & 3372&  &  \\
\hline
\textbf{Total} & \totaltests & 21&24&29& 23 &\unknown&  & &  143 & 1320 & 13579 &1562& \infy &   &  \\
\hline
\multicolumn{16}{|l|}{ }\\
\multicolumn{16}{|l|}{\textbf{Automatically-generated unit tests} }\\
\hline
JodaTime & \jodatimeautotests & 586 &815& 966 & 534 & \unknown&& & 131  & 1139 & 9000 & 448 & \infy & &  \\
XML Security& \xmlsecurityautotests& 167 & 171 & 171 & 129 &&  &  & 50 & 430 & 4174 & 133 & \infy & & \\
Crystal & \crystalautotests & 159 & 162 & 164 & 55 & \unknown& & & 103 & 949& 9436  & 2477 & \infy & & \\
Synoptic & \synopticautotests & 3 & 7 & 10 &2& \unknown& &  &81& 770  & 6311 & 454 & \infy & & \\
\hline
\textbf{Total} & \totalautotests &915&1155& 1311 & 720& & & &365 &3288 & 28921 & 3512 & \infy & & \\
\hline
\end{tabular}
\caption{Experimental results. Column ``\#Tests'' shows the total number
of tests, taken from Table~\ref{tab:subjects}. Column ``\#Detected Dependent Tests''
shows the number of detected dependent tests in each subject program.
Columns ``Randomized'', ``Exhaustive'' and ``Dependence-Aware'' show the results
of applying the randomized algorithm, the exhaustive $k$-bounded algorithm and the dependence-aware
$k$-bounded algorithms, respectively. 
When evaluating the randomized algorithm, we use $numtrials$ =
$\smalltrialnum, \mediumtrialnum, \trialnum$ in the algorithm (Figure~\ref{fig:randalgorithm}).
Column ``Analysis Costs (second)''
shows the time cost (in seconds) of each algorithm under
different settings.
}
\label{tab:results}
\end{table*}


Dependent tests reach beyond theory and appear in real-world programs.  
In some cases, they are intentional, developers are aware of them and
document them, but in other cases they are inadvertent. 
Test dependence can cause problems, not only when test suites are reordered,
but even when they are
executed in the intended order.
This section presents concrete examples of test dependence found in
well-known open source programs. Figure~\ref{fig:example-summary}
summarizes the projects we studied and the results: The table
summarizes the number of tests in the suites produced by the
developers (\emph{MT}), the number of tests we generated automatically
with Randoop (\emph{AT}), and the corresponding numbers of dependent
tests in those test suites (\emph{MTD} and \emph{ATD}, respectively). 
The discussion of the examples in this section is distinguished by
the problems caused by test dependence (\emph{Kind}): when faults are masked because
tests make incorrect assumptions about the global environment (Section~\ref{sec:mask}); 
when tests do not
respect required initialization protocols (Section~\ref{sec:examples:initialization}); and when
undocumented test dependence leads to spurious bug reports (Section~\ref{sec:spurious}).
We also describe dependent tests in an automatically-generated test
suite (Section~\ref{sec:autogen}).
While this list---and associated set of examples---certainly is not exhaustive, it shows that there are
several classes of dependence-related problems that have practical
relevance.



\subsection{Masking Faults}\label{sec:mask}

\emph{Masking} is a particularly perplexing problem caused by
dependence.
The negative effect of masking is that it hides a fault in the
program, \emph{exactly} when the test suite is executed in its default
order. 
%So while manifest dependences can reveal such a problem, the
%underlying fault is in the program and affects first-order testing and
%use of the program.
%In its simplest form, masking occurs when parts of a program or tests assume that
%global state has correctly been initialized before these parts can
%ever execute. When this assumption is incorrect, because
%initialization is not implemented correctly, the interactions of
%different parts of the program might jointly modify the global state
%in ways that lead to intricate and subtle faults.
Masking occurs when a test case $t$ (a) \emph{should}
reveal a fault, (b) only does so when executed in a specific environment
$\env_R$, but (c) tests executed before $t$ in a test suite always
generate environments different from
$\env_R$.
%To express this more
More precisely and without loss of generality, assume any
environment with only a single variable. Then let $T =
\suite{t_1,\dots,t_n}$ be the test suite, and let $t_i, 1 < i \leq n$
be the test that should reveal the fault in environment $\env_R$. A
dependency $t_k \prec t_i, k < i$ masks the fault if
$\exec{\suite{t_1,\dots,t_{i-1}}}{\env_0} \neq \env_R$.

The following two examples illustrate masking in
practice.

\paragraph{CLI: A Long-Standing Bug}

\begin{figure}
% \lstset{language=Java,numbers=left}
%\lstset{language=Java}
\lstset{belowskip=0ex,escapechar={@},numbers=left,numberstyle=\small\ttfamily}
\begin{lstlisting}
public final class OptionBuilder {
  @\itshape\color{red}
  private static String argName;@
  
  private static void reset() {
    ...
    @\itshape\color{red}argName = "arg";@
    ...
  }
   
  public static Option create(String opt){
    Option option = 
      new Option(opt, description);
    ...
    option.setArgName(argName);
    @\itshape\color{red}OptionBuilder.reset();@
    return option;
  }
}
\end{lstlisting}
\caption{Fault-related code from \code{Option\-Build\-er.java}}
\label{fig:option_builder}
\end{figure}

A straightforward example of fault masking occurs in the Apache CLI
library.\footnote{\url{http://commons.apache.org/cli/}}
Two test cases fail when run in isolation:
\code{test13666} and \code{test\-Op\-tion\-With\-out\-Short\-For\-mat2} in test
classes \code{Bugs\-Test} and \code{Help\-For\-mat\-ter\-Test},
respectively.

A detailed study of the code under test revealed that both 
tests fail due to the same hidden dependence. The fault is located in 
\code{OptionBuilder.java} and is caused by not initializing a global
variable early enough.
Figure~\ref{fig:option_builder} shows code that
illustrates the fault. 
%
By default,
\code{argName} is initialized to \code{null} (line 2), and only set to
its intended default value \code{"arg"} by the \code{create()} method
via calling \code{reset()} (line 15). 
Consequently, if clients of CLI do not explicitly initialize the value of
\code{argName}, the first option created will have \code{null} rather
than \code{"arg"} as its argument name.
%In CLI, there are two types of options: options with and without
%argument names. If an option without argument is created first,
%this fault will not lead to a failure, because the \code{null} value
%will be ignored. Consecutive calls to \code{create()} can rely on
%\code{reset()} to establish the desired default value.

Both dependent tests
% \code{test13666} and \code{test27635} (or \code
% {test\-Op\-tion\-With\-out\-Short\-For\-mat2}) 
can reveal this fault, since they create an option with 
the default argument as the first thing in their execution. However,
in the default order of test execution, 
%the test classes \code{BugTest} and \code{Help\-For\-mat\-ter\-Test} both
%contain other 
tests that create options with explicit arguments execute \emph{before} 
these dependent tests.
% \code{test13666}
% and \code{test27635} respectively. 
%Thus, when the tests in these classes are 
%executed in order, the tests executed before \code{test13666}
%and \code{test27635} call \code{create()} 
Thus, the tests that are executed before call \code{create()} at least once, which
sets the default \code{argName} value, thus masking the fault.

%\todo{JW}{The following paragraph is not really necessary here. We
%want to illustrate how these things happen. If we have time and space,
%I think a general discussion section about the relevance of dependence
%might benefit from this, though}
%
%\todo{SZ}{I am on the side of keeping the following text. It is very
%impressive about the effect of dependent tests, making the whole
%test dependence story stronger.}

This fault is reported in the bug
database several times,\footnote{\url{https://issues.apache.org/jira/browse/CLI-26} \url{https://
issues.apache.org/jira/browse/CLI-186} \url{https://issues.apache.org/jira/browse/
CLI-187}} starting on March 13, 2004 (CLI-26). The report is marked as resolved
\emph{three years} later on March 15, 2007, but is then reopened as CLI-186 on
July 31, 2009. On this report, one of the developers commented:
\begin{quote}
I reproduced the issue, it requires a dedicated test case since it is tied to the initialization 
of a static field in OptionBuilder.
\end{quote}
Despite the realization that a dedicated test is required, no such
test was ever created.
About one month later, the bug is duplicated as CLI-187, and the
actual fix happens one 
year later on June 19, 2010, about six years after the bug was first reported (and four years
total on the open-issue list).
%total ``awareness'' of the fault
%of years. %The fix consists of adding the following code to \code{OptionBuilder.java}:

%We associated these dependent tests with a bug first reported by a
%user in 2004, marked as ``resolved''---but not actually resolved---in 2007,
%reopened in 2009 by another user, and finally identified and
%fixed in 2010.  This bug sat on the shelf for about six years and
%would have been identified much earlier and much more easily by
%considering test dependence.

\newcommand{\jodatime}{JodaTime\xspace}
\paragraph{\jodatime: Complex interactions that mask faults}
\label{sec:jodatime}
\newcommand{\periodType}{\texttt{Period\-Type}}
\newcommand{\durationFieldType}{\texttt{Duration\-Field\-Type}}
\newcommand{\forFields}{\texttt{for\-Fields}}

\jodatime{}\footnote{\url{http://joda-time.sourceforge.net/}} is an open source
date and time library intended to improve upon the weaknesses of the
date and time facilities provided by the standard JDK.
%written to enhance the capabilities provided by the standard
%JDK such as allowing multiple calendar systems. 
It is a mature project that has been under active development
for more than eight years.

\jodatime\ uses intricate caching mechanisms that are high\-ly complex
and coupled.  All dependences we found are complex,
in two cases even requiring a
specific ordering of \emph{three} tests to manifest.

In a simple dependence, \jodatime{} caches \periodType{} objects, which 
% Caching is done by using a
% global \texttt{HashMap} that holds the \periodType{}s that are created by
% \forFields{} method. 
contain an array of
\durationFieldType{}s (e.g., week, month). 
The order of \durationFieldType{}s in the array is an
important of the data representation, and 
two \periodType{}s with the same \durationFieldType{}s in a different
order are not equal internally in \jodatime, even though they are equal
to \jodatime clients.
%However, this implementation detail should be transparent to the user: As long
%as two \periodType{} objects have the same \durationFieldType{}s, they should represent
%the same period. 
To make this internal detail transparent to users of \jodatime, 
new \periodType{}s are normalized before they are cached. However, a fault in the code 
% checking for existing objects
makes it possible to insert non-normalized \periodType{}s into the
cache, leading to cache misses when searching for correctly normalized
\periodType{}s.
%This is even acknowledged by the developers: \forFields{}
%method first creates a \periodType{} using method argument \texttt{types} (the
%ordering provided by the user) and checks whether the cache already contains
%this ordering. However, if this fails, then another \periodType{} with
%normalized ordering (\texttt{checkedType}) is created at the end of the method 
%and the cache is rechecked as shown in Figure~\ref{fig:jodatime_forFields}.


%\begin{figure}
% \lstset{language=Java,numbers=left}
%%\lstset{language=Java}
%\begin{lstlisting}
%PeriodType forFields
%  (DurationFieldType[] types) {
%  ...
%  PeriodType input =
%    new PeriodType(null, types, null);
%  ...
%  // recheck cache in case
%  // initial array order was wrong
%  PeriodType check = ...
%  PeriodType checkedType = cache.get(check);
%  if (checkedType != null) {
%    cache.put(input, checkedType);
%    return checkedType;
%  }
%  cache.put(input, type);
%  return type;
%}
%\end{lstlisting}
%\caption{Fault related code from \code{Pe\-ri\-od\-Type.for\-Fields}
%\newline (rev. 3937d82f6670e5a30b2809b13cb6d05a7e606037)}
%\label{fig:jodatime_forFields}
%\end{figure}
%
%
%In spite to the developers' extra check, a small mistake in
%Figure~\ref{fig:jodatime_forFields} creates a bug in the program. Note that the
%developers are putting the \texttt{input} variable (the ordering provided by
%the user) as the `key' of the cache.
%As a result, if \forFields{} method is called with a wrongly ordered
%\texttt{types} parameter first, then a \periodType{} with wrong ordering will be
%added to the cache. Later, calling the same method with correct ordering of
%the same content causes a cache miss.

A test that checks for correct normalization when
caching objects
%The developers even have a very simple test case that checks for this.
%\texttt{Test\-Period\-Type.test\-For\-Fields4} method creates two
%\durationFieldType{}s: first with wrong ordering and the second with correct
%ordering, both representing the same period. The test creates the corresponding
%\periodType{}s by calling \forFields{} method with these \durationFieldType{}s.
%Finally, the objects retrieved from \forFields{} method are asserted for
%equality. 
fails in isolation but passes when the entire test suite
executes in the default order; this happens
because a prior test creates the expected
\periodType{}, and thus it is already in the cache for the
later test.
This behavior has been reported as a bug and has been fixed by the
developers.
%never fails during the development due to a simple dependency with the previous
%test in the same class. \texttt{test\-For\-Fields3} method creates the same
%\periodType{} content --- which will be used in the next test --- with the
%correct ordering for some other purpose. As a result, this \periodType{} is added with
%the correct ordering to the cache, which guarantees correct retrieval for both
%the correct and wrong ordering in the next test.


%The test case that would catch the bug was written the same revision it is
%introduced. However, since the developers never ran that test in isolation the
%bug lived for more than six years. During this time period, there have been 773
%commits for the project. Even the buggy file (\periodType{}) is
%changed for nine times and the buggy method (\forFields{}) is changed
%once. Finally, the bug gets reported by a
%user\footnote{\url{http://sourceforge.net/mailarchive/message.php?msg_id=28501345}}
%in December 06, 2011 and is fixed the same
%day\footnote{\url{https://github.com/JodaOrg/joda-time/compare/b609d7d66d...d6791cb5f9}}.
%During the same commit, the developers also removed the dependency for
%the related test by creating a unique
%\periodType{} for that test. The actual fix contains changing two
%variables:
%two instances of variable \texttt{input} (possibly wrong ordering) to variable
%\texttt{check} (to correct ordering) at the end in
%Figure~\ref{fig:jodatime_forFields}.

After inspecting the code, we reported the more complex dependence of
three tests to the developers of \jodatime. They confirmed the
phenomenon, but contended that it is due to interactions that are not
intended in the design of the library~\cite{jodatime}. In particular, one of the
methods, \code{DateTimeZone.setProvider()}, is only supposed to be
called a single time to initialize the library.  In practice, multiple
tests initialize the library, which leaves incorrect values in the cache
and causes other tests to fail under some execution orders.
%However, the tests call it more
%than once, which causes at least two cases to %break the caching
%%mechanism by leaving 
%leave incorrect values in the cache, causing the tests to fail.


\subsection{Poor Test Construction}\label{sec:examples:initialization}

Based on our interaction with the \jodatime developers, this last
dependence does not
mask a fault in the program.  Instead, it represents a less severe consequence of test
dependence that suggest that a test, or a test suite, 
has been constructed poorly in some dimension.  While test dependences that mask faults
correspond to a defect
in the program source, these dependences correspond to defects in the test code.
%
%In contrast to the previous section
%where the dependences led to defects in the program source, this section concerns defects
%in the test source.

%In some sense, dependences that are due to missing initialization are
%the dual to dependences that mask faults.  Both reveal problems in source code.
%However, masked faults reside in the program source, while incorrect
%initialization is a fault that resides in the test suite.

The test dependences presented in this section arise due to incorrect initialization
of program state by one or more tests. In the first case,
%
%The following two examples show two common patterns where incorrect
%initialization leads to test dependence.
%The first example is probably the most common. 
tested program code relies on a
global variable that is a part of the environment, but the test does
not properly initialize it.  In the second case, a test should but
does not call
an initialization function before later invocations to a complex library.
This flaw in the test code is masked because the default test suite execution
order includes other tests that initialize the library.  The defect is
inconsequential until and unless the flawed test is reordered, either manually or by
a downstream tool, to execute before any other initializing test.

%The second example employs a common pattern for complex
%libraries that requires a call to an initialization function before
%any other part of the library can be used.
%In both cases, other tests perform the required setup, and because
%they occur before the dependent tests in the normal execution order,
%no tests fail under normal circumstances.

\paragraph{Crystal: Global Variables Considered Harmful}
Crystal~\cite{crystal} is a tool that
pro-actively examines developers' code and precisely identifies and reports on textual, compilation, and behavioral conflicts.

The latest release of Crystal contains 81 human-written unit tests. 
Of those, 75 are fully automated, and 18 exhibit
dependences.
All these dependencies are caused by incomplete initialization of the
environment when testing methods of three distinct classes
(\code{Data\-Source, Lo\-cal\-State\-Re\-sult, Con\-flict\-Daemon}).
In all cases, one test initializes the environment correctly, and all
other tests rely on that test executing first. 

A short conversation with the developers confirmed that this was not
intentional and most likely happened because the developers were not
aware of the potential dependency caused by the use of global
variables. Since we pointed out this problem, the developers treat the
dependencies as undesirable and opened a bug report to have this issue
resolved.\footnote{\url{https://code.google.com/p/crystalvc/issues/detail?id=57}}

%Dependent tests in Crystal fall into the following three groups, which
%share the same root cause of global data dependence across multiple tests:
%
%\begin{itemize}
%
%\item 9 dependent tests come from the \CodeIn{DataSourceTest} class.
%In that class, a test method \CodeIn{testSetField} initializes a global variable \CodeIn{data}
%and other test methods read the value of the \CodeIn{data} variable.
%As a result, when a test using \CodeIn{data} is executed in isolation or executed
%before the \CodeIn{testSetField} method, a \CodeIn{NullPointerException} is
%thrown.
%
%\item 7 dependent tests come from class \CodeIn{LocalStateResultTest}.
%In that class, a test method \CodeIn{testLocalStateResult} initializes a global variable
%\CodeIn{localState} and other test methods use that variable. Therefore,
%7 tests using the \CodeIn{localState} variable exhibits
%a \CodeIn{NullPointerException} when they are executed before \CodeIn{testLocalStateResult}.
%\todo{KM}{I see no difference between the first item and this one. They all
%seem to happen due to one test initializing a global variable and the others
%reading the same global variable}
%
%\item 1 dependent test comes from class \CodeIn{ConflictDaemonTest}. This
%test uses a shared global variable which requires other tests in the
%same test class to initialize, \todo{KM}{Again, the same thing. I believe we
%should say that all dependencies are due to initialization of a global variable
%and then explain all of them as three sentences (instead of bullet pointing
%them as they were really different)}
%\end{itemize}



\paragraph{XML Security: Global Initialization}

%Are these test dependences realistic, or part of the modifications SIR
%made? by SZ: they are realistic, we use the original version without
%any modification from SIR people


%on a global variable. Take \code{test\_Y1} as an example. This test passes when being executed
%with other tests, but fails by throwing an \code{InvalidCanonicalizerException}
%when executed individually.
%The root cause of such behavior difference is that, in XML-security, the \code{Init.init()} method initializes
%the static field \code{Canonicalizer.canonicalizerHash}, and test \code{test\_Y1} needs to use
%that static field to create a \code{Canonicalizer} instance. 
%When executing this test in the programmer-fixed order, method \code{Init.init()} has been called by
%other tests executed before \code{test\_Y1}, so that test \code{test\_Y1} passes.
%However, without calling \code{Init.init()} first,
%\code{test\_Y1} fails to create the \code{Canonicalizer} instance.
%
%Based on the dumped error message in the \code{InvalidCano-\\nicalizerException}:
%
%\begin{quote}
%``You must initialize the xml-security library correctly before you use it.
%Call the static method ``org.apache.xml.security.Init.init()'' to do that before you use any functionality
%from that library''
%\end{quote}

%We speculate that programmers should realize this potential dependence, but they
%overlook to enforce \code{test\_Y1} to be executed in a desirable order. Instead,
%programmers may have put an implicit assumption that tests in a suite can be executed in isolation
%and miss to add the necessary preconditions for \code{test\_Y1}. 

% vim:wrap:wm=8:bs=2:expandtab:ts=4:tw=70:


\subsection{Spurious Bug Reports and Bug Fixes}\label{sec:spurious}
Sometimes developers introduce dependent tests intentionally because it is
easier, more efficient or more convenient to write unit tests for some modules
in that way~\cite{kapfhammeretal:FSE:2003, whittakeretal:2012}.
%DB-testing}.
Even though the developers are aware of these instances
when they create them, this knowledge can get lost, 
and other people who are not aware of these dependences can get confused 
when they run a subset of the test suite that manifests the
dependences.

As a result, they
might report bugs backed by the failing tests, although this is exactly the expected
behavior. If the dependence is not documented clearly and
correctly, it can take a considerable amount of time to work out that
these reported failures are spurious. Or worse, the developers may try
to fix a bug that is not there.

\paragraph{Eclipse SWT: Causing Spurious Bug Reports}
\newcommand{\ite}{\texttt{Invalid\-Thread\-Access\-Exception}}

The Eclipse Standard Widget Toolkit
(SWT)\footnote{\url{http://eclipse.org/swt/}} is a cross-platform GUI
library developed within the Eclipse framework.
%\todo{KM}{Year here\ldots}.
%It has been developed to combine best parts of Sun's Abstract Window Toolkit (AWT)
%and Swing: native look and feel and native performance.
%
Due to the difficulty of obtaining source, compiling and running
test suites with the SWT project, we only examined some test cases
manually, after a bug report indicated test dependence. The
numbers reported in Figure~\ref{fig:example-summary} are the number
of tests we manually examined, and the number of dependencies we found
among those respectively.

As is common practice in GUI toolkits, SWT permits only one
\texttt{Display} object per thread. Attempting to create multiple
\texttt{Display}s in a single thread causes an \ite{}. 
%In other words each thread is responsible of disposing its \texttt{Display} after it is done with it. 
To permit the reuse of \texttt{Display}s, SWT provides two 
methods: \texttt{Display.getDefault} and \texttt{new Shell}. These
methods return the existing \texttt{Display} or create a new one if none exists.


In the test suite of SWT, all tests except those in the class \texttt{Test\_org\_eclipse\_swt\_widgets\_Display}
(\texttt{TestDisplay} for short) retrieve the current \texttt{Display} by using
one of the latter methods. On the other hand, all tests in
\texttt{TestDisplay} create their \texttt{Display} at the beginning of the test
and dispose of it at the end. 

%\texttt{TestDisplay.setup} contains the following
%comment:
%\begin{quote}
%There can only be one Display object per thread. If a second Display is created
%on the same thread, an InvalidThreadAccessException is thrown. 
% \\ Each test will create its own Display and must dispose of it before
% completing.
%\end{quote}


In September 2003, a user reported a
bug,\footnote{\url{https://bugs.eclipse.org/bugs/show_bug.cgi?id=43500}}
stating that tests throw an \ite{}
if she runs any other test before \texttt{DisplayTest}. 
The cause of this is simple: any other test creates, but does not
dispose of a \code{Display} object. Then the tests in
\code{TestDisplay} attempt to create a new object, which fails, as one
is already associated with the current thread.
Since this is the expected and desired behavior, the bug report is
spurious (except maybe it points to a problem in the test suite,
rather than the code).


%Let us examine the bug
%report: running any other test would create a \texttt{Display} (through one of
%the latter two methods) and would not dispose it. Thus, when these test
%complete, the main thread owns a \texttt{Display}. At this moment, when the same
%thread tries to run \texttt{DisplayTest} and thus tries to create another
%\texttt{Display}, an \ite{} is thrown. However, note that this is really the
%intended case when a thread attempts to create multiple \texttt{Display}s. In
%other words, this dependency leads to a spurious bug: there is a change in the
%test outcome when the order of tests are changed, however this does not
%correspond to a bug in the program. Nevertheless, understanding this dependency
%--- even though the comment on \texttt{DisplayTest.setup} existed --- takes
%about a month for the developers. One of the developers closes the bug with the
%following comment:
%\begin{quote}
%Turns out that the tests really are order-dependent - the Display tests must 
%be run first. It's not an SWT bug or anything, it's just the way the tests are 
%written, and I think it would be weird to code around it. \\
%\ldots I'm not going to make any code changes, but for the `fix' I have added a big 
%comment in the AllTests method saying that the Display tests must go first.
%\end{quote}
%We believe that the way this bug is handled shows that the
%dependencies between tests can lead to confusion even when there is no real bug. 


% This led to a spurious bug report. This is actually a good example,
% because it shows how hard it is to tell the difference between a bug
% and dependent test.
% So what do we fix? The "bug" or the tests?


\subsection{Dependence in Auto Generated Tests}
\label{sec:autogen}
\newcommand{\pub}{\texttt{Prop\-er\-ty\-Utils\-Bean}}
\newcommand{\fhm}{\texttt{Fast\-Hash\-Map}}
\newcommand{\cub}{\texttt{ConvertUtilsBean}}

As shown in Table~\ref{tab:results}, in most
projects, a large fraction of the remaining tests are dependent, while
in some projects, there are almost no dependent tests.

Take the Beanutils program as an example, Randoop generates a test suite consisting of
2692 unit tests for a recent release of Beanutils (version 1.8.3).
The prototype tool we outline in Section~\ref{sec:impl}
detected 299 dependent tests in this test
suite.

After a close inspection of the automatically generated test code, we found
the primary reason for the dependencies is missing initialization 
(cf.~Sec.~\ref{sec:examples:initialization}).
%is the unintended program state before test execution.
Specifically, 248 tests attempt to retrieve values from a cache before
anything has been added to the cache. This particular dependence could be
fixed by adding a single line of setup code to each test.
Most of the other dependencies could be fixed with similarly low effort, too.
However, this particular fix requires understanding of at least part
of the program semantics, which is a feat beyond the abilities of
current test generation tools.
%and fully automatically.

%implicitly assume a particular program state (
%the caching state) when executing in the generated order. In Beanutils,
%\pub{} --- which can be
%accessed as a singleton --- contains a global cache from \texttt{Class} to
%\fhm{}. These 248 tests retrieve the value for \cub{}
%from this cache and assert that the result is not \texttt{null}, without adding
%anything to the cache first. However, some other generated tests call
%\texttt{Prop\-er\-ty\-Utils.get\-Prop\-er\-ty\-Editor\-Class}, which adds
%a \{\cub{}, \fhm{}\} pair to the cache internally. As a result, the former tests
%fail when run in isolation (since the cache is empty and it returns null), however
%pass when run within the whole test suite. Reasons for the remaining 51 dependent
%tests are similar, which we omit here for brevity.


Given the high ratio of dependent tests in the automatically generated
test suite, we speculate that the following two phenomena could be
reasons for this.

First, developers usually know a lot about the intended purpose of a
program when they write tests for it. This knowledge helps them to
build well-structured and coherent test suites.
Automated tools, on the other hand, have no such knowledge. One
possible consequence of this is illustrated by the example: the
automated tool does not understand the cache protocol and thus does
not know that it must add values to the cache first. 

%unit
%tests, programmers tend to put logically-related code in the same unit test to test certain software functionality. By contrast, automated test generation tools are often not aware of the underlying program structure nor the test execution environment when creating new ones. In particular, random test
%generation tools like Randoop invokes tested methods
%with little guidance. Thus, a generated test is more likely to depend on the
%execution of others.

%are more likely to ``interleave''
%with each other, such as, invoking the same static method mutating program states.
%\todo{JW}{I do not understand this explanation at all. What has this
%interleaving to do with dependences?}

Second, it is often hard for automated tools to understand that
specific parts of the code depend on the environment, and thus may not
explicitly generate code that sets up the environment correctly. If,
at the same time, other tests are generated that as a side effect
create the needed environment, test dependence ensues.

%Second, test frameworks like JUnit offer constructs \code{@Before}
%(\code{@After}) to permit programmers to abstract common execution environment
%construction (de-construction) code for each unit test. Such mechanism prevents
%dependent tests exhibiting to some extent. However, to the best of our knowledge,
%most automated test generation tools do not leverage
%such mechanism to enforce generated unit tests to execute in an intended environment.

%\todo{JW}{While this is true, I'm not sure if this is the right plave
%to put it}
% \textbf{Kivanc has investigated this, but i can not find the email now.}
% \todo{KM:}{ This is not completely true. I just have the stack trace for the
% first dependency which seems to be due to a mistake in caching. However, I
% don't know if this is a bug, or any information about other dependencies.}



On the other hand, test dependence in automatically generated test suites is 
even more troublesome than in human-written suites. The reason for this is
that all automated test generation tools we are aware of produce tests
that are hard to read for humans, are undocumented, and their intent
cannot easily be gleaned from naming conventions and other aids
developers normally use. While there is some work to alleviate this problem, it
still remains difficult to determine whether a failed test points to a bug
in the program or a dependent test~\cite{fraseretal:ISSTA:2011}.

%We already showed some evidence that test dependence is not uncommon
%in human-written tests. Given the increasing importance of
%automatically generated tests, we also wanted to at least get a
%glimpse of what is happening in that area.
%As a very preliminary, and by no means exhaustive or conclusive
%investigation, we applied Randoop to all the projects for which the
%source was readily available (this excludes SWT).

%Why this strong division happens, and whether the differences between
%the programs can be used to derive guidelines for better testing is an
%interesting question left to future work.

%\todo{JW}{I no longer think the following paragraph is true}
%This is at once surprising and troublesome. It is surprising, because
%in our experience test dependence occurs either because it is too much
%hassle to write proper test setup code for every single test, or
%because developers are not aware that global state is relevant to the
%code that is being tested. The first point should not at all be
%relevant to automated techniques, as the effort of generating boiler
%plate code is negligible compared to the cost of figuring out useful
%parts of the code to test. The second aspect is fairly well amenable
%to static analysis. Thus overall, there is no reason why automated
%tools could not avoid test dependence altogether.



%\section{How does the theory relate to our examples}



\section{Algorithm and Implementation}
\label{sec:algorithm-tool}

In this section we present an algorithm and a prototype tool to detect dependent
tests.
In the worst case, a naive, exhaustive search would execute all $n!$
permutations of the test suite to detect dependent tests. While this
is not feasible for realistic $n$, our approximate algorithm uses 
our intuition that many dependences can be found by running only short subsequences of
test suites, and introduces a bound $k$ on the length
of subsequences. That effectively bounds the execution time to
$O(n^k)$, which for small $k$ is tractable. At the same time, our
prototype tool and the experiments we conducted with it, suggest that
many dependences can be found for small $k$.

\subsection{Algorithm}
\label{sec:algorithm}

\newcommand{\testlist}[0]{\ensuremath{T^k_i}}
\newcommand{\executeTestsInOrder}[1]{\result{#1}{\env_0}}
Since the general form of the dependent test detection problem is
NP-complete, we do not expect to find an efficient algorithm for it.
Instead, we developed an algorithm to approximate solutions by
detecting a subset of dependent
tests. %, focusing on specific, less-general forms of the problem. 
For tractability, our algorithm in Figure~\ref{fig:dtalgorithm} bounds the length of
test execution sequences, and thus the number of permutations to
execute. Instead of executing all permutations of the
whole test suite, we execute all possible $k$-tuples for a bounding
parameter $k$.

Given a test suite $T = \suite{t_1, t_2, \ldots, t_n}$, our algorithm
executes $\exec{T}{\env_0}$ to obtain the \emph{expected result}
$\result{T}{\env_0}$ of each test (line 2). The environment $\env_0$
is the environment provided by the test execution framework.
It then executes every $k$-tuple \testlist\ of tests as
$\exec{\testlist}{\env_0}$, and 
checks whether any result $\result{\testlist}{\env_0}$ differs
from the expected result, i.e.
that there is a dependence in $\testlist$
(lines 3--10). The algorithm returns the set of all tests $t_i \in T$
that have at least one dependence.

It is easy to extend this algorithm to return the shortest sequence of
tests that manifest a dependency for a given test $t_i$, for example
by reducing manifesting sequences with Delta
Debugging~\cite{Zeller:2002}.

%\todo{SZ}{david: sorry that I do not quite understand your comments well. why equal
%to length k if DD is not used? suppose a list of test: t1, t2, t3, t4 all pass
%in this order, but t2 fails when execute t3, t2, t4 in a 3-permutation order.
%So preTest = \{t3\}, and the length of it does not equal to k. Do I understand
%this wrong?  DN: Sai, you are correct about DD, my mistake.  But the Output in the algorithm
%is at least confusing to me.  Is preTests ordered?  The union statement in line 9 is
%probably where I'm confused; we are using angle-brackets for suites, and this isn't
%a suite, correct?}
%\todo{SZ}{david: sorry that I did not realize angle-brackets have used in the theory part. now correct this issue, and put the "preTests" parts in the practical consideration section}

\begin{figure}[t]
\textbf{Input}: a test suite $\mathit{T}$, an execution length $\mathit{k}$\\
\textbf{Output}: a set of dependent tests $\mathit{depTests}$\\
\vspace{-5mm}
\begin{algorithmic}[1]
\STATE $\mathit{depTests}$ $\leftarrow$ $\emptyset$
\STATE $\mathit{expectedResults}$ $\leftarrow$
 \executeTestsInOrder{T}
\FOR{each $\testlist$ in getPossibleExecOrder($\mathit{T}$, $\mathit{k}$)}
\STATE $\mathit{execResults}$ $\leftarrow$
\executeTestsInOrder{\testlist}
\FOR{each test $\mathit{t}$ in $\testlist$}
\IF{$\mathit{execResults}$[$\mathit{t}$] $\neq$ $\mathit{expectedResults}$[$\mathit{t}$]}
\STATE $\mathit{depTests}$ $\leftarrow$ $\mathit{depTests}$ $\cup$ $\mathit{t}$
\ENDIF
\ENDFOR
\ENDFOR
\RETURN $\mathit{depTests}$
%\ENDWHILE
\end{algorithmic}
\vspace{-3mm}
\caption{$k$-bounded approximation algorithm to detect dependent
tests. 
``getPossibleExecOrder'' returns all permutations of tests from $T$ of length $k$. 
} 
\label{fig:dtalgorithm}
\end{figure}


\subsection{Tool Implementation}
\label{sec:tool}

We implemented our $k$-bounded dependent test detection algorithm 
in a prototype tool.\footnote{Available at: \url{http://testisolation.googlecode.com}} 
%written in the form of JUnit 3.X.
The tool is fully-automated and needs only a test suite and the
bounding parameter $k$ as inputs. 
Our
current implementation supports JUnit 3.x tests.
%exhaustively executes every $k$-tuple
%of tests, and compares execution results to identify possible dependence cases. When
%comparing the observed result of a test in an execution order with
%its intended result (corresponding to line 6 in Figure~\ref{fig:dtalgorithm}),
We consider JUnit test results to be the same when the tests either
both pass, or exactly the same exception or assertion violation leads
to test failure.
%if both the observed result and intended result are passing or
%exactly the same exceptions are thrown otherwise. 
The tool creates a fresh JVM for each \testlist, thus, ignoring
external state such as files and OS services, the environment
that the test suites are executed in is always the same $\env_0$.
This ensures that there is no interaction between
different \testlist\ through shared memory.

We used the prototype to verify the dependent tests reported by
users, developers, other researchers, and us, and to find new dependent
tests in the
example programs in Section~\ref{sec:examples} using isolated execution ($k = 1$)
and pairwise execution ($k = 2$).

All the dependent tests reported in Figure~\ref{fig:example-summary},
except for two dependent tests in JodaTime and the dependences in SWT, 
can already be found by isolated execution. Since we could not run the
test suite of SWT, we could not check these dependences with our tool.
During manual bug diagnosis in JodaTime, we identified two test dependences that require
\emph{three} tests to manifest. While these are easy to reproduce, we
did not check that our tool finds them, because the time needed to
run our naive algorithm on JodaTime with $k=3$ is measured in months.

While we believe that most test dependences can be found with small
$k$. This is in part because the set of dependent tests that can be
found with a bound $k$ is always a subset of the set of dependent
tests that can be found with any bound $k' > k$. Additionally, our
intuition and preliminary exploration seem to indicate that small $k$
find many dependences, while larger $k$ do not. However, in principle
it is conceivable
that any number of chain dependences with chains longer than any tried $k$ exist
in all the libraries we analyzed.


%However, due to the computational complexity of the general dependent test
%detection problem, it is difficult to know precisely how many dependent
%tests exist in a test suite. Thus, we do not
%yet have empirical data that shows how many percentages of dependence tests
%our tool can catch. Giving a reasonable estimation
%is one of our future work.

%we do not yet have strong empirical data that shows our algorithm catches X
%percentage or Y of the worst test dependences. It is one of our future work.


% The tool is publicly
% available\footnote{\url{http://testisolation.googlecode.com}}.
% The source code and user manual of our tool is publicly available at:
% \url{http://testisolation.googlecode.com}


% vim:wrap:wm=8:bs=2:expandtab:ts=4:tw=70:


%\subsection{Practical Considerations}
%\label{sec:practical}
%In principle, there is no \emph{ground truth} for the order of test
execution.
Therefore, we assert that the
\emph{programmer-defined} execution order, and consequently the test
results from executing the test suite in that order, are the ground
truth for our experiments.
%would naturally serves 
%the ``truth'' for our definitions of test dependence, and records
%the results from that execution order as intended results (line 2, Figure~\ref{fig:dtalgorithm}).

When a dependent test is identified, programmers may wish to know
a minimal list of other tests on which the identified test depends. 
Given an execution sequence that manifests the dependence, Delta
Debugging (also implemented in our tool) can
be used to return a shortest subsequence 
that still manifests the dependence~\cite{Zeller:2002}. 
%to minimize the recorded
%test list before the dependent test was executed.
%\todo{SZ}{is it clear? or need more explanation?}


In practice, another possible way to help detect potential dependent tests is
to leverage programmers' domain knowledge or employ some program analyses
to identify a subset of tests that are likely to contain dependent tests,
and run the algorithm only on that subset instead of the whole suite.

\todo{sz}{need a summary sentence here for the whole section 5.}

%\todo{DN}{I'm on the side of removing DD if reasonable, and coming back
%to the idea later, maybe in a ``practical considerations'' section/subsection}
%In addition, the algorithm employs Delta debugging~\cite{Zeller:2002}
%to minimize the test set that are executed before a test in
%an execution (lines 7--8). Together with the minimized
%dependent test set, the test revealing with different behaviors
%are added to the output (line 9).


%\todo{JW}{We should mention that we used the tool to find/verify the
%examples. We should also mention that isolation corresponds to $k=1$
%and we did pair-wise (corresponding to $k=2$).}
%\todo{JW}{
%For reverser execution, as far as I remember, we didn't use it to for
%the actual examples we have. But we might claim that it is useful for
%identifying particular kinds of deps. But it would be better if we had
%an example for that.}



\section{Conclusions}
\label{sec:questions}

\input{openquestions}

\subsection*{Acknowledgments} Bilge Soran was a participant in the project
that led to the initial result.  Yuriy Brun and Colin Gordon provided advice about
the formal notation.  Reid Holmes and Laura Inozemtseva identified the initial \jodatime dependence.  Mark Grechanik, Adam Porter, Michal
Young, and Reid Holmes provided timely and insightful comments on a draft.

\bibliographystyle{plain}
\bibliography{references}

%\section{Unused text snippets}
%

\begin{itemize}

\item \todo{KM}{I kind of understand what this paragraph is saying.
However the many minor mistakes in the writing make it very hard to
follow up.} Patterns of dependent tests.
In many cases, there is a \textit{N -- 1} \todo{KM}{The first time I read this,
I read as N minus 1, which is the incorrect way. Maybe write down as ``N to 1''}
dependence relationship, in which $N$-th \todo{KM}{Is ``th'' really needed?} distinct tests depends\todo{KM}{This should be either ``tests depend on'' (more likely) or ``test depends on'', but I couldn't decide} on the same test, which probably is used to set up the environment. For
such cases, the 1 depend test \todo{KM}{``1 dependent test'' or ``first
dependent test''} should be moved to the common \CodeIn{setUp}
\todo{KM}{Consistency: consider @Before} method.
Less frequently, there is a \textit{1 -- N} dependence relationship
\todo{KM}{Have we ever seen this, or is this purely theoretical?}, in which one
test depends on $N$ tests to set up its testing environment.  In one subject, a newly-added test changes the shared variable state of an existing test. Although the newly-added test is executed after the existing test and reveals the same behavior when executing in isolation, the existing test exhibit different behavior if it is executed after the newly-added test.


\item 
\todo{JW}{If we really want to discuss this, this should be connected
to the theory section, as all this follows from theory. A ``finding''
might be that these differences actually matter in practice. And I
don't think we checked that.}
Different techniques have their own strength in detecting dependent tests. We
have investigate three methods (i.e., executing in isolation, executing in a
reversed order\todo{KM}{Did we (do we) really do this?}, and executing in
k-permutation) to identify dependent tests, and found each method complements others. There exist certain tests that can only be found by one method
but missed by the other two.  Executing tests in isolation found more dependent tests
than executing in a reversed order, and executing every $k$-permutation is
infeasible in practice due to the exponentially large number of possible combinations.

\end{itemize}


\end{document}
% vim:wrap:wm=8:bs=2:expandtab:ts=4:tw=70:


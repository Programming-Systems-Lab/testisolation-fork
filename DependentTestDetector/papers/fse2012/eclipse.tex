\newcommand{\ite}{\texttt{Invalid\-Thread\-Access\-Exception}}

The Eclipse Standard Widget Toolkit
(SWT)\footnote{\url{http://eclipse.org/swt/}} is a cross-platform GUI
library developed within the Eclipse framework.
%\todo{KM}{Year here\ldots}.
%It has been developed to combine best parts of Sun's Abstract Window Toolkit (AWT)
%and Swing: native look and feel and native performance.
%
Due to the difficulty of obtaining source, compiling and running
test suites with the SWT project, we only examined some test cases
manually, after a bug report indicated test dependence. The
numbers reported in Figure~\ref{fig:example-summary} are the number
of tests we manually examined, and the number of dependencies we found
among those respectively.

As is common practice in GUI toolkits, SWT permits only one
\texttt{Display} object per thread. Attempting to create multiple
\texttt{Display}s in a single thread causes an \ite{}. 
%In other words each thread is responsible of disposing its \texttt{Display} after it is done with it. 
To permit the reuse of \texttt{Display}s, SWT provides two 
methods: \texttt{Display.getDefault} and \texttt{new Shell}. These
methods return the existing \texttt{Display} or create a new one if none exists.


In the test suite of SWT, all tests except those in the class \texttt{Test\_org\_eclipse\_swt\_widgets\_Display}
(\texttt{TestDisplay} for short) retrieve the current \texttt{Display} by using
one of the latter methods. On the other hand, all tests in
\texttt{TestDisplay} create their \texttt{Display} at the beginning of the test
and dispose of it at the end. 

%\texttt{TestDisplay.setup} contains the following
%comment:
%\begin{quote}
%There can only be one Display object per thread. If a second Display is created
%on the same thread, an InvalidThreadAccessException is thrown. 
% \\ Each test will create its own Display and must dispose of it before
% completing.
%\end{quote}


In September 2003, a user reported a
bug,\footnote{\url{https://bugs.eclipse.org/bugs/show_bug.cgi?id=43500}}
stating that tests throw an \ite{}
if she runs any other test before \texttt{DisplayTest}. 
The cause of this is simple: any other test creates, but does not
dispose of a \code{Display} object. Then the tests in
\code{TestDisplay} attempt to create a new object, which fails, as one
is already associated with the current thread.
Since this is the expected and desired behavior, the bug report is
spurious (except maybe it points to a problem in the test suite,
rather than the code).


%Let us examine the bug
%report: running any other test would create a \texttt{Display} (through one of
%the latter two methods) and would not dispose it. Thus, when these test
%complete, the main thread owns a \texttt{Display}. At this moment, when the same
%thread tries to run \texttt{DisplayTest} and thus tries to create another
%\texttt{Display}, an \ite{} is thrown. However, note that this is really the
%intended case when a thread attempts to create multiple \texttt{Display}s. In
%other words, this dependency leads to a spurious bug: there is a change in the
%test outcome when the order of tests are changed, however this does not
%correspond to a bug in the program. Nevertheless, understanding this dependency
%--- even though the comment on \texttt{DisplayTest.setup} existed --- takes
%about a month for the developers. One of the developers closes the bug with the
%following comment:
%\begin{quote}
%Turns out that the tests really are order-dependent - the Display tests must 
%be run first. It's not an SWT bug or anything, it's just the way the tests are 
%written, and I think it would be weird to code around it. \\
%\ldots I'm not going to make any code changes, but for the `fix' I have added a big 
%comment in the AllTests method saying that the Display tests must go first.
%\end{quote}
%We believe that the way this bug is handled shows that the
%dependencies between tests can lead to confusion even when there is no real bug. 


% This led to a spurious bug report. This is actually a good example,
% because it shows how hard it is to tell the difference between a bug
% and dependent test.
% So what do we fix? The "bug" or the tests?

\section{Introduction}


In a test suite, all the test cases should be independent:
no test should affect any other test's result, and running
the tests in any order should produce the same test results.
The assumption of test independence is important so that
tests behave consistently as designed. However,
as shown in our previous work~\cite{}, this test independence
assumption hold does not hold in practice.
A test's result may depend on whether it runs after
some other tests. For example, consider 
a simple test suite containing two tests \CodeIn{A} and \CodeIn{B},
where running \CodeIn{A} and then \CodeIn{B}
leads to \CodeIn{A} passing, while running \CodeIn{B} and
then \CodeIn{A} leads to \CodeIn{A} failing. We call
\CodeIn{A} a dependent test (in the context of this test suite),
since its result depends on whether it runs after \CodeIn{B} or not.


Test dependence can have non-trivial impacts. First,
test dependence can \textit{mask faults} in a program.
Specifically, executing a test suite in the default order does not
expose the fault, whereas executing the same test suite in
a different order does. Second, 
test dependence can lead to spurious bug reports.
When a test should pass but fails after reordering due to
the dependence, people who are not aware of the dependence
can get confused and might report bugs. 



Our previous work has provide concrete evidence to
show that dependent tests exist, and they can both lead
to false alarms and missed alarms. However, it is still
unclear that whether test dependence can affect the
results of downstream testing techniques, and if it
impacts, how should we cope with the impact.

Int this paper, we focus on investigating the impact of
test dependence to downstream testing techniques, and
describing techniques to to cope with the impact of test dependence.

\subsection{Evaluating the Impact}

Test dependence can impact downstream testing
techniques that change a test suite and thereby change a test's
execution environment. Examples of such techniques include
test selection techniques (that identify a subset of the input
test suite to run during regression testing)~\cite{},
test prioritization techniques (that reorder the input to
discover defects sooner)~\cite{}, and test parallelization
techniques (that schedule the input tests for execution across
multiple CPUs)~\cite{}.

Most of these downstream testing techniques implicitly
assume that there are no test dependences in the input test
suite. Violation of this assumption, as we show happens in
practice, can cause unexpected output. As an example, test
prioritization may produce a reordered sequence of tests that
do not return the same results as they do when executed in
the default order.

To understand whether and how test dependence can impact
the downstream testing techniques, we implemented XXX
test prioritization, XXX test selection, and XXX test parallelization
techniques, and evalauted each technique
on XX real-world subject programs containing known dependent tests.
Our empirical results indicate that the outcome of
every downstream technique we have evaluated is affected by 
test dependence. \todo{more details here.}

Our findings also suggest that test dependence should no longer
be ignored in designing a downstream testing technique that
may change the original test execution order.



%\subsection{Evaluating the Impact}

\subsection{Coping with the Impact}

When test dependence arises, developers wish to understand
its root cause and resolve the test dependence quickly, and
tool users wish to keep the tool outcome consistent.

To address these two problems, we propose two techniques to
cope with the impact of test dependence.

\begin{enumerate}
\item To help developers understand the dependence root cause,
we present a dynamic analysis to explain why the result of
a test depends on another. The key of our dynamic analysis
is monitoring the access of memory locations and external
resources (e.g., files) for each dependent test, and then
cross-checking the accessing information by other tests.
\todo{need specific above, need revise below.}
To make the explanation concise and informative, we further
augument the field accessing information will the actual
method-call chains, and condense low-level memory access
information to the high-level field access information.

\item To enable downstream testing techniques to produce
consistent results in the presence of test dependence, we
we present a family of techniques to augument each downstream
testing technique to respect test dependence. The key of
our technique is to make each technique be aware of test dependence:
when a dependent test has been selected to execute, it ensures
all other tests it depends on will be executed before the
dependent test.

\end{enumerate}


We evalauted both techniques on real-world programs with dependent
tests. We sent the generated dependence root cause explanation
to program developers, and got positive feedback. All downstream
testing techniques produce consistent and correct results
after being augumented by our technique.

\subsection{Contributions}

This paper makes the following contributions.

\begin{itemize}

\item \textbf{Impact Assessment of Test Dependence.}
We present an empirical study to assess the impact of test dependence
on \prionum test prioritization, \selnum test selection, and \parnum
test parallelization techniques. We evaluated these 
testing techniques on \subjnum subject programs with a test suite containing
dependent tests. The results indicate that the results of
all techniques are affected by test dependence. This also
suggests that test dependence should no longer be overlooked
in designing new testing techniques (Section~\ref{sec:impact}).

\item \textbf{Techniques to Cope with Test Dependence.}
We propose a family of techniques to cope with test dependence.
Our techniques enhance existing test prioritization, selection,
and parallelization techniques to make them respect the test dependence
(Section~\ref{sec:cope}). 
%First, we describe a dynamic analysis to localize the dependence
%root cause and generate a concise report to help developers
%understand why the test dependence arises (Section~\ref{sec:coperoot}).

\item \textbf{Empirical Evaluation.} We evalauted the proposed
techniques on \subjnum subject programs containing test dependence, and
showed that the enhanced testing techniques are aware of test dependence;
they consistently produce the same results without compromising
the effectiveness (Section~\ref{sec:evaluation}).

\end{itemize}


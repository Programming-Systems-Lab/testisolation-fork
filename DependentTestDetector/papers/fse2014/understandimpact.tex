\section{Evaluating the Impact}

\newcommand{\jt}{Joda-Time\xspace}

\newcommand{\jfreecharttests}{2234\xspace}%change the total num
\newcommand{\jodatimetests}{3875\xspace}
\newcommand{\xmlsecuritytests}{108\xspace}
\newcommand{\crystaltests}{75\xspace}
\newcommand{\synoptictests}{118\xspace}
\newcommand{\totaltests}{4176\xspace}

\newcommand{\jfreechartautotests}{2946\xspace}
\newcommand{\jodatimeautotests}{2639\xspace}
\newcommand{\xmlsecurityautotests}{665\xspace}
\newcommand{\crystalautotests}{3198\xspace}
\newcommand{\synopticautotests}{2467\xspace}
\newcommand{\totalautotests}{8969\xspace}

\label{sec:impact}

This section describes our empirical evaluation of
the impact of test dependence on test prioritization,
test selection, and test parallelization techniques.

\begin{table}
\centering
\setlength{\tabcolsep}{0.25\tabcolsep}
\begin{tabular}{|l|l|c|c|l|}
%\toprule
\hline
\textbf{Program} & \textbf{LOC} & \textbf{\#Tests} & \textbf{\#Auto Tests} & \textbf{Revision}
\\
\hline
%JFreechart & 92253 & \jfreecharttests & \jfreechartautotests& 1.0.15\\
%if we add JFreechart, need to change the total num, and num of subject program
%\midrule
\jt & 27183 & \jodatimetests
% 3875 is retrieved by running mvn test on the related revision
& -- &  b609d7d66d\\
XML Security & 18302 & \xmlsecuritytests & \xmlsecurityautotests& version 1.0.4 \\ 
Crystal & 4676 & \crystaltests & \crystalautotests& trunk version\\
Synoptic & 28872 & \synoptictests & \synopticautotests&  trunk version\\ 
JFreechart& xxx & xxx & xxx &  xxx \\ 
%\bottomrule
\hline
%\textbf{Total}& &  & &  \\ 
%\hline
\end{tabular}
\caption{Subject programs used in our evaluation.
Column ``\#Tests'' shows the number of human-written
unit tests. Column
``\#Auto Tests'' shows the number of 
unit tests generated by Randoop~\cite{PachecoLET2007}.
}
\label{tab:subjects}
\end{table}

\subsection{Subject Programs}

Table~\ref{tab:subjects} lists the programs and
tests used in our evaluation. We used these subject
programs because they have been developed for
a considerable amount of time (3--10 years) and each
of them includes a well-written unit test suite.

\jt~\cite{jodatime} is an open source
date and time library. It is a mature project that
has been under active development
for ten years. XML Security~\cite{xmlsecurity}
is a component library implementing XML signature and encryption
standards. XML Security is included in
the SIR repository~\cite{sir} and has been used widely
as a subject program in the software testing community.
Crystal~\cite{crystal} is a tool that
pro-actively examines developers' code and
identifies textual, compilation, and behavioral conflicts.
Synoptic~\cite{synoptic} is a tool to mine a finite state
machine model representation of a system from logs.
All of the subject programs' test suites are designed to be executed in
a single JVM, rather than requiring separate processes per test case~\cite{vmvm}.

Given the increasing importance of automated test generation
tools~\cite{PachecoLET2007, ZhangSBE2011, Csallner:2004, fraseretal:ISSTA:2011},
we also want to investigate dependent tests in automatically-generated
test suites. For each subject program, we use
Randoop~\cite{PachecoLET2007}, a state-of-the-art automated
test generation tool, to create a suite of 5,000 tests.
Randoop automatically drops textually-redundant tests 
and outputs a subset of the generated tests as
shown in Table~\ref{tab:subjects}.

We discarded the automatically-generated test suite of
\jt, since many tests in it are non-deterministic ---
they depend on the current time.

\todo{explain two versions}

\subsection{Methodology}

Each subject program

\todo{explain why these subject programs}

\subsection{Results}

This section presents the evaluation results
for test prirotization, test selection, and
test parallelization techniques.

\subsubsection{Impact on Test Prioritization}

\begin{table}
\centering
\setlength{\tabcolsep}{1.25\tabcolsep}
\begin{tabular}{|l|l|l|l|l|l|}
%\toprule
\hline
\textbf{Subject Program} & T1 & T3 & T4 & T5 & T7 \\
\hline
\jt& 0 & 0 & 0 & 0 & 0\\
XML Security& 0 & 0 & 0 & 0 & 0 \\
Crystal& 6 & 0 & 2 & 1 & 1 \\
Synoptic& 0 & 1 & 0 & 0 & 0 \\
JFreechart& 0 & 1 & 0 & 0 & 0 \\
%\bottomrule
\hline
\textbf{Total} & 6 & 1 & 2 & 1 & 1\\
\hline
 &  &  &  &  & \\
\hline
\jt& 0 & 0 & 0 & 0 & 0\\
XML Security& 0 & 0 & 0 & 0 & 0 \\
Crystal& 6 & 0 & 2 & 1 & 1 \\
Synoptic& 0 & 1 & 0 & 0 & 0 \\
JFreechart& 0 & 1 & 0 & 0 & 0 \\
%\bottomrule
\hline
\textbf{Total} & 6 & 1 & 2 & 1 & 1\\
\hline
%\textbf{Total}& &  & &  \\ 
%\hline
\end{tabular}
\caption{Results of evaluating the five test prioritization techniques
in Table~\ref{tab:testprio} on four human-written unit test suites.
Each cell shows the number of dependent tests
that do not return the same results as they do when executed
in the default, unprioritized order. \todo{revise the number here}
}
\label{tab:testprioresult}
\end{table}

\subsubsection{Impact on Test Selection}

\begin{table}
\centering
\setlength{\tabcolsep}{1.25\tabcolsep}
\begin{tabular}{|l|l|l|l|l|l|}
%\toprule
\hline
\textbf{Subject Program} & T1 & T3 & T4 & T5 & T7 \\
\hline
\jt& 0 & 0 & 0 & 0 & 0\\
XML Security& 0 & 0 & 0 & 0 & 0 \\
Crystal&  & 0 &  &  &  \\
Synoptic& 0 &  & 0 & 0 & 0 \\
JFreechart& 0 &  & 0 & 0 & 0 \\
%\bottomrule
\hline
\textbf{Total} &  &  &  &  & \\
\hline
 &  &  &  &  & \\
\hline
\jt& 0 & 0 & 0 & 0 & 0\\
XML Security& 0 & 0 & 0 & 0 & 0 \\
Crystal&  & 0 &  &  & 1 \\
Synoptic& 0 & 1 & 0 & 0 & 0 \\
JFreechart& 0 & 1 & 0 & 0 & 0 \\
%\bottomrule
\hline
\textbf{Total} &  &  &  &  & 1\\
\hline
%\textbf{Total}& &  & &  \\ 
%\hline
\end{tabular}
\caption{Results of evaluating the XXX test selection techniques
in Table~\ref{tab:testprio} on four human-written unit test suites.
Each cell shows the number of dependent tests
that do not return the same results as they do when executed
in the original test suite. \todo{revise the number here}
}
\label{tab:testselresult}
\end{table}

\subsubsection{Impact on Test Parallelization}

\begin{table}
\centering
\setlength{\tabcolsep}{1.25\tabcolsep}
\begin{tabular}{|l|l|l|l|l|l|}
%\toprule
\hline
\textbf{Subject Program} & P1 & P2 & P3 & P4 & T7 \\
\hline
\jt& 0 & 0 & 0 & 0 & 0\\
XML Security& 0 & 0 & 0 & 0 & 0 \\
Crystal&  & 0 &  &  &  \\
Synoptic& 0 &  & 0 & 0 & 0 \\
JFreechart& 0 &  & 0 & 0 & 0 \\
%\bottomrule
\hline
\textbf{Total} &  &  &  &  & \\
\hline
 &  &  &  &  & \\
\hline
\jt& 0 & 0 & 0 & 0 & 0\\
XML Security& 0 & 0 & 0 & 0 & 0 \\
Crystal&  & 0 &  &  & 1 \\
Synoptic& 0 & 1 & 0 & 0 & 0 \\
JFreechart& 0 & 1 & 0 & 0 & 0 \\
%\bottomrule
\hline
\textbf{Total} &  &  &  &  & 1\\
\hline
%\textbf{Total}& &  & &  \\ 
%\hline
\end{tabular}
\caption{Results of evaluating the XXX test parallelization techniques
in Table~\ref{tab:testprio} on four human-written unit test suites.
Each cell shows the number of dependent tests
that do not return the same results as they do when executed
in the original test suite. \todo{revise the number here}
}
\label{tab:testparresult}
\end{table}

\subsection{Discussion}

\subsubsection{Root Causes of Test Dependence}

\subsubsection{Threats to Validity}

\subsubsection{Experimental Conclusion}

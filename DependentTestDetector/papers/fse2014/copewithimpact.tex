\section{Coping With the Impact}
\label{sec:cope}

Test dependence can cause non-trivial repercussions
to both developers and downstream testing techniques.
When test dependence arises, developers should quickly
identify its root causes and downstream testing techniques
should respect the potential dependence and keep the
test execution results consistent.

This section presents a family of techniques to cope with the
impact of test dependence. First, we present an automated
technique to identify the root cause of test dependence
(Section~\ref{sec:coperoot}). Second, to alleviate the
impact on downstream testing techniques, we present a family
of techniques to enhance existing test prioritization,
test selection, and test parallelization techniques (Section~\ref{sec:copeenhance}).

All proposed techniques are empirically evaluated on
our subject programs (Section~\ref{sec:evaluation}).

\subsection{Automatically Explaining Test Dependence}
\label{sec:coperoot}


Test dependencies arise largely because of improper access
to shared global variables or external resources (e.g., files).
However, realizing the root causes of test dependence is often no
obvious.
The global variables or external resources 
accessed by two tests are usually buried deep in
the program code, and the assertions do not directly check
them, but rather check values that have been computed from
them. In any non-trivial real-world program, this deep
nesting effectively hides potential dependencies from developers,
and they may only become aware of them when a subtle bug
leads them there. This section presents an automated
technique that generates a concise report to explain the
root cause of such test dependence.


\subsubsection{Explanation Technique}

Our technique, called \dtexplain, contains three steps:

\begin{enumerate}
\item \textbf{Simplification}. For a dependent test $t$,
\dtexplain uses Delta Debugging to isolate the shortest test
execution sequence $T_{seq}$ that if executed before $t$, will lead
$t$ to exhibit a different result than the result of
executing $t$ alone.

\item \textbf{Execution with Instrumentation}. \dtexplain instruments
the tested program, $t$, and all tests in $T_{seq}$ to monitor
the access to shared global variables and files. \dtexplain 
executes $t$ and $T_{seq}$ in two different order. \dtexplain
first executes $t$ alone to observe its accessed variables
and files. Then, \dtexplain executes $T_{seq}$ before $t$
and observes variables and files accessed by both $T_{seq}$
and $t$.

\item \textbf{Summarization}. \dtexplain summarizes the
variable-accessing information to generate a report to describe
which fields $t$ and $T_{seq}$ may have improper access.
If the field access occurs in the library code, \dtexplain
``bubbles up" it to the application code.
\dtexplain also presents the method-call sequence
that leads to the improper field access. The detailed
algorithm for this step is given in Figure~\ref{fig:summarization}.
\end{enumerate}

We next explain the algorithm for the thrid step with
a concrete example in Figure~\ref{fig:dtex}.


\subsection{Enhancing Downstream Testing Techniques}
\label{sec:copeenhance}

\subsubsection{Enhancing Test Prioritization}

\subsubsection{Enhancing Test Selection}

\subsubsection{Enhancing Test Parallelization}

\subsection{Empirical Evaluation}
\label{sec:evaluation}

\subsubsection{Subject Programs}

use the same subject programs

\subsubsection{Understanding Dependence Root Causes}

\subsubsection{Evaluating Dependance-Aware Test Prioritization, Selection, and Parallelization Techniques}

\subsection{Discussion}

\subsubsection{Threats to Validity}

There are several threats to the validity
of our experiments. First, \dtexplain 
only considers test dependencies arised
from improper field access. It may not
produce useful results for other types of
test dependencies~\cite{}. Second,
\todo{evaluating on the test coverage, no
bug finding abilities yet.}

\subsubsection{Experimental Conclusions}

We have two chief findings. \textbf{(1)} \dtexplain
generates concise report to explain why
test dependence arises, and such report
helps developers to understand the root cause
of test dependence. \textbf{(2)} All
enhanced downstream testing techniques
output consistent results on test suites
containing dependent tests.

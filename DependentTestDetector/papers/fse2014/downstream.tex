\section{Background}
\label{sec:background}
We first present necessary background on three
categories of downstream testing techniques.
For each category, we also describe the techniques
we have evaluated in this paper.


\subsection{Test Prioritization}
\label{sec:backgroundprio}

Test prioritization techniques schedule test cases
for execution in an order that attempts to
increase their effectiveness at meeting some performance goal.
Various goals are possible; one involves
rate of fault detection --- a measure of how quickly
faults are detected within the testing process. An
improved rate of fault detection during testing can
provide faster feedback on the system under test and
let software engineers begin correcting faults earlier
than might otherwise be possible. One application
of prioritization techniques involves regression testing --
the retesting of software following modifications;
in this context, prioritization techniques can take advantage of information gathered about the previous
execution of test cases to obtain test case orderings.

In the testing literature, many
techniques have been developed, primarily by using test execution
information, to prioritize test cases. These techniques
fall into three major categories: (1) techniques
that order test cases based on their total coverage of
code components; (2) techniques that order test
cases based on their coverage of code components
not previously covered; (3) techniques that order test
cases based on their estimated ability to reveal faults
in the code components that they cover. In addition to the
test execution information, the third
category requires a comprehensive history of known
faults, which are often absent in practice and
approximated by seeded faults~\cite{}.


\begin{table}
\centering
\setlength{\tabcolsep}{0.25\tabcolsep}
\begin{tabular}{|l|l|}
%\toprule
\hline
\textbf{Label} & \textbf{Technique Description} \\
\hline
T1 & Randomized ordering \\
T3 & Prioritize on coverage of statements \\
T4 & Prioritize on coverage of statements not yet covered\\
T5 & Prioritize on coverage of methods\\
T7 & Prioritize on coverage of functions not yet covered \\
%\bottomrule
\hline
%\textbf{Total}& &  & &  \\ 
%\hline
\end{tabular}
\caption{Five test prioritization techniques used
to assess the impact of dependent tests. These five
techniques are introduced in Table 1
of~\cite{Elbaum:2000:PTC:347324.348910}. (We use
the same labels as in~\cite{Elbaum:2000:PTC:347324.348910}. We did not
implement the other 9 test prioritization techniques
introduced in~\cite{Elbaum:2000:PTC:347324.348910}, since
they require a fault history that is not
available for our subject programs.)
}
\label{tab:testprio}
\end{table}



Test prioritization techniques
may change the default execution order of tests in a suite.
The assumption of test independence 
is critical so that tests behave consistently
as designed.  However, this critical assumption is
rarely questioned, investigated, or even mentioned.
A total of 31 papers on test prioritization have been  
published in the research track of five major software
engineering conferences
(ICSE, FSE, ISSTA, ASE, and ICST) or in two major
software engineering journals
(TOSEM and TSE) between 2000 and 2013~\cite{alltestprior}.
Of these,
27 papers explicitly or implicitly assumed test independence,
3 papers acknowledged that the potential dependences between tests
may affect the prioritization output~\cite{Kim:2002:HTP:581339.581357,
Qu:2008:CRT, Rothermel:2004:TSC},
and only 1 paper considered test dependence in the design of
test prioritization algorithms~\cite{10.1109/TSE.2012.26}.
Anecdotally, researchers have told us that test dependence
is not a significant concern in designing and evaluating
test prioritization techniques. 


In this paper, we focus on evaluating 5 test prioritization
techniques from the first and second categories (summarized
in Table~\ref{tab:testprio}).
We wish to empirically investigate the validity of this unverified
conventional wisdom, in order to understand whether
test dependence can affect the results of test prioritization
techniques.



\subsection{Test Selection}
\label{sec:backgroundsel}

Regression testing is the process of validating modified
software to detect whether new errors
have been introduced into previously tested code and
to provide confidence that modifications
are correct. Since regression testing is an expensive process,
researchers have proposed regression test selection (for short,
test selection) techniques as a way to reduce some of this expense.
These techniques attempt to reduce costs by selecting and running
only a subset of the test cases in a program's existing test suite.


Since test selection techniques execute only a subset of the original
test suite, they may change the execution environment
of each individual test. 
Further, test selection is often used together with
test prioritization by prioritizing the selected
subset of tests~\cite{}. The test independence assumption
is so important to keep the selected tests behaving the
same as in the original test suite. Unfortunately,
like test prioritization, this critical assumption is
rarely questioned, investigated, or even mentioned.
A total of XXX papers on test selection have been  
published in the research track of five major software
engineering conferences
(ICSE, FSE, ISSTA, ASE, and ICST) or in two major
software engineering journals
(TOSEM and TSE) between 2000 and 2013~\cite{alltestprior}.
Of these,
XXX papers explicitly or implicitly assumed test independence,
XXX papers acknowledged that the potential dependences between tests
may affect the selection output~\cite{},
and only XXX paper considered test dependence in the design of
test selection algorithms~\cite{}.


In this paper, we focus on two popular test selection techniques (listed
in Table~\ref{tab:testsel}) and conduct empirical evaluations
to investigate whether test dependence will affect the results of
each test selection technique. For each test selection technique,
we also evalaute it while combined with each of the test
prioritization technique listed in Table~\ref{tab:testprio}.

\begin{table}
\centering
\setlength{\tabcolsep}{0.25\tabcolsep}
\begin{tabular}{|l|l|}
%\toprule
\hline
\textbf{Label} & \textbf{Technique Description} \\
\hline
S1 & Select tests covering new/deleted/modified statements\\
S2 & Select tests covering new/deleted/modified methods\\
%\bottomrule
\hline
%\textbf{Total}& &  & &  \\ 
%\hline
\end{tabular}
\caption{Two test selection techniques used
to assess the impact of dependent tests. These
techniques are introduced in~\cite{}. These
two techniques form the basis of many other
popular test selection techniques~\cite{}.
}
\label{tab:testprio}
\end{table}



\subsection{Test Parallelization}
\label{sec:backgroundpar}

Test execution parallelization (for short, test parallelization)
schedules the input tests for execution across
multiple CPUs to reduce the test time.

Test parallelization techniques are adopted in
industry. For example, Visual Studio 2010 (and later)
supports a model of executing tests in parallel on a multi-CPU/core machine~\cite{}.
\todo{more evidence here}

The test independence assumption is critical to test
execution parallelization techniques, since scheduling tests
to multiple machines may change the execution environment
of each test and then affect the result. However, this critical
assumption still remains unverified: to the best of our
knowledge, all existing test parallelization techniques and
tools~\cite{} implicitly assume that each test in a test
suite is independent from one another.

In this paper, we focus on XXX test parallelization techniques
(listed in Table~\ref{testpar}) and empirically investigate whether test dependence will affect its
results.

\begin{table}
\centering
\setlength{\tabcolsep}{0.25\tabcolsep}
\begin{tabular}{|l|l|}
%\toprule
\hline
\textbf{Label} & \textbf{Technique Description} \\
\hline
P1 & Parallelize on test id\\
P2 & Random parallelization\\
P3 & Parallelize on test execution time\\
%\bottomrule
\hline
%\textbf{Total}& &  & &  \\ 
%\hline
\end{tabular}
\caption{Three test parallelization techniques used
to assess the impact of dependent tests. These
techniques are supported in industrial-strength
tools~\cite{}. \todo{explain each technique here.}
Each parallelization technique parameterized
by the number of available machines: $k$. We evaluate
each technique with $k$ = 2, 4, 8, and 16.
}
\label{tab:testpar}
\end{table}

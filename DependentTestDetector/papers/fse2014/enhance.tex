\subsubsection{Enhancing Test Prioritization}

ffslfjslfj

%http://en.wikibooks.org/wiki/LaTeX/Algorithms

\begin{figure}[t]
\textbf{Input}: a test suite $\mathit{T}$\\
\textbf{Output}: a set of dependent tests $\mathit{dependentTests}$\\
\vspace{-5mm}
\begin{algorithmic}[1]
\STATE $\mathit{dependentTests}$ $\leftarrow$ $\emptyset$
\STATE $\mathit{expectedResults}$ $\leftarrow$ $T{T}{env0}$
\FOR{each $\mathit{ts}$ in getPossibleExecOrder($\mathit{T}$)}
\STATE $\mathit{execResults}$ $\leftarrow$ $T{\mathit{ts}}{env0}$
\FOR{each test $\mathit{t}$ in $\mathit{ts}$}
\IF{$\mathit{execResults}$[$\mathit{t}$] $\neq$ $\mathit{expectedResults}$[$\mathit{t}$]}
\STATE $\mathit{dependentTests}$ $\leftarrow$ $\mathit{dependentTests}$ $\cup$ $\mathit{t}$
\ENDIF
\ENDFOR
\ENDFOR
\RETURN $\mathit{dependentTests}$
%\ENDWHILE
\end{algorithmic}

% getPossibleExecOrder($T$, $k$): returns a set of test suites, each of size
% $\le k$; each suite is composed of tests selected from $T$ without replacement.\\

\vspace{-3mm}
\caption {An example for reference.
}
\label{fig:algorithm}
\end{figure}

Figure~\ref{fig:algorithm} says something

\subsubsection{Enhancing Test Selection}

\subsubsection{Enhancing Test Parallelization}

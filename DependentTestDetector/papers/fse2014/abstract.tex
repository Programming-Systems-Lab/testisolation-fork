\begin{abstract}

In a test suite, all the test cases should be independent:
no test should affect any other test's result, and running
the tests in any order should produce the same test results.
The assumption of test independence is important so that
tests behave consistently as designed. In addition, many
downstream testing techniques assume test independence,
such as test prioritization, test selection, and test
parallelization. Our recent work shows this critical
assumption often does not hold in practice. However, the impact
of test dependence is still unclear. In this paper, we
empircally investigates the impact of test dependence
and described techniques to cope with the impact.
We present three results.

First, an empirical study of the impact of test dependence
on 5 test prioritization, 5 test selection, and 3 test parallelization
algorithms. The study suggests that test dependence
can affect the results of \textit{all} these downstream
testing algorithms.

Second, we present two techniques to cope with the impact
of test dependence. The first technique monitors the
test execution of a test suite and generates a concise report
to describe the root cause of test dependence.
%The report permits developers understand why and how
%test dependence arises.
The second technique enhances existing test prioritization,
test selection, and test parallelization algorithms to ensure
they respect the test dependence.

Third, we describe an experimental evaluation and a user
study to show that the proposed techniques for coping with
the test dependence impact are useful and effective. They
help developers understand why and how test dependence arises
and enable downstream techniques output consistent results
in the presence of test dependence.

\end{abstract}


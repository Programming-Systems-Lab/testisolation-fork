
As such deeper knowledge is acquired, we may better understand
what contributes to creating test dependence. Is this
  phenomenon linked to particular testing levels (unit, integration,
  system, etc.), specified testing techniques and frameworks, the programming
languages and/or the software development process employed, the relationship
and communication structures between developers and testers on a team, etc.  In turn,
any insights gained in these dimensions may lead to more systematic
approaches to dealing with test dependence.

%A detailed study of the code under test revealed that both 
%tests fail due to the same hidden dependence. 
The fault is located in 
\code{OptionBuilder.java} and is caused by not initializing a global
variable early enough.
Figure~\ref{fig:option_builder} shows code that
illustrates the fault. 

%
By default,
\code{argName} is initialized to \code{null} (line 2), and only set to
its intended default value \code{"arg"} by the \code{create()} method
via calling \code{reset()} (line 15). 
Consequently, if clients of CLI do not explicitly initialize the value of
\code{argName}, the first option created will have \code{null} rather
than \code{"arg"} as its argument name.
%In CLI, there are two types of options: options with and without
%argument names. If an option without argument is created first,
%this fault will not lead to a failure, because the \code{null} value
%will be ignored. Consecutive calls to \code{create()} can rely on
%\code{reset()} to establish the desired default value.

Both dependent tests
% \code{test13666} and \code{test27635} (or \code
% {test\-Op\-tion\-With\-out\-Short\-For\-mat2}) 
can reveal this fault, since they create an option with 
the default argument as the first thing in their execution. However,
in the default order of test execution, 
%the test classes \code{BugTest} and \code{Help\-For\-mat\-ter\-Test} both
%contain other 
tests that create options with explicit arguments execute \emph{before} 
these dependent tests.
% \code{test13666}
% and \code{test27635} respectively. 
%Thus, when the tests in these classes are 
%executed in order, the tests executed before \code{test13666}
%and \code{test27635} call \code{create()} 
Thus, the tests that are executed before call \code{create()} at least once, which
sets the default \code{argName} value, thus masking the fault.



%%%this file includes many commented out sentences

%In this paper, we
%show through a set of substantive real-world examples that test dependence arises in practice. 
%that only manifests when a sequence of three tests are run in a specified, non-default order.
%approximates solutions to this problem.

%.  For example, we identify several situations where test dependence masks
%program faults: in these situations, running the test suite in the default order does not expose a fault

%but running the suite in a different order does.  
%We also argue that existing
%tools rarely ``surface'' test dependences in a direct way, making it harder for developers
%to observe them.


%To a lesser degree, we describe how two trends in software testing may interact
%with test dependence: one, downstream testing tools such as selection, prioritization,
%and parallelization are increasingly common, and may assume that the
%suites they take as input have no test dependences; and, two, automated test
%generation tools are becoming more common, and we provide some initial
%evidence that test dependence appears to be orders of magnitude more common
%in automatically-generated test suites than in manually-produced suites.
%

%WE show that, in practice,
%test dependence does occur sometimes and does have grave consequences.
%We further argue, that given the increasing importance of second-order
%testing techniques, such as prioritization and parallelization, which
%are directly affected by test dependence, this topic deserves
%attention from the research community. As a first step, 


%: (a) given the lack of


%Further, considering the increasing importance of
%automated test generation techniques, we wanted to get an impression
%of whether test dependence occurs in automatically generated test
%suites. We applied Randoop to the same
%set of programs and 
%attention to test dependence, reporting solely on potential but
%unrealized dependence, that is, ``false positives'', might be of little value; and
%(b) computational advantages arise from computing manifest rather than
%potential dependence.

%In practice, test suites that are thought to include only independent
%tests but that manifest dependence can cause problems including:
%\begin{itemize}
%\item masking faults in the program that are not exposed in one execution order but that are
%in another order; 
%\item exhibiting unexpected
%results if reordered (for instance, by downstream testing techniques such as
%test selection or prioritization), a likely indication of poor test construction; and,
%\item reporting of spurious bugs.
%% if the tests are intended to be dependent but the dependence
%%is undocumented. \todo{DN}{I'm thinking of removing this bullet from the intro.
%%It's a bit different, in that the test writers understand the dependence, of course.
%%I'm just afraid that it'll increase confusion instead of clarifying things.}
%\end{itemize}

%As an example of the first category, the JodaTime library
%(Section~\ref{sec:jodatime}) defines a complex caching system.  Its
%test suite includes tests that check the rather complicated function that normalizes
%object states into cache keys.
%However, an unexpected test dependence between two tests masks a bug in this code. The default
%test execution order exercises an unintended path for one test because an object is
%already cached due to a previously executed test; the fault is exposed if that object
%is not initially cached, as would happen
%if the two tests
%are executed in the reverse order.
%Examples of the other categories are
%described in Section~\ref{sec:examples}.

%Test suites with unknown dependent tests may also exhibit unexpected
%results if reordered by downstream testing techniques such as
%test selection or prioritization  In addition, undocumented
%but desired test dependence can
%lead to spurious bug reports.

%
%Changing a test suite's execution order can
%increase the potential to change the execution environment for a given test:
%different tests may be executed before that given test, and they may
%produce an environment that may cause the test to have a different result.

%
%It is justified to argue that developers 
%should take the utmost care to initialize their tests correctly and completely. In principle,
%such setup code could be part of each individual test case. Frameworks

%The frameworks provide effective mechanisms for setting test execution
%environments, but they do not ensure that these mechanisms are used properly.
%And the more complex a programs structure
%is, the more likely it is that some initialization of global variables
%will be forgotten. By identifying test dependence
%as a more broadly discussed issue, and by providing algorithms and
%tools for identifying test dependences, we hope to reduce their


%Yet no framework can ensure that these methods are used
%correctly. 
%Since we are most interested in practical issues, we argue that
%developers are as likely to make mistakes when writing tests, as they
%are when writing code. 
%\todo{mark it red to avoid been overlooked}{frequency and their cost.}

%\subsection{Downstream Testing Tools}

%In essence, these tools may have an unstated precondition---''the input
%must have no test dependences''---that may not be checked or satisfied in some cases.
%An alternative would be for the tools to detect and eliminate dependences.
%
%If this selection happened for regression testing, developers may be
%led to investigate only modified and newly added code to find the
%fault. But it is possible that the fault lies
%elsewhere and has not been discovered due to the dependence in the
%test suite. \todo{KM}{I don't buy this. Once you have the failing test, it
%should not be very hard for a reasonable developer to find the real reason
%behind the failure.}

%
%The two fundamental operations that such tools can apply to test
%suites are sub-suite selection, and suite reordering. Our
%formalization in Section~\ref{sec:formalism} and the examples in
%Section~\ref{sec:examples} show that both these operations can
%produce test suites that exhibit manifest dependences, because all
%these operations lead to tests being executed in potentially different
%environments. While there are some differences between selection and
%re-ordering, these are not significant for the following discussion. 
%Therefore, in Section~\ref{sec:related} we consider only
%related work in test prioritization as a representative of this
%entire class of techniques. %, especially due to its concern with reordering.
%

%The value of studying manifest dependence lies in the fact that
%it can impact second
%order techniques, such as test prioritization or parallelization.
%One premise of such techniques is usually that executing tests in any
%order or in parallel will produce the same results. Therefore, we
%consider as test dependence, effects that cause the results of tests
%to differ when they are executed in different environments.
%Such test dependence may arise when the test results rely on a particular
%context that may unexpectedly differ from one execution order to
%another (Figure~\ref{fig:downstream}).  
%For example, if test \code{A} assumes that a global variable has been
%initialized by some other test, executing test \code{A} before those
%tests may cause different test results.
%Conversely, tests are independent when
%they either do not rely on context at all, or assure the correct
%context before executing.



% \todo{KM}{I don't think what this paragraph says is true. I recommend cutting
% it (the above paragraph also contains what it tries to say).} Our examples also
% show extensive use of test infrastructure that can reduce test dependence: in
% JUnit\footnote{\url{http://www.junit.org}}, \code{setUp()} and
% \code{tearDown()} as well as \code{@before} and \code{@after}
% annotations help developers significantly but allowing them to more
% easily establish the execution environment.  However, these and
% related features are mechanisms only and are not intended to, and do
% not, enforce any policies to ensure that developers use the
% mechanisms consistently and effectively.
% 
% Why do some tests allow varying execution environments?  Can't this
% be easily avoided?
% We contend
% that this is for the same reasons that developers still create bugs,
% still don't always initialize program variables, still don't always check
% array lengths before indexing, etc.  By identifying test dependence
% as a more broadly discussed issue, and by providing algorithms and
% tools for identifying test dependences, we hope to reduce their
% frequency and their cost.

%While superficially straightforward, reasoning about test dependence
%%and the potential causes and consequences of test dependence is
%is intricate and non-trivial. We introduce a formalism to help
%understand and reason about test dependence.  The two key aspects
%of the formalism are (a) defining
%test
%suites as ordered sequences of tests and (b) making explicit the
%context in which tests are run. The formalism provides a precise
%basis for defining test dependence, for proving the (NP-hard) complexity
%of determining if a suite can manifest test dependences, and for
%algorithms that can efficiently identify important classes of dependences.
%
%
%may arise due to the context 
%in which tests are
%executed (Section~\ref{sec:formalism} formalizes context). Hence, the
%notion and use of context is the second fundamental part of our
%formalism.


%We raise awareness of the problems caused by the test
%  independence assumption but demonstrating through theory and
%  manifestations of test dependence that the consequences can be
%  severe. 
%  \item We define a formalism to reason about test suites as sequences
%  and show how dependences arise in theory and practice.
%  \item We lay a foundation for efficient heuristic algorithms to
%  detect dependences in existing test suites and show with some
%  examples that heuristics rather than exhaustive algorithms already
%  have signigificant benefits.


%\todo{JW}{I couldn't fit this into the rewritten intro. I might like
%to use in in Sec 4 or 5}
%The two ways
%  of altering context that we address here are \emph{isolation} and
%  \emph{ordering}.  By isolation, we mean executing each test in a
%  test suite separately: for example, in a different instance of JUnit
%  or in a different virtual machine.  This isolates, and may provide a
%  different context for a test, by ensuring that the initial context
%  is reinitialized for each test.  In contrast, most conventional
%  approaches execute tests in a sequence in the same context, giving
%  (for example) the second test an execution context that can in
%  principle depend in part on how the first test may have modified the
%  context.  By ordering (which as we show in
%  Section~\ref{sec:formalism} is strictly more general than
%  isolation), we mean that the sequence in which tests in a test suite
%  are executed can be varied.  A different ordering of test execution
%  can cause a given test to execute in a different context and,
%  perhaps, provide a different result.

%\todo{DN}{I think we will need to go over the paper later on in the process to make
%really sure we are consistent about test dependence definitions and our writing.
%If I am clear, we define dependence between two tests; we often (informally?)
%talk about a suite with dependence(s).}
%\todo{JW}{Informally, you are correct. I'll make sure to clarify this
%in the text}
%\todo{DN}{We are also inconsistency about ``dependence'' vs. ``dependency'' vs. ``dependencies'' and
%such.  Probably not a big deal, but if we have time...}
%\todo{DN}{General comment about the intuition below -- it's too long now, which makes it less intuitive.}


%As shown in Section~\ref{sec:examples}, most real examples of test dependence we have seen to date relate
%to mistakes in the initialization of test environments.
%\footnote{Test frameworks such as JUnit 
%provide \emph{mechanisms}, such as \code{@Before} and \code{@After} annotations, to help developers more easily
%establish execution conditions.  They are not intended to, and
%do not, \emph{enforce} any policies
%to ensure that developers use these mechanisms consistently and effectively.}

%Consider the two examples (Figure~\ref{fig:dep_examples})
%of the basic
%way dependencies between tests arise in practice. \code{test2} checks
%the value of a variable that has been assigned elsewhere. If the tests
%are executed in the order \code{test1, test2}, both tests will pass (assuming \code{test1} does not
%change the value of \code{a} after the initial assignment),
%while running \code{test2} first will make it fail.  To determine
%potential test dependences would require an analysis of the read/write
%behavior of the tests (for example, under what conditions, if any, does \code{test1} change the value of \code{a}?); such analyses are well-known to be
%imprecise and/or incomplete.    


% to cleanly set all execution conditions for each test
%case. Methods marked with the
%annotations \code{@Before} and \code{@After} are executed before and
%after each test case and are intended to initialize and clean the
%execution conditions for each test case.
%}

%When not all state in the environment is cleanly initialized, test
%dependences can arise, because then the actual state when a test case
%executes can change depending on, for example,


%Very informally, dependence between tests, like the one described above, arises when test cases do not include all% relevant execution conditions.
%In particular, when they compute their result based on a shared global data, and this shared global data (we call it \emph{environment}) is initialized external to the test case.

%\todo{DN}{I'd love to see if we can reduce or eliminate the discussion of JUnit here.  The following
%paragraph, for example, seems confusing in the following sense: if junit has this, why is there a
%dependence problem?  This is obvious to us, but I don't want people thinking about that here. Moved it
%to a footnote -- before/after -- and rewrote a little}
%

%In this section we first give an intuition what causes test
%dependence and what its consequence may be. Then we formally define
%our notion of test dependence, and lastly we demonstrate how these
%notions lead to the \emph{test dependence detection} problem, and how
%we can solve it.

%\todo{JW}{Find a better section heading or remove the heading}
%\subsection{Intuition}



%
%rely on executing in a
%particular context, for example some test may assume that global variables have
%been initialized to specific values, without confirming the expected
%context before they execute. 

%\todo{DN}{I'm at present inclined to move the example to here, but the test structure/results
%text.  That is, use the example to describe those rather than vice versa.  Sort of like: (a) here
%is a simple example; (b) note that a complete analysis of test dependence would require a full
%analysis of the code in test1 to determine possible values for a; (c) instead of considering that
%complicated analysis that would describe the potential for dependence, we use the test oracles/results
%to represent the outcome of the execution, in a given environment.  Or possibly move the
%potential/actual text up instead, and describe how we address it, then the example?}




%Consider the two examples in Fig.~\ref{fig:dep_examples}. The example
%in Fig.~\ref{fig:dep_examples:direct} shows the most basic
%way dependencies between tests arise in practice. \code{test2} checks
%the value of a variable that has been assigned elsewhere. If the tests
%are executed in the order \code{test1, test2}, both tests will pass,
%while running \code{test2} first will make it fail.

%

%\todo{SZ}{for example (b), perphas we can make it more clearer as
%follows:  test1 \{a++\}, test2\{a++\}, ..., testn \{assert a == n-1\}}
%
%\todo{JW}{While this is also a dependence, it does \emph{not} enforce
%any particular order on the first n-1 tests.}
%
%\todo{KM}{I second Jochen. We discussed this quite a bit with him and I think
%he came up with the most readable and easiest example that forces the execution
%of only t1-t1-\ldots-tn to pass and all other orders to fail.}

%Test dependences arise when tests rely on state that is generated
%by other tests. %CLI is a case in point. 
%Most examples we found are quite direct dependecies on
%global variables, where one test implicitly relies on a global variable to be in
%a certain state before executing the sequence of methods to be tested. 
%At a more abstract level, we can think of test dependence as
%read/write conflicts between different transactions. Each test reads
%and writes variables. If a test implicitly assumes a variable to be in
%a particular state, but does not ensure this state before it executes,
%the test may fail.
%\todo{JW}{
%Thinking of the causes for test dependences as read/write conflicts
%might go far. Intuitively, I think the cases that cause parallel
%transactions to abort are the \emph{good} cases for us (because each
%test ensures that it has written the values it needs). All other cases
%seem potentially hazardous. I'll dig into this tomorrow}


%This is in strong contrast to most
%existing work that considers test suites in general as sets, and thus
%ignores the ordering aspect that is important here.
%Informally, a test dependence arises when the results of executing
%test suites in different orders differ. The following formal
%definitions make our notion of test dependence precise.
%\todo{DN}{I took out the ``informally'' sentence or so here, since we've done that
%already a number of times, and now we're going formal!}


%So while manifest dependences can reveal such a problem, the
%underlying fault is in the program and affects first-order testing and
%use of the program.
%In its simplest form, masking occurs when parts of a program or tests assume that
%global state has correctly been initialized before these parts can
%ever execute. When this assumption is incorrect, because
%initialization is not implemented correctly, the interactions of
%different parts of the program might jointly modify the global state
%in ways that lead to intricate and subtle faults.

\section{Conclusion and Future Work}
\label{sec:questions}

Test independence is broadly assumed but rarely addressed, and
test dependence has largely been ignored in previous
research on software testing. 
To understand dependent tests, we described one of the first studies on
real-world dependent tests. We showed that 
test dependence \textit{does} arise in practice, and could have
costly repercussions. We also
formalized the dependent test detection
problem. To detect dependent tests, we designed
and implemented three algorithms to identify manifest test dependence
in a test suite. Our empirical evaluation revealed
dependent tests in every subject program
we studied, from both human-written and automatically-generated test
suites.

Our future work should focus on the following directions:

\vspace{1mm}

\noindent \textbf{{Investigating impact of dependent tests.}}
Dependent tests can compromise the application of
testing techniques such as test generation, selection,
prioritization, and parallelization, since
most current testing techniques just assume independence and
make no statement about what happens when this
assumption is not true. However,
more comprehensive empirical studies should measure  
the extent of this impact.
%
Another open question is how should
testing techniques handle test dependence.
One straightforward way 
might be to augment such techniques to respect a
defined partial order among tests. This partial order
can be derived from knowledge about dependent tests,
or be detected by our \ourtool tool.
%Like contrived examples of test
%dependence itself, it is easy to produce simple examples where
%downstream techniques produce incorrect output when applied to dependent
%tests.
%under the assumption that the input tests have no dependences.
%However, 



\vspace{1mm}

\noindent \textbf{{Eliminating dependent tests.}}
As reflected in our study (Section~\ref{sec:study}),
the practice of eliminating dependent tests
remains mostly manual and ad hoc --- software developers
usually manually hardcode test
execution orders in a configuration file or
simply merge or remove tests.
A more flexible and robust methodology for
dependent test elimination should be developed.
This question also applies to automated test generators.
While there is some work to alleviate
this problem~\cite{fraseretal:ISSTA:2011}, the question
of removing automatically-generated dependent tests
still remains open.

%\todo{Do not forget to check words: subset, and subsequence through the paper.  Use them properly.}

%On the other hand, 
%almost all automated test generation
%techniques we are aware of produce tests
%that are hard to read for humans, are undocumented, and their intent
%cannot easily be gleaned from naming conventions and other aids
%developers normally use. Therefore, it requires more effort
%from developers to identify the root cause of dependence
%and then remove the dependence. While there is some work to alleviate
%this problem~\cite{fraseretal:ISSTA:2011}, the question
%of eliminating  automatically-generated dependent tests
%still remains open.


%As discussed in our experiments, it appears that test
%dependence in automatically generated test suites is 
%even more troublesome than in human-written suites. 


\vspace{1mm}


\noindent \textbf{{Preventing dependent tests.}}
%Detecting dependent tests is not obvious in most
%cases. Thus, a natural question is how could
%software developers prevent dependent tests when
%writing testing code.
One possible way is encouraging to
use advanced testing frameworks that support test dependence~\cite{testng},
so that developers can explicitly specify test
dependence when writing tests.
%However, using different testing frameworks may
%bring up the backward-compatibility issue to the existing tests.

Stylized coding patterns can also be useful. Developers
should be encouraged to write tests ``defensively'' by
specifying necessary test execution pre-conditions and
using less (or properly mocking) global variables or shared resources. 
There is already some work aiming at automating this
process to prevent the potential
for dependences by refactoring programs to use
less global state~\cite{wlokaetal:FSE:2009}. 


%The source code of our tool implementation is publicly
%available at: \url{http://testisolation.googlecode.com}.

\section{Conclusion}
\label{sec:questions}

Test independence is broadly assumed but rarely addressed, and
test dependence has largely been ignored in previous
research on software testing. 
To understand dependent tests, we described one of the first studies on
real-world dependent tests. We showed that 
test dependence \textit{does} arise in practice, and could have
non-trivial negative repercussions. We also
formalized the the dependent test detection
problem. To detect dependent tests, we designed
and implemented three algorithms to identify manifest test dependence
in a test suite. Our empirical evaluation revealed
previously-unknown dependent tests in every subject program
we studied, from both human-written and automatically-generated test
suites.

\begin{comment}
The results of this paper offer initial answers to the four questions we posed
in the introduction. First, test dependence \textit{does}
arise in practice, from both human-written test suites and automatically-generated
test suites. Second, dependent tests can have
non-trivial negative repercussions. Third, we speculate that this
problem is not thoroughly examined before due to the
lack of empirical evidence, a rigorous problem definition,
and detection algorithms; alternatively people may simply believe that well-designed
test suites do not have this problem.
Fourth,
detecting manifest dependent tests is a NP-complete problem,
for which efficient algorithms are unlikely to exist. However,
in practice, a simple algorithm like the randomized algorithm
described in this paper can work fairly well.
\end{comment}

%The source code of our tool implementation is publicly
%available at: \url{http://testisolation.googlecode.com}.

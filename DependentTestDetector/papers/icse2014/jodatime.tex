\newcommand{\periodType}{\texttt{Period\-Type}}
\newcommand{\durationFieldType}{\texttt{Duration\-Field\-Type}}
\newcommand{\forFields}{\texttt{for\-Fields}}

\jodatime{}\footnote{\url{http://joda-time.sourceforge.net/}} is an open source
date and time library intended to improve upon the weaknesses of the
date and time facilities provided by the standard JDK.
%written to enhance the capabilities provided by the standard
%JDK such as allowing multiple calendar systems. 
It is a mature project that has been under active development
for more than eight years.

\jodatime\ uses intricate caching mechanisms that are high\-ly complex
and coupled.  All dependences we found are complex,
in two cases even requiring a
specific ordering of \emph{three} tests to manifest.

In a simple dependence, \jodatime{} caches \periodType{} objects, which 
% Caching is done by using a
% global \texttt{HashMap} that holds the \periodType{}s that are created by
% \forFields{} method. 
contain an array of
\durationFieldType{}s (e.g., week, month). 
The order of \durationFieldType{}s in the array is an
important of the data representation, and 
two \periodType{}s with the same \durationFieldType{}s in a different
order are not equal internally in \jodatime, even though they are equal
to \jodatime clients.
%However, this implementation detail should be transparent to the user: As long
%as two \periodType{} objects have the same \durationFieldType{}s, they should represent
%the same period. 
To make this internal detail transparent to users of \jodatime, 
new \periodType{}s are normalized before they are cached. However, a fault in the code 
% checking for existing objects
makes it possible to insert non-normalized \periodType{}s into the
cache, leading to cache misses when searching for correctly normalized
\periodType{}s.
%This is even acknowledged by the developers: \forFields{}
%method first creates a \periodType{} using method argument \texttt{types} (the
%ordering provided by the user) and checks whether the cache already contains
%this ordering. However, if this fails, then another \periodType{} with
%normalized ordering (\texttt{checkedType}) is created at the end of the method 
%and the cache is rechecked as shown in Figure~\ref{fig:jodatime_forFields}.


%\begin{figure}
% \lstset{language=Java,numbers=left}
%%\lstset{language=Java}
%\begin{lstlisting}
%PeriodType forFields
%  (DurationFieldType[] types) {
%  ...
%  PeriodType input =
%    new PeriodType(null, types, null);
%  ...
%  // recheck cache in case
%  // initial array order was wrong
%  PeriodType check = ...
%  PeriodType checkedType = cache.get(check);
%  if (checkedType != null) {
%    cache.put(input, checkedType);
%    return checkedType;
%  }
%  cache.put(input, type);
%  return type;
%}
%\end{lstlisting}
%\caption{Fault related code from \code{Pe\-ri\-od\-Type.for\-Fields}
%\newline (rev. 3937d82f6670e5a30b2809b13cb6d05a7e606037)}
%\label{fig:jodatime_forFields}
%\end{figure}
%
%
%In spite to the developers' extra check, a small mistake in
%Figure~\ref{fig:jodatime_forFields} creates a bug in the program. Note that the
%developers are putting the \texttt{input} variable (the ordering provided by
%the user) as the `key' of the cache.
%As a result, if \forFields{} method is called with a wrongly ordered
%\texttt{types} parameter first, then a \periodType{} with wrong ordering will be
%added to the cache. Later, calling the same method with correct ordering of
%the same content causes a cache miss.

A test that checks for correct normalization when
caching objects
%The developers even have a very simple test case that checks for this.
%\texttt{Test\-Period\-Type.test\-For\-Fields4} method creates two
%\durationFieldType{}s: first with wrong ordering and the second with correct
%ordering, both representing the same period. The test creates the corresponding
%\periodType{}s by calling \forFields{} method with these \durationFieldType{}s.
%Finally, the objects retrieved from \forFields{} method are asserted for
%equality. 
fails in isolation but passes when the entire test suite
executes in the default order; this happens
because a prior test creates the expected
\periodType{}, and thus it is already in the cache for the
later test.
This behavior has been reported as a bug and has been fixed by the
developers.
%never fails during the development due to a simple dependency with the previous
%test in the same class. \texttt{test\-For\-Fields3} method creates the same
%\periodType{} content --- which will be used in the next test --- with the
%correct ordering for some other purpose. As a result, this \periodType{} is added with
%the correct ordering to the cache, which guarantees correct retrieval for both
%the correct and wrong ordering in the next test.


%The test case that would catch the bug was written the same revision it is
%introduced. However, since the developers never ran that test in isolation the
%bug lived for more than six years. During this time period, there have been 773
%commits for the project. Even the buggy file (\periodType{}) is
%changed for nine times and the buggy method (\forFields{}) is changed
%once. Finally, the bug gets reported by a
%user\footnote{\url{http://sourceforge.net/mailarchive/message.php?msg_id=28501345}}
%in December 06, 2011 and is fixed the same
%day\footnote{\url{https://github.com/JodaOrg/joda-time/compare/b609d7d66d...d6791cb5f9}}.
%During the same commit, the developers also removed the dependency for
%the related test by creating a unique
%\periodType{} for that test. The actual fix contains changing two
%variables:
%two instances of variable \texttt{input} (possibly wrong ordering) to variable
%\texttt{check} (to correct ordering) at the end in
%Figure~\ref{fig:jodatime_forFields}.

After inspecting the code, we reported the more complex dependence of
three tests to the developers of \jodatime. They confirmed the
phenomenon, but contended that it is due to interactions that are not
intended in the design of the library~\cite{jodatime}. In particular, one of the
methods, \code{DateTimeZone.setProvider()}, is only supposed to be
called a single time to initialize the library.  In practice, multiple
tests initialize the library, which leaves incorrect values in the cache
and causes other tests to fail under some execution orders.
%However, the tests call it more
%than once, which causes at least two cases to %break the caching
%%mechanism by leaving 
%leave incorrect values in the cache, causing the tests to fail.

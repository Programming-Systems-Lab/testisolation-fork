%\subsection*{Test Suites as Collections}


%and it is even the
%characterization of test suites used in Wikipedia (which also contradicts the IEEE Standard by stating that
%``[c]ollections of test cases are sometimes incorrectly termed a test plan, a test script, or even a test scenario.'')~\cite{wiki:test-suites}.
%It has been difficult --- and seems unnecessary for our results --- to identify precisely when ``collections''
%came into use for test suites.  It seems likely that this arose from the now-common
%terminology for containers that group multiple elements together: sets, lists,
%hash tables, arrays, etc.

%\subsection{Syntactic and Semantic Test Dependencies}


%Santelices et al.\ define a formal model of how changes might interact
%at the source code level and present a technique for detecting such
%interactions that arise at
%run-time~\cite{Santelices:2010:PDR:1828417.1828487}.  In contrast to
%our approach, they identify changes that interact rather than tests
%that depend upon each other.

%There are results that identify dependences that
%may surface, for example, when there is aggressive testing of parameter
%settings by a single test case; one (of many) approaches uses
%bounded model checking to vary the parameters~\cite{Sullivan:2004:SAB:1013886.1007531}.



%\todo{DN}{Young says: ``If you are referring to the work I am familiar with, I think what has
%been treated in some depth is combinations of parameter settings in a
%single test case.  That's a sort of dependence as well (e.g., the
%parameters might enable a couple of features that interact in nasty
%ways), but it doesn't involve variations in the ordering of operations
%or test cases.''  I can't find anything on this, but I'm probably searching
%wrong.  If something finds a good citation or two, please include it and write
%something here.  If not, delete this entirely.}

%Another kind of dependence helps address
%the testing of configurable software, which can be combinatorial with respect to the
%set of configurable options~\cite{Cohen97theaetg,Cohen:2003}.  

%In
%practice, there are often far fewer
%configurations that are used and thus should be tested.  This
%often structures the configuration space in a way that allows
%potential dependences to be explored more
%efficiently, as a test suite that binds multiple configuration option values
%can test all configurations that share those settings.
%The dependences considered in this approach are not between tests, but rather within
%the configuration option space.


%Another set of results searches for
%efficient (small) test suites that aggressively exercise potentially unexpected
%interactions --- dependences --- among the components of the
%program~\cite{Cohen97theaetg,Cohen:2003}.  These approaches are used for configurable software where
%in principle there are a combinatorial set of instances to test based on variations of the
%configuration options.  The objective is to temper the combinatorial blow-up by identifying
%components sharing some configurable option values; for example, 
%if a block of code is included in such system instances, a test that exercises that block
%can be used as a proxy for interactions between the shared options.
%
%These approaches tend to use covering arrays as a way to determine which of
%the dependencies among components are/are not exercised.  In the case of configurable
%software, there may be a combinatorial number of interactions to be considered.
%These approaches focus on generating a set of effective tests based on program interactions,
%rather than our focus on identifying dependent tests based on ordering, environment, and the
%results of the tests.
%
%
%In principle it could be combinatorial, and the idea of covering
%arrays is that most of the interactions that matter involve just a
%small number of choices.  If there are things that break only for a
%single setting of parameters A, B, C, D, E, F, G, H, we're hosed $A!-(B but
%if something breaks whenever B has value 1 and G has value 2, then
%something like covering arrays has a chance.  



%
%The research we propose is basic research that impacts most
%aspects of testing, ranging from (automatic) test generation through
%regression testing, test case selection and ending with
%considerations on the right test granularity. 
%Similar to the distinguished paper of Staats et
%al.~\cite{staatsetal:ICSE:2011}, we propose to give a rigorous
%foundation to an important aspect of software testing that is
%present but rarely examined in detail in current research.

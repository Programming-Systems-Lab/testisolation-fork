\documentclass[letterpaper]{sig-alternate}
\usepackage{graphicx}
\usepackage[draft]{fixme}
\usepackage{url}
\usepackage{amsfonts}
\usepackage{amssymb}
\usepackage{amsmath}
\usepackage{algorithmic}
\usepackage{booktabs}
\usepackage{listings}
\usepackage[T1]{fontenc}
\usepackage{lmodern}
\usepackage{color}
\usepackage[draft]{hyperref}
\usepackage{xspace}
\usepackage{subfigure}
\usepackage{bold-extra}

\usepackage{color}
\newcommand{\CodeIn}[1]{{\small\texttt{#1}}}

%\newcommand{\todo}[1]{{\color{red}\bfseries [[#1]]}}

 % Add line between figure and text
\makeatletter
\def\topfigrule{\kern3\p@ \hrule \kern -3.4\p@} % the \hrule is .4pt high
\def\botfigrule{\kern-3\p@ \hrule \kern 2.6\p@} % the \hrule is .4pt high
\def\dblfigrule{\kern3\p@ \hrule \kern -3.4\p@} % the \hrule is .4pt high
\makeatother
 % If there is a line, you can get away with reducing the separation between
 % figures and text.  Don't do this without the line, though.
\addtolength{\textfloatsep}{-.5\textfloatsep}
\addtolength{\dbltextfloatsep}{-.5\dbltextfloatsep}
\addtolength{\floatsep}{-.5\floatsep}
\addtolength{\dblfloatsep}{-.5\dblfloatsep}

\newtheorem{definition}{Definition}
\newtheorem{theorem}{Theorem}
\newtheorem{corollary}{Corollary}
%\newtheorem{proof}{Proof}
\newcommand{\pass}{\ensuremath{\mathit{PASS}}\xspace}
\newcommand{\fail}{\ensuremath{\mathit{FAIL}}\xspace}

% commands for formalization
\newcommand{\suites}[0]{\ensuremath{\mathcal{S}\xspace}}
\newcommand{\environs}[0]{\ensuremath{\mathcal{E}\xspace}}
\newcommand{\manifest}[1]{\ensuremath{\prec_{#1}}}
\newcommand{\suite}[1]{\ensuremath{ \langle  #1 \rangle }}
\newcommand{\env}[0]{\ensuremath{\mathbf{E}}\xspace}
\newcommand{\exec}[2]{\ensuremath{\varepsilon(#1,#2)}}
\newcommand{\result}[2]{\ensuremath{R(#1|#2)}}

%commands must be the last package imported
% This package provides the \todo{Name}{Comment} command
\usepackage{commands}
%remove syntax highlighting from Java code
\lstset{basicstyle=\ttfamily,tabsize=2,keywordstyle=\ttfamily,stringstyle=\ttfamily,commentstyle=\ttfamily,
captionpos=b,numberstyle=\small\ttfamily,numbersep=1ex,
keywordstyle=\color{red}}

%\renewcommand{\todo}[2]{}

\author{
Author names\\ 
\affaddr{Department of Computer Science \& Engineering}\\ 
\affaddr{University of Washington, Seattle, USA} \\ 
\email{\{emails\}@cs.washington.edu}
}

%\title{Reexamining the Test Independence Assumption}
%Does Test Execution Order Matter?
%\title{Test Dependence: Theory and Observations}
\title{Test Dependence: Theory and Manifestation}

\begin{document}
\maketitle

\begin{abstract}

In a test suite, all the test cases should be independent:  no test
should affect any other test's result, and running the tests in any order
should produce the same test results.
Techniques such as test %selection and
prioritization generally assume that the tests in a suite are independent.
%but they do not justify that assumption.
Test dependence is a little-studied phenomenon.
This paper presents five results related to test dependence.

First, we characterize the test dependence that arises in practice.
We studied \dtnum real-world dependent tests from \repnum
issue tracking systems.
Our study
%  identifies common characteristics of dependent tests.  It 
shows that test dependence can be hard for programmers to identify.
It also shows that test dependence can cause
non-trivial consequences, such as masking
program faults and leading to spurious bug reports.

Second, we formally define test dependence in terms of
test suites as ordered sequences of tests along with explicit
environments in which these tests are executed.
We formulate the problem
of detecting dependent tests and prove that a useful special
case is NP-complete. 

Third, guided by the study of real-world dependent tests, we
propose and compare four algorithms to detect
dependent tests in a test suite. 

Fourth, we applied our dependent test detection algorithms
to \subjnum real-world programs and found
dependent tests 
in each human-written
and automatically-generated test suite.

Fifth, we empirically assessed the impact of
dependent tests on five test prioritization techniques
and found that dependent tests affect the output 
of all five techniques; that is, the reordered suite 
fails even though the original suite did not.

%the follwoing claim might be too strong
%Our tool revealed a large number of previously-unknown dependent tests.
%In our study, on average \todo{xx}\% of the human-written tests are
%dependent and \todo{xx}\% of the automatically-generated tests
%are dependent.

\smallskip
\smallskip


\noindent
\textbf{Categories and Subject Descriptors}:  
D.2.5 [Software Engineering]: Testing and Debugging.
\\
\textbf{General Terms: }
Reliability, Experimentation.
\\
\textbf{Keywords: }
Test dependence, detection algorithms, empirical studies.

\end{abstract}


\label{dummy-label-for-etags}


%\category{D.2.5}{Testing and Debugging}
%\keywords{Software testing; test dependence; test selection; test prioritization}


\section{Introduction}

%\todo{Is ``dependent test'' a term that can be applied to one test in
%  isolation?  Or is a pair of tests dependent on one another if the result
%  of one of them changes if the other one is run first?  The paper doesn't
%  make this clear anywhere in sections 1 and 2, and this leads to some
%  confusion for the reader.}

%Informally, a \emph{dependent test} produces different test
%results when executed in different environments. 
Consider a test suite containing two tests \code{A}
and \code{B}, where running \code{A} and then \code{B} leads
to \code{A} passing, while running \code{B} and then
\code{A} leads to \code{A} failing. We call \code{A}
an \textit{order-dependent} test (in the context of this test suite), since its result depends on
whether it runs after \code{B} or not.



In a test suite, all the test cases should be independent:
no test should affect any other test's result, and
running the tests in any order should produce the same test results.
%Practitioners are well aware of test dependence:  coding
%guidelines~\cite{unit-test-def,Massol:2003} and
%standards~\cite{IEEE:829-1998,IEEE:829-2008} say to avoid or document it,
%and tools support those goals~\cite{junitordering,depunit,testng, easymock, randomjunit}.
%%\todo{Cite a mocking framework}.
%Researchers are also aware of test
%dependence~\cite{Csallner:2004, Steimann:2013, Gray:1994:QGB:191843.191886,Chays:2000:FTD:347324.348954,kapfhammeretal:FSE:2003,Wang:2007:AGC, Samimi:2013:DM}.
%%\todo{Cite research about creating mocks.}
%Nonetheless, much testing research and practice
%assumes test independence;
%this includes techniques for test selection~\cite{harroldetal:OOPSLA:2001,RenCR2006},
%test prioritization~\cite{Elbaum:2000:PTC:347324.348910},
%and test parallelization~\cite{Misailovic:2007}.
%% These are not offected:
%%, test factoring~\cite{Saff:2005}, and test carving~\cite{Elbaum:2006}.
%theoretical results, algorithms, and tools may behave unexpectedly
%in the presence of dependent tests.
%A test suite that contains dependent tests affects the applications
%of these techniques.
%These techniques produce incorrect results if run on a test suite that contains dependent tests. 
%\todo{The claim of the above sentence is too strong, and may irritate readers, given we do not have strong evidence.
%I would tone down it as: dependent tests can
%affect the results of these techniques. Further, the ``correctness''
%of a test selection/prioritization technique is decided by whether
%a test has been selected or not, rather than whether the test should
%maintain the same resutls or not.}
The assumption of test independence 
is important so that testing techniques behave
as designed.
Consider a test prioritization or selection algorithm.  By design, if its
input is a passing test suite, then its output should be a passing test
suite.  If the suite contains an order-dependent test, then prioritization
or selection can introduce test failures, which violates the design
requirement.

Many techniques assume test independence, including test
prioritization~\cite{Elbaum:2000:PTC:347324.348910,
Rummel:2005:TPR:1066677.1067016, Srivastava:2002:EPT:566172.566187, Jiang:2009:ART},
test selection~\cite{harroldetal:OOPSLA:2001, Orso:2004:SRT,
Briand:2009:ART, Zhang:2012:RMT, Nanda:2011:RTP, hsu09may},
test execution~\cite{Kim:2013:OUT, Misailovic:2007},
test factoring~\cite{Saff:2005, Wu:2010:LRV}, test carving~\cite{Elbaum:2006},
and experimental debugging techniques~\cite{Zeller:2002,
Steimann:2013, Zhang:2013:IMF}.
%often implicitly assume no test dependences in
%a test suite. 
%\todo{check the following sentencee, people may ask
%is the above paper list representative enough? be aware.}
%\todo{I think it would be more compelling for this section to only cite
%  papers that do not mention but implicitly assume test independence.  The
%  current writing sounds like you might be cherry-picking:  why this set of
%  papers?  Are the ratios representative?  But if you just give a list (as
%  long as possible) of papers that don't assume it, then that makes the
%  point you want to, and no one will misinterpret the list as trying to be
%  representative.}
However, this critical assumption is
rarely questioned, investigated, or even mentioned:
none of the above papers explicitly mentions the assumption
as a limitation or a threat to validity.
Between 2000 and 2013, 31 papers on test prioritization were
published in the research track of
ICSE, FSE, ISSTA, ASE, and ICST or in
TOSEM and TSE~\cite{alltestprior}.
Of these,
27 papers explicitly or implicitly assumed test independence,
3 papers acknowledged that the potential dependences between tests
may affect\todo{in what way?  be specific} the prioritization output~\cite{Kim:2002:HTP:581339.581357,
Qu:2008:CRT, Rothermel:2004:TSC},
and only 1 paper considered test dependence in the design of
test prioritization algorithms~\cite{10.1109/TSE.2012.26}.

Anecdotally, a number of researchers have told us that they believe test dependence
is not a significant concern in practice.
We wish to investigate the validity of this unverified conventional wisdom,
in order to understand whether test dependence arises in practice, 
the repercussions of dependent tests, and how to 
detect dependent tests.

%is generally ignored.
%Is this acceptable, because test dependence does
%not arise in practice?
%Is it because even when test dependence arises, there are few
%negative repercussions?
%Is it because no one has studied this problem or thought to examine it?
%Is it because the problem is important but is too hard to analyze or understand?

\subsection{Manifest Test Dependence}

%To explore these questions, 
This paper focuses on test
dependence that manifests as a
difference in test result (i.e., passing or failing) as determined by the testing oracle.
We adopt the results of the default
%% True, but a distraction.
% , usually implicit,
order of execution of a test suite as the
expected results; these are the results that a developer sees when running
the suite in the standard way. A test is dependent when there exists a possibly
reordered subsequence of the original test suite, in which
the test's result (determined by its existing testing
oracles) differs from its expected result in the
original test suite.
%
That is, manifest test dependence
%we focus on a \emph{manifest} perspective of test dependence,
requires a concrete order of the test suite that
produces {different} results than expected.  
%
%



This paper uses \textit{dependent test} as a shorthand for
\textit{manifest order-dependent test}
unless otherwise noted.
A single test may consist of setup and teardown
code, multiple statements, and multiple assertions
distributed through the test.



%\subsection{Causes of Dependent Tests}

\subsection{Causes and Repercussions}
%\todo{I merged the two original subsections into the following one. looks better?}
%\todo{also some text editing below}
%\subsection{Repercussions of Dependent Tests}
\label{sec:intro-repercussions}

%\todo{Fold this section into Section~\ref{sec:intro-repercussions}??}

Test dependence results from interactions with other tests,
as reflected in the execution environment.
Tests may make \textit{implicit} assumptions about their
execution environment --- values of global variables,
contents of files, etc. A dependent test
manifests when another test alters the execution
environment in a way that invalidates those assumptions.

%A test has the potential to yield
%different test results when executed in different environments
%--- global variables with different values, differences in the file system, etc.

Why does this happen?
%As
%suggested by the principle of unit testing~\cite{Greiler:2013:SAT, Massol:2003},
Each test ought to initialize (or mock) the execution environment
and/or any resources it will use.
Likewise, after test execution, it should reset the
execution environment and external resources
to avoid affecting other tests' execution.
However, developers sometimes
%are as likely to
make mistakes when writing tests.
Even though frameworks such as
JUnit provide ways to set up the environment for a test execution and clean
up the environment afterward,
they cannot ensure that it is done
properly. This means that tests, like other code,
sometimes have unintended and unexpected behaviors.
%And
%as programs increase in complexity, so may tests, which may
%increase the frequency of such problems in tests, which may
%in turn increase the frequency of test dependence.
%\edit{check the grammar of the above sentence}



%\todo{Where does this paragraph belong?  Here?}
%This principle is adopted and confirmed by many
%real-world developers (Sections~\ref{sec:study} and~\ref{sec:expdiscussion}).
%When it needs to interact with the execution environment,
%it should mock or carefully resetting external resources
%
%Ideally, each test should not depend on its environment, because it
%initializes any resources it will use 
%Likewise, the test should not modify its environment, because of mocks or
%resetting resources after test execution. 
%




% Our study of \dtnum real-world, confirmed dependent
% tests 
% % from \repnum software issue tracking systems
% (Section~\ref{sec:study}) identified two 
Here are three consequences of the fact that a dependent
test gives different results depending on when it is executed
during testing.

\textbf{(1)}
Dependent tests can
\emph{mask faults in a program}. Specifically, executing a test suite in the
default order does not expose the fault, whereas
executing the same test suite in a different order does. 
% We found a
One bug~\cite{clibug} in the Apache CLI library~\cite{cli}
was masked by two dependent tests
for 3 years (Section~\ref{sec:repercussion}).

\textbf{(2)}
Test dependences can lead to \emph{spurious bug reports}.
When a dependent test fails, it usually represents
%\edit{change to use the word of "represents", good? using
%"reveals" may sound like dependent test is good to improve code quality}
a weakness in the test
suite (such as failure to perform proper initialization) rather than a bug
in the program. 
When a test should pass but
it fails after reordering due to the dependence,
people who are not aware of the dependence can get confused
and might report bugs.
% about the failing test,
%even though this is exactly the intended behavior.
%Programmers made these errors even though frameworks such as
%JUnit provide ways to set up the environment for a test execution and clean
%up the environment afterward.
%
As an example, the Eclipse developers
investigated a bug report~\cite{eclipsebug} in SWT for
more than a month before realizing that the 
bug report was invalid and was caused by test dependences
(i.e., a test should pass, but it failed when a user
ran tests in a different order).
%were intentional,
%allowing them to close the bug report without a change to the system.
%


%Second, guided by the findings of our study, we design two algorithms
%to detect manifest dependent tests. By applying our algorithms
%to \todo{xx} open-source programs and their test suites, we 
%found a large number of unknown dependent tests, .

\textbf{(3)}
Dependent tests can \textit{interfere with downstream testing
techniques} that change a test suite and thereby change a test's execution environment.
Examples of such techniques include
test selection techniques (which identify a subset of
the input test suite to run during
regression testing)~\cite{harroldetal:OOPSLA:2001, Orso:2004:SRT,
Briand:2009:ART, Zhang:2012:RMT, Nanda:2011:RTP, hsu09may},
test prioritization techniques (which reorder the
input to discover defects sooner)~\cite{Elbaum:2000:PTC:347324.348910, Kim:2002:HTP:581339.581357, Rummel:2005:TPR:1066677.1067016, Srivastava:2002:EPT:566172.566187, Jiang:2009:ART},
%test parallelization techniques (which schedule the input
%tests for execution across multiple
%CPUs)~\cite{Misailovic:2007},
test execution techniques~\cite{Kim:2013:OUT},
test factoring~\cite{Saff:2005, Wu:2010:LRV} and test carving~\cite{Elbaum:2006} (which
convert large system tests into smaller unit tests),
%test generation (which re-executes suites as it builds them up)~\cite{PachecoE2005,RobinsonEPAL2011},
experimental debugging techniques (such as Delta
Debugging~\cite{Zeller:2002, Steimann:2013, Zhang:2013:IMF} and mutation
analysis~\cite{Zhang:2012:RMT, Schuler:2009:EMT, Zhang:2013:FMT},
which run a set of tests repeatedly), etc. 
Most of these techniques implicitly assume that
there are no test dependences in the input test suite. Violation of
this assumption, as we show happens in practice, can cause unexpected
output. %\todo{change erroneous to different?} 
As an example, test prioritization may produce a reordered sequence
of tests that do not
return the same results as they do when executed in
the default order. Section~\ref{sec:impact}
provides empirical evidence to show that
dependent tests do affect the output of five test prioritization
techniques.


\subsection{Contributions}
\label{sec:contributions}

This paper addresses and questions
conventional wisdom about the test independence assumption. 
This paper makes the following contributions:

\begin{itemize}

  \item \textbf{Study.} We describe a study of \dtnum real-world
  dependent tests from \repnum software issue tracking
  systems to characterize dependent tests that
  arise in practice.  Test dependence can have
  potentially non-trivial repercussions and can be hard to identify
  (Section~\ref{sec:study}).

\item \textbf{Formalization.} We formalize test dependence
  in terms of test suites as ordered sequences of tests and explicit execution
  environments for test suites.  The formalization enables reasoning about test dependence
  as well as a proof that finding manifest dependent tests is an NP-complete
  problem (Section~\ref{sec:formalism}).

  \todo{I edited the following paragraph, adding the heuristic algorithm}
  \item \textbf{Algorithms.} We present four algorithms
  to detect dependent tests: one reversing,
  one randomized, one exhaustive bounded, and one that prunes the search
  space using dynamic analyses.
  All four algorithms are \emph{sound} but \emph{incomplete}:
  every dependent test they identify is real, but the algorithms
  do not guarantee to find all dependent tests (Section~\ref{sec:detecting}). 
  %\edit{check above when the algorithm section is written}

  \item \textbf{Evaluation.} We implemented our algorithms in a prototype
  tool, called \ourtool (Section~\ref{sec:impl}).
  \ourtool detected 27 previously-unknown dependent tests in human-written
  unit tests in \subjnum real-world subject programs.
  % (and even more in automatically-generated tests).
  The developers confirmed all of these as
  undesired (Section~\ref{sec:evaluation}).

  %\item \textbf{Assessment.} 
  \item \textbf{Impact Assessment.} We implemented five test prioritization
  techniques and evaluated them on \subjnum subject programs
  that contain dependent tests. The results show that all
  five test prioritization techniques are affected by\todo{be more specific
    about what the effect is} dependent tests
  (Section~\ref{sec:evaluation}).

  % \textit{every} subject program we studied, from both  and automatically-generated
  % unit tests (Section~\ref{sec:evaluation}).
  %been discovered before, showing that on average \todo{xx}\% of the human-written
  %unit tests are dependent and \todo{xx}\% of the automatically-generated
  %unit tests are dependent
  %Finally, we discuss a set of open questions and other possible impacts of dependent
  %tests in Section~\ref{sec:discussion}.
\end{itemize}
\vspace{-4mm}
\paragraph{Implications}
\todo{I move the implications from conclusions to here}
Our findings are of utility to practitioners and researchers.
Both can learn that test dependence is a real problem that should not be
ignored any longer, because it leads to false positive and false negative
test results.
Practitioners can adjust their practice based on what code patterns most
often lead to test dependence, and they can use our tool to 
find dependent tests.
Researchers are posed important but challenging new problems, such as how
to adapt testing methodologies to account for dependent tests and how to detect
and correct all dependent tests.



%  LocalWords:  Kapfhammer Soffa subsequence SWT CLI NUM dependences

%  LocalWords:  teardown


\section{Related Work}
\label{sec:related}

%Denoting a group of test cases as a ``suite of test programs'' began around the
%mid-1970's~\cite[p.~217]{brown:CSUR:1974}; similar terms include
%``testcase dataset''~\cite{milleretal:ICRS:1975} and ``scenario,''
%which an IEEE Standard defines as ``groups of test cases;
%synonyms are script, set, or suite''~\cite[p.~10]{IEEE:829-1998}.

Treating test suites explicitly as \emph{mathematical sets} of tests dates at least
to Howden~\cite[p.~554]{howden:ToC:1975} and remains common in the literature.
The execution order of tests in a suite is usually not considered:
%or informally, suggesting that the potential of executing a given test
%in different contexts is immaterial to those results: 
that is, test independence is assumed. We next discuss some
existing definitions of test dependence, techniques that
assume test dependence, and tools that support specifying
test dependence.


\subsection{Test Dependence}

Definitions in the testing literature are generally clear that the
conditions under which a test is executed may affect its result. 
The
importance of context in testing has been explored in some depth in
some domains including databases~\cite{Gray:1994:QGB:191843.191886,Chays:2000:FTD:347324.348954,
kapfhammeretal:FSE:2003}, with results about test
generation, test adequacy criteria, etc., and mobile
applications~\cite{Wang:2007:AGC}.
For the database domain, Kapfhammer and Soffa formally
define independent test suites and distinguish them from
other suites that ``can capture more of an application's
interaction with a database while requiring the constant monitoring of
database state and the potentially frequent re-computations of test
adequacy''~\cite[p.~101]{kapfhammeretal:FSE:2003}.
By contrast, our definition differs from that of Kapfhammer
and Soffa by considering
test results rather than program and database states
(which may not be visible to users).
%Considering only manifest test dependences allows
%us to more easily situate this research in the empirical domain (Section~\ref{sec:formaldiscussion}).

The IEEE Standard for Software and System Test
Documentation (829-1998) \S 11.2.7, ``Intercase
Dependencies,'' says in its entirety: ``List the identifiers of
test cases that must be executed prior to this test
case. Summarize
the nature of the dependences''~\cite{IEEE:829-1998}.  The succeeding version of this
standard (829-2008) adds a single sentence: ``If
test cases are documented (in a tool or otherwise) in the order in
which they need to be executed, the Intercase Dependencies for most or
all of the cases may not be needed''~\cite{IEEE:829-2008}.


%In addition to the work by Kapfhammer and
%Soffa~\cite{kapfhammeretal:FSE:2003},
%there are a handful of categorical references that
%acknowledge that tests can
%be dependent based on context, suggesting 
%ways to document or find situations where the independence
%assumption fails to hold.  


%McGregor and Korson discuss interaction tests that
%are intended to identify ``two methods that may directly or indirectly
%cause each other to produce incorrect results'' and suggest constructing such
%interaction tests by identifying the values shared via parameter passing
%between methods
% that two or more test cases share~\cite[p~.69]{mcgregoretal:CACM:1994}.

Bergelson and Exman characterize a form of test dependence informally:
given two tests that each pass, the composite
execution of these tests may still
fail~\cite[p.~38]{bergelsonetal:EEE:2006}.  That is, if 
\suite{t_1} executed by itself passes and \suite{t_2} executed by itself passes,
executing the sequence \suite{t_1, t_2} in the same context may fail.
However, they do not provide any empirical evidence of
test dependence nor any detection algorithms.

Some practitioners acknowledge test dependence as a possible, albeit low probability, event:
\begin{quote}
Unit testing \dots  
requires that we test the unit in isolation. That is, we
want to be able to say, \emph{to a very high degree of confidence} [emphasis added], that
any actual results obtained from the execution of test cases are
purely the result of the unit under test. The introduction of
other units may color our results~\cite{unit-test-def}.
\end{quote}
They further note that other tests, as well as stubs and drivers,
may ``interfere with the straightforward
execution of one or more test cases.''


Compared with these informal definitions,
we formalize test dependence, and provide empirical evidence
to show that test dependence does arise in practice, and could
have costly repercussions.
%They give an informal definition of what it means for the execution of a
%test to influence the outcome of another test.  We define
%this precisely, and we also define manifest test dependence in terms
%of execution environments
%and test execution order rather than in terms of code use.

%Other definitions of test dependence are primarily considered
%to be \textit{syntactic} dependences between program units, for example
%methods calling other methods, and classes using other classes~\cite{bergelsonetal:EEE:2006,briandetal:SEKE:2002}. 
%\emph{Syntactic} dependence here means that a unit \code{A} cannot be
%compiled and executed without unit \code{B} being present. If we test
%such a unit \code{A} without convincing ourselves first that \code{B}
%is correct, a test failure for \code{A} is harder to interpret,
%because it could just as well indicate a fault in \code{B}.
%Zhang and Ryder extend this notion to \emph{semantic} dependences,
%which is closer to our approach~\cite{zhangetal:TR:2006}. 
%They use a notion of
%``test outcome'' to determine whether or not syntactically dependent
%classes or methods can influence each others results, and consider
%only those that can to be semantically dependent.
%They give an informal definition of what it means for the execution of a
%test to influence the outcome of another test.  We define
%this precisely, and we also define manifest test dependence in terms
%of execution environments
%and test execution order rather than in terms of code use.


\subsection{Techniques Assuming Test Independence}

The assumption of test independence lies at the heart of most,
if not all, techniques for automated regression test selection,
test case prioritization, test generation, coverage-based
fault localization, etc. 


Test prioritization seeks to reorder a test suite to detect
software defects more quickly. 
Early work in test
prioritization~\cite{Wong:1997:SER:851010.856115,Rothermel:1999:TCP:519621.853398}
laid the foundation for the most commonly used problem definition:
consider the set of all permutations of a test suite and find the best
award value for an objective function over that
set~\cite{Elbaum:2000:PTC:347324.348910}.  The most common objective
functions favor permutations where more faults in the underlying
program  are found with running fewer tests.
Test independence is often explicitly asserted as a
requirement for most test selection and prioritization work (e.g.,~\cite[p.~1500]{Rummel:2005:TPR:1066677.1067016}).
%For some test selection and prioritization work,
%test independence is even explicitly asserted as a requirement.
%For example, Rummel et al.\ states in
%A number of studies carefully evaluation various prioritization techniques
%empirically~\cite[\emph{et
%alia}]{Rothermel:1999:TCP:519621.853398,Do:2010:ETC:1907658.1908088}. 
Evaluations of selection and prioritization techniques
~\cite[\emph{et alia}]{Rothermel:1999:TCP:519621.853398,Do:2010:ETC:1907658.1908088}
are based in part on the test independence
assumption as well as the assumption that the set of faults in the underlying
program is known beforehand; the possibility that test dependence may unmask additional faults in the program is not studied.

%\begin{quote}
%A test suite contains a tuple of tests \suite{T_1 $\ldots$ T_R} that execute in a specified order.  We require that each test is
%independent so that there are no test execution ordering dependencies.  This requirement enables our prioritization algorithm to
%re-order the tests in any sequence that maximizes the suite's
%ability to isolate defects.  The assumption of test dependence
%is acceptable because the JUnit test execution framework
%provides \code{setUp} and \code{tearDown} methods that execute before
%and after a test case and can be used to clear application
%state.
%\end{quote}

Most automated test generation
techniques~\cite{PachecoLET2007, Wang:2007:AGC,
ZhangSBE2011} do not take test dependence
into consideration. As shown in our experiments (Section~\ref{sec:evaluation}),
a large number of tests generated by Randoop are dependent.
We speculate that these dependences arise because automated
test generation techniques generally create new tests
based on the program state after executing the previous test,
for the sake of test diversity and efficiency. 
To the best of our knowledge, only JCrasher~\cite{Csallner:2004}
provides a mode to clear the environment changes caused
by a previous test. Such functionality helps eliminate
potential test dependence, but may make generated
tests less behaviorally-diverse --- as they cannot be constructed
on top of previous tests. Exploring how to
incorporate test dependence into the design of automated
test generator is our future work.

Coverage-based fault localization techniques~\cite{Jones:2002:VTI}
often treat a test suite as a collection of test cases
whose result is \textit{independent} of the order of their
execution. They can also be impacted by test dependence.
In a recent evaluation of several coverage-based fault locators,
 Steimann et al.\ found fault locators' accuracy has been significantly
 affected by tests failed due to the violation of the test
 independence assumption~\cite{Steimann:2013}. 
 %For example, if a test depends on a static field whose value is set by
 %previous test cases. 
 Compared to our work, Steimann et al.'s
 work focuses on identifying possible threats to validity
 in evaluating coverage-based fault localization, and does
 not present any formalism, study, or detection algorithms
 for dependent tests.


%define a test suite as a
%collection of test cases whose result is \textit{independent}
%of the order of their execution~\cite{Steimann:2013}.

As shown in Sections~\ref{sec:study} and~\ref{sec:evaluation},
the test independence assumption often does not hold for either
human-written or automatically-generated tests. Thus, techniques
that rely on this assumption may need to be reformulated.

\subsection{Tools Supporting Test Dependence}
\label{sec:supporting}

Testing frameworks provide mechanisms
for developers to define the context for tests.
JUnit, for example, provides means to
automatically execute setup and clean-up tasks
(\code{setUp()} and \code{tearDown()} in JUnit
3.x, and methods annotated with \code{@Before} and \code{@After} in
JUnit 4.x). Ensuring that these mechanisms are used properly, however, is
beyond the scope of any framework, although the latest release of JUnit
(version 4.11)
supports executing tests in lexicographic order by test method name~\cite{junitordering}.


Only a few tools explicitly consider test dependence, by
allowing developers to annotate dependent tests and
provide supporting mechanisms to ensure that the test execution framework
respects those annotations.  DepUnit~\cite{depunit}
allows developers to define soft and hard dependences. Soft dependences control
test ordering, while hard dependences in addition control whether specific tests are
run at all.  TestNG~\cite{testng} 
allows dependence annotations and supports a variety of execution policies
that respect these dependences
such as sequential execution
in a single thread, execution of a single test class per thread, etc.\
What distinguishes our work from these approaches is that, while they allow dependences
to be made explicit and respected during execution, they do not help developers
\emph{identify} dependences.  A tool that finds dependences
(Section~\ref{sec:impl}) could co-exist
with such frameworks by generating annotations for them.

Our previous work~\cite{DBLP:conf/sigsoft/MusluSW11} proposed
an algorithm to find bugs by executing each unit
test in isolation. With a different focus,
this work investigates the validity of the test independence assumption
rather than finding new bugs,
and presents four new results.
Further, as indicated by our study and experiments, most dependent
tests reveal weakness in the test code rather than bugs in the program. Thus,
using test dependence may not achieve a high return in finding bugs.

%  LocalWords:  Howden Kapfhammer Soffa dependences Intercase Bergelson
%  LocalWords:  Exman JCrasher Steimann setUp tearDown DepUnit TestNG


%\section{Dependent Tests in Practice}
\section{Real-World Dependent Tests}
\label{sec:study}

\newcommand{\unum}{{{14}}\xspace}
\newcommand{\svratio}{{{61}}}
\newcommand{\svnum}{{{59}}\xspace}
\newcommand{\unfixed}{{{58}}\xspace}


% are known to occur in practice, but
Little is known about the characteristics of dependent tests.
This section qualitatively studies
concrete examples of test dependence found in
well-known open source software. 


\subsection{Sources and Study Methodology}

We examined five
% well-known, publicly-accessible 
software issue
tracking systems: Apache \cite{apachebug},
Eclipse~\cite{eclipsebug}, JBoss~\cite{jbossbug},
Hibernate~\cite{hibernatebug}, and Codehaus~\cite{codehausbug}.
Each issue tracking system serves tens of projects.
% , and
% holds thousands of bug reports, feature requests, improvement
% suggestions, etc.

For each issue tracking system, we searched for four phrases
(``dependent test'', ``test dependence'', ``test execution order'',
``different test outcome'') and manually examined the matched results. For each match, we read the
description of the issue report, the discussions between reporters
and developers, and the fixing patches (if available). This information
helped us understand whether the report is about test dependence.
%--- a test manifesting different results under different
%test execution orders. 
Each dependent test candidate was examined by
at least two people and the whole process consisted of several
rounds of (re-)study and cross checking. We ignored reports
that are described vaguely, and we excluded tests whose results are
affected by non-determinism (e.g., multi-threading).
In total, we examined the first 450 matched reports, of which 53
reports are about test dependence (some reports contain multiple
dependent tests).
All collected dependent tests are publicly available 
at: \url{http://homes.cs.washington.edu/~szhang/dependent\_tests.html}


\subsection{Findings}
\label{sec:studyfindings}

\begin{table*}[t]
\vspace{1mm}
\centering
\small{
\setlength{\tabcolsep}{.15\tabcolsep}
\begin{tabular}{|c||c|c|c|c||c|c|c|c|c||c|c|c|c|c||c|c|c|c|}
\hline
%1&2&3&4&5&6&7&8&9&10&11&12&13&14\\
\textbf{Issue}&\multicolumn{4}{|c||}{\textbf{Dependent Tests}}&\multicolumn{5}{|c||}{\textbf{\# Involved Tests for}}&\multicolumn{5}{|c||}{\textbf{Resolution}}&\multicolumn{4}{|c|}{\textbf{Root Cause}}\\
\cline{2-5}\cline{11-19}
\textbf{Tracking} &Total&\multicolumn{3}{|c||}{Severity}&\multicolumn{5}{|c||}{\textbf{Manifestation}}&
&\multicolumn{4}{|c||}{Patch Location}&Static&File & Data-& Unknown\\
\cline{3-10}\cline{12-15}
\textbf{System}&Number&Major&Minor&Trivial& Self &1 test&2 tests&3 tests & Unknown&Days&Code&Test&Doc&Unfixed&Variable&System& base &\\
\hline
Apache&26&22&3&1&0&5&18&1&2&93&5&20&0&1&9&3&8 &6\\
\hline
Eclipse&59&0&59&0&0&0&49&1&9&48&1&8&49&1&49&0&0 &10\\
\hline
JBoss&6&6&0&0&0&0&3&0&3&44&0&2&0&4&1&0& 0 & 5\\
\hline
Hibernate&3&1&1&1&0&0&3&0&0&6&0&1&0&2&0&0& 2 & 1\\
\hline
Codehaus&2&2&0&0&1&1&0&0&0&3&0&1&0&1&0&1&0 &1\\
\hline
\hline
\textbf{Total} & \dtnum &31&63&2&1&6&73&2&\unum&194&6&32&49&9&\svnum&4&10&23\\
\hline
\end{tabular}
}
\vspace{-2mm}
\caption{{\label{tab:studyresults}
Real-world dependent tests.
%Column ``Total Number'' shows the total number of identified dependent tests.
Column ``Severity'' is the developers' assessment of the importance of the
test dependence.
Column ``\# Involved Tests for Manifestation'' is the number of tests needed
to manifest the dependence. Column ``Self'' shows the number of
tests that depend on themselves. Column ``Days'' is the
average days taken by developers to resolve a dependent test.
Column ``Patch Location'' shows how developers resolved the dependent tests:
by modifying program code, by modifying test code, by adding
code comments, or not fixed.
%In column ``Dependence Root Cause'', ``other'' execution environment
%differences include language, locale, and databases.
}
}
%\todo{Split ``unfixed'' into ``documented'' and ``unfixed''}
\end{table*}

%  LocalWords:  JBoss Codehaus


Table~\ref{tab:studyresults} summarizes the dependent tests.


\subsubsection{Characteristics}


We summarize three characteristics of dependent tests:
manifestation, root cause, and developer actions.

\vspace{1mm}
\noindent \textbf{{Manifestation: at least \pertange of the dependent
tests in the study can be manifested by 2 or fewer tests.}}
A dependent test is manifested if there exists a possibly reordered
subsequence of the original test suite, such that the test
produces a different
result than when run in the original suite.
We measure the size of the reported subsequence 
in the issue report.
If the test produces a different result when run
in isolation, the number of tests to manifest
the dependent test is 1.
If the test produces a different result
when run after one other test (often, the subsequence is
running these two tests in the opposite order as the full original test
suite), then the number of tests to manifest the dependent test is 2.
Among the \dtnum studied dependent tests, we found only 2 of them
require 3 tests to manifest the dependence.
One other test depends on itself:
running the test twice produces different results than running it once,
because this test side-effects a database it reads.
We count this special case separately in the ``Self'' column
of Table~\ref{tab:studyresults}.

For the remaining \unum dependent tests, the number of involved tests
is unknown, since the relevant information is missing
or vaguely described in the issue tracking systems. For example,
some reports simply stated that ``running \textit{all} tests in one class before
test \emph{t} makes \emph{t} fail'' or ``randomizing the test execution order
makes test \emph{t} fail''.

%Due to the extremely intricated
%environment needed for each open-source project, we were unable
%to reproduce all such dependent tests and minimize the involved tests.


% In theory,
% given a $n$-sized test suite, dependent test can occur in any
% length of permutations. However, among \dtnum collected tests,
% 86 (82\%) of them can be manifested by running no more than
% 2 tests. 

%\todo{Discuss the ``unknown'' column.  What happened?  Could we not
%  reproduce it at all?  Why not?}

\vspace{1mm}
\noindent \textbf{{Root cause: at least \svratio\% of the dependent tests
in the study arise because of improper access to shared static
variables.}} Among \dtnum dependent tests, \svnum (\svratio\%) of them
arise due to inappropriate access to
shared static variables; 4 (4\%) of them arise
due to inappropriate access to the file system, and 10 (10\%) of them arise
due to inappropriate access to a database.
The root cause for the remaining 23 (25\%) tests is not apparent
in the issue tracking system.

\vspace{1mm}
\noindent \textbf{{Developer actions: dependent tests
often indicate flaws in the test code, and developers usually
modify the test code to remove them.}}
% In some cases, dependent tests are intentional and developers
% document them, but in other cases they are
% inadvertent.
Among \dtnum dependent tests, developers considered 
94 (98\%) to be major or minor problems, and the 
developers' discussions showed that the developers thought that the test
dependence should be removed.
Nonetheless, developers fixed only 38 (40\%) of the \dtnum dependent tests.
Another 49 (51\%) were ``fixed'' 
by adding comments to the test code to document the existing dependence.
For the remaining 9 (9\%) unfixed tests,
developers thought they were not important enough given the limited
development time, so they simply closed the issue report without taking
any action.

%% Do we have evidence of this?  I'm inclined to omit the discussion.  -MDE
% The primary reason is that
% % developers often \textit{intentionally} introduce test dependence because 
% it is
% easier and more convenient to write the test code. 

%There is no statistical
%significant correlation between severity and fixing.
A dependent test usually
reveals a flaw in the test code rather than the program code:
only 16\% of the code fixes (6 out of 38) are
on the program code.
In all 6 cases, the developers changed
code that performs static variable initialization, which ensures that
each dependent test will not read an undesired value.
Section~\ref{sec:repercussion} gives an example.
The other 32 code fixes were in the test code:
28 (87\%) of the dependent tests were fixed by manually specifying
the test execution order in a test script or a configuration file,
3 (10\%) of them were simply deleted by developers
from the test suite, and the remaining 1 (3\%) test was merged with its
initializing test.


%Second, many popular testing
%frameworks such as JUnit does not support to explicitly specify
%test dependence in the test code\footnote{In fact, the execution order of
%JUnit tests depends on the underlying JVM implementation~\cite{junitordering}}.
%It is non-trivial 



%Test dependence can cause problems, not only
%when test suites are reordered, but even when they are
%executed in the intended order.



\subsubsection{Manifestation of Dependent Tests}
\label{sec:repercussion}

\begin{table}
\centering
\setlength{\tabcolsep}{0.15\tabcolsep}
\begin{tabular}{|c||c|c|}
%\toprule
\hline
\textbf{Issue Tracking} & \textbf{Revealing Weakness} & \textbf{Masking Faults} \\
\textbf{System } & \textbf{in a Test Suite} & \textbf{in a Program} \\
\hline
Apache &24 & 2 \\
\hline
Eclipse & 59 & 0 \\
\hline
JBoss& 6 & 0 \\
\hline
Hibernate & 3 & 0 \\
\hline
Codehaus & 2 & 0 \\
\hline
\hline
\textbf{Total}  & 94 & 2 \\
\hline
\end{tabular}
\caption{
Classification of the collected dependent tests
based on its repercussions.
}
\label{tab:reper}
\end{table}


A dependent test may manifest as a false alarm or a missed alarm
(Table~\ref{tab:reper}).

\vspace{1mm}

\noindent \textbf{False alarm.} Most of
the dependent tests (94 out of \dtnum) 
result in false alarms:
%indicate a weakness in the test suite rather than the
%tested code:  
the test should pass but fails after reordering due to the dependence.
The test dependence arises due to incorrect initialization
of program state by one
or more tests. Typically, one test initializes
a global variable or the execution environment, and another
test does not perform any initialization, but
relies on the program state after the first test's execution.
Such dependence in the test code is often masked because
the initializing test always executes before other tests in the
default execution order. The dependent tests are not revealed
until the initializing test is reordered to execute
after other tests. 
%In this category, the test dependency
%is introduced \textit{unintentionally} by developers. 
%the default test execution order includes tests that initialize the library.  The defect is
%inconsequential until and unless the flawed test is reordered, either manually or by
%a downstream tool, to execute before any other initializing test.

%\vspace{1mm}

Sometimes developers introduce dependent tests intentionally because it is
more efficient or convenient~\cite{kapfhammeretal:FSE:2003, whittakeretal:2012}.
%DB-testing}.
Even though the developers are aware of these dependences
when they create tests, this knowledge can get lost.
Other people who are not aware of these dependences can get confused 
when they run a subset of the test suite that manifests the
dependent tests, and might report bugs about the failing tests,
even though this is exactly the intended behavior. 
If the dependence is not documented clearly and
correctly, it can take a considerable amount of time to work out that
these reported failures are spurious. 
The Eclipse issue tracking system contains at least
49 such dependent tests.
%Or worse, the developers may try
%to fix a bug that is not there.
In September 2003, a user filed a
bug report in SWT~\cite{swt}~\cite{eclipsebug},
stating that 49 tests were failing unexpectedly
if she ran any other test before \code{TestDisplay} --- 
a test suite creates a new \code{Display} object and tests it.
However, this bug report was spurious and was
caused by undocumented test dependence.
All 49 failing tests are dependent tests with the same
root cause: in SWT, only one global \code{Display}
object is allowed; the user ran tests that
create but do not dispose of a \code{Display} object, while
the tests in \code{TestDisplay} attempt to create
a new \code{Display} object, which fails, as one
is already created. This is the desired behavior of SWT,
and points to a weakness in the test suite.
% rather
%than the code.

\vspace{1mm}

\noindent \textbf{Missed alarm}. In rare cases,
dependent tests can hide a fault in the
program, \emph{exactly} when the test suite is executed in its default
order. Masking occurs when a test case $t$ \emph{should}
reveal a fault, but tests executed before $t$ in a test suite always
generate environments in which $t$ passes accidently and
does not reveal the fault. 
Tests in this category result in \textit{missed alarms} ---
a test should fail but passes due to the dependence.


\begin{figure}[t]
%\noindent \textbf{\small{Fault-related code in CLI:}}
%\vspace{-2mm}
\begin{CodeOut}
\begin{alltt} 
public final class OptionBuilder \{
  \textbf{private static String argName = null;}
  private static void reset() \{
    ...
    \textbf{argName = "arg";}
    ...
  \}
\}
\end{alltt}
\end{CodeOut}
\vspace*{-15pt}
\caption{Simplified fault-related code in CLI~\cite{cli} (revision 661513).
The fault was masked by two dependent tests for over 3 years.
}
\label{fig:option_builder}
\end{figure}

%  LocalWords:  OptionBuilder argName arg CLI


We found two such dependent tests in
the Apache CLI library~\cite{cli}.
Figure~\ref{fig:option_builder} shows the simplified fault-related
code. The fault is due to improper initialization of the static variable
\CodeIn{argName}. The static variable \CodeIn{argName} should be set
to its default value \CodeIn{"arg"} by CLI's clients via calling
method \CodeIn{reset()}. Otherwise, \CodeIn{argName}'s
default value remains \CodeIn{null} and should \emph{not} be
used in creating an \CodeIn{OptionBuilder} object.
In CLI, two test cases 
\code{Bugs\-Test.test13666} and \code{Bugs\-Test.test27635}
can reveal this potential fault by directly initializing
a \CodeIn{OptionBuilder} object without calling \CodeIn{reset()}.
These two tests fail when run in isolation,
but both pass when run in the default order. This is because
in the default order, tests running \emph{before} these
two tests call \CodeIn{reset()} at least once, which sets
the value of \CodeIn{argName} and masks the fault.

%Both dependent tests can reveal this fault,  but
%the default order of test execution makes both tests pass
%accidentally. 

Such dependent tests have a non-trivial impact in practice.
This fault was reported in the bug database several times~\cite{clibug},
starting on March 13, 2004 (CLI-26). The report was marked as resolved
\emph{three years} later on March 15, 2007 when developers
realized the test dependence. The developers fixed this
fault by adding a static initialization block which
calls \CodeIn{reset()} in class \CodeIn{OptionBuilder}.

%\edit{where should we emphasize that masking faults is an
%orthognonal issue of fixing the dependence on code or test?}


\subsubsection{Implications for Dependent Test Detection}

We summarize the main implications of our findings.

\noindent \textbf{{Dependent tests exist in practice, but
they are not easy to identify.}}
None of the dependent tests we studied can be identified by
running the existing test suite in the default order. 
Every dependent test was reported when the
test suite was reordered, either accidentally by a user or
by a testing tool. This indicates the need
for a tool to detect dependent tests.
%dependent test detection techniques should
%explicitly search for such dependent tests.

\vspace{1mm}
\noindent \textbf{Dependent test detection techniques
can bound the search space to a small number of tests.}
In theory, a technique needs to exhaustively execute
all $n!$ permutations of a $n$-sized
test suite to detect all dependent tests. This is
not feasible for realistic $n$.  Our study shows that
most dependent tests can be manifested by executing
no more than 2 tests together. Thus, a practical technique
can focus on running only short subsequences (whose
length is bounded by a parameter $k$)
of a test suite. This will reduce the number of permutations
to $O(n^k)$, which is tractable for small $k$ and $n$.

\vspace{1mm}
\noindent \textbf{Dependent test detection techniques
should focus on analyzing accesses to global variables.}
Dependent tests can result from many
interactions with the execution environment, including
global variables, file systems, databases, network, etc.
However, as reflected by our study, more than half of the
real-world dependent tests are caused
by improper static variable accesses. This implies that a technique
may achieve a high return by focusing on global variables.


%\vspace{1mm}
%\noindent \textbf{Dependent test fixing tool
%Test dependence reveals flaws in the test code.}
%This indicates that a potential dependent test fixing tool should target
%the test code


\subsection{Threats to validity}

Our findings apply in the context of our study and methodology and may not
apply to arbitrary programs.
The applications we studied are all written in 
Java and have JUnit test suites.  

We accepted the developers' judgment regarding which tests are dependent,
the severity of each dependent test, and how many tests are needed
to manifest the dependence.  We did not intentionally ignore
any test dependence in the issue tracking system.
However, a limitation is that the developers might have made a mistake,
might not have marked a test dependence in a way we found it
(different search terms might discover additional dependent tests), and are
unlikely to have found all the dependent tests in those projects. 


%  LocalWords:  JBoss Codehaus reproducibility multi dependences SWT CLI
%  LocalWords:  TestDisplay test13666 subsequences subsuite



%\section{What's the essence of what causes the dependences?}
\section{Theory}
\label{sec:formalism}

%\todo{DN}{I think we will need to go over the paper later on in the process to make
%really sure we are consistent about test dependence definitions and our writing.
%If I am clear, we define dependence between two tests; we often (informally?)
%talk about a suite with dependence(s).}
%\todo{JW}{Informally, you are correct. I'll make sure to clarify this
%in the text}
%\todo{DN}{We are also inconsistency about ``dependence'' vs. ``dependency'' vs. ``dependencies'' and
%such.  Probably not a big deal, but if we have time...}
%\todo{DN}{General comment about the intuition below -- it's too long now, which makes it less intuitive.}

A standard textbook 
states that ``[a] test case includes not only input data but
also any relevant \emph{execution conditions}
\dots''~\cite[p.~152, emphasis added]{pezze-young:2007}.   
This characterization is
consistent with the example that 
 piqued our interest in test dependence: we seren\-dip\-itously identified a bug in an open-source
system when we found that running individual tests one-by-one---each
in a newly initialized environment---produced different
results from running the entire test suite normally (\emph{i.e.}, with a single initialized environment followed by
the sequential execution of each test in order)~\cite{DBLP:conf/sigsoft/MusluSW11}.  These tests
shared global variables, and the test results varied depending on the
values stored in these variables.  That is, relevant execution conditions---specifically, pertinent parts
of the implicit \emph{environment} comprising global variables, the file system, operating system services, etc.---were neglected.
%As shown in Section~\ref{sec:examples}, most real examples of test dependence we have seen to date relate
%to mistakes in the initialization of test environments.
%\footnote{Test frameworks such as JUnit 
%provide \emph{mechanisms}, such as \code{@Before} and \code{@After} annotations, to help developers more easily
%establish execution conditions.  They are not intended to, and
%do not, \emph{enforce} any policies
%to ensure that developers use these mechanisms consistently and effectively.}

To characterize the relevant execution conditions precisely, 
our formalism below explicitly represents the notions of
(a) the order in which test cases are executed and (b) the environment in which a test suite is executed.  

Consider two examples 
of how test dependences arise in terms of order and environment (Figure~\ref{fig:dep_examples}).
In the leftmost example, \code{test2} checks
the value of a variable that has been assigned elsewhere. If the tests
are executed in the order \suite{\code{test1},\code{test2}}, both tests will pass,
% Kivanc: Cut: There is no \ldots in test1, so I don't think below explanation
% is needed.
% (assuming \code{test1} does not
% change the value of \code{a} after the initial assignment)
while running \code{test2} first will make it fail.    The rightmost  example extends this
principle to multiple tests. While none of the $n-1$ tests prior to
\code{testn} will fail, they all must execute in this particular order
for \code{testn} to pass. 

The global variables involved are usually buried deep in
the program code, and the assertions do not directly check them,
but rather check values that have been computed from
them. In any non-trivial real-world program, this
deep nesting effectively hides potential dependencies from developers,
and they may only become aware of them when a subtle bug leads them
there.  Therefore, we explicitly
distinguish potential test dependences (Definition~\ref{def:dependency})---those that could cause a variation in test suite results 
under \emph{some} environment and order---and manifest test
dependences (Definition~\ref{def:manifest})---those that are guaranteed to cause a
variation in test suite results under a \emph{specific} environment and order.  

%Consider the two examples (Figure~\ref{fig:dep_examples})
%of the basic
%way dependencies between tests arise in practice. \code{test2} checks
%the value of a variable that has been assigned elsewhere. If the tests
%are executed in the order \code{test1, test2}, both tests will pass (assuming \code{test1} does not
%change the value of \code{a} after the initial assignment),
%while running \code{test2} first will make it fail.  To determine
%potential test dependences would require an analysis of the read/write
%behavior of the tests (for example, under what conditions, if any, does \code{test1} change the value of \code{a}?); such analyses are well-known to be
%imprecise and/or incomplete.    


% to cleanly set all execution conditions for each test
%case. Methods marked with the
%annotations \code{@Before} and \code{@After} are executed before and
%after each test case and are intended to initialize and clean the
%execution conditions for each test case.
%}

%When not all state in the environment is cleanly initialized, test
%dependences can arise, because then the actual state when a test case
%executes can change depending on, for example,


%Very informally, dependence between tests, like the one described above, arises when test cases do not include all% relevant execution conditions.
%In particular, when they compute their result based on a shared global data, and this shared global data (we call it \emph{environment}) is initialized external to the test case.

%\todo{DN}{I'd love to see if we can reduce or eliminate the discussion of JUnit here.  The following
%paragraph, for example, seems confusing in the following sense: if junit has this, why is there a
%dependence problem?  This is obvious to us, but I don't want people thinking about that here. Moved it
%to a footnote -- before/after -- and rewrote a little}
%

%In this section we first give an intuition what causes test
%dependence and what its consequence may be. Then we formally define
%our notion of test dependence, and lastly we demonstrate how these
%notions lead to the \emph{test dependence detection} problem, and how
%we can solve it.

%\todo{JW}{Find a better section heading or remove the heading}
%\subsection{Intuition}



%
%rely on executing in a
%particular context, for example some test may assume that global variables have
%been initialized to specific values, without confirming the expected
%context before they execute. 

%\todo{DN}{I'm at present inclined to move the example to here, but the test structure/results
%text.  That is, use the example to describe those rather than vice versa.  Sort of like: (a) here
%is a simple example; (b) note that a complete analysis of test dependence would require a full
%analysis of the code in test1 to determine possible values for a; (c) instead of considering that
%complicated analysis that would describe the potential for dependence, we use the test oracles/results
%to represent the outcome of the execution, in a given environment.  Or possibly move the
%potential/actual text up instead, and describe how we address it, then the example?}




%Consider the two examples in Fig.~\ref{fig:dep_examples}. The example
%in Fig.~\ref{fig:dep_examples:direct} shows the most basic
%way dependencies between tests arise in practice. \code{test2} checks
%the value of a variable that has been assigned elsewhere. If the tests
%are executed in the order \code{test1, test2}, both tests will pass,
%while running \code{test2} first will make it fail.
\begin{figure}
\subfigure[Direct dependence\label{fig:dep_examples:direct}]{
\begin{minipage}{.47\columnwidth}
\code{test1 \{ $\mathtt{v_1}$ = 4 \}}

\code{test2 \{
  assert $\mathtt{v_1}$==4 \}
}

\mbox{ }\\

\mbox{ } 
\vspace{0.5em}
\end{minipage}
}
\subfigure[Chain dependence\label{fig:dep_example:chain}]{
\begin{minipage}{.5\columnwidth}
\code{test1 \{ $\mathtt{v_1}$ = 1 \}}

\code{test2 \{
  $\mathtt{v_2}$ = $\mathtt{v_1}$ + 1 \}}

\code{...}

\code{testn \{ \\
\mbox{}\hspace{1ex} assert $\mathtt{v_{n-1}}$ == n-1 \}}
\vspace{0.5em}
\end{minipage}
}
\caption{Examples for basic causes of test
dependences}\label{fig:dep_examples}
\end{figure}
%

%\todo{SZ}{for example (b), perphas we can make it more clearer as
%follows:  test1 \{a++\}, test2\{a++\}, ..., testn \{assert a == n-1\}}
%
%\todo{JW}{While this is also a dependence, it does \emph{not} enforce
%any particular order on the first n-1 tests.}
%
%\todo{KM}{I second Jochen. We discussed this quite a bit with him and I think
%he came up with the most readable and easiest example that forces the execution
%of only t1-t1-\ldots-tn to pass and all other orders to fail.}

%Test dependences arise when tests rely on state that is generated
%by other tests. %CLI is a case in point. 
%Most examples we found are quite direct dependecies on
%global variables, where one test implicitly relies on a global variable to be in
%a certain state before executing the sequence of methods to be tested. 
%At a more abstract level, we can think of test dependence as
%read/write conflicts between different transactions. Each test reads
%and writes variables. If a test implicitly assumes a variable to be in
%a particular state, but does not ensure this state before it executes,
%the test may fail.
%\todo{JW}{
%Thinking of the causes for test dependences as read/write conflicts
%might go far. Intuitively, I think the cases that cause parallel
%transactions to abort are the \emph{good} cases for us (because each
%test ensures that it has written the values it needs). All other cases
%seem potentially hazardous. I'll dig into this tomorrow}

\subsection{Definitions}
\label{sec:definitions}

We express test dependences through the results of executing
\emph{ordered} sequences of tests in a given \emph{environment}.

%This is in strong contrast to most
%existing work that considers test suites in general as sets, and thus
%ignores the ordering aspect that is important here.
%Informally, a test dependence arises when the results of executing
%test suites in different orders differ. The following formal
%definitions make our notion of test dependence precise.
%\todo{DN}{I took out the ``informally'' sentence or so here, since we've done that
%already a number of times, and now we're going formal!}

\begin{definition}[Environment]
An \emph{environment} \env for the execution of a test
consists of all values of global variables, files,
operating
system services, etc. that
can be accessed by the test and program code exercised by the test
case.
%
%The set of all possible environments is denoted $\environs$.
\end{definition}

\begin{definition}[Test]
%\todo{DN}{I'm not sure what ``fixed, well-defined inputs means'' and I'm not sure we need it.
%Why not just ``a sequence of program statements''?}
%
%\todo{JW}{Abstractly, all programs take inputs and compute something
%based on them (unless they are a constant function or
%non-deterministic). A test case is not complete without well-defined
%inputs, simply because you can't run the program. The should be
%well-defined and fixed, so that you can repeatedly run your test,
%always get the same result, and be sure what that result should be
%according to your spec. While I agree that this is not well
%formulated, I think if we ask serious testing people to read this and
%we ignore the whole input thing (a lot of test generation only deals
%with generating inputs), we might be in trouble.}

A test is a sequence of program statements, executed with fixed,
well-defined inputs, and an oracle that
decides whether a test passes or fails.
\end{definition}

%\todo{DN}{Should we footnote this next oracle discussion?  And shorten it, since some of the
%soundness/completeness issues are with respect to specifications, which we don't mention/address.
%Should we?}

%\todo{JW}{I realized while I wrote this that eventually we will have
%to include oracles. That's why I at least wanted to mention something
%here. But I don't think it will be possible (and sensible) to do fully
%integrate that now. The paragraph below is a summary of why we need to
%talk about this.}

%Generally, oracles are an important aspect of testing. Here, however,
%only two facts are relevant. First, Staats et al.\ discuss that oracles are
%often neither sound nor complete as we
%define them below often imply unsound oracles. Second, 
%In particular, that means that an oracle can decide that a
%test passes, while the program is incorrect. Test dependences
Simplifying from Staats
et al.~\cite{staatsetal:ICSE:2011}, and without loss of generality,
we consider an oracle to be a boolean predicate over tests and environments.

%While oracles in practice, and specifications in theory, play an
%important role in testing, we do not incorporate them in our
%formalism, because explicit specifications often do not exist, and for
%our purposes the oracle judgement, rather than its full definition, is sufficient.

\begin{definition}[Test Suite]
A test suite\/ $T$ is an $n$-tuple (i.e., ordered sequence) of tests
\suite{t_1, t_2, \dots, t_n}.

%When it is clear which test suite we are talking about, or the details
%of the suite are not important, we use $T$ to denote the entire test
%suite $(t_1, \dots, t_n)$.
\end{definition}

\begin{definition}[Test Execution]
Let\/ $T$ be a test suite and\/ \environs\ the set of all possible
environments.
The function\/ $\varepsilon: T \times \environs \rightarrow
\environs$ is called test
execution. $\varepsilon$ maps the execution of a test\/ $ t \in T$ 
in an environment\/ $\env \in \environs$ to the new (potentially updated)
environment\/ $\env'$.

For the execution of test suites\/ $T = \suite{t_1, t_2, \dots, t_n}$
we use the shorthand\/
$\exec{T}{\env}$ for $\exec{t_n}{\exec{t_{n-1}}{\dots \exec{t_1}
{\env} \dots }}$.
\end{definition}

\begin{definition}[Test Result]
The result of a test $t$ executed in an environment\/ $\env$,
denoted\/ \result{t}{\env} (and sometimes referred to 
as an oracle judgment), is defined by the test's oracle
and is either \pass or \fail.

The result of a test suite\/ \suite{t_1,\dots,t_n}, executed in an
environment\/ \env, denoted\/ \result{\suite{t_1,\dots,t_n}}{\env} is a
sequence of results\/ \suite{o_1,\dots,o_n} with $o_i \in \{\pass,\fail\}$.

%For test outcomes of sequences where all individual outcomes are
%either \pass or \fail, we use the notation $(\pass^*)$ and $(\fail^*)$,
%respectively.

For example, $\result{\suite{t_1, t_2}}{\env_1} = \suite{\fail, \pass}$ represents that 
given the environment\/ $\env_1$, $t_1$ fails and\/ $t_2$ passes.
\end{definition}


%\todo{DN}{Do we need the following paragraph?  I think the dot notation is unnecessary,
%since it's always a test suite, even if not the ``original'' one that we put in this
%place.  Also, I would -- again -- like to remove the references to Junit, VMs, etc. here.}
%
%The notation \result{\cdot}{\env} implies that only the explicitly specified
%tests are run in a given VM, execution, and environment \env. 
%In specific frameworks, such as JUnit, a single test would 
%include any automatically executed setup and the actual test method.

%The terminology for \pass and \fail is general:
%a test passes when its outcome is as expected and
%no assertion is violated; a test fails under all other circumstances.\footnote{In
%the JUnit framework, for example, this means that the outcomes \emph{failure} and \emph{error}
%would both be treated as \fail in the formalism.}

%The execution of tests can change the environment if one of the
%variables in it is modified. We write \exec{T}{\env} for the environment
%that derives from executing test suite $T$ in the original environment
%$\env$. $\exec{T}{\env} \neq \env$ is a necessary but not
%sufficient condition for test order dependencies.

\begin{definition}[Potential Test Dependence] \label{def:dependency}
Given a test suite\/ $T$,
a test\/ $t_l \in T$ is \emph{potentially dependent} on test\/ $t_k
\in T$, if and only if\/
$\exists \env : \result{T}{\env} = \suite{o_1,\dots, o_n} \wedge
\result{\suite{t_k,t_l}}{\env} = \suite{o_k, o_l} \wedge
\result{t_l}{\env} = \neg o_l$.
We write\/ $t_k \prec t_l$ when\/ $t_l$ is potentially dependent on\/ $t_k$.
\end{definition}

This definition is \emph{dynamic} because dependence arises only
if there exists an environment in which actual test results would differ.
It is \emph{potential} as it only requires the existence of such an
environment, but does not
guarantee that the test suite will ever be executed in the context
of such an environment.

%One direct implication of this definition is that $t_k \prec t_l
%\rightarrow \exec{t_k}{\env} \neq \env$, that is $t_k$ must modify the
%environment. Additionally, $t_l$ must read values from the
%environment.
%\todo{DN}{I'm not certain if the rightarrow is implication.  I'm not sure
%why this implication is crucial.}

%\begin{theorem}\label{theorem:pairs}
%
%$( t_1, \dots, t_k ) \prec t_n \Rightarrow \exists_{t \in (t_1, \dots,
%t_k)} : t \prec t_n $
%
%\end{theorem}
%
%\begin{proof}
%Proof by contradiction.
%
%Let $s = (t_1, \dots, t_k)$ be a test sequence of length greater than
%one, and let $s$ be the shortest sequence such that $ s \prec t_n$.
%I.e. there is no real prefix or suffix of $s$ that $t_n$ depends on.
%
%That means that there is an environment $\env_x$, such that $R(s,t_n :
%\env_x) \neq R(s:\env_x) \circ R(t_n:\env_x)$ 
%and for all environments $\env$
%$R(t_2, \dots, t_k, t_n:\env) = R(t_2,\dots, t_k:\env) \circ
%R(t_n:\env)$. In particular, this holds also for the environment
%$ \env_1 = \gamma(t_1, \env_x)$. 
%
%$R(s,t_n :\env_x) \neq R(s:\env_x) \circ R(t_n:\env_x)$
%and $\env$
%$R(t_2, \dots, t_k, t_n:\env_1) = R(t_2,\dots, t_k:\env_1) \circ
%R(t_n:\env_1)$ implies that $R(t_1,t_n:\env_x) \neq R(t_1:\env_x) \circ
%R(t_n:\env_x)$, which contradicts the
%hypothesis that there is no prefix of $s$ that $t_n$ depends on.
%\end{proof}
%
%Theorem~\ref{theorem:pairs} is useful for theoretical work because it
%reduces the complexity of the structures we have to study to pairs of
%tests. From a practical point of view, however, things are not quite
%as easy, because the proof only states the existence of environments
%that will expose the dependency, but does not constructively describe
%how to find these environments. Further, the whole theoretical
%framework by its nature cannot relate this to the way environments are
%constructed by actual testing frameworks such as JUnit. 
%
%With the above proof it is easy to see that the following corollary
%also holds:
%
%\begin{corollary}
%$t_k \prec (t_m, t_n) \Rightarrow t_k \prec t_m \vee t_k \prec t_n$
%\end{corollary}
%
%\begin{proof}
%Case 1: If the outcome of $t_m$ differs, it directly implies $t_k \prec t_m$.
%
%Case 2: The outcome of $t_m$ is the same, but $t_n$ differs.
%$ t_k \prec (t_m,t_n) \Rightarrow (t_k, t_m) \prec t_n \Rightarrow t_k
%\prec t_n$ 
%
%\end{proof}
%
%The above definitions and the properties of order dependencies that
%follow from these definitions allow some reasoning about order
%dependencies already. From a practical point of view, however, it
%would be desirable that the relation implied by the order dependence
%definition were a partial order. However, I do not think that that is
%true. The following theorem frames this:
%
%\begin{theorem}[Cyclic dependencies]
%There exist test suites $S$ such that for some $t_m, t_n \in S: t_m
%\prec t_n \wedge t_n \prec t_m$
%\end{theorem}
%
%This is an artifact of the fact that we allow reordering and our
%definition of order dependency hinges on the \emph{outcome} of
%executions.
%
%Intuitively, test dependencies $t \prec r$ between two tests $t$ and $r$ arise
%when $r$ expects some variable $v$ of the environment to have a specific
%value, but in the environment $\gamma(t,\env)$ $v$ has a different
%value (for some arbitrary $\env$).
%This situation can arise either by $r$ assigning an unexpected value
%to $v$, or $r$ failing to assign the expected value.
%
%\subsubsection{Manifest Dependencies}
%
%
%For practical reasons, the pure existence of an environment that
%exposes dependencies is not sufficient. Most of the time we are more
%concerned whether or not dependencies will become apparent in the
%environment we are actually dealing with.
%
%\todo{JW}{ We are aiming at defining enough theory for useful
%algorithms. In practice we derive our environments from some sort of
%default environment provided by the test framework. While we don't
%know the exact shape of that environment, assuming deterministic
%programs and tests, all other environments created through test
%execution derive from this initial environment. This must have some
%sort of impact on the complexity of our problem.}

%\begin{definition}[Subsequence]
%Given an ordered se\-quence $T = \suite{t_1, \dots t_n}$, an ordered
%sequence $S = \suite{t_i,\dots,t_k}$ is a subsequence of $T$, if and
%only if the length of $S$ is less than or equal to the length of $T$,
%the elements of $S$ are also elements of $T$ and 
%preserve the order of the elements in $T$.
%Analogously to sets, we write $S \subseteq T$ if $S$ is a subsequence of
%$T$.
%\end{definition}

We refine this definition of dependence to require a concrete environment guaranteed
to \emph{manifest} a dependence:
\begin{definition}[Manifest Dependence] \label{def:manifest}
Given a test suite\/ $T$, two dependent tests\/ $t_i, t_j \in T$,
the dependence\/ $t_i \prec t_j$ \emph{manifests} in a given
environment\/
$\env$ if\/ $\exists {S \subseteq T}: t_i, t_j \in S \wedge
\result{T}{\env}
= \suite{o_1, \dots, o_n} \wedge \result{S}{\env} =
\suite{\dots,o_i,o_j} \wedge \result{t_j}{\env} = \neg o_j$. We
write\/ $t_i \manifest{\env} t_j$ for manifest dependence.\footnote{$S \subseteq T$ means that $S$ is a subsequence of
$T$.}
\end{definition}

Note that the dependent tests $t_i$\/ and $t_j$ do not have to be
adjacent in the original test suite, but that they must be adjacent in
the shortest test suite that manifests the dependence.

The intuition behind manifest dependences is that in practice we
do not construct arbitrary environments to execute tests in. Rather,
we use the natural environment $\env_0$ provided by frameworks such as JUnit,
and the only modifications of this environment happen through the
tests and the tested code. Hence, potential dependences manifest only
if there is a sequence of tests $S^*$ whose execution
$\exec{S^*}{\env_0}$  produces the
environment $\env'$ that will reveal the dependency.
The algorithm we propose in Section~\ref{sec:algorithm-tool} detects
dependences by running tests and checking for different test results,
hence it can only detect manifest dependences.
%
%Note that this definition defines manifest dependencies with regard to
%a given test suite $T$ and a given environment $\env$. Thus it
%reduces the space of possible dependencies considerably. In practice,
%the given environment $\env$ is what the text execution framework
%provides \emph{before} it executes the first test.
To improve algorithms that are affected by test dependences, we are
interested in the shortest test suite $S^* \subseteq T$ that manifests a
dependence, because these define the partial order of test execution
that such techniques must respect.

%\begin{theorem}[Dependency bound]
%Let $U$ by a true sub-sequence of length $k$ of 
%a given test suite $T$, and let $U$
%be the shortest such sub-sequence that manifests a dependency with
%test t, i.e. $ U \manifest{\env} t$.
%Then there are at least $k$ variables responsible for manifesting the
%dependency.
%\end{theorem}
%
%\todo{JW}{ I'm not sure if this theorem actually holds. I'm working on
%it, but if anyone has ideas and insights, they are certainly welcome.}
%\todo{KM}{ I don't think this is true, consider the following: \\ 
%Initial environment: a = 0 \\ 
%Test1: {assert a == 3} \\
%Test2: {a = a + 1, assert true} \\
%Test3: {a = a + 2, assert true} \\
%In this setup Test1 depends on any order of Test2 and Test3. This is the minimum
%dependence (i.e., running only Test2 or Test3 won't suffice). However, the
%dependency only requires one variable, `a'. }
%
%\begin{proof}
%Proof by induction.
%Let $k=1$. Trivial. By definition there must be at least one variable
%that creates and manifests the dependency.
%
%Let $k=n+1$. Assume there are less than $k$ variables involved. 
%\end{proof}

In later sections we often talk about executing tests in isolation, or
executing all tests in a test suite in isolation. This is an important
approximation to detecting test dependences.

\begin{definition}[Test Isolation]
The result of executing a test\/ $t$ in isolation, given an
environment\/
$\env_0$ is the result\/ \result{t}{\env_0} of executing that test in
the given environment.  

The result of executing all tests in a test suite\/ \suite{t_1, \dots,
t_m} in isolation is the
sequence of results\/ \suite{\result{t_1}{\env_0}, \dots,
\result{t_n}{\env_0}}.
\end{definition}

\subsection{Detecting Dependent Tests}

From a practical perspective, techniques that affect the ordering of
test suites must respect dependences. Otherwise their results cannot
be interpreted correctly in the presence of dependences. Detecting
dependences in existing test suites is thus an interesting problem.
In the following, we first give a precise definition of the problem of
detecting dependent tests, and then prove that in general this problem
is NP-complete. In Section~\ref{sec:algorithm-tool} we outline an
algorithm that approximates solutions efficiently.

\begin{definition}[Dependent Test Detection Problem]
Given a set suite\/ $T = \suite{t_1, \dots, t_n}$ and an environment\/
$\env_0$, for a given test\/ $t_i \in T$, is there a test suite\/ $S
\subseteq T$ that manifests a test dependence involving\/ $t_i$? 
\end{definition}

%To prove that this problem is NP-complete,
We prove that this problem is NP-hard by reducing the NP-complete Exact Cover problem
to the Dependent Test Detection
problem~\cite{karp:NP:1972}. 
Then we provide a linear time algorithm to verify any answer to the
question.
%Then we sketch an exponential
%time algorithm that can solve the problem.
Together these two parts prove the the Dependent Test Detection Problem is NP-complete.

\begin{theorem}
The problem of finding a test suite that manifests a dependence is
NP-hard.
\end{theorem}

\begin{proof}
%We prove this claim by reducing Exact Cover to Dependent Test
%Detection.
In the Exact Cover problem,
we are given a set $X$ = \{$x_1, x_2, x_3, \dots, x_m$\} and a collection $S$ of subsets of $X$.
The goal is to identify a sub-collection $S^*$ of $S$ such that \textit{each}
element in $X$ is contained in \textit{exactly} one subset in $S^*$.  

Assume a set $V = \{v_1, v_2, v_3, \dots, v_m\}$ of variables,
and a set $S = \{S_1, S_2, \dots, S_n\}$ with $S_i \subseteq V$ for $ 1\leq i
\leq n$. 

We now construct a tested program $P$, and a test suite
$T = \suite{t_1, t_2, \dots t_n , t_{n+1}}$ as follows:

\begin{itemize}

\item $P$ consists of $m$ global variables 
$v_1, v_2,\dots, v_m$, each with initial value 1.

\item 
For $1 \le i \le n$, $t_i$ is constructed as follows:
for $1 \le j \le m$, if $x_j \in S_i$, then adding a
single assignment statement \CodeIn{$v_j$ = $v_j$ - 1} to $t_i$.

$t_{n+1}$ consists only of the oracle
\CodeIn{assert($v_1$ != 0 || $v_2$ != 0 \dots || $v_m$ !=0)}.

\end{itemize}

In the above construction, the tests $t_i$ for $1 \le i \le n$ 
will always pass. The only
test that may fail and thus exhibit different behavior is $t_{n+1}$, which 
\emph{only} fails when each variable $v_i$ appears exactly
once in a test case.

For the given test $t_{n+1}$, if we can
find a sequence \suite{t_{i_1}, t_{i_2},\dots, t_{i_j}}
that makes $t_{n+1}$ fail, the subsets $S^*$ corresponding
to each $t_{i_j}$ are an exact cover of $V$.
\end{proof}

In practice, the structure of the proof directly translates to the
structure of test suites. $t_{n+1}$ is the dependent test, $S$ is
defined by the tests that write variables used by $t_{n+1}$, and every
exact cover of $S$ represents an independent shortest test suite that
is a manifest dependency of $t_{n+1}$.

To complete the proof that Dependent Test Detection is NP-complete, we
provide an algorithm to verify solutions to the problem, that is
linear in the size of the test suite.
Given a test suite $T$ and a test suite $S \subseteq T$ that is said
to manifest a dependency on $t_i$, we first execute $T$, then $S$, and
compare the result for $t_i$ in both executions. 
If the results differ the solution is correct, if they do not differ,
the solution is rejected.
Since in the worst case we have to execute $2n$ tests, the complexity
of this algorithm is linear.




\subsection{Discussion}

This formalism has dual intents:
to lay a foundation for reasoning about test dependence
in a precise way; and
to be consistent with and to allow for approximate and practical algorithms and tools~(Section~\ref{sec:algorithm-tool}).

This second intent, of course, requires a balance of theory and
practice.  First, the dynamic nature of our our view on dependences 
allows us to avoid the complexity issues that come with a static
approach. With a static approach, it would be essential 
to decide how to address undecidability. The most
likely and common approach being to choose soundness with respect to all
possible executions and accepting the consequent imprecision of the analysis.
Second, our focus on manifest dependence, when realized in a tool will
only identify true positives, although it may miss some
dependences (false negatives).  It is often easier to have tools
with this kind of property accepted by practitioners than some other
kinds.  Third, the manifest test dependence problem is NP-complete;
although that is daunting (but less so than undecidability),
approximate algorithms can be defined for large classes of NP-complete
problems.  

The examples in the following section and the algorithm and tool following that
give a better flavor for
why we made these decisions.
The degree to which these are the ``right'' (or at least effective) decisions
is itself an empirical
question beyond the scope of this paper.

%.  The following section we will illustrate these factors in action.
%We discuss several examples in detail,
%showing how these different factors contribute to test dependence in
%real-world applications. 


%The potential for test dependence arises from the test structure and
%the oracle:
%%from the test results: 
%what
%global state do the tests read and write, and does that global state contribute
%to the computed result evaluated by the oracle?  
%At the same time, the \emph{potential} for a test dependence
%is realized only if 
%the values derived from the context \emph{actually} affect the test results.
%affect program state that
%is checked by the oracle can a dependence on the environment affect
%the outcome of a test.
%This potential \emph{manifests} when test results differ between
%executions in different environments.

%The fact that manifestation of test dependence depends on both test
%structures and test results means that dependences can silently propagate
%through sequences of tests before they become apparent.

%The abstract examples above, and the concrete examples presented 
%in Section~\ref{sec:examples} share some common features that
%ultimately lead to dependences. 
%%All applications and libraries we studied 
%They rely on global variables to some extent, and the
%tests that check behavior that depends on these variables usually
%assume these variables to be in some state. This state is typically
%defined by the default execution order of the test suite, and rarely
%established explicitly before each test.

%\todo{sz}{does that make sense to put the following text (needs slight revise)
% after Section 4. I feel they are more relevant to concrete examples.}

%To summarize, the features that contribute to the test dependences
%we discuss are:
%\begin{itemize}
%\item Test results depend on global state.
%\item Tests do not check their preconditions explicitly, but rely on
%the test suite to ensure them.
%%\item 
%%\todo{JW}{The following point has lead to a lot of confusion. We have
%to clarify or remove it}
%The strength of test oracles. Stronger oracles are more likely
%to cause dependence than weaker oracles. 
%%in the sense that they check for concrete
%%values rather than conditions. 
%For example, a check for $x = 5$
%rather than $x > 0$, is more likely to fail, while the latter might
%still be sufficient to check
%whether a specific branch of the program was executed. 
%\todo{KM}{I did not like
%this argument much. Though I agree that having weak oracles would reduce the
%dependences, it is still possible (in theory) to write dependent tests with
%weak oracles.}
%\end{itemize}



% vim:wrap:wm=8:bs=2:expandtab:ts=4:tw=70:



\section{Manifestations}
\label{sec:examples}

\newcommand{\unknown}{N/A\xspace}
\newcommand{\ignore}{---\xspace}
\newcommand{\infy}{$\infty$\xspace}

\begin{table*}
\centering
\setlength{\tabcolsep}{0.12\tabcolsep}
\begin{tabular}{|l|c|c|C|C|C|c|c|c|c|c|c|c|c|c|c|c|c|c|}
%\toprule
\hline
\textbf{Subject} & \textbf{\#} & \multicolumn{8}{c|}{\textbf{\# Detected Dependent Tests}} & \multicolumn{8}{c|}{\textbf{Analysis Cost (seconds)}}\\
%\midrule
\cline{3-18}
\textbf{Programs} & \textbf{Tests} & \textbf{Rev} & \multicolumn{3}{c|}{\textbf{Randomized}} & \multicolumn{2}{c|}{\textbf{Exhaustive }} & \multicolumn{2}{c|}{\textbf{Dep-Aware}} & \textbf{Rev}& \multicolumn{3}{|c|}{\textbf{Randomized}} & \multicolumn{2}{c|}{\textbf{Exhaustive }} & \multicolumn{2}{c|}{\textbf{Dep-Aware}} \\
%\cline{3-8}\cline{10-15}
& & & \smalltrialnum & \mediumtrialnum & \trialnum& \; $k$=1 & $k$=2 & \quad $k$=1 \;\; \quad & $k$=2 && \smalltrialnum & \mediumtrialnum & \trialnum & \; $k$=1 & $k$=2 &  \quad $k$=1 \quad \quad & $k$=2  \\
\hline
%\bottomrule
\multicolumn{18}{|l|}{ }\\
\multicolumn{18}{|l|}{\textbf{Human-written unit tests} }\\
\hline
%JFreechart & \jfreecharttests & 6 & 8 & 8 & 0 & $\ge$0 * & 0 & $\ge$0 * &  66  & 625 & 6097 & 694 & 2$\times$$10^6$ *  &310  &  1$\times$$10^6$ *\\
%the data of jfreechart is above, MUST update the total column
\jt & \jodatimetests & 2 & 1 & 1 & 6 & 2 & $\ge$2 * & 2& $\ge$2 * & 18&   57 & 528 & 5538 &1265& 4$\times$$10^6$ * & 291 & 5$\times$$10^5$ *  \\
XML Security& \xmlsecuritytests & 0 & 1 & 4 & 4 &4 &4 & 4 & 4  & 18&65 & 594 & 5977 & 106 &  11927 & 93 & 3322  \\
Crystal & \crystaltests & 18 & 18 & 18 & 18 &17&18& 17 & 18 & 3 &14& 131 & 1304 & 166 & 7323 & 95  & 4155 \\
Synoptic & \synoptictests & 1 & 1 &1  & 1 & 0 &1 & 0 & 1 & 2 &  7 & 67 & 760& 25 & 3372& 24 & 1797 \\
\hline
\textbf{Total} & \totaltests & 21 & 21&24&\textbf{29}& 23 & $\ge$24 & 23 & $\ge$25 &41&  143 & 1320 & 13579 &1562&  4$\times$$10^6$ *& 503  & 5$\times$$10^5$ *\\
\hline
\multicolumn{18}{|l|}{ }\\
\multicolumn{18}{|l|}{\textbf{Automatically-generated unit tests} }\\
\hline
%JFreechart & \jfreechartautotests& \ignore & \ignore & \ignore & \ignore & \ignore & \ignore & \ignore & \ignore & \ignore & \ignore & \ignore & \ignore & \ignore &  \ignore \\
\jt & \jodatimeautotests &\ignore & \ignore & \ignore & \ignore & \ignore & \ignore & \ignore & \ignore & \ignore & \ignore & \ignore & \ignore & \ignore & \ignore & \ignore &  \ignore \\
XML Security& \xmlsecurityautotests&138& 167 & 171 & 171 & 129 & $\ge$129 * & 128  & $\ge$128 *   & 7& 50 & 430 & 4174 & 133 & 1$\times$$10^5$ * & 128 & 5$\times$$10^4$ * \\
Crystal & \crystalautotests & 75 & 159 & 162 & 164 & 55 & $\ge$55 * & 55 & $\ge$55 *  & 22 & 103 & 949& 9436  & 2477 & 8$\times$$10^6$ *& 2297 & 1$\times$$10^6$ * \\
Synoptic & \synopticautotests &3 & 3 & 7 & 10 &2& $\ge$2 * & 2 & $\ge$2 *   & 13 &81& 770  & 6311 & 454 & 1$\times$$10^6$ *& 454 & 2$\times$$10^4$ * \\
\hline
\textbf{Total} & \totalautotests &216 &329 &340 & \textbf{345} & 186 & $\ge$186  & 185 & $\ge$185  &42&234&2149& 19921& 3064 & 1$\times$$10^7$ *& 2879& 1$\times$$10^6$ * \\
\hline
\end{tabular}
\caption{Experimental results.  Column ``\# Tests'' shows the total number
of tests, taken from Table~\ref{tab:subjects}. Column ``\# Detected Dependent Tests''
shows the number of detected dependent tests in each test suite.
% Columns ``Rev'', ``Randomized'', ``Exhaustive'' and ``Dep-Aware'' show the results
% of applying the reversal algorithm, randomized algorithm, exhaustive $k$-bounded algorithm, and the \dependenceaware{}, respectively.
%$k$-bounded algorithm, respectively. 
When evaluating the randomized algorithm, we used $\mathit{numtrials}$ =
$\smalltrialnum$, $\mediumtrialnum$, and $\trialnum$ (Figure~\ref{fig:randalgorithm}).
%``\unknown'' means the technique does not scale to the test
%suite (i.e., requiring more than 1 day to execute all test permutations),
%and thus the exact number of dependent tests is unknown.
``\ignore'' means the test suite is not evaluated due to its non-determinism.
%Column ``Analysis Cost''
%shows the time cost of each algorithm.
An asterisk (*) means the algorithm did not finish
within 1 day:
the number of dependent tests is those discovered before timing out, and 
the time estimation methodology is described in Section~\ref{sec:performance}.
\tinyrelax
}
\label{tab:results}
\end{table*}

%  LocalWords:  Joda numtrials


Dependent tests reach beyond theory and appear in real-world programs.  
In some cases, they are intentional, developers are aware of them and
document them, but in other cases they are inadvertent. 
Test dependence can cause problems, not only when test suites are reordered,
but even when they are
executed in the intended order.
This section presents concrete examples of test dependence found in
well-known open source programs. Figure~\ref{fig:example-summary}
summarizes the projects we studied and the results: The table
summarizes the number of tests in the suites produced by the
developers (\emph{MT}), the number of tests we generated automatically
with Randoop (\emph{AT}), and the corresponding numbers of dependent
tests in those test suites (\emph{MTD} and \emph{ATD}, respectively). 
The discussion of the examples in this section is distinguished by
the problems caused by test dependence (\emph{Kind}): when faults are masked because
tests make incorrect assumptions about the global environment (Section~\ref{sec:mask}); 
when tests do not
respect required initialization protocols (Section~\ref{sec:examples:initialization}); and when
undocumented test dependence leads to spurious bug reports (Section~\ref{sec:spurious}).
We also describe dependent tests in an automatically-generated test
suite (Section~\ref{sec:autogen}).
While this list---and associated set of examples---certainly is not exhaustive, it shows that there are
several classes of dependence-related problems that have practical
relevance.



\subsection{Masking Faults}\label{sec:mask}

\emph{Masking} is a particularly perplexing problem caused by
dependence.
The negative effect of masking is that it hides a fault in the
program, \emph{exactly} when the test suite is executed in its default
order. 
%So while manifest dependences can reveal such a problem, the
%underlying fault is in the program and affects first-order testing and
%use of the program.
%In its simplest form, masking occurs when parts of a program or tests assume that
%global state has correctly been initialized before these parts can
%ever execute. When this assumption is incorrect, because
%initialization is not implemented correctly, the interactions of
%different parts of the program might jointly modify the global state
%in ways that lead to intricate and subtle faults.
Masking occurs when a test case $t$ (a) \emph{should}
reveal a fault, (b) only does so when executed in a specific environment
$\env_R$, but (c) tests executed before $t$ in a test suite always
generate environments different from
$\env_R$.
%To express this more
More precisely and without loss of generality, assume any
environment with only a single variable. Then let $T =
\suite{t_1,\dots,t_n}$ be the test suite, and let $t_i, 1 < i \leq n$
be the test that should reveal the fault in environment $\env_R$. A
dependency $t_k \prec t_i, k < i$ masks the fault if
$\exec{\suite{t_1,\dots,t_{i-1}}}{\env_0} \neq \env_R$.

The following two examples illustrate masking in
practice.

\paragraph{CLI: A Long-Standing Bug}

\begin{figure}
% \lstset{language=Java,numbers=left}
%\lstset{language=Java}
\lstset{belowskip=0ex,escapechar={@},numbers=left,numberstyle=\small\ttfamily}
\begin{lstlisting}
public final class OptionBuilder {
  @\itshape\color{red}
  private static String argName;@
  
  private static void reset() {
    ...
    @\itshape\color{red}argName = "arg";@
    ...
  }
   
  public static Option create(String opt){
    Option option = 
      new Option(opt, description);
    ...
    option.setArgName(argName);
    @\itshape\color{red}OptionBuilder.reset();@
    return option;
  }
}
\end{lstlisting}
\caption{Fault-related code from \code{Option\-Build\-er.java}}
\label{fig:option_builder}
\end{figure}

A straightforward example of fault masking occurs in the Apache CLI
library.\footnote{\url{http://commons.apache.org/cli/}}
Two test cases fail when run in isolation:
\code{test13666} and \code{test\-Op\-tion\-With\-out\-Short\-For\-mat2} in test
classes \code{Bugs\-Test} and \code{Help\-For\-mat\-ter\-Test},
respectively.

A detailed study of the code under test revealed that both 
tests fail due to the same hidden dependence. The fault is located in 
\code{OptionBuilder.java} and is caused by not initializing a global
variable early enough.
Figure~\ref{fig:option_builder} shows code that
illustrates the fault. 
%
By default,
\code{argName} is initialized to \code{null} (line 2), and only set to
its intended default value \code{"arg"} by the \code{create()} method
via calling \code{reset()} (line 15). 
Consequently, if clients of CLI do not explicitly initialize the value of
\code{argName}, the first option created will have \code{null} rather
than \code{"arg"} as its argument name.
%In CLI, there are two types of options: options with and without
%argument names. If an option without argument is created first,
%this fault will not lead to a failure, because the \code{null} value
%will be ignored. Consecutive calls to \code{create()} can rely on
%\code{reset()} to establish the desired default value.

Both dependent tests
% \code{test13666} and \code{test27635} (or \code
% {test\-Op\-tion\-With\-out\-Short\-For\-mat2}) 
can reveal this fault, since they create an option with 
the default argument as the first thing in their execution. However,
in the default order of test execution, 
%the test classes \code{BugTest} and \code{Help\-For\-mat\-ter\-Test} both
%contain other 
tests that create options with explicit arguments execute \emph{before} 
these dependent tests.
% \code{test13666}
% and \code{test27635} respectively. 
%Thus, when the tests in these classes are 
%executed in order, the tests executed before \code{test13666}
%and \code{test27635} call \code{create()} 
Thus, the tests that are executed before call \code{create()} at least once, which
sets the default \code{argName} value, thus masking the fault.

%\todo{JW}{The following paragraph is not really necessary here. We
%want to illustrate how these things happen. If we have time and space,
%I think a general discussion section about the relevance of dependence
%might benefit from this, though}
%
%\todo{SZ}{I am on the side of keeping the following text. It is very
%impressive about the effect of dependent tests, making the whole
%test dependence story stronger.}

This fault is reported in the bug
database several times,\footnote{\url{https://issues.apache.org/jira/browse/CLI-26} \url{https://
issues.apache.org/jira/browse/CLI-186} \url{https://issues.apache.org/jira/browse/
CLI-187}} starting on March 13, 2004 (CLI-26). The report is marked as resolved
\emph{three years} later on March 15, 2007, but is then reopened as CLI-186 on
July 31, 2009. On this report, one of the developers commented:
\begin{quote}
I reproduced the issue, it requires a dedicated test case since it is tied to the initialization 
of a static field in OptionBuilder.
\end{quote}
Despite the realization that a dedicated test is required, no such
test was ever created.
About one month later, the bug is duplicated as CLI-187, and the
actual fix happens one 
year later on June 19, 2010, about six years after the bug was first reported (and four years
total on the open-issue list).
%total ``awareness'' of the fault
%of years. %The fix consists of adding the following code to \code{OptionBuilder.java}:

%We associated these dependent tests with a bug first reported by a
%user in 2004, marked as ``resolved''---but not actually resolved---in 2007,
%reopened in 2009 by another user, and finally identified and
%fixed in 2010.  This bug sat on the shelf for about six years and
%would have been identified much earlier and much more easily by
%considering test dependence.

\newcommand{\jodatime}{JodaTime\xspace}
\paragraph{\jodatime: Complex interactions that mask faults}
\label{sec:jodatime}
\newcommand{\periodType}{\texttt{Period\-Type}}
\newcommand{\durationFieldType}{\texttt{Duration\-Field\-Type}}
\newcommand{\forFields}{\texttt{for\-Fields}}


\jodatime\ uses intricate caching mechanisms that are high\-ly complex
and coupled.  All dependences we found are complex,
in two cases even requiring a
specific ordering of \emph{three} tests to manifest.

In a simple dependence, \jodatime{} caches \periodType{} objects, which 
% Caching is done by using a
% global \texttt{HashMap} that holds the \periodType{}s that are created by
% \forFields{} method. 
contain an array of
\durationFieldType{}s (e.g., week, month). 
The order of \durationFieldType{}s in the array is an
important of the data representation, and 
two \periodType{}s with the same \durationFieldType{}s in a different
order are not equal internally in \jodatime, even though they are equal
to \jodatime clients.
%However, this implementation detail should be transparent to the user: As long
%as two \periodType{} objects have the same \durationFieldType{}s, they should represent
%the same period. 
To make this internal detail transparent to users of \jodatime, 
new \periodType{}s are normalized before they are cached. However, a fault in the code 
% checking for existing objects
makes it possible to insert non-normalized \periodType{}s into the
cache, leading to cache misses when searching for correctly normalized
\periodType{}s.
%This is even acknowledged by the developers: \forFields{}
%method first creates a \periodType{} using method argument \texttt{types} (the
%ordering provided by the user) and checks whether the cache already contains
%this ordering. However, if this fails, then another \periodType{} with
%normalized ordering (\texttt{checkedType}) is created at the end of the method 
%and the cache is rechecked as shown in Figure~\ref{fig:jodatime_forFields}.


%\begin{figure}
% \lstset{language=Java,numbers=left}
%%\lstset{language=Java}
%\begin{lstlisting}
%PeriodType forFields
%  (DurationFieldType[] types) {
%  ...
%  PeriodType input =
%    new PeriodType(null, types, null);
%  ...
%  // recheck cache in case
%  // initial array order was wrong
%  PeriodType check = ...
%  PeriodType checkedType = cache.get(check);
%  if (checkedType != null) {
%    cache.put(input, checkedType);
%    return checkedType;
%  }
%  cache.put(input, type);
%  return type;
%}
%\end{lstlisting}
%\caption{Fault related code from \code{Pe\-ri\-od\-Type.for\-Fields}
%\newline (rev. 3937d82f6670e5a30b2809b13cb6d05a7e606037)}
%\label{fig:jodatime_forFields}
%\end{figure}
%
%
%In spite to the developers' extra check, a small mistake in
%Figure~\ref{fig:jodatime_forFields} creates a bug in the program. Note that the
%developers are putting the \texttt{input} variable (the ordering provided by
%the user) as the `key' of the cache.
%As a result, if \forFields{} method is called with a wrongly ordered
%\texttt{types} parameter first, then a \periodType{} with wrong ordering will be
%added to the cache. Later, calling the same method with correct ordering of
%the same content causes a cache miss.

A test that checks for correct normalization when
caching objects
%The developers even have a very simple test case that checks for this.
%\texttt{Test\-Period\-Type.test\-For\-Fields4} method creates two
%\durationFieldType{}s: first with wrong ordering and the second with correct
%ordering, both representing the same period. The test creates the corresponding
%\periodType{}s by calling \forFields{} method with these \durationFieldType{}s.
%Finally, the objects retrieved from \forFields{} method are asserted for
%equality. 
fails in isolation but passes when the entire test suite
executes in the default order; this happens
because a prior test creates the expected
\periodType{}, and thus it is already in the cache for the
later test.
This behavior has been reported as a bug and has been fixed by the
developers.
%never fails during the development due to a simple dependency with the previous
%test in the same class. \texttt{test\-For\-Fields3} method creates the same
%\periodType{} content --- which will be used in the next test --- with the
%correct ordering for some other purpose. As a result, this \periodType{} is added with
%the correct ordering to the cache, which guarantees correct retrieval for both
%the correct and wrong ordering in the next test.


%The test case that would catch the bug was written the same revision it is
%introduced. However, since the developers never ran that test in isolation the
%bug lived for more than six years. During this time period, there have been 773
%commits for the project. Even the buggy file (\periodType{}) is
%changed for nine times and the buggy method (\forFields{}) is changed
%once. Finally, the bug gets reported by a
%user\footnote{\url{http://sourceforge.net/mailarchive/message.php?msg_id=28501345}}
%in December 06, 2011 and is fixed the same
%day\footnote{\url{https://github.com/JodaOrg/joda-time/compare/b609d7d66d...d6791cb5f9}}.
%During the same commit, the developers also removed the dependency for
%the related test by creating a unique
%\periodType{} for that test. The actual fix contains changing two
%variables:
%two instances of variable \texttt{input} (possibly wrong ordering) to variable
%\texttt{check} (to correct ordering) at the end in
%Figure~\ref{fig:jodatime_forFields}.

After inspecting the code, we reported the more complex dependence of
three tests to the developers of \jodatime. They confirmed the
phenomenon, but contended that it is due to interactions that are not
intended in the design of the library~\cite{jodatime}. In particular, one of the
methods, \code{DateTimeZone.setProvider()}, is only supposed to be
called a single time to initialize the library.  In practice, multiple
tests initialize the library, which leaves incorrect values in the cache
and causes other tests to fail under some execution orders.
%However, the tests call it more
%than once, which causes at least two cases to %break the caching
%%mechanism by leaving 
%leave incorrect values in the cache, causing the tests to fail.


\subsection{Poor Test Construction}\label{sec:examples:initialization}

Based on our interaction with the \jodatime developers, this last
dependence does not
mask a fault in the program.  Instead, it represents a less severe consequence of test
dependence that suggest that a test, or a test suite, 
has been constructed poorly in some dimension.  While test dependences that mask faults
correspond to a defect
in the program source, these dependences correspond to defects in the test code.
%
%In contrast to the previous section
%where the dependences led to defects in the program source, this section concerns defects
%in the test source.

%In some sense, dependences that are due to missing initialization are
%the dual to dependences that mask faults.  Both reveal problems in source code.
%However, masked faults reside in the program source, while incorrect
%initialization is a fault that resides in the test suite.

The test dependences presented in this section arise due to incorrect initialization
of program state by one or more tests. In the first case,
%
%The following two examples show two common patterns where incorrect
%initialization leads to test dependence.
%The first example is probably the most common. 
tested program code relies on a
global variable that is a part of the environment, but the test does
not properly initialize it.  In the second case, a test should but
does not call
an initialization function before later invocations to a complex library.
This flaw in the test code is masked because the default test suite execution
order includes other tests that initialize the library.  The defect is
inconsequential until and unless the flawed test is reordered, either manually or by
a downstream tool, to execute before any other initializing test.

%The second example employs a common pattern for complex
%libraries that requires a call to an initialization function before
%any other part of the library can be used.
%In both cases, other tests perform the required setup, and because
%they occur before the dependent tests in the normal execution order,
%no tests fail under normal circumstances.

\paragraph{Crystal: Global Variables Considered Harmful}
%Dependent tests in Crystal fall into the following three groups, which
%share the same root cause of global data dependence across multiple tests:
%
%\begin{itemize}
%
%\item 9 dependent tests come from the \CodeIn{DataSourceTest} class.
%In that class, a test method \CodeIn{testSetField} initializes a global variable \CodeIn{data}
%and other test methods read the value of the \CodeIn{data} variable.
%As a result, when a test using \CodeIn{data} is executed in isolation or executed
%before the \CodeIn{testSetField} method, a \CodeIn{NullPointerException} is
%thrown.
%
%\item 7 dependent tests come from class \CodeIn{LocalStateResultTest}.
%In that class, a test method \CodeIn{testLocalStateResult} initializes a global variable
%\CodeIn{localState} and other test methods use that variable. Therefore,
%7 tests using the \CodeIn{localState} variable exhibits
%a \CodeIn{NullPointerException} when they are executed before \CodeIn{testLocalStateResult}.
%\todo{KM}{I see no difference between the first item and this one. They all
%seem to happen due to one test initializing a global variable and the others
%reading the same global variable}
%
%\item 1 dependent test comes from class \CodeIn{ConflictDaemonTest}. This
%test uses a shared global variable which requires other tests in the
%same test class to initialize, \todo{KM}{Again, the same thing. I believe we
%should say that all dependencies are due to initialization of a global variable
%and then explain all of them as three sentences (instead of bullet pointing
%them as they were really different)}
%\end{itemize}



\paragraph{XML Security: Global Initialization}

%Are these test dependences realistic, or part of the modifications SIR
%made? by SZ: they are realistic, we use the original version without
%any modification from SIR people

XML Security\footnote{\url{http://projects.apache.org/projects/xml_security_java.html}}
is a component library implementing XML signature and encryption
standards. Each released
version of XML Security has a human-written JUnit test suite that
achieves fairly high statement coverage.

Four stable released versions (1.0.2, 1.0.4, 1.0.5d2, and 1.0.71) of XML Security
have been incorporated in the Soft\-ware-artifact Infrastructure Repository
(SIR).\footnote{\url{http://sir.unl.edu}}
We found that at least two out of the four versions contain dependent tests. Specifically, in versions 1.0.4 and 1.0.5d2, \code{test\_Y1}, \code{test\_Y2}, and \code{test\_Y3}
in class \code{ExclusiveC14NInterop} show dependent behavior.
Since the dependences are the same in both versions, in the further
discussion and in Figure~\ref{fig:example-summary}, we consider only
version 1.0.4.

For all three dependences, the cause of the dependence is the same: before any
method in the library can be used, the global initialization function 
\code{Init.init()} has to be called. Internally, it initializes
the static field that the code tested by the dependent tests rely
on.

Given that the error when executing the dependent tests clearly explains the
cause of the error, we speculate that developers either simply forgot to
initialize the tests properly, or expected that these tests would always execute
in the order defined
in the test suite.

%on a global variable. Take \code{test\_Y1} as an example. This test passes when being executed
%with other tests, but fails by throwing an \code{InvalidCanonicalizerException}
%when executed individually.
%The root cause of such behavior difference is that, in XML-security, the \code{Init.init()} method initializes
%the static field \code{Canonicalizer.canonicalizerHash}, and test \code{test\_Y1} needs to use
%that static field to create a \code{Canonicalizer} instance. 
%When executing this test in the programmer-fixed order, method \code{Init.init()} has been called by
%other tests executed before \code{test\_Y1}, so that test \code{test\_Y1} passes.
%However, without calling \code{Init.init()} first,
%\code{test\_Y1} fails to create the \code{Canonicalizer} instance.
%
%Based on the dumped error message in the \code{InvalidCano-\\nicalizerException}:
%
%\begin{quote}
%``You must initialize the xml-security library correctly before you use it.
%Call the static method ``org.apache.xml.security.Init.init()'' to do that before you use any functionality
%from that library''
%\end{quote}

%We speculate that programmers should realize this potential dependence, but they
%overlook to enforce \code{test\_Y1} to be executed in a desirable order. Instead,
%programmers may have put an implicit assumption that tests in a suite can be executed in isolation
%and miss to add the necessary preconditions for \code{test\_Y1}. 

% vim:wrap:wm=8:bs=2:expandtab:ts=4:tw=70:


\subsection{Spurious Bug Reports and Bug Fixes}\label{sec:spurious}
Sometimes developers introduce dependent tests intentionally because it is
easier, more efficient or more convenient to write unit tests for some modules
in that way~\cite{kapfhammeretal:FSE:2003, whittakeretal:2012}.
%DB-testing}.
Even though the developers are aware of these instances
when they create them, this knowledge can get lost, 
and other people who are not aware of these dependences can get confused 
when they run a subset of the test suite that manifests the
dependences.

As a result, they
might report bugs backed by the failing tests, although this is exactly the expected
behavior. If the dependence is not documented clearly and
correctly, it can take a considerable amount of time to work out that
these reported failures are spurious. Or worse, the developers may try
to fix a bug that is not there.

\paragraph{Eclipse SWT: Causing Spurious Bug Reports}
\newcommand{\ite}{\texttt{Invalid\-Thread\-Access\-Exception}}

The Eclipse Standard Widget Toolkit
(SWT)\footnote{\url{http://eclipse.org/swt/}} is a cross-platform GUI
library developed within the Eclipse framework.
%\todo{KM}{Year here\ldots}.
%It has been developed to combine best parts of Sun's Abstract Window Toolkit (AWT)
%and Swing: native look and feel and native performance.
%
Due to the difficulty of obtaining source, compiling and running
test suites with the SWT project, we only examined some test cases
manually, after a bug report indicated test dependence. The
numbers reported in Figure~\ref{fig:example-summary} are the number
of tests we manually examined, and the number of dependencies we found
among those respectively.

As is common practice in GUI toolkits, SWT permits only one
\texttt{Display} object per thread. Attempting to create multiple
\texttt{Display}s in a single thread causes an \ite{}. 
%In other words each thread is responsible of disposing its \texttt{Display} after it is done with it. 
To permit the reuse of \texttt{Display}s, SWT provides two 
methods: \texttt{Display.getDefault} and \texttt{new Shell}. These
methods return the existing \texttt{Display} or create a new one if none exists.


In the test suite of SWT, all tests except those in the class \texttt{Test\_org\_eclipse\_swt\_widgets\_Display}
(\texttt{TestDisplay} for short) retrieve the current \texttt{Display} by using
one of the latter methods. On the other hand, all tests in
\texttt{TestDisplay} create their \texttt{Display} at the beginning of the test
and dispose of it at the end. 

%\texttt{TestDisplay.setup} contains the following
%comment:
%\begin{quote}
%There can only be one Display object per thread. If a second Display is created
%on the same thread, an InvalidThreadAccessException is thrown. 
% \\ Each test will create its own Display and must dispose of it before
% completing.
%\end{quote}


In September 2003, a user reported a
bug,\footnote{\url{https://bugs.eclipse.org/bugs/show_bug.cgi?id=43500}}
stating that tests throw an \ite{}
if she runs any other test before \texttt{DisplayTest}. 
The cause of this is simple: any other test creates, but does not
dispose of a \code{Display} object. Then the tests in
\code{TestDisplay} attempt to create a new object, which fails, as one
is already associated with the current thread.
Since this is the expected and desired behavior, the bug report is
spurious (except maybe it points to a problem in the test suite,
rather than the code).


%Let us examine the bug
%report: running any other test would create a \texttt{Display} (through one of
%the latter two methods) and would not dispose it. Thus, when these test
%complete, the main thread owns a \texttt{Display}. At this moment, when the same
%thread tries to run \texttt{DisplayTest} and thus tries to create another
%\texttt{Display}, an \ite{} is thrown. However, note that this is really the
%intended case when a thread attempts to create multiple \texttt{Display}s. In
%other words, this dependency leads to a spurious bug: there is a change in the
%test outcome when the order of tests are changed, however this does not
%correspond to a bug in the program. Nevertheless, understanding this dependency
%--- even though the comment on \texttt{DisplayTest.setup} existed --- takes
%about a month for the developers. One of the developers closes the bug with the
%following comment:
%\begin{quote}
%Turns out that the tests really are order-dependent - the Display tests must 
%be run first. It's not an SWT bug or anything, it's just the way the tests are 
%written, and I think it would be weird to code around it. \\
%\ldots I'm not going to make any code changes, but for the `fix' I have added a big 
%comment in the AllTests method saying that the Display tests must go first.
%\end{quote}
%We believe that the way this bug is handled shows that the
%dependencies between tests can lead to confusion even when there is no real bug. 


% This led to a spurious bug report. This is actually a good example,
% because it shows how hard it is to tell the difference between a bug
% and dependent test.
% So what do we fix? The "bug" or the tests?


\subsection{Dependence in Auto Generated Tests}
\label{sec:autogen}
\newcommand{\pub}{\texttt{Prop\-er\-ty\-Utils\-Bean}}
\newcommand{\fhm}{\texttt{Fast\-Hash\-Map}}
\newcommand{\cub}{\texttt{ConvertUtilsBean}}

Take the Beanutils program as an example, Randoop generates a test suite consisting of
2692 tests, in which \ourtool detected 299 dependent tests.

After a close inspection of the automatically-generated test code, we found
the primary reason for the dependencies is missing initialization 
(cf.~Sec.~\ref{sec:humantest}).
%is the unintended program state before test execution.
Specifically, 248 tests attempt to retrieve values from a cache before
anything has been added to the cache. This particular dependence could be
fixed by adding a single line of setup code to each test.
Most of the other dependencies could be fixed with similarly low effort, too.
However, this particular fix requires understanding of at least part
of the program semantics, which is a feat beyond the abilities of
current test generation tools.
%and fully automatically.

%implicitly assume a particular program state (
%the caching state) when executing in the generated order. In Beanutils,
%\pub{} --- which can be
%accessed as a singleton --- contains a global cache from \texttt{Class} to
%\fhm{}. These 248 tests retrieve the value for \cub{}
%from this cache and assert that the result is not \texttt{null}, without adding
%anything to the cache first. However, some other generated tests call
%\texttt{Prop\-er\-ty\-Utils.get\-Prop\-er\-ty\-Editor\-Class}, which adds
%a \{\cub{}, \fhm{}\} pair to the cache internally. As a result, the former tests
%fail when run in isolation (since the cache is empty and it returns null), however
%pass when run within the whole test suite. Reasons for the remaining 51 dependent
%tests are similar, which we omit here for brevity.


Given the high ratio of dependent tests in the automatically generated
test suite, we speculate more reasons.
%that the following two phenomena could be
%reasons for this.

First, developers usually have a good understanding
of the code under test. This knowledge helps them to
build well-structured and coherent test suites.
Automated tools, on the other hand, have no such knowledge. One
possible consequence of this is illustrated by the example: the
automated tool does not understand the cache protocol and thus does
not know that it must add values to the cache first. 

%unit
%tests, programmers tend to put logically-related code in the same unit test to test certain software functionality. By contrast, automated test generation tools are often not aware of the underlying program structure nor the test execution environment when creating new ones. In particular, random test
%generation tools like Randoop invokes tested methods
%with little guidance. Thus, a generated test is more likely to depend on the
%execution of others.

%are more likely to ``interleave''
%with each other, such as, invoking the same static method mutating program states.
%\todo{JW}{I do not understand this explanation at all. What has this
%interleaving to do with dependences?}

Second, it is often hard for automated tools to
generate code that sets up the execution environment correctly,
since doing so requires understanding which specific parts
of the environment have been affected.
%to understand that
%specific parts of the code depend on the environment, and thus may not
%explicitly . 
Without proper setup code, if, at the same time, 
other tests are generated that as a side effect
create the needed environment, test dependence arises.

%Second, test frameworks like JUnit offer constructs \code{@Before}
%(\code{@After}) to permit programmers to abstract common execution environment
%construction (de-construction) code for each unit test. Such mechanism prevents
%dependent tests exhibiting to some extent. However, to the best of our knowledge,
%most automated test generation tools do not leverage
%such mechanism to enforce generated unit tests to execute in an intended environment.

%\todo{JW}{While this is true, I'm not sure if this is the right plave
%to put it}
% \textbf{Kivanc has investigated this, but i can not find the email now.}
% \todo{KM:}{ This is not completely true. I just have the stack trace for the
% first dependency which seems to be due to a mistake in caching. However, I
% don't know if this is a bug, or any information about other dependencies.}




%We already showed some evidence that test dependence is not uncommon
%in human-written tests. Given the increasing importance of
%automatically generated tests, we also wanted to at least get a
%glimpse of what is happening in that area.
%As a very preliminary, and by no means exhaustive or conclusive
%investigation, we applied Randoop to all the projects for which the
%source was readily available (this excludes SWT).

%Why this strong division happens, and whether the differences between
%the programs can be used to derive guidelines for better testing is an
%interesting question left to future work.

%\todo{JW}{I no longer think the following paragraph is true}
%This is at once surprising and troublesome. It is surprising, because
%in our experience test dependence occurs either because it is too much
%hassle to write proper test setup code for every single test, or
%because developers are not aware that global state is relevant to the
%code that is being tested. The first point should not at all be
%relevant to automated techniques, as the effort of generating boiler
%plate code is negligible compared to the cost of figuring out useful
%parts of the code to test. The second aspect is fairly well amenable
%to static analysis. Thus overall, there is no reason why automated
%tools could not avoid test dependence altogether.



%\section{How does the theory relate to our examples}





\section{Algorithm and Implementation}
\label{sec:algorithm-tool}

In this section we present an algorithm and a prototype tool to detect dependent
tests.
In the worst case, a naive, exhaustive search would execute all $n!$
permutations of the test suite to detect dependent tests. While this
is not feasible for realistic $n$, our approximate algorithm uses 
our intuition that many dependences can be found by running only short subsequences of
test suites, and introduces a bound $k$ on the length
of subsequences. That effectively bounds the execution time to
$O(n^k)$, which for small $k$ is tractable. At the same time, our
prototype tool and the experiments we conducted with it, suggest that
many dependences can be found for small $k$.

\subsection{Algorithm}
\label{sec:algorithm}
%%
%% This is file `algorithm.sty',
%% generated with the docstrip utility.
%%
%% The original source files were:
%%
%% algorithms.dtx  (with options: `algorithm')
%% This is a generated file.
%% 
%% Copyright (C) 1994-2004   Peter Williams <pwil3058@bigpond.net.au>
%% Copyright (C) 2005-2009   Rog�rio Brito <rbrito@ime.usp.br>
%% 
%% This document file is free software; you can redistribute it and/or
%% modify it under the terms of the GNU Lesser General Public License as
%% published by the Free Software Foundation; either version 2 of the
%% License, or (at your option) any later version.
%% 
%% This document file is distributed in the hope that it will be useful, but
%% WITHOUT ANY WARRANTY; without even the implied warranty of
%% MERCHANTABILITY or FITNESS FOR A PARTICULAR PURPOSE.  See the GNU Lesser
%% General Public License for more details.
%% 
%% You should have received a copy of the GNU Lesser General Public License
%% along with this document file; if not, write to the Free Software
%% Foundation, Inc., 59 Temple Place - Suite 330, Boston, MA 02111-1307,
%% USA.
%% 
\NeedsTeXFormat{LaTeX2e}[1999/12/01]
\ProvidesPackage{algorithm}
   [2009/08/24 v0.1 Document Style `algorithm' - floating environment]
\RequirePackage{float}
\RequirePackage{ifthen}
\newcommand{\ALG@within}{nothing}
\newboolean{ALG@within}
\setboolean{ALG@within}{false}
\newcommand{\ALG@floatstyle}{ruled}
\newcommand{\ALG@name}{Algorithm}
\newcommand{\listalgorithmname}{List of \ALG@name s}
% Declare Options:
% * first: appearance
\DeclareOption{plain}{
  \renewcommand{\ALG@floatstyle}{plain}
}
\DeclareOption{ruled}{
  \renewcommand{\ALG@floatstyle}{ruled}
}
\DeclareOption{boxed}{
  \renewcommand{\ALG@floatstyle}{boxed}
}
% * then: numbering convention
\DeclareOption{part}{
  \renewcommand{\ALG@within}{part}
  \setboolean{ALG@within}{true}
}
\DeclareOption{chapter}{
  \renewcommand{\ALG@within}{chapter}
  \setboolean{ALG@within}{true}
}
\DeclareOption{section}{
  \renewcommand{\ALG@within}{section}
  \setboolean{ALG@within}{true}
}
\DeclareOption{subsection}{
  \renewcommand{\ALG@within}{subsection}
  \setboolean{ALG@within}{true}
}
\DeclareOption{subsubsection}{
  \renewcommand{\ALG@within}{subsubsection}
  \setboolean{ALG@within}{true}
}
\DeclareOption{nothing}{
  \renewcommand{\ALG@within}{nothing}
  \setboolean{ALG@within}{true}
}
\DeclareOption*{\edef\ALG@name{\CurrentOption}}
% ALGORITHM
%
\ProcessOptions
\floatstyle{\ALG@floatstyle}
\ifthenelse{\boolean{ALG@within}}{
  \ifthenelse{\equal{\ALG@within}{part}}
     {\newfloat{algorithm}{htbp}{loa}[part]}{}
  \ifthenelse{\equal{\ALG@within}{chapter}}
     {\newfloat{algorithm}{htbp}{loa}[chapter]}{}
  \ifthenelse{\equal{\ALG@within}{section}}
     {\newfloat{algorithm}{htbp}{loa}[section]}{}
  \ifthenelse{\equal{\ALG@within}{subsection}}
     {\newfloat{algorithm}{htbp}{loa}[subsection]}{}
  \ifthenelse{\equal{\ALG@within}{subsubsection}}
     {\newfloat{algorithm}{htbp}{loa}[subsubsection]}{}
  \ifthenelse{\equal{\ALG@within}{nothing}}
     {\newfloat{algorithm}{htbp}{loa}}{}
}{
  \newfloat{algorithm}{htbp}{loa}
}
\floatname{algorithm}{\ALG@name}
\newcommand{\listofalgorithms}{\listof{algorithm}{\listalgorithmname}}
\endinput
%%
%% End of file `algorithm.sty'.


\subsection{Tool Implementation}
\label{sec:tool}

We implemented our $k$-bounded dependent test detection algorithm 
in a prototype tool.\footnote{Available at: \url{http://testisolation.googlecode.com}} 
%written in the form of JUnit 3.X.
The tool is fully-automated and needs only a test suite and the
bounding parameter $k$ as inputs. 
Our
current implementation supports JUnit 3.x tests.
%exhaustively executes every $k$-tuple
%of tests, and compares execution results to identify possible dependence cases. When
%comparing the observed result of a test in an execution order with
%its intended result (corresponding to line 6 in Figure~\ref{fig:dtalgorithm}),
We consider JUnit test results to be the same when the tests either
both pass, or exactly the same exception or assertion violation leads
to test failure.
%if both the observed result and intended result are passing or
%exactly the same exceptions are thrown otherwise. 
The tool creates a fresh JVM for each \testlist, thus, ignoring
external state such as files and OS services, the environment
that the test suites are executed in is always the same $\env_0$.
This ensures that there is no interaction between
different \testlist\ through shared memory.

We used the prototype to verify the dependent tests reported by
users, developers, other researchers, and us, and to find new dependent
tests in the
example programs in Section~\ref{sec:examples} using isolated execution ($k = 1$)
and pairwise execution ($k = 2$).

All the dependent tests reported in Figure~\ref{fig:example-summary},
except for two dependent tests in JodaTime and the dependences in SWT, 
can already be found by isolated execution. Since we could not run the
test suite of SWT, we could not check these dependences with our tool.
During manual bug diagnosis in JodaTime, we identified two test dependences that require
\emph{three} tests to manifest. While these are easy to reproduce, we
did not check that our tool finds them, because the time needed to
run our naive algorithm on JodaTime with $k=3$ is measured in months.

While we believe that most test dependences can be found with small
$k$. This is in part because the set of dependent tests that can be
found with a bound $k$ is always a subset of the set of dependent
tests that can be found with any bound $k' > k$. Additionally, our
intuition and preliminary exploration seem to indicate that small $k$
find many dependences, while larger $k$ do not. However, in principle
it is conceivable
that any number of chain dependences with chains longer than any tried $k$ exist
in all the libraries we analyzed.


%However, due to the computational complexity of the general dependent test
%detection problem, it is difficult to know precisely how many dependent
%tests exist in a test suite. Thus, we do not
%yet have empirical data that shows how many percentages of dependence tests
%our tool can catch. Giving a reasonable estimation
%is one of our future work.

%we do not yet have strong empirical data that shows our algorithm catches X
%percentage or Y of the worst test dependences. It is one of our future work.


% The tool is publicly
% available\footnote{\url{http://testisolation.googlecode.com}}.
% The source code and user manual of our tool is publicly available at:
% \url{http://testisolation.googlecode.com}


% vim:wrap:wm=8:bs=2:expandtab:ts=4:tw=70:




%\vspace{-1mm}
\section{Conclusion and Future Work}
\label{sec:questions}


Test independence is broadly assumed but rarely addressed, and
test dependence has largely been ignored in previous
research on software testing. 
To understand dependent tests, we described one of the first studies on
real-world dependent tests. We showed that 
test dependence \textit{does} arise in practice, and could 
have non-trivial repercussions. We also
formalized the dependent test detection
problem. To detect dependent tests, we designed
and implemented three algorithms to identify manifest test dependence
in a test suite. Our experiments revealed
dependent tests in every subject program
we studied, from both human-written and automatically-generated test
suites. The revealed dependent tests interfere with
five existing test prioritization techniques.
Our tool is publicly available at
\url{https://testisolation.googlecode.com/}.
% \todo{XXX}.

Our findings are of utility to practitioners and researchers.
Both can learn that test dependence is a real problem that should not be
ignored any longer, because it leads to false positive and false negative
test results.
Practitioners can adjust their practice based on what code patterns most
often lead to test dependence, and they can use our tool to 
find dependent tests.
Researchers are posed important but challenging new problems, such as how
to adapt testing methodologies to account for dependent tests how to detect
and correct all dependent tests.

%\enlargethispage{5pt}

As future work, we plan to study
the impact of dependent tests on other
downstream testing techniques, such as test selection and
test parallelization.
We also plan to develop
a general methodology to eliminate dependent tests.
Another future direction is investigating how to
prevent dependent tests.
%
One possible way is encouraging developers to
use advanced testing frameworks that support test dependence~\cite{testng},
so that developers can explicitly specify test
dependence when writing tests.
%
Stylized coding patterns may also be useful. Developers
should be encouraged to write tests ``defensively'' by
specifying necessary test execution pre-conditions and
using less (or properly mocking) global variables or shared resources. 

%There is already some work aiming at automating this
%process to prevent the potential
%for dependences by refactoring programs to use
%less global state~\cite{wlokaetal:FSE:2009}. 

%This question also applies to automated test generators.
%While there is some work to alleviate
%this problem~\cite{vmvm, RobinsonEPAL2011,fraseretal:ISSTA:2011}, the question
%of removing automatically-generated dependent tests
%still remains open.

\begin{comment}
Our future work should focus on the following directions:

\vspace{1mm}

\noindent \textbf{{Investigating impact of dependent tests
on other downstream testing techniques.}}
We have shown that dependent tests can compromise the application of
five existing test prioritization techniques.
%assumption is not true. However,
We plan to conduct more comprehensive empirical studies to
measure the impact of dependent tests to other
downstream techniques, such as test selection,
test parallelization, test factoring, experimental
debugging techniques, and mutation analysis.
We are also interested in how to enhance these
testing techniques to support test dependence.
%
%Another open question is how
%testing techniques should handle test dependence.
One straightforward way 
might be to augment such techniques to respect a
defined partial order among tests. This partial order
can be derived from knowledge about dependent tests,
or can be detected by our \ourtool tool.
%Like contrived examples of test
%dependence itself, it is easy to produce simple examples where
%downstream techniques produce incorrect output when applied to dependent
%tests.
%under the assumption that the input tests have no dependences.
%However, 



\vspace{1mm}

\noindent \textbf{{Eliminating dependent tests.}}
As reflected in our study (Section~\ref{sec:study}),
the practice of eliminating dependent tests
remains mostly manual and ad hoc --- software developers
usually manually hardcode test
execution orders in a configuration file or
simply merge or remove tests.
A more flexible and robust methodology for
dependent test elimination should be developed.
This question also applies to automated test generators.
While there is some work to alleviate
this problem~\cite{vmvm, RobinsonEPAL2011,fraseretal:ISSTA:2011}, the question
of removing automatically-generated dependent tests
still remains open.

%\todo{Do not forget to check words: subset, and subsequence through the paper.  Use them properly.}

%On the other hand, 
%almost all automated test generation
%techniques we are aware of produce tests
%that are hard to read for humans, are undocumented, and their intent
%cannot easily be gleaned from naming conventions and other aids
%developers normally use. Therefore, it requires more effort
%from developers to identify the root cause of dependence
%and then remove the dependence. While there is some work to alleviate
%this problem~\cite{fraseretal:ISSTA:2011}, the question
%of eliminating  automatically-generated dependent tests
%still remains open.


%As discussed in our experiments, it appears that test
%dependence in automatically generated test suites is 
%even more troublesome than in human-written suites. 


\vspace{1mm}


\noindent \textbf{{Preventing dependent tests.}}
%Detecting dependent tests is not obvious in most
%cases. Thus, a natural question is how could
%software developers prevent dependent tests when
%writing testing code.
One possible way is encouraging developers to
use advanced testing frameworks that support test dependence~\cite{testng},
so that developers can explicitly specify test
dependence when writing tests.
%However, using different testing frameworks may
%bring up the backward-compatibility issue to the existing tests.

Stylized coding patterns can also be useful. Developers
should be encouraged to write tests ``defensively'' by
specifying necessary test execution pre-conditions and
using less (or properly mocking) global variables or shared resources. 
There is already some work aiming at automating this
process to prevent the potential
for dependences by refactoring programs to use
less global state~\cite{wlokaetal:FSE:2009}. 

\end{comment}

%The source code of our tool implementation is publicly
%available at: \url{http://testisolation.googlecode.com}.



\subsection*{Acknowledgments} Bilge Soran was a participant in the project
that led to the initial result.  Yuriy Brun and Colin Gordon provided advice about
the formal notation.  Reid Holmes and Laura Inozemtseva identified the initial \jodatime dependence.  Mark Grechanik, Adam Porter, Michal
Young, and Reid Holmes provided timely and insightful comments on a draft.

\bibliographystyle{plain}
\bibliography{references}


%\subsection{Practical Considerations}
%\label{sec:practical}
%In principle, there is no \emph{ground truth} for the order of test
execution.
Therefore, we assert that the
\emph{programmer-defined} execution order, and consequently the test
results from executing the test suite in that order, are the ground
truth for our experiments.
%would naturally serves 
%the ``truth'' for our definitions of test dependence, and records
%the results from that execution order as intended results (line 2, Figure~\ref{fig:dtalgorithm}).

When a dependent test is identified, programmers may wish to know
a minimal list of other tests on which the identified test depends. 
Given an execution sequence that manifests the dependence, Delta
Debugging (also implemented in our tool) can
be used to return a shortest subsequence 
that still manifests the dependence~\cite{Zeller:2002}. 
%to minimize the recorded
%test list before the dependent test was executed.
%\todo{SZ}{is it clear? or need more explanation?}


In practice, another possible way to help detect potential dependent tests is
to leverage programmers' domain knowledge or employ some program analyses
to identify a subset of tests that are likely to contain dependent tests,
and run the algorithm only on that subset instead of the whole suite.

\todo{sz}{need a summary sentence here for the whole section 5.}

%\todo{DN}{I'm on the side of removing DD if reasonable, and coming back
%to the idea later, maybe in a ``practical considerations'' section/subsection}
%In addition, the algorithm employs Delta debugging~\cite{Zeller:2002}
%to minimize the test set that are executed before a test in
%an execution (lines 7--8). Together with the minimized
%dependent test set, the test revealing with different behaviors
%are added to the output (line 9).


%\todo{JW}{We should mention that we used the tool to find/verify the
%examples. We should also mention that isolation corresponds to $k=1$
%and we did pair-wise (corresponding to $k=2$).}
%\todo{JW}{
%For reverser execution, as far as I remember, we didn't use it to for
%the actual examples we have. But we might claim that it is useful for
%identifying particular kinds of deps. But it would be better if we had
%an example for that.}



%\section{Unused text snippets}
%

\begin{itemize}

\item \todo{KM}{I kind of understand what this paragraph is saying.
However the many minor mistakes in the writing make it very hard to
follow up.} Patterns of dependent tests.
In many cases, there is a \textit{N -- 1} \todo{KM}{The first time I read this,
I read as N minus 1, which is the incorrect way. Maybe write down as ``N to 1''}
dependence relationship, in which $N$-th \todo{KM}{Is ``th'' really needed?} distinct tests depends\todo{KM}{This should be either ``tests depend on'' (more likely) or ``test depends on'', but I couldn't decide} on the same test, which probably is used to set up the environment. For
such cases, the 1 depend test \todo{KM}{``1 dependent test'' or ``first
dependent test''} should be moved to the common \CodeIn{setUp}
\todo{KM}{Consistency: consider @Before} method.
Less frequently, there is a \textit{1 -- N} dependence relationship
\todo{KM}{Have we ever seen this, or is this purely theoretical?}, in which one
test depends on $N$ tests to set up its testing environment.  In one subject, a newly-added test changes the shared variable state of an existing test. Although the newly-added test is executed after the existing test and reveals the same behavior when executing in isolation, the existing test exhibit different behavior if it is executed after the newly-added test.


\item 
\todo{JW}{If we really want to discuss this, this should be connected
to the theory section, as all this follows from theory. A ``finding''
might be that these differences actually matter in practice. And I
don't think we checked that.}
Different techniques have their own strength in detecting dependent tests. We
have investigate three methods (i.e., executing in isolation, executing in a
reversed order\todo{KM}{Did we (do we) really do this?}, and executing in
k-permutation) to identify dependent tests, and found each method complements others. There exist certain tests that can only be found by one method
but missed by the other two.  Executing tests in isolation found more dependent tests
than executing in a reversed order, and executing every $k$-permutation is
infeasible in practice due to the exponentially large number of possible combinations.

\end{itemize}


\end{document}
% vim:wrap:wm=8:bs=2:expandtab:ts=4:tw=70:


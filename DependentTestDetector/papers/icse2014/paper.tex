\documentclass{sig-alternate}
\usepackage{microtype}
\usepackage{graphicx}
\usepackage[draft]{fixme}
\usepackage{url}
\usepackage{amsfonts}
\usepackage{amssymb}
\usepackage{amsmath}
\usepackage{algorithmic}
\usepackage{booktabs}
\usepackage{listings}
\usepackage[T1]{fontenc}
\usepackage{lmodern}
\usepackage{color}
\usepackage[draft]{hyperref}
\usepackage{xspace}
\usepackage{subfigure}
\usepackage{bold-extra}
\usepackage{bm, array}
\usepackage{balance}
\usepackage{float}
\usepackage{caption}
\captionsetup[table]{font=bf}
\captionsetup[figure]{font=bf}
\floatstyle{plaintop}
\restylefloat{table}
%\usepackage[tableposition=top]{caption}

%\makeatletter
%\def\footnotesize{\@setsize\footnotesize{9pt}\viipt\@viipt}
%\makeatother

\newcolumntype{C}{>{\centering\arraybackslash}p{2em}}

\usepackage{color}
%\usepackage{times}
\usepackage{graphicx}
\usepackage{epsf}
\usepackage{verbatim}
%\usepackage{psfig}
\usepackage{cite}
\usepackage{url}
\usepackage{color}
\usepackage{alltt}

\newcommand{\StateProjection}{static analysis}
\newcommand{\CurAspectJSubjectCount}{12}

\newcommand{\Add}{\CodeIn{add}}
\newcommand{\AVTree}{\CodeIn{AVTree}}
\newcommand{\Assignment}[3]{$\langle$ \Object{#1}, \Object{#2}, \Object{#3} $\rangle$}
\newcommand{\BinaryTreeRemove}{\CodeIn{BinaryTree\_remove}}
\newcommand{\BinaryTree}{\CodeIn{BinaryTree}}
\newcommand{\Caption}{\caption}
\newcommand{\Char}[1]{`#1'}
\newcommand{\CheckRep}{\CodeIn{checkRep}}
\newcommand{\ClassC}{\CodeIn{C}}
\newcommand{\CodeIn}[1]{{\small\texttt{#1}}}
\newcommand{\CodeOutSize}{\scriptsize}
\newcommand{\Comment}[1]{}
\newcommand{\Ensures}{\CodeIn{ensures}}
\newcommand{\ExtractMax}{\CodeIn{extractMax}}
\newcommand{\FAL}{field-ordering}
\newcommand{\FALs}{field-orderings}
\newcommand{\Fact}{observation}
\newcommand{\Get}{\CodeIn{get}}
\newcommand{\HashSet}{\CodeIn{HashSet}}
\newcommand{\HeapArray}{\CodeIn{HeapArray}}
\newcommand{\Intro}[1]{\emph{#1}}
\newcommand{\Invariant}{\CodeIn{invariant}}
\newcommand{\JUC}{\CodeIn{java.\-util.\-Collections}}
\newcommand{\JUS}{\CodeIn{java.\-util.\-Set}}
\newcommand{\JUTM}{\CodeIn{java.\-util.\-TreeMap}}
\newcommand{\JUTS}{\CodeIn{java.\-util.\-TreeSet}}
\newcommand{\JUV}{\CodeIn{java.\-util.\-Vector}}
\newcommand{\JMLPlusJUnit}{JML+JUnit}
\newcommand{\Korat}{Korat}
\newcommand{\Left}{\CodeIn{left}}
\newcommand{\Lookup}{\CodeIn{lookup}}
\newcommand{\MethM}{\CodeIn{m}}
\newcommand{\Node}[1]{\CodeIn{N}$_#1$}
\newcommand{\Null}{\CodeIn{null}}
\newcommand{\Object}[1]{\CodeIn{o}\ensuremath{_#1}}
\newcommand{\PostM}{\MethM$_{post}$}
\newcommand{\PreM}{\MethM$_{pre}$}
\newcommand{\Put}{\CodeIn{put}}
\newcommand{\Remove}{\CodeIn{remove}}
\newcommand{\RepOk}{\CodeIn{repOk}}
\newcommand{\Requires}{\CodeIn{requires}}
\newcommand{\Reverse}{\CodeIn{reverse}}
\newcommand{\Right}{\CodeIn{right}}
\newcommand{\Root}{\CodeIn{root}}
\newcommand{\Set}{\CodeIn{set}}
\newcommand{\State}[1]{2^{#1}}
\newcommand{\TestEra}{TestEra}
\newcommand{\TreeMap}{\CodeIn{TreeMap}}

\newenvironment{CodeOut}{\begin{scriptsize}}{\end{scriptsize}}
\newenvironment{SmallOut}{\begin{small}}{\end{small}}

\newcommand{\pairwiseEquals}{PairwiseEquals}
\newcommand{\monitorEquals}{MonitorEquals}
%\newcommand{\monitorWField}{WholeStateW}
\newcommand{\traverseField}{WholeState}
\newcommand{\monitorSMSeq}{ModifyingSeq}
\newcommand{\monitorSeq}{WholeSeq}

\newcommand{\IntStack}{\CodeIn{IntStack}}
\newcommand{\UBStack}{\CodeIn{UBStack}}
\newcommand{\BSet}{\CodeIn{BSet}}
\newcommand{\BBag}{\CodeIn{BBag}}
\newcommand{\ShoppingCart}{\CodeIn{ShoppingCart}}
\newcommand{\BankAccount}{\CodeIn{BankAccount}}
\newcommand{\BinarySearchTree}{\CodeIn{BinarySearchTree}}
\newcommand{\LinkedList}{\CodeIn{LinkedList}}

\newcommand{\Book}{\CodeIn{Book}}
\newcommand{\Library}{\CodeIn{Library}}

\newcommand{\Jtest}{Jtest}
\newcommand{\JCrasher}{JCrasher}
\newcommand{\Daikon}{Daikon}
\newcommand{\JUnit}{JUnit}

\newcommand{\trie}{trie}

\newcommand{\Perl}{Perl}


\newcommand{\SubjectCount}{11}
\newcommand{\DSSubjectCount}{two}

\newcommand{\Equals}{\CodeIn{equals}}
\newcommand{\Pairwise}{PairwiseEquals}
\newcommand{\Subgraph}{MonitorEquals}
\newcommand{\Concrete}{WholeState}
\newcommand{\ModSeq}{ModifyingSeq}
\newcommand{\Seq}{WholeSeq}
\newcommand{\Aeq}{equality}

\newcommand{\Meaning}[1]{\ensuremath{[\![}#1\ensuremath{]\!]}}
\newcommand{\Pair}[2]{\ensuremath{\langle #1, #2 \rangle}}
\newcommand{\Triple}[3]{\ensuremath{\langle #1, #2, #3 \rangle}}
\newcommand{\SetSuch}[2]{\ensuremath{\{ #1 | #2 \}}}
%\Comment{
%\newtheorem{definition}{Definition}
%\newtheorem{theorem}[definition]{Theorem}
%}
\newcommand{\Equiv}[2]{\ensuremath{#1 \EquivSTRel{} #2}}
\newcommand{\EquivME}{\Equiv}
\newcommand{\EquivST}{\Equiv}
\newcommand{\EquivSTRel}{\ensuremath{\cong}}
\newcommand{\Redundant}[2]{\ensuremath{#1 \lhd #2}}
\newcommand{\VB}{\ensuremath{\mid}}
\newcommand{\MES}{method-entry state}

\newcommand{\Small}[1]{{\small{#1}}}

\newcommand{\CenterCell}[1]{\multicolumn{1}{c|}{#1}}
\newcommand{\Fix}[1]{{\large\textbf{FIX}}#1{\large\textbf{FIX}}}

\newcommand{\CodeInS}[1]{{\scriptsize\texttt{#1}}}
\newcommand{\CodeInFN}[1]{{\footnotesize\texttt{#1}}}
\newcommand{\CodeOutFN}{\footnotesize}

\newcommand{\SmallSpace}{\vspace*{-1.4ex}}
\newcommand{\Item}{\SmallSpace\item}
\newenvironment{Itemize}{\begin{itemize}}{\end{itemize}\SmallSpace}
\newenvironment{Enumerate}{\begin{enumerate}}{\end{enumerate}\SmallSpace}

% Permit line breaking
\newcommand{\dependenceaware}{de\-pend\-ence-aware}

%\newtheorem{definition}{Definition}
%\newtheorem{theorem}[definition]{Theorem}

%\newcommand{\Item}{\vspace*{-0.5ex}\item\vspace*{-0.5ex}}
%\newenvironment{Itemize}{\begin{itemize}\vspace*{-1ex}}{\end{itemize}\vspace*{-1ex}}
%\newenvironment{Enumerate}{\begin{enumerate}\vspace*{-1ex}}{\end{enumerate}\vspace*{-1ex}}
\newenvironment{Definition}{\begin{definition}\vspace*{-1.5ex}}{\end{definition}\vspace*{-1.5ex}}

% Local Variables:
% mode:latex
% tex-main-file:"ase04.tex"
% End:

%\newcommand{\CodeIn}[1]{{\small\texttt{#1}}}
\newcommand{\tinyrelax}{\vspace{1mm}}
\newcommand{\smallrelax}{\vspace{2mm}}

\newcommand{\tinysqueeze}{\vspace{-1mm}}
\newcommand{\smallsqueeze}{\vspace{-2mm}}
\newcommand{\bigsqueeze}{\vspace{-5mm}}

\newcommand{\todo}[1]{{\color{red}\bfseries [[{#1}]]}}
\newcommand{\edit}[1]{{\color{blue}\bfseries [[{#1}]]}}
%\newcommand{\todo}[1]{{}}
%\newcommand{\edit}[1]{{}}
%\newcommand{\CodeIn}[1]{{\small\texttt{#1}}}
\newcommand{\code}[1]{{\small\texttt{#1}}}
 % Add line between figure and text
%\makeatletter
%\def\topfigrule{\kern3\p@ \hrule \kern -3.4\p@} % the \hrule is .4pt high
%\def\botfigrule{\kern-3\p@ \hrule \kern 2.6\p@} % the \hrule is .4pt high
%\def\dblfigrule{\kern3\p@ \hrule \kern -3.4\p@} % the \hrule is .4pt high
\makeatother
 % If there is a line, you can get away with reducing the separation between
 % figures and text.  Don't do this without the line, though.
\addtolength{\textfloatsep}{-.5\textfloatsep}
\addtolength{\dbltextfloatsep}{-.5\dbltextfloatsep}
\addtolength{\floatsep}{-.5\floatsep}
\addtolength{\dblfloatsep}{-.5\dblfloatsep}

\newtheorem{definition}{Definition}
\newtheorem{theorem}{Theorem}
\newtheorem{corollary}{Corollary}
%\newtheorem{proof}{Proof}
\newcommand{\pass}{\ensuremath{\mathit{PASS}}\xspace}
\newcommand{\fail}{\ensuremath{\mathit{FAIL}}\xspace}
\newcommand{\dtnum}{{{96}}\xspace}
\newcommand{\repnum}{{{5}}\xspace}
\newcommand{\subjnum}{{{4}}\xspace}
\newcommand{\ourtool}{DTDetector\xspace}
\newcommand{\jodatime}{Joda-Time\xspace}
\newcommand{\dtrate}{82\%\xspace}

% commands for formalization
\newcommand{\suites}[0]{\ensuremath{\mathcal{S}\xspace}}
\newcommand{\environs}[0]{\ensuremath{\mathcal{E}\xspace}}
\newcommand{\alltests}[0]{\ensuremath{\mathcal{T}\xspace}}
\newcommand{\manifest}[1]{\ensuremath{\prec_{#1}}}
\newcommand{\suite}[1]{\ensuremath{ \langle  #1 \rangle }}
\newcommand{\env}[0]{\ensuremath{\mathbf{E}}\xspace}
\newcommand{\execfunc}{\ensuremath{exec}\xspace}
\newcommand{\exec}[2]{\ensuremath{\execfunc(#1,#2)}}
\newcommand{\result}[2]{\ensuremath{R(#1|#2)}}

\newcommand{\question}[2]{\medskip\noindent\hspace{\parindent}\begin{minipage}{0.9\columnwidth}\textit{#1}
\textsc{#2}
\end{minipage}
\medskip}

%% Bring items closer together in list environments
%% This doesn't work with an optional argument to the list environment.
% Prevent infinite loops
\let\Itemize =\itemize    
\let\Enumerate =\enumerate
\let\Description =\description
% Zero the vertical spacing parameters
\def\Nospacing{\itemsep=0pt\topsep=0pt\partopsep=0pt\parskip=0pt\parsep=0pt}
% Redefine the environments in terms of the original values
\renewenvironment{itemize}{\Itemize\Nospacing}{\endlist}
\renewenvironment{enumerate}{\Enumerate\Nospacing}{\endlist}
\renewenvironment{description}{\Description\Nospacing}{\endlist}

\setlength{\leftmargini}{.5\leftmargini}
%commands must be the last package imported
% This package provides the \todo{Name}{Comment} command
%\usepackage{commands}
%remove syntax highlighting from Java code
\lstset{basicstyle=\ttfamily,tabsize=2,keywordstyle=\ttfamily,stringstyle=\ttfamily,commentstyle=\ttfamily,
captionpos=b,numberstyle=\small\ttfamily,numbersep=1ex,
keywordstyle=\color{red}}

%\renewcommand{\todo}[2]{}

\author{
Sai Zhang, Darioush Jalali, Jochen Wuttke, K{\i}van{\c{c}}
Mu{\c{s}}lu, Wing Lam, \\ Michael D. Ernst, David Notkin\\ 
\affaddr{Department of Computer Science \& Engineering, University of Washington, Seattle, USA} \\ 
\email{\{szhang, darioush, wuttke, kivanc, winglam, mernst\}@cs.washington.edu}
}

\title{Empirically Revisiting\\the Test Independence Assumption}

%\title{Does Test Execution Order Affect Test Results? \\ Understanding %and Detecting Order-Dependent Tests}
%\vspace{-2mm}
%\subtitle{Understanding and Detecting Dependent Tests}

\begin{document}
\conferenceinfo{ISSTA}{'14, July 21--25, 2014, San Jose, CA, USA}
\CopyrightYear{2014}
\crdata{978-1-4503-2645-2/14/07}
\maketitle



\begin{abstract}

In a test suite, all the test cases should be independent:  no test
should affect any other test's result, and running the tests in any order
should produce the same test results.
Techniques such as test %selection and
prioritization generally assume that the tests in a suite are independent.
%but they do not justify that assumption.
Test dependence is a little-studied phenomenon.
This paper presents five results related to test dependence.

First, we characterize the test dependence that arises in practice.
We studied \dtnum real-world dependent tests from \repnum
issue tracking systems.
Our study
%  identifies common characteristics of dependent tests.  It 
shows that test dependence can be hard for programmers to identify.
It also shows that test dependence can cause
non-trivial consequences, such as masking
program faults and leading to spurious bug reports.

Second, we formally define test dependence in terms of
test suites as ordered sequences of tests along with explicit
environments in which these tests are executed.
We formulate the problem
of detecting dependent tests and prove that a useful special
case is NP-complete. 

Third, guided by the study of real-world dependent tests, we
propose and compare four algorithms to detect
dependent tests in a test suite. 

Fourth, we applied our dependent test detection algorithms
to \subjnum real-world programs and found
dependent tests 
in each human-written
and automatically-generated test suite.

Fifth, we empirically assessed the impact of
dependent tests on five test prioritization techniques
and found that dependent tests affect the output 
of all five techniques; that is, the reordered suite 
fails even though the original suite did not.

%the follwoing claim might be too strong
%Our tool revealed a large number of previously-unknown dependent tests.
%In our study, on average \todo{xx}\% of the human-written tests are
%dependent and \todo{xx}\% of the automatically-generated tests
%are dependent.

\smallskip
\smallskip


\noindent
\textbf{Categories and Subject Descriptors}:  
D.2.5 [Software Engineering]: Testing and Debugging.
\\
\textbf{General Terms: }
Reliability, Experimentation.
\\
\textbf{Keywords: }
Test dependence, detection algorithms, empirical studies.

\end{abstract}


\label{dummy-label-for-etags}


%\category{D.2.5}{Testing and Debugging}
%\keywords{Software testing; test dependence; test selection; test prioritization}


\section{Introduction}

%\todo{Is ``dependent test'' a term that can be applied to one test in
%  isolation?  Or is a pair of tests dependent on one another if the result
%  of one of them changes if the other one is run first?  The paper doesn't
%  make this clear anywhere in sections 1 and 2, and this leads to some
%  confusion for the reader.}

%Informally, a \emph{dependent test} produces different test
%results when executed in different environments. 
Consider a test suite containing two tests \code{A}
and \code{B}, where running \code{A} and then \code{B} leads
to \code{A} passing, while running \code{B} and then
\code{A} leads to \code{A} failing. We call \code{A}
an \textit{order-dependent} test (in the context of this test suite), since its result depends on
whether it runs after \code{B} or not.



In a test suite, all the test cases should be independent:
no test should affect any other test's result, and
running the tests in any order should produce the same test results.
%Practitioners are well aware of test dependence:  coding
%guidelines~\cite{unit-test-def,Massol:2003} and
%standards~\cite{IEEE:829-1998,IEEE:829-2008} say to avoid or document it,
%and tools support those goals~\cite{junitordering,depunit,testng, easymock, randomjunit}.
%%\todo{Cite a mocking framework}.
%Researchers are also aware of test
%dependence~\cite{Csallner:2004, Steimann:2013, Gray:1994:QGB:191843.191886,Chays:2000:FTD:347324.348954,kapfhammeretal:FSE:2003,Wang:2007:AGC, Samimi:2013:DM}.
%%\todo{Cite research about creating mocks.}
%Nonetheless, much testing research and practice
%assumes test independence;
%this includes techniques for test selection~\cite{harroldetal:OOPSLA:2001,RenCR2006},
%test prioritization~\cite{Elbaum:2000:PTC:347324.348910},
%and test parallelization~\cite{Misailovic:2007}.
%% These are not offected:
%%, test factoring~\cite{Saff:2005}, and test carving~\cite{Elbaum:2006}.
%theoretical results, algorithms, and tools may behave unexpectedly
%in the presence of dependent tests.
%A test suite that contains dependent tests affects the applications
%of these techniques.
%These techniques produce incorrect results if run on a test suite that contains dependent tests. 
%\todo{The claim of the above sentence is too strong, and may irritate readers, given we do not have strong evidence.
%I would tone down it as: dependent tests can
%affect the results of these techniques. Further, the ``correctness''
%of a test selection/prioritization technique is decided by whether
%a test has been selected or not, rather than whether the test should
%maintain the same resutls or not.}
The assumption of test independence 
is important so that testing techniques behave
as designed.
Consider a test prioritization or selection algorithm.  By design, if its
input is a passing test suite, then its output should be a passing test
suite.  If the suite contains an order-dependent test, then prioritization
or selection can introduce test failures, which violates the design
requirement.

Many techniques assume test independence, including test
prioritization~\cite{Elbaum:2000:PTC:347324.348910,
Rummel:2005:TPR:1066677.1067016, Srivastava:2002:EPT:566172.566187, Jiang:2009:ART},
test selection~\cite{harroldetal:OOPSLA:2001, Orso:2004:SRT,
Briand:2009:ART, Zhang:2012:RMT, Nanda:2011:RTP, hsu09may},
test execution~\cite{Kim:2013:OUT, Misailovic:2007},
test factoring~\cite{Saff:2005, Wu:2010:LRV}, test carving~\cite{Elbaum:2006},
and experimental debugging techniques~\cite{Zeller:2002,
Steimann:2013, Zhang:2013:IMF}.
%often implicitly assume no test dependences in
%a test suite. 
%\todo{check the following sentencee, people may ask
%is the above paper list representative enough? be aware.}
%\todo{I think it would be more compelling for this section to only cite
%  papers that do not mention but implicitly assume test independence.  The
%  current writing sounds like you might be cherry-picking:  why this set of
%  papers?  Are the ratios representative?  But if you just give a list (as
%  long as possible) of papers that don't assume it, then that makes the
%  point you want to, and no one will misinterpret the list as trying to be
%  representative.}
However, this critical assumption is
rarely questioned, investigated, or even mentioned:
none of the above papers explicitly mentions the assumption
as a limitation or a threat to validity.
Between 2000 and 2013, 31 papers on test prioritization were
published in the research track of
ICSE, FSE, ISSTA, ASE, and ICST or in
TOSEM and TSE~\cite{alltestprior}.
Of these,
27 papers explicitly or implicitly assumed test independence,
3 papers acknowledged that the potential dependences between tests
may affect\todo{in what way?  be specific} the prioritization output~\cite{Kim:2002:HTP:581339.581357,
Qu:2008:CRT, Rothermel:2004:TSC},
and only 1 paper considered test dependence in the design of
test prioritization algorithms~\cite{10.1109/TSE.2012.26}.

Anecdotally, a number of researchers have told us that they believe test dependence
is not a significant concern in practice.
We wish to investigate the validity of this unverified conventional wisdom,
in order to understand whether test dependence arises in practice, 
the repercussions of dependent tests, and how to 
detect dependent tests.

%is generally ignored.
%Is this acceptable, because test dependence does
%not arise in practice?
%Is it because even when test dependence arises, there are few
%negative repercussions?
%Is it because no one has studied this problem or thought to examine it?
%Is it because the problem is important but is too hard to analyze or understand?

\subsection{Manifest Test Dependence}

%To explore these questions, 
This paper focuses on test
dependence that manifests as a
difference in test result (i.e., passing or failing) as determined by the testing oracle.
We adopt the results of the default
%% True, but a distraction.
% , usually implicit,
order of execution of a test suite as the
expected results; these are the results that a developer sees when running
the suite in the standard way. A test is dependent when there exists a possibly
reordered subsequence of the original test suite, in which
the test's result (determined by its existing testing
oracles) differs from its expected result in the
original test suite.
%
That is, manifest test dependence
%we focus on a \emph{manifest} perspective of test dependence,
requires a concrete order of the test suite that
produces {different} results than expected.  
%
%



This paper uses \textit{dependent test} as a shorthand for
\textit{manifest order-dependent test}
unless otherwise noted.
A single test may consist of setup and teardown
code, multiple statements, and multiple assertions
distributed through the test.



%\subsection{Causes of Dependent Tests}

\subsection{Causes and Repercussions}
%\todo{I merged the two original subsections into the following one. looks better?}
%\todo{also some text editing below}
%\subsection{Repercussions of Dependent Tests}
\label{sec:intro-repercussions}

%\todo{Fold this section into Section~\ref{sec:intro-repercussions}??}

Test dependence results from interactions with other tests,
as reflected in the execution environment.
Tests may make \textit{implicit} assumptions about their
execution environment --- values of global variables,
contents of files, etc. A dependent test
manifests when another test alters the execution
environment in a way that invalidates those assumptions.

%A test has the potential to yield
%different test results when executed in different environments
%--- global variables with different values, differences in the file system, etc.

Why does this happen?
%As
%suggested by the principle of unit testing~\cite{Greiler:2013:SAT, Massol:2003},
Each test ought to initialize (or mock) the execution environment
and/or any resources it will use.
Likewise, after test execution, it should reset the
execution environment and external resources
to avoid affecting other tests' execution.
However, developers sometimes
%are as likely to
make mistakes when writing tests.
Even though frameworks such as
JUnit provide ways to set up the environment for a test execution and clean
up the environment afterward,
they cannot ensure that it is done
properly. This means that tests, like other code,
sometimes have unintended and unexpected behaviors.
%And
%as programs increase in complexity, so may tests, which may
%increase the frequency of such problems in tests, which may
%in turn increase the frequency of test dependence.
%\edit{check the grammar of the above sentence}



%\todo{Where does this paragraph belong?  Here?}
%This principle is adopted and confirmed by many
%real-world developers (Sections~\ref{sec:study} and~\ref{sec:expdiscussion}).
%When it needs to interact with the execution environment,
%it should mock or carefully resetting external resources
%
%Ideally, each test should not depend on its environment, because it
%initializes any resources it will use 
%Likewise, the test should not modify its environment, because of mocks or
%resetting resources after test execution. 
%




% Our study of \dtnum real-world, confirmed dependent
% tests 
% % from \repnum software issue tracking systems
% (Section~\ref{sec:study}) identified two 
Here are three consequences of the fact that a dependent
test gives different results depending on when it is executed
during testing.

\textbf{(1)}
Dependent tests can
\emph{mask faults in a program}. Specifically, executing a test suite in the
default order does not expose the fault, whereas
executing the same test suite in a different order does. 
% We found a
One bug~\cite{clibug} in the Apache CLI library~\cite{cli}
was masked by two dependent tests
for 3 years (Section~\ref{sec:repercussion}).

\textbf{(2)}
Test dependences can lead to \emph{spurious bug reports}.
When a dependent test fails, it usually represents
%\edit{change to use the word of "represents", good? using
%"reveals" may sound like dependent test is good to improve code quality}
a weakness in the test
suite (such as failure to perform proper initialization) rather than a bug
in the program. 
When a test should pass but
it fails after reordering due to the dependence,
people who are not aware of the dependence can get confused
and might report bugs.
% about the failing test,
%even though this is exactly the intended behavior.
%Programmers made these errors even though frameworks such as
%JUnit provide ways to set up the environment for a test execution and clean
%up the environment afterward.
%
As an example, the Eclipse developers
investigated a bug report~\cite{eclipsebug} in SWT for
more than a month before realizing that the 
bug report was invalid and was caused by test dependences
(i.e., a test should pass, but it failed when a user
ran tests in a different order).
%were intentional,
%allowing them to close the bug report without a change to the system.
%


%Second, guided by the findings of our study, we design two algorithms
%to detect manifest dependent tests. By applying our algorithms
%to \todo{xx} open-source programs and their test suites, we 
%found a large number of unknown dependent tests, .

\textbf{(3)}
Dependent tests can \textit{interfere with downstream testing
techniques} that change a test suite and thereby change a test's execution environment.
Examples of such techniques include
test selection techniques (which identify a subset of
the input test suite to run during
regression testing)~\cite{harroldetal:OOPSLA:2001, Orso:2004:SRT,
Briand:2009:ART, Zhang:2012:RMT, Nanda:2011:RTP, hsu09may},
test prioritization techniques (which reorder the
input to discover defects sooner)~\cite{Elbaum:2000:PTC:347324.348910, Kim:2002:HTP:581339.581357, Rummel:2005:TPR:1066677.1067016, Srivastava:2002:EPT:566172.566187, Jiang:2009:ART},
%test parallelization techniques (which schedule the input
%tests for execution across multiple
%CPUs)~\cite{Misailovic:2007},
test execution techniques~\cite{Kim:2013:OUT},
test factoring~\cite{Saff:2005, Wu:2010:LRV} and test carving~\cite{Elbaum:2006} (which
convert large system tests into smaller unit tests),
%test generation (which re-executes suites as it builds them up)~\cite{PachecoE2005,RobinsonEPAL2011},
experimental debugging techniques (such as Delta
Debugging~\cite{Zeller:2002, Steimann:2013, Zhang:2013:IMF} and mutation
analysis~\cite{Zhang:2012:RMT, Schuler:2009:EMT, Zhang:2013:FMT},
which run a set of tests repeatedly), etc. 
Most of these techniques implicitly assume that
there are no test dependences in the input test suite. Violation of
this assumption, as we show happens in practice, can cause unexpected
output. %\todo{change erroneous to different?} 
As an example, test prioritization may produce a reordered sequence
of tests that do not
return the same results as they do when executed in
the default order. Section~\ref{sec:impact}
provides empirical evidence to show that
dependent tests do affect the output of five test prioritization
techniques.


\subsection{Contributions}
\label{sec:contributions}

This paper addresses and questions
conventional wisdom about the test independence assumption. 
This paper makes the following contributions:

\begin{itemize}

  \item \textbf{Study.} We describe a study of \dtnum real-world
  dependent tests from \repnum software issue tracking
  systems to characterize dependent tests that
  arise in practice.  Test dependence can have
  potentially non-trivial repercussions and can be hard to identify
  (Section~\ref{sec:study}).

\item \textbf{Formalization.} We formalize test dependence
  in terms of test suites as ordered sequences of tests and explicit execution
  environments for test suites.  The formalization enables reasoning about test dependence
  as well as a proof that finding manifest dependent tests is an NP-complete
  problem (Section~\ref{sec:formalism}).

  \todo{I edited the following paragraph, adding the heuristic algorithm}
  \item \textbf{Algorithms.} We present four algorithms
  to detect dependent tests: one reversing,
  one randomized, one exhaustive bounded, and one that prunes the search
  space using dynamic analyses.
  All four algorithms are \emph{sound} but \emph{incomplete}:
  every dependent test they identify is real, but the algorithms
  do not guarantee to find all dependent tests (Section~\ref{sec:detecting}). 
  %\edit{check above when the algorithm section is written}

  \item \textbf{Evaluation.} We implemented our algorithms in a prototype
  tool, called \ourtool (Section~\ref{sec:impl}).
  \ourtool detected 27 previously-unknown dependent tests in human-written
  unit tests in \subjnum real-world subject programs.
  % (and even more in automatically-generated tests).
  The developers confirmed all of these as
  undesired (Section~\ref{sec:evaluation}).

  %\item \textbf{Assessment.} 
  \item \textbf{Impact Assessment.} We implemented five test prioritization
  techniques and evaluated them on \subjnum subject programs
  that contain dependent tests. The results show that all
  five test prioritization techniques are affected by\todo{be more specific
    about what the effect is} dependent tests
  (Section~\ref{sec:evaluation}).

  % \textit{every} subject program we studied, from both  and automatically-generated
  % unit tests (Section~\ref{sec:evaluation}).
  %been discovered before, showing that on average \todo{xx}\% of the human-written
  %unit tests are dependent and \todo{xx}\% of the automatically-generated
  %unit tests are dependent
  %Finally, we discuss a set of open questions and other possible impacts of dependent
  %tests in Section~\ref{sec:discussion}.
\end{itemize}
\vspace{-4mm}
\paragraph{Implications}
\todo{I move the implications from conclusions to here}
Our findings are of utility to practitioners and researchers.
Both can learn that test dependence is a real problem that should not be
ignored any longer, because it leads to false positive and false negative
test results.
Practitioners can adjust their practice based on what code patterns most
often lead to test dependence, and they can use our tool to 
find dependent tests.
Researchers are posed important but challenging new problems, such as how
to adapt testing methodologies to account for dependent tests and how to detect
and correct all dependent tests.



%  LocalWords:  Kapfhammer Soffa subsequence SWT CLI NUM dependences

%  LocalWords:  teardown


%\section{Dependent Tests in Practice}
\section{Real-World Dependent Tests}
\label{sec:study}

\newcommand{\unum}{{{14}}\xspace}
\newcommand{\svratio}{{{61}}}
\newcommand{\svnum}{{{59}}\xspace}
\newcommand{\unfixed}{{{58}}\xspace}


% are known to occur in practice, but
Little is known about the characteristics of dependent tests.
This section qualitatively studies
concrete examples of test dependence found in
well-known open source software. 


\subsection{Sources and Study Methodology}

We examined five
% well-known, publicly-accessible 
software issue
tracking systems: Apache \cite{apachebug},
Eclipse~\cite{eclipsebug}, JBoss~\cite{jbossbug},
Hibernate~\cite{hibernatebug}, and Codehaus~\cite{codehausbug}.
Each issue tracking system serves tens of projects.
% , and
% holds thousands of bug reports, feature requests, improvement
% suggestions, etc.

For each issue tracking system, we searched for four phrases
(``dependent test'', ``test dependence'', ``test execution order'',
``different test outcome'') and manually examined the matched results. For each match, we read the
description of the issue report, the discussions between reporters
and developers, and the fixing patches (if available). This information
helped us understand whether the report is about test dependence.
%--- a test manifesting different results under different
%test execution orders. 
Each dependent test candidate was examined by
at least two people and the whole process consisted of several
rounds of (re-)study and cross checking. We ignored reports
that are described vaguely, and we excluded tests whose results are
affected by non-determinism (e.g., multi-threading).
In total, we examined the first 450 matched reports, of which 53
reports are about test dependence (some reports contain multiple
dependent tests).
All collected dependent tests are publicly available 
at: \url{http://homes.cs.washington.edu/~szhang/dependent\_tests.html}


\subsection{Findings}
\label{sec:studyfindings}

\begin{table*}[t]
\vspace{1mm}
\centering
\small{
\setlength{\tabcolsep}{.15\tabcolsep}
\begin{tabular}{|c||c|c|c|c||c|c|c|c|c||c|c|c|c|c||c|c|c|c|}
\hline
%1&2&3&4&5&6&7&8&9&10&11&12&13&14\\
\textbf{Issue}&\multicolumn{4}{|c||}{\textbf{Dependent Tests}}&\multicolumn{5}{|c||}{\textbf{\# Involved Tests for}}&\multicolumn{5}{|c||}{\textbf{Resolution}}&\multicolumn{4}{|c|}{\textbf{Root Cause}}\\
\cline{2-5}\cline{11-19}
\textbf{Tracking} &Total&\multicolumn{3}{|c||}{Severity}&\multicolumn{5}{|c||}{\textbf{Manifestation}}&
&\multicolumn{4}{|c||}{Patch Location}&Static&File & Data-& Unknown\\
\cline{3-10}\cline{12-15}
\textbf{System}&Number&Major&Minor&Trivial& Self &1 test&2 tests&3 tests & Unknown&Days&Code&Test&Doc&Unfixed&Variable&System& base &\\
\hline
Apache&26&22&3&1&0&5&18&1&2&93&5&20&0&1&9&3&8 &6\\
\hline
Eclipse&59&0&59&0&0&0&49&1&9&48&1&8&49&1&49&0&0 &10\\
\hline
JBoss&6&6&0&0&0&0&3&0&3&44&0&2&0&4&1&0& 0 & 5\\
\hline
Hibernate&3&1&1&1&0&0&3&0&0&6&0&1&0&2&0&0& 2 & 1\\
\hline
Codehaus&2&2&0&0&1&1&0&0&0&3&0&1&0&1&0&1&0 &1\\
\hline
\hline
\textbf{Total} & \dtnum &31&63&2&1&6&73&2&\unum&194&6&32&49&9&\svnum&4&10&23\\
\hline
\end{tabular}
}
\vspace{-2mm}
\caption{{\label{tab:studyresults}
Real-world dependent tests.
%Column ``Total Number'' shows the total number of identified dependent tests.
Column ``Severity'' is the developers' assessment of the importance of the
test dependence.
Column ``\# Involved Tests for Manifestation'' is the number of tests needed
to manifest the dependence. Column ``Self'' shows the number of
tests that depend on themselves. Column ``Days'' is the
average days taken by developers to resolve a dependent test.
Column ``Patch Location'' shows how developers resolved the dependent tests:
by modifying program code, by modifying test code, by adding
code comments, or not fixed.
%In column ``Dependence Root Cause'', ``other'' execution environment
%differences include language, locale, and databases.
}
}
%\todo{Split ``unfixed'' into ``documented'' and ``unfixed''}
\end{table*}

%  LocalWords:  JBoss Codehaus


Table~\ref{tab:studyresults} summarizes the dependent tests.


\subsubsection{Characteristics}


We summarize three characteristics of dependent tests:
manifestation, root cause, and developer actions.

\vspace{1mm}
\noindent \textbf{{Manifestation: at least \pertange of the dependent
tests in the study can be manifested by 2 or fewer tests.}}
A dependent test is manifested if there exists a possibly reordered
subsequence of the original test suite, such that the test
produces a different
result than when run in the original suite.
We measure the size of the reported subsequence 
in the issue report.
If the test produces a different result when run
in isolation, the number of tests to manifest
the dependent test is 1.
If the test produces a different result
when run after one other test (often, the subsequence is
running these two tests in the opposite order as the full original test
suite), then the number of tests to manifest the dependent test is 2.
Among the \dtnum studied dependent tests, we found only 2 of them
require 3 tests to manifest the dependence.
One other test depends on itself:
running the test twice produces different results than running it once,
because this test side-effects a database it reads.
We count this special case separately in the ``Self'' column
of Table~\ref{tab:studyresults}.

For the remaining \unum dependent tests, the number of involved tests
is unknown, since the relevant information is missing
or vaguely described in the issue tracking systems. For example,
some reports simply stated that ``running \textit{all} tests in one class before
test \emph{t} makes \emph{t} fail'' or ``randomizing the test execution order
makes test \emph{t} fail''.

%Due to the extremely intricated
%environment needed for each open-source project, we were unable
%to reproduce all such dependent tests and minimize the involved tests.


% In theory,
% given a $n$-sized test suite, dependent test can occur in any
% length of permutations. However, among \dtnum collected tests,
% 86 (82\%) of them can be manifested by running no more than
% 2 tests. 

%\todo{Discuss the ``unknown'' column.  What happened?  Could we not
%  reproduce it at all?  Why not?}

\vspace{1mm}
\noindent \textbf{{Root cause: at least \svratio\% of the dependent tests
in the study arise because of improper access to shared static
variables.}} Among \dtnum dependent tests, \svnum (\svratio\%) of them
arise due to inappropriate access to
shared static variables; 4 (4\%) of them arise
due to inappropriate access to the file system, and 10 (10\%) of them arise
due to inappropriate access to a database.
The root cause for the remaining 23 (25\%) tests is not apparent
in the issue tracking system.

\vspace{1mm}
\noindent \textbf{{Developer actions: dependent tests
often indicate flaws in the test code, and developers usually
modify the test code to remove them.}}
% In some cases, dependent tests are intentional and developers
% document them, but in other cases they are
% inadvertent.
Among \dtnum dependent tests, developers considered 
94 (98\%) to be major or minor problems, and the 
developers' discussions showed that the developers thought that the test
dependence should be removed.
Nonetheless, developers fixed only 38 (40\%) of the \dtnum dependent tests.
Another 49 (51\%) were ``fixed'' 
by adding comments to the test code to document the existing dependence.
For the remaining 9 (9\%) unfixed tests,
developers thought they were not important enough given the limited
development time, so they simply closed the issue report without taking
any action.

%% Do we have evidence of this?  I'm inclined to omit the discussion.  -MDE
% The primary reason is that
% % developers often \textit{intentionally} introduce test dependence because 
% it is
% easier and more convenient to write the test code. 

%There is no statistical
%significant correlation between severity and fixing.
A dependent test usually
reveals a flaw in the test code rather than the program code:
only 16\% of the code fixes (6 out of 38) are
on the program code.
In all 6 cases, the developers changed
code that performs static variable initialization, which ensures that
each dependent test will not read an undesired value.
Section~\ref{sec:repercussion} gives an example.
The other 32 code fixes were in the test code:
28 (87\%) of the dependent tests were fixed by manually specifying
the test execution order in a test script or a configuration file,
3 (10\%) of them were simply deleted by developers
from the test suite, and the remaining 1 (3\%) test was merged with its
initializing test.


%Second, many popular testing
%frameworks such as JUnit does not support to explicitly specify
%test dependence in the test code\footnote{In fact, the execution order of
%JUnit tests depends on the underlying JVM implementation~\cite{junitordering}}.
%It is non-trivial 



%Test dependence can cause problems, not only
%when test suites are reordered, but even when they are
%executed in the intended order.



\subsubsection{Manifestation of Dependent Tests}
\label{sec:repercussion}

\begin{table}
\centering
\setlength{\tabcolsep}{0.15\tabcolsep}
\begin{tabular}{|c||c|c|}
%\toprule
\hline
\textbf{Issue Tracking} & \textbf{Revealing Weakness} & \textbf{Masking Faults} \\
\textbf{System } & \textbf{in a Test Suite} & \textbf{in a Program} \\
\hline
Apache &24 & 2 \\
\hline
Eclipse & 59 & 0 \\
\hline
JBoss& 6 & 0 \\
\hline
Hibernate & 3 & 0 \\
\hline
Codehaus & 2 & 0 \\
\hline
\hline
\textbf{Total}  & 94 & 2 \\
\hline
\end{tabular}
\caption{
Classification of the collected dependent tests
based on its repercussions.
}
\label{tab:reper}
\end{table}


A dependent test may manifest as a false alarm or a missed alarm
(Table~\ref{tab:reper}).

\vspace{1mm}

\noindent \textbf{False alarm.} Most of
the dependent tests (94 out of \dtnum) 
result in false alarms:
%indicate a weakness in the test suite rather than the
%tested code:  
the test should pass but fails after reordering due to the dependence.
The test dependence arises due to incorrect initialization
of program state by one
or more tests. Typically, one test initializes
a global variable or the execution environment, and another
test does not perform any initialization, but
relies on the program state after the first test's execution.
Such dependence in the test code is often masked because
the initializing test always executes before other tests in the
default execution order. The dependent tests are not revealed
until the initializing test is reordered to execute
after other tests. 
%In this category, the test dependency
%is introduced \textit{unintentionally} by developers. 
%the default test execution order includes tests that initialize the library.  The defect is
%inconsequential until and unless the flawed test is reordered, either manually or by
%a downstream tool, to execute before any other initializing test.

%\vspace{1mm}

Sometimes developers introduce dependent tests intentionally because it is
more efficient or convenient~\cite{kapfhammeretal:FSE:2003, whittakeretal:2012}.
%DB-testing}.
Even though the developers are aware of these dependences
when they create tests, this knowledge can get lost.
Other people who are not aware of these dependences can get confused 
when they run a subset of the test suite that manifests the
dependent tests, and might report bugs about the failing tests,
even though this is exactly the intended behavior. 
If the dependence is not documented clearly and
correctly, it can take a considerable amount of time to work out that
these reported failures are spurious. 
The Eclipse issue tracking system contains at least
49 such dependent tests.
%Or worse, the developers may try
%to fix a bug that is not there.
In September 2003, a user filed a
bug report in SWT~\cite{swt}~\cite{eclipsebug},
stating that 49 tests were failing unexpectedly
if she ran any other test before \code{TestDisplay} --- 
a test suite creates a new \code{Display} object and tests it.
However, this bug report was spurious and was
caused by undocumented test dependence.
All 49 failing tests are dependent tests with the same
root cause: in SWT, only one global \code{Display}
object is allowed; the user ran tests that
create but do not dispose of a \code{Display} object, while
the tests in \code{TestDisplay} attempt to create
a new \code{Display} object, which fails, as one
is already created. This is the desired behavior of SWT,
and points to a weakness in the test suite.
% rather
%than the code.

\vspace{1mm}

\noindent \textbf{Missed alarm}. In rare cases,
dependent tests can hide a fault in the
program, \emph{exactly} when the test suite is executed in its default
order. Masking occurs when a test case $t$ \emph{should}
reveal a fault, but tests executed before $t$ in a test suite always
generate environments in which $t$ passes accidently and
does not reveal the fault. 
Tests in this category result in \textit{missed alarms} ---
a test should fail but passes due to the dependence.


\begin{figure}[t]
%\noindent \textbf{\small{Fault-related code in CLI:}}
%\vspace{-2mm}
\begin{CodeOut}
\begin{alltt} 
public final class OptionBuilder \{
  \textbf{private static String argName = null;}
  private static void reset() \{
    ...
    \textbf{argName = "arg";}
    ...
  \}
\}
\end{alltt}
\end{CodeOut}
\vspace*{-15pt}
\caption{Simplified fault-related code in CLI~\cite{cli} (revision 661513).
The fault was masked by two dependent tests for over 3 years.
}
\label{fig:option_builder}
\end{figure}

%  LocalWords:  OptionBuilder argName arg CLI


We found two such dependent tests in
the Apache CLI library~\cite{cli}.
Figure~\ref{fig:option_builder} shows the simplified fault-related
code. The fault is due to improper initialization of the static variable
\CodeIn{argName}. The static variable \CodeIn{argName} should be set
to its default value \CodeIn{"arg"} by CLI's clients via calling
method \CodeIn{reset()}. Otherwise, \CodeIn{argName}'s
default value remains \CodeIn{null} and should \emph{not} be
used in creating an \CodeIn{OptionBuilder} object.
In CLI, two test cases 
\code{Bugs\-Test.test13666} and \code{Bugs\-Test.test27635}
can reveal this potential fault by directly initializing
a \CodeIn{OptionBuilder} object without calling \CodeIn{reset()}.
These two tests fail when run in isolation,
but both pass when run in the default order. This is because
in the default order, tests running \emph{before} these
two tests call \CodeIn{reset()} at least once, which sets
the value of \CodeIn{argName} and masks the fault.

%Both dependent tests can reveal this fault,  but
%the default order of test execution makes both tests pass
%accidentally. 

Such dependent tests have a non-trivial impact in practice.
This fault was reported in the bug database several times~\cite{clibug},
starting on March 13, 2004 (CLI-26). The report was marked as resolved
\emph{three years} later on March 15, 2007 when developers
realized the test dependence. The developers fixed this
fault by adding a static initialization block which
calls \CodeIn{reset()} in class \CodeIn{OptionBuilder}.

%\edit{where should we emphasize that masking faults is an
%orthognonal issue of fixing the dependence on code or test?}


\subsubsection{Implications for Dependent Test Detection}

We summarize the main implications of our findings.

\noindent \textbf{{Dependent tests exist in practice, but
they are not easy to identify.}}
None of the dependent tests we studied can be identified by
running the existing test suite in the default order. 
Every dependent test was reported when the
test suite was reordered, either accidentally by a user or
by a testing tool. This indicates the need
for a tool to detect dependent tests.
%dependent test detection techniques should
%explicitly search for such dependent tests.

\vspace{1mm}
\noindent \textbf{Dependent test detection techniques
can bound the search space to a small number of tests.}
In theory, a technique needs to exhaustively execute
all $n!$ permutations of a $n$-sized
test suite to detect all dependent tests. This is
not feasible for realistic $n$.  Our study shows that
most dependent tests can be manifested by executing
no more than 2 tests together. Thus, a practical technique
can focus on running only short subsequences (whose
length is bounded by a parameter $k$)
of a test suite. This will reduce the number of permutations
to $O(n^k)$, which is tractable for small $k$ and $n$.

\vspace{1mm}
\noindent \textbf{Dependent test detection techniques
should focus on analyzing accesses to global variables.}
Dependent tests can result from many
interactions with the execution environment, including
global variables, file systems, databases, network, etc.
However, as reflected by our study, more than half of the
real-world dependent tests are caused
by improper static variable accesses. This implies that a technique
may achieve a high return by focusing on global variables.


%\vspace{1mm}
%\noindent \textbf{Dependent test fixing tool
%Test dependence reveals flaws in the test code.}
%This indicates that a potential dependent test fixing tool should target
%the test code


\subsection{Threats to validity}

Our findings apply in the context of our study and methodology and may not
apply to arbitrary programs.
The applications we studied are all written in 
Java and have JUnit test suites.  

We accepted the developers' judgment regarding which tests are dependent,
the severity of each dependent test, and how many tests are needed
to manifest the dependence.  We did not intentionally ignore
any test dependence in the issue tracking system.
However, a limitation is that the developers might have made a mistake,
might not have marked a test dependence in a way we found it
(different search terms might discover additional dependent tests), and are
unlikely to have found all the dependent tests in those projects. 


%  LocalWords:  JBoss Codehaus reproducibility multi dependences SWT CLI
%  LocalWords:  TestDisplay test13666 subsequences subsuite



\section{Formalizing Test Dependence}
\label{sec:formalism}


The result of a test not only depends on
its input data but also its \emph{execution conditions}.
To characterize the relevant execution conditions, 
our formalism represents
(a)~the order in which test cases are executed and (b)~the environment in which a test suite is executed.  


\subsection{Definitions}
\label{sec:definitions}

We express test dependences through the results of executing
\emph{ordered} sequences of tests in a given \emph{environment}.


\begin{definition}[Environment]
An \emph{environment} \env for the execution of a test
consists of all values of global variables, files,
operating
system services, etc. that
can be accessed by the test and program code exercised by the test
case.
\end{definition}

We use $\env_0$ to represent the initial environment, such
as a fresh JVM initialized by frameworks like JUnit
before executing any test.


\begin{definition}[Test]

A test is a sequence of executable program statements, and an oracle
--- a Boolean predicate that
decides whether the test passes or fails.
\end{definition}

For simplicity, our definition does not consider non-deter\-min\-istic
tests, non-terminating tests, and tests aborting the JVM.


\begin{definition}[Test Suite]
A test suite\/ $T$ is an $n$-tuple (i.e., ordered sequence) of tests
\suite{t_1, t_2, \dots, t_n}.

%When it is clear which test suite we are talking about, or the details
%of the suite are not important, we use $T$ to denote the entire test
%suite $(t_1, \dots, t_n)$.
\end{definition}


\begin{definition}[Test Execution]
Let\/ \alltests\ be the set of all possible
tests and\/ \environs\ the set of all possible
environments.
The function\/ $\execfunc: \alltests \times \environs \rightarrow
\environs$ is called test
execution. $\execfunc$ maps the execution of a test\/ $ t \in
\alltests$ 
in an environment\/ $\env \in \environs$ to a new (potentially updated)
environment\/ $\env' \in \environs$.

Given a test suite\/ $T = \suite{t_1, t_2, \dots, t_n}$,
we use the shorthand\/
$\exec{T}{\env}$ for $\exec{t_n}{\exec{t_{n-1}}{{\dots \exec{t_1}
{\env} \dots }}}$, to represent its execution.
\end{definition}

\begin{definition}[Test Result]
The result of a test $t$ executed in an environment\/ $\env$,
denoted\/ \result{t}{\env}, is defined by the test's oracle
and is either \pass or \fail.

The result of a test suite\/ $T$ = \suite{t_1,\dots,t_n}, executed in an
environment\/ \env, denoted\/ \result{\suite{t_1,\dots,t_n}}{\env}, is a
sequence of results\/ \suite{o_1,\dots,o_n} with $o_i \in \{\pass,\fail\}$.
We use \result{T}{\env}[$t$] to denote the result of a test $t \in T$.

%For test outcomes of sequences where all individual outcomes are
%either \pass or \fail, we use the notation $(\pass^*)$ and $(\fail^*)$,
%respectively.

For example, $\result{\suite{t_1, t_2}}{\env_1} = \suite{\fail, \pass}$ represents that if
$t_1$ then $t_2$ are run, starting with the environment\/ $\env_1$, then\/
$t_1$ fails and\/ $t_2$ passes.
\end{definition}

A manifest order-dependent test (for short, dependent test)
is one that can be exposed by 
reordering existing test cases.
A dependent test $t$ manifests only
if there are two test suites $S_1$ and $S_2$ which
are two permutations of the original test suite $T$,
in which $t$ exhibits a different result
in the execution $\exec{S_1}{\env_0}$
than in the execution $\exec{S_2}{\env_0}$.
%produces an
%environment that will make $t$ exhibit a
%different result than in the environment produced by $\exec{S_2}{\env_0}$.

\begin{definition}[Manifest Order-Dependent Test] \label{def:manifest}
Given a test suite\/ $T$, a test $t \in T$ is a
manifest order-dependent test if $\exists$ two test suites
$S_1, S_2 \in$ permutations($T$):
\result{S_1}{\env_0}[$t$] $\neq$
\result{S_2}{\env_0}[$t$].
%
%${S_1 \subseteq T - t}$
%and ${S_2 \subseteq T - t}$: \result{t}{f(S_1, \env_0)} $\neq$
%\result{t}{f(S_2, \env_0)}.
%\footnote{$S \subseteq T - t$ represents that $S$ does not
%contain repetitive tests, and for each
%test $t' \in S$ such that $t' \in T$ and $t' \neq t$.}
\end{definition}

It would be possible to consider a test dependent if reordering could
affect any internal computation or heap value (non-manifest dependence);
but these internal details, such as order of elements in a hash table,
might never affect any test result: they could be false dependences.
Another alternative would be to ask
whether it is possible to write a new dependent test for an existing
test suite; but the answer to this question is trivially ``yes''.
This paper focuses on manifest dependence and works with real, existing
test suites to determine the practical impact and prevalence of dependent
tests.


\subsection{The Dependent Test Detection Problem}

We prove that the problem of detecting dependent tests
 is NP-complete.


\begin{definition}[Dependent Test Detection Problem]
Given a set suite\/ $T = \suite{t_1, \dots, t_n}$ and an initial environment\/
$\env_0$, is $t \in T$ a dependent test for $T$?
%there a test suite\/ $S
%\subseteq T$ that manifests a test dependence involving\/ $t_i$? 
\end{definition}

We prove that this problem is NP-hard by reducing the NP-complete Exact Cover problem
to the Dependent Test Detection
problem~\cite{karp:NP:1972}. 
Then we provide a linear-time algorithm to verify any answer to the
question.
%Then we sketch an exponential
%time algorithm that can solve the problem.
Together these two parts prove that the Dependent Test Detection Problem is NP-complete.

\begin{theorem}
The problem of determining whether a test is a dependent test for
a test suite is NP-hard.
\end{theorem}

\begin{proof}
Due to space limits, we omit the NP-hard proof. Interested
readers can refer to~\cite{testdependence} for details.
\end{proof}

%

\begin{proof}
%We prove this claim by reducing Exact Cover to Dependent Test
%Detection.
In the Exact Cover problem,
we are given a set $X$ = \{$x_1, x_2, x_3, \dots, x_m$\} and a collection $S$ of subsets of $X$.
The goal is to identify a sub-collection $S^*$ of $S$ such that \textit{each}
element in $X$ is contained in \textit{exactly} one subset in $S^*$.  

Assume a set $V = \{v_1, v_2, v_3, \dots, v_m\}$ of variables,
and a set $S = \{S_1, S_2, \dots, S_n\}$ with $S_i \subseteq V$ for $ 1\leq i
\leq n$. 

We now construct a tested program $P$, and a test suite
$T = \suite{t_1, t_2, \dots t_n , t_{n+1}}$ as follows:

\begin{itemize}

\item $P$ consists of $m$ global variables 
$v_1, v_2,\dots, v_m$, each with initial value 1.

\item 
For $1 \le i \le n$, $t_i$ is constructed as follows:
for $1 \le j \le m$, if $v_j \in S_i$, then add a
single assignment statement \CodeIn{$v_j$ = $v_j$ - 1} to $t_i$.

$t_{n+1}$ consists only of the oracle
\CodeIn{assert($v_1$ != 0 || $v_2$ != 0 \dots || $v_m$ !=0)}.

\end{itemize}

In the above construction, the tests $t_i$ for $1 \le i \le n$ 
will always pass. The only
test that may fail and thus exhibit different behavior is $t_{n+1}$, which 
\emph{only} fails when each variable $v_i$ appears exactly
once in a test case.

For the given test $t_{n+1}$, if we can
find a sequence \suite{t_{i_1}, t_{i_2},\dots, t_{i_j}}
that makes $t_{n+1}$ fail, the subsets $S^*$ corresponding
to each $t_{i_j}$ are an exact cover of $V$.

In practice, the structure of the proof directly translates to the
structure of test suites. $t_{n+1}$ is the dependent test, $S$ is
defined by the tests that write variables used by $t_{n+1}$, and every
exact cover of $S$ represents an independent shortest test suite that
is a manifest dependency of $t_{n+1}$.
\end{proof}

To complete the proof that Dependent Test Detection is NP-complete, we
provide an algorithm to verify a solution to the problem, that is
linear in the size of the test suite.
Given a test suite $T$, a test $t \in T$ and a sequence
$S \subseteq T$ that manifests a dependency on $t$, we first execute $T$, then $S$, and
compare the result for $t$ in both executions. 
If the results differ the solution is correct; if they do not differ,
the solution is rejected.
Since in the worst case we have to execute $2n$ tests, the complexity
of this algorithm is linear. 


\subsection{Discussion}
\label{sec:formaldiscussion}

For the sake of simplicity, our formalism only
considers deterministic tests,
and excludes tests whose results might be affected by
non-determinism such as thread scheduling
and timing issues.
Our formalism excludes self-dependence, 
when executing the same test twice
may lead to different results. Our empirical study
indicates that self-dependent tests
are rare in practice. In addition, typical
downstream testing techniques such as test selection and
prioritization do not usually execute a test twice within the same JVM.


% vim:wrap:wm=8:bs=2:expandtab:ts=4:tw=70:

%  LocalWords:  dependences tuple



\section{Detecting Dependent Tests}
\label{sec:detecting}

\newcommand{\smalltrialnum}{10\xspace}
\newcommand{\mediumtrialnum}{100\xspace}
\newcommand{\trialnum}{1000\xspace}

\newcommand{\testlist}[0]{\ensuremath{T^k_i}}
\newcommand{\executeTestsInOrder}[1]{\result{#1}{\env_0}}

\begin{figure}[t]
\textbf{Input}: a test suite $\mathit{T}$\\
\textbf{Output}: a set of dependent tests $\mathit{dependentTests}$\\
\vspace{-5mm}
\begin{algorithmic}[1]
\STATE $\mathit{dependentTests}$ $\leftarrow$ $\emptyset$
\STATE $\mathit{expectedResults}$ $\leftarrow$ $\result{T}{\env_0}$
\FOR{each $\mathit{ts}$ in getPossibleExecOrder($\mathit{T}$)}
\STATE $\mathit{execResults}$ $\leftarrow$ $\result{ts}{\env_0}$
\FOR{each test $\mathit{t}$ in $\mathit{ts}$}
\IF{$\mathit{execResults}$[$\mathit{t}$] $\neq$ $\mathit{expectedResults}$[$\mathit{t}$]}
\STATE $\mathit{dependentTests}$ $\leftarrow$ $\mathit{dependentTests}$ $\cup$ $\mathit{t}$
\ENDIF
\ENDFOR
\ENDFOR
\RETURN $\mathit{dependentTests}$
%\ENDWHILE
\end{algorithmic}

% getPossibleExecOrder($T$, $k$): returns a set of test suites, each of size
% $\le k$; each suite is composed of tests selected from $T$ without replacement.\\

\vspace{-3mm}
\caption {The algorithm to detect dependent tests.
The getPossibleExecOrder function is instantiated
by different algorithms in Figures~\ref{fig:randalgorithm},
~\ref{fig:exhaustivealgorithm}, and~\ref{fig:impralg}.
}
\label{fig:basealgorithm}
\end{figure}


\begin{figure}[t]
getPossibleExecOrder($T$):\\
\vspace{-5mm}
\begin{algorithmic}[1]
\FOR{$i$ in 1..$\mathit{numtrials}$}
\STATE \textbf{yield} shuffle($T$)
\ENDFOR
\end{algorithmic}

\vspace{-3mm}
\caption {The randomized algorithm to detect dependent tests.
It uses the algorithm of Figure~\ref{fig:basealgorithm}, re-defining
the getPossibleExecOrder function.
Our experiments use $\mathit{numtrials} = \smalltrialnum,
\mediumtrialnum, \trialnum$.}
\label{fig:randalgorithm}
\end{figure}

An exhaustive search would execute all $n!$
permutations of the test suite to detect dependent tests.
However, this is not feasible for realistic $n$.
Since the general form of the dependent test detection problem is
NP-complete, we do not expect to find an efficient algorithm for it.

To approximate the exact solution, this section
presents three approximate algorithms that find a \textit{subset} of
all dependent tests.
%In this section we present two algorithms to detect dependent
%tests. 
Section~\ref{sec:randomized} describes a randomized algorithm
that repeatedly executes all the tests of a suite in random order.
Section~\ref{sec:basic} describes an exhaustive bounded algorithm that
executes all possible sequences of $k$ tests for a bounding parameter $k$
(specified by users).
Section~\ref{sec:advalgorithm} describes a \dependenceaware{} $k-$bounded algorithm.
The \dependenceaware{} algorithm dynamically collects the fields that each test
reads or writes, and uses such collected information to reduce the search space.
All algorithms are \textit{sound} but \textit{incomplete}:
every dependent test they find is real, but they do not
any guarantee to find every dependent test.

\subsection{Randomized Algorithm}
\label{sec:randomized}

Figure~\ref{fig:randalgorithm} shows the algorithm.
Given a test suite $T = \suite{t_1, t_2, \ldots, t_n}$, this algorithm
first executes $T$ with its default order
to obtain the \emph{expected result} of each test (line 2).
Then, it randomizes the original
test execution order (line 3), and then executes each test
again to observe its result (line 4). The algorithm checks
whether the result of any test differs from the
expected result (lines 5--8). 

In our implementation, this algorithm repeatedly
shuffles the test suite, and executes each test until
no more dependent tests are identified within a
pre-defined number of iterations. % (default: \smalltrialnum iterations).
Our experiments use $\smalltrialnum,
\mediumtrialnum, \trialnum$ as the iteration numbers.

%\todo{How did we choose 10 iterations?  It might be good to try with some
%  different number such as 20 and report that there was no difference.
%  More generally, when choosing an arbitrary number, it adds credibility to
%  either give a justification for the number or to do an experiment to show
%  that the number is a reasonable choice.}


\subsection{Exhaustive Bounded Algorithm}
\label{sec:basic}

\begin{figure}[t]
\textbf{Auxiliary methods}:\\
kPermutations($T$, $k$): returns all $k$-permutations of $T$; that is, all
sequences of $k$ elements of $T$ without repetition

\medskip

getPossibleExecOrder($T$):\\
\vspace{-5mm}
\begin{algorithmic}[1]
\RETURN kPermutations($T$, $k$)
\end{algorithmic}

\vspace{-3mm}
\caption {The exhaustive $k$-bounded algorithm to detect dependent tests.
It uses the algorithm of Figure~\ref{fig:basealgorithm}, re-defining the
getPossibleExecOrder function.
%Our experiments use $k=1$ and $k=2$ to bound the length of
%test execution. 
} 
\label{fig:exhaustivealgorithm}
\end{figure}


%To detect all possible dependent tests, 

This algorithm uses the findings of our study
(Section~\ref{sec:study})
that most dependent tests can be found by running only short
subsequences of test suites. For example,
in our study, \pertange of the real-world dependent tests
can be found by running no more than 2 tests together.
Instead of executing all permutations of the
whole test suite, our algorithm (Figure~\ref{fig:exhaustivealgorithm})
executes all $k$-permutations for a bounding
parameter $k$.
By doing so, the algorithm reduces
the number of permutations to execute
to $O(n^k)$, which for small $k$ is tractable. 


Figure~\ref{fig:exhaustivealgorithm} shows the algorithm.
Given a test suite $T = \suite{t_1, t_2, \ldots, t_n}$, the
algorithm executes every $k$-permutations of tests,
and checks whether the result of any test differs
from the expected result (lines 3--10). Finally, the algorithm returns the set
of all tests $t_i \in T$
that exhibit different results.





\subsection{Dependence-Aware Bounded Algorithm}
\label{sec:advalgorithm}

%\begin{figure}
%\centering
%
%\strut \hspace{-25mm} Initially: \code{x = y = 1;}
%
%\subfigure{
%\begin{minipage}{.43\columnwidth}
%\code{Test1 \\  if(x == 0) \{ y = 0;\}\\ }
%\end{minipage}
%}
%\subfigure{
%\begin{minipage}{.43\columnwidth}
%\code{Test2\\ assert y == 1;}
%\vspace{0.9em}
%\end{minipage}
%}
%\subfigure{
%\begin{minipage}{.43\columnwidth}
%\code{Test3\\ x = 0;}
%
%\vspace{0.5em}
%\end{minipage}
%}
%\subfigure{
%\begin{minipage}{.43\columnwidth}
%\code{Test4 \\  if(x == 0) \{ y = 1;\}\\ }
%\end{minipage}
%}
%
%\strut \hspace{-3mm} Field read and write information by each test when executing
%in the order of Test1, Test2, Test3, and Test4.
%\vspace{1mm}
%
%\begin{tabular}{|c|l|l|l|l|}
%\toprule
%\hline
%\textbf{Global Fields } & \textbf{Test1} & \textbf{Test2} & \textbf{Test3}& \textbf{Test4}\\
%\hline
%\code{x} & Read & & Write& Read\\
%\hline
%\code{y} & & Read & &Write \\
%\hline
%\end{tabular}
%
%\vspace{4mm}
%
%Number of permutations that need to be executed.
%\setlength{\tabcolsep}{1.3\tabcolsep}
%\begin{tabular}{|c|c|c|}
%\toprule
%\hline
%\textbf{$k$ value} & \textbf{Exhaustive } &
%\textbf{Dependence-Aware }  \\
%\hline
%1 & 4 & 1\\
%\hline
%2 & 12 & 5\\
%\hline
%3 & 24 & 18 \\
%\hline
%4 & 24 & 20 \\
%\hline
%\hline
%\textbf{Total} & 64 & 44 \\
%\hline
%\end{tabular}
%
%
%\Caption{Example tests to illustrate the \dependenceaware{}
%$k$-bounded dependent test detection algorithm (Figure~\ref{fig:impralg}).}
%\label{fig:rwexample}
%\end{figure}

%The exhaustive $k-$bounded algorithm cannot scale to a realistic test suite.
%For example, it would take months to 
%execute all 3-permutations
%in Joda-Time's test suite (3875 tests, Table~\ref{tab:subjects}).

The \dependenceaware{} $k$-bounded algorithm
detects the same number of dependent tests
as the exhaustive $k$-bounded algorithm does (when using the same $k$),
but it uses dynamic analyses to improve the efficiency.
%Information obtained from shorter permutations is used to avoid running lonnger permutations; 
The intuition of this algorithm is that, given
a test permutation, for \textit{each} test,
if \textit{every global field} (and other external
resources like a {file}) it reads
is \textit{not} written by any test executed \textit{before} it,
all tests in the permutation are guranteed to produce
the same results as in the default order (since each
test does not interfere another). Thus, the permutation
can be safely ignored.

%a permutation can safely be ignored if all global fields a test accesses
%5are the same as a previously executed permutation. 


Figure~\ref{fig:impralg} shows the algorithm, which redefines
the getPossibleExecOrder function in Figure~\ref{fig:basealgorithm}.
The redefined getPossibleExecOrder function first records
fields that each test reads and writes when executed in \textit{isolation}
(line 1).  When generating all possible test permutations
up to length $k$, the algorithm checks whether
\textit{all} global fields \textit{each} test (in the generated permutation)
may read are not written by any test executed before it (lines 6--10).
If so, all tests in the permutation
must produce the same results as executed in isolation,
and the algorithm can safely discard this permutation without
executing it. Otherwise, the algorithm adds the generated
permutation to the result set (line 9), and runs the algorithm in Figure~\ref{fig:basealgorithm}
to identify dependent tests. %among it.
Due to space limits, we omit the proof of the correctness
of the \dependenceaware{} $k$-bounded algorithm. Interested
readers can refer to~\cite{proof-dependence-aware} for proof
and examples.

The \dependenceaware{} $k$-bounded algorithm has two major benefits.
First, it clusters tests by the fields they
read and write. Only tests reading or writing
the same global field(s), rather than \textit{all} tests
in a suite, are treated as potentially dependent.
Second, for tests reading or writing the same global
field(s), some permutations can be ignored by checking
the global fields each test may access. 

\begin{figure}[t]
\textbf{Auxiliary methods}:\\
recordFieldAccess($\mathit{t}$): executes a test $\mathit{t}$ in
a fresh JVM and returns the fields it reads and writes.\\
%by a test $t$. \\% in an ordered sequence of tests $ts$. \\


\vspace{-2mm}

getPossibleExecOrder($T$):\\
\vspace{-5mm}
\begin{algorithmic}[1]
\FOR { each $\mathit{t}$ in $\mathit{T}$ }
\STATE $\langle \mathit{reads}_t, \mathit{writes}_t\rangle$ $\leftarrow$ recordFieldAccess($\mathit{t}$)\\ 
\ENDFOR

\STATE $\mathit{result}$ $\leftarrow$ $\emptyset$ \\
\FOR{each $\mathit{ts}$ in kPermutations($\mathit{T}$, $\mathit{k}$)}
	\FOR {each $\mathit{t_i}$ in $\mathit{ts}$ \{ $\mathit{i}$ is the index of $\mathit{t_i}$ in $\mathit{ts}$ \}} 
		\STATE $\mathit{previousWrites}$ $\leftarrow$ $\bigcup_{\mathit{j} < \mathit{i}}  \mathit{writes}_{t_j} $ \\
		\IF {$\mathit{previousWrites} \cap \mathit{reads}_{t_i} \neq \emptyset$}
			\STATE $\mathit{result} \leftarrow \mathit{results} \cup \mathit{ts}$
		\ENDIF
	\ENDFOR

\ENDFOR
\RETURN $\mathit{result}$
\end{algorithmic}

\vspace{-3mm}
\caption {The \dependenceaware{} $k$-bounded algorithm to detect dependent tests.
It uses the algorithm of Figure~\ref{fig:basealgorithm}, re-defining the
getPossibleExecOrder function.
%Our experiments uses $k=1$, \todo{experiment setting, how
%large k it can scale to.}. 
} 
\label{fig:impralg}
\end{figure}


%in which every test must
%reveal the same result to reduce the total number
%of permutations which need to be explored.

%To better illustrate the second benefit, Figure~\ref{fig:rwexample}
%shows an example
%consisting of 2 global fields and 4 tests. In the default execution
%order, Test 1 reads the \textit{initial} value of \code{x};
%Test2 reads the \textit{initial} value of \code{y}, and Test4
%reads the value of \code{x} written by Test3.
%Based on such recorded information, when creating permutations with $k=1$,
%our algorithm determines that only
%Test4 needs to be executed, since Test4, if executed in isolation,
%will read the \textit{initial} value of \code{x} rather than the \code{x}
%value written by Test3. Similarly, when creating permutations with $k=2$,
%only 5 out of 12 permutations ($\langle$Test1, Test4$\rangle$,
%$\langle$Test2, Test4$\rangle$, $\langle$Test4, Test1$\rangle$,
%$\langle$Test4, Test2$\rangle$, and $\langle$Test4, Test3$\rangle$) need to be executed.
%Due to space limits, we omit test permutations when
%$k=3$ and $k=4$, and list the number of permutations that need
%to execute in Figure~\ref{fig:rwexample}.

%\todo{illustrate why simply group tests by read/write is incorrect.}


%\todo{give an intuition of why it can improve scalability, since
%not every test will overlap with each other.}



%In case a test $t_2$ does not potentially \todo{?} depend on a previously executed test $t_1$, this means 
%that all permutations of tests in which $t_2$ follows $t_1$ do not need to be considered for potential \todo{?} dependency
%given that all tests on which $t_1$ potentially \todo{?} depends on are present as a prefix of that permutation, 
%in the same order that the test suite was initially executed, thus reducing the number of permutations which
%need to be explored.


%\begin{comment}



%  LocalWords:  getPossibleExecOrder numtrials lastWriteTS previousWrites
%  LocalWords:  lastWrites


\section{Tool Implementation}
\label{sec:impl}


We implemented our three dependent test detection algorithms in
a prototype tool, called \ourtool. \ourtool
supports JUnit 3.x/4.x tests. %, and is fully-automated.

To ensure there is no interaction between
different runs, \ourtool launches a fresh JVM
when executing a test permutation, and after a run it resets resources,
such as deleting any temporary files that were created.
When comparing the observed result of
a test in a permutation with its expected result,
\ourtool considers two JUnit test results to be the same when the
tests either both pass, or exhibit exactly the same exception
(from the same line of code) or assertion violation.

To implement the \dependenceaware{} $k$-bounded
algorithm, \ourtool uses ASM~\cite{asm} to perform load-time bytecode
instrumentation. \ourtool inserts code to monitor each
static field access (including read and write), and
monitors each file access by
installing a Java \code{SecurityManager} which provides 
file-level read/write information.
Each test produces a trace file containing both
field and file access information, after being executed
on a \ourtool-instrumented program.
%Both field and file access information are recorded
%in a trace file after executing a \ourtool-instrumented test.

\ourtool does not perform any sophisticated points-to or shape
analyses. It uses the side-effect annotations
provided by Javari~\cite{QuinonezTE2008} to determine the immutable
classes.  Then, it conservatively treats both read
and write to a mutable static field as a write effect,
since a read access to a static field may mutate objects
reachable from the field in the heap.
\ourtool assumes that the JDK is stateless,
and thus does not track field access in JDK classes. 


%\ourtool does not instrument JDK classes.
%It uses the side-effect annotations
%provided by the Javarifier~\cite{QuinonezTE2008} tool to determine whether
%an object will be mutated or not. If a test uses an object
%as a mutable reference in a Javarifier-annotated JDK call, \ourtool
%treats the test has a write effect to the object.
%For JDK classes not covered by
%Javarifier's annotations, \ourtool assumes these classes (and their object
%instances) are stateless. 

Optionally, users can also specify a
list of ``dependence-free'' fields (e.g., a static field for
logging or counting), which can never be the root cause of manifest test dependence.
\ourtool ignores accesses to these fields.

%In order to make the problem tractable, it was assumed that objects reachable from the static fields do not alias one another. 

%Mutable and immutable objects must be treated differently, as even a ``read'' access to a static
%field will provide access to all the heap reachable from the object stored in that field. Such accesses must be treated as a read followed by a write. This issue seriously impacts the effectiveness of the dependency-aware algorithm. Classes known to only have immutable instances were provided as input to \ourtool. The instrumented Javari JDK provided by the Checker framework \cite{??}, classes marked as immutable by developer comments, and manual inspection of widely used classes (e.g., \code{java.awt.Color}, \code{java.util.Locale}) are the source of this information.

%In practice many objects are not mutated even though this is not prohibited by the Java language (e.g., primitive arrays being used as constants). The algorithm was also instructed to ignore such fields, determined by manually inspecting fields which followed the Java naming convention of all-capital letters for constants, and fields commonly referenced by test code. 


%This is important since in presence of aliasing, writing to one static field may effect others. This was not an issue in the subject programs we explored.



%The other assumption made was that the JDK does not contain state itself, as the JDK was not instrumented. This is not entirely true and was an issue in \todo{explain this issue}. 


%\ourtool also instruments
%calls to the \CodeIn{clone()} method and reflection
%methods, to record the objects they may create.
%\edit{a few sentences about security manager, and file accesses here}

%In addition, \ourtool implements the Delta debugging
%algorithm~\cite{Zeller:2002} to
%return the shortest sequence of tests that manifest a dependent test.
%This is useful to understand
%a manifested dependent test in a shuffled test suite.

The source code of \ourtool is available at:\\ \url{http://testisolation.googlecode.com}.

% vim:wrap:wm=8:bs=2:expandtab:ts=4:tw=70:

%\todo{one improvement space: distinguish different values}

%\todo{I prefer to talk about file systems, optimizations
%that use user-provided annotations, the prefix of a permutation,
%and use multiple executions to
%guide selection in the implementation section.}

%  LocalWords:  ASM tradeoff


\section{Empirical Evaluation}
\label{sec:evaluation}


\newcommand{\jt}{Joda-Time\xspace}

\newcommand{\jfreecharttests}{2234\xspace}%change the total num
\newcommand{\jodatimetests}{3875\xspace}
\newcommand{\xmlsecuritytests}{108\xspace}
\newcommand{\crystaltests}{75\xspace}
\newcommand{\synoptictests}{118\xspace}
\newcommand{\totaltests}{4176\xspace}

\newcommand{\jfreechartautotests}{2946\xspace}
\newcommand{\jodatimeautotests}{2639\xspace}
\newcommand{\xmlsecurityautotests}{665\xspace}
\newcommand{\crystalautotests}{3198\xspace}
\newcommand{\synopticautotests}{2467\xspace}
\newcommand{\totalautotests}{8969\xspace}


\begin{table}
\centering
\setlength{\tabcolsep}{0.4\tabcolsep}
\begin{tabular}{|l|l|c|c|l|}
%\toprule
\hline
\textbf{Program} & \textbf{LOC} & \textbf{\#Tests} & \textbf{\#Auto Tests} & \textbf{Revision}
\\
\hline
%\midrule
JodaTime & 27183 & \jodatimetests
% 3875 is retrieved by running mvn test on the related revision
& \jodatimeautotests&  b609d7d66d\\
XML Security & 18302 & \xmlsecuritytests & \xmlsecurityautotests& version 1.0.4 \\ 
Crystal & 4676 & \crystaltests & \crystalautotests& trunk version\\
Synoptic & 28872 & \synoptictests & \synopticautotests&  trunk version\\ 
%\bottomrule
\hline
%\textbf{Total}& &  & &  \\ 
%\hline
\end{tabular}
\caption{Subject programs used in our evaluation.
Column ``\#Tests'' shows the number of human-written
unit tests associated with each program. Column
``\#Auto Tests'' shows the number of automatically-generated
unit tests for each program, by Randoop~\cite{PachecoLET2007}.
}
\label{tab:subjects}
\end{table}

\newcommand{\unknown}{N/A\xspace}
\newcommand{\ignore}{---\xspace}
\newcommand{\infy}{$\infty$\xspace}

\begin{table*}
\centering
\setlength{\tabcolsep}{0.12\tabcolsep}
\begin{tabular}{|l|c|c|C|C|C|c|c|c|c|c|c|c|c|c|c|c|c|c|}
%\toprule
\hline
\textbf{Subject} & \textbf{\#} & \multicolumn{8}{c|}{\textbf{\# Detected Dependent Tests}} & \multicolumn{8}{c|}{\textbf{Analysis Cost (seconds)}}\\
%\midrule
\cline{3-18}
\textbf{Programs} & \textbf{Tests} & \textbf{Rev} & \multicolumn{3}{c|}{\textbf{Randomized}} & \multicolumn{2}{c|}{\textbf{Exhaustive }} & \multicolumn{2}{c|}{\textbf{Dep-Aware}} & \textbf{Rev}& \multicolumn{3}{|c|}{\textbf{Randomized}} & \multicolumn{2}{c|}{\textbf{Exhaustive }} & \multicolumn{2}{c|}{\textbf{Dep-Aware}} \\
%\cline{3-8}\cline{10-15}
& & & \smalltrialnum & \mediumtrialnum & \trialnum& \; $k$=1 & $k$=2 & \quad $k$=1 \;\; \quad & $k$=2 && \smalltrialnum & \mediumtrialnum & \trialnum & \; $k$=1 & $k$=2 &  \quad $k$=1 \quad \quad & $k$=2  \\
\hline
%\bottomrule
\multicolumn{18}{|l|}{ }\\
\multicolumn{18}{|l|}{\textbf{Human-written unit tests} }\\
\hline
%JFreechart & \jfreecharttests & 6 & 8 & 8 & 0 & $\ge$0 * & 0 & $\ge$0 * &  66  & 625 & 6097 & 694 & 2$\times$$10^6$ *  &310  &  1$\times$$10^6$ *\\
%the data of jfreechart is above, MUST update the total column
\jt & \jodatimetests & 2 & 1 & 1 & 6 & 2 & $\ge$2 * & 2& $\ge$2 * & 18&   57 & 528 & 5538 &1265& 4$\times$$10^6$ * & 291 & 5$\times$$10^5$ *  \\
XML Security& \xmlsecuritytests & 0 & 1 & 4 & 4 &4 &4 & 4 & 4  & 18&65 & 594 & 5977 & 106 &  11927 & 93 & 3322  \\
Crystal & \crystaltests & 18 & 18 & 18 & 18 &17&18& 17 & 18 & 3 &14& 131 & 1304 & 166 & 7323 & 95  & 4155 \\
Synoptic & \synoptictests & 1 & 1 &1  & 1 & 0 &1 & 0 & 1 & 2 &  7 & 67 & 760& 25 & 3372& 24 & 1797 \\
\hline
\textbf{Total} & \totaltests & 21 & 21&24&\textbf{29}& 23 & $\ge$24 & 23 & $\ge$25 &41&  143 & 1320 & 13579 &1562&  4$\times$$10^6$ *& 503  & 5$\times$$10^5$ *\\
\hline
\multicolumn{18}{|l|}{ }\\
\multicolumn{18}{|l|}{\textbf{Automatically-generated unit tests} }\\
\hline
%JFreechart & \jfreechartautotests& \ignore & \ignore & \ignore & \ignore & \ignore & \ignore & \ignore & \ignore & \ignore & \ignore & \ignore & \ignore & \ignore &  \ignore \\
\jt & \jodatimeautotests &\ignore & \ignore & \ignore & \ignore & \ignore & \ignore & \ignore & \ignore & \ignore & \ignore & \ignore & \ignore & \ignore & \ignore & \ignore &  \ignore \\
XML Security& \xmlsecurityautotests&138& 167 & 171 & 171 & 129 & $\ge$129 * & 128  & $\ge$128 *   & 7& 50 & 430 & 4174 & 133 & 1$\times$$10^5$ * & 128 & 5$\times$$10^4$ * \\
Crystal & \crystalautotests & 75 & 159 & 162 & 164 & 55 & $\ge$55 * & 55 & $\ge$55 *  & 22 & 103 & 949& 9436  & 2477 & 8$\times$$10^6$ *& 2297 & 1$\times$$10^6$ * \\
Synoptic & \synopticautotests &3 & 3 & 7 & 10 &2& $\ge$2 * & 2 & $\ge$2 *   & 13 &81& 770  & 6311 & 454 & 1$\times$$10^6$ *& 454 & 2$\times$$10^4$ * \\
\hline
\textbf{Total} & \totalautotests &216 &329 &340 & \textbf{345} & 186 & $\ge$186  & 185 & $\ge$185  &42&234&2149& 19921& 3064 & 1$\times$$10^7$ *& 2879& 1$\times$$10^6$ * \\
\hline
\end{tabular}
\caption{Experimental results.  Column ``\# Tests'' shows the total number
of tests, taken from Table~\ref{tab:subjects}. Column ``\# Detected Dependent Tests''
shows the number of detected dependent tests in each test suite.
% Columns ``Rev'', ``Randomized'', ``Exhaustive'' and ``Dep-Aware'' show the results
% of applying the reversal algorithm, randomized algorithm, exhaustive $k$-bounded algorithm, and the \dependenceaware{}, respectively.
%$k$-bounded algorithm, respectively. 
When evaluating the randomized algorithm, we used $\mathit{numtrials}$ =
$\smalltrialnum$, $\mediumtrialnum$, and $\trialnum$ (Figure~\ref{fig:randalgorithm}).
%``\unknown'' means the technique does not scale to the test
%suite (i.e., requiring more than 1 day to execute all test permutations),
%and thus the exact number of dependent tests is unknown.
``\ignore'' means the test suite is not evaluated due to its non-determinism.
%Column ``Analysis Cost''
%shows the time cost of each algorithm.
An asterisk (*) means the algorithm did not finish
within 1 day:
the number of dependent tests is those discovered before timing out, and 
the time estimation methodology is described in Section~\ref{sec:performance}.
\tinyrelax
}
\label{tab:results}
\end{table*}

%  LocalWords:  Joda numtrials


%We evaluated two aspects of \ourtool's
%effectiveness, answering the following
%research questions:
Our evaluation answers the following research questions:

\vspace{-1mm}

\begin{enumerate}
\vspace{-1mm}
\item How many dependent tests can each detection
algorithm detect in
real-world programs (Section~\ref{sec:detectedtests})?

\item How long does each algorithm in \ourtool take to detect dependent
tests (Section~\ref{sec:performance})?

\item Which algorithm is the most cost-effective in detecting
dependent tests (Section~\ref{sec:algcomparison})?
%\item How does \ourtool's effectiveness compare to an alternative
%approach based on test execution order randomization
%(Section~\ref{sec:random})?

\item Can dependent tests interfere with downstream testing techniques
such as test prioritization (Section~\ref{sec:impact})?

\end{enumerate}

\subsection{Subject Programs}


Table~\ref{tab:subjects} lists the programs and
tests used in our evaluation. We used these subject
programs because they have been developed for
a considerable amount of time (3--10 years) and each
of them includes a well-written unit test suite.

\jt~\cite{jodatime} is an open source
date and time library. It is a mature project that
has been under active development
for ten years. XML Security~\cite{xmlsecurity}
is a component library implementing XML signature and encryption
standards. XML Security is included in
the SIR repository~\cite{sir} and has been used widely
as a subject program in the software testing community.
Crystal~\cite{crystal} is a tool that
pro-actively examines developers' code and
identifies textual, compilation, and behavioral conflicts.
Synoptic~\cite{synoptic} is a tool to mine a finite state
machine model representation of a system from logs.

Given the increasing importance of automated test generation
tools~\cite{PachecoLET2007, ZhangSBE2011, Csallner:2004, fraseretal:ISSTA:2011},
we also want to investigate dependent tests in automatically-generated
test suites. For each subject program, we use
Randoop~\cite{PachecoLET2007}, a state-of-the-art automated
test generation tool, to create a suite of 5,000 tests.
Randoop automatically drops textually-redundant tests 
and outputs a subset of the generated tests as
shown in Table~\ref{tab:subjects}.

We discarded the automatically-generated test suite of
\jt, since many tests in it are non-deterministic ---
they depend on the current time.


\subsection{Evaluation Procedure}

We evaluated each algorithm 
on both the human-written test suite 
and the automatically-generated test suite
of each subject program in Table~\ref{tab:subjects}.

%We run the three algorithms proposed
%in Section~\ref{sec:detecting} on both
%human-written and automatically-generated test suites
%of each subject program.

We ran the randomized algorithm \smalltrialnum, \mediumtrialnum,
and \trialnum times on each test suite, and recorded
the total number of detected dependent tests and time cost
for each setting. The choice of \trialnum times is based
on a practical guideline for using randomized algorithms
in software engineering, as summarized in~\cite{Arcuri:2011}.
%
For the exhaustive $k$-bounded algorithm
and the \dependenceaware{} $k$-bounded algorithm,
we use isolated execution ($k = 1$) and
pairwise execution ($k = 2$). The choice of $k$ is
based on the results of our empirical
study (Section~\ref{sec:study}) that a small $k$
can find most realistic dependent tests.

We provided \ourtool with a list of 39 ``dependence-free'' fields
for the 4 subject programs. This manual step cost
about 30 minutes in total.

%\edit{say a few sentences here about the manual part, such as
%the approximate manual time cost in
%listing the immutable fields.}

We examined each output dependent test manually to make
sure the test dependence is not caused by non-deter\-min\-istic
factors, such as multi-threading.

Our experiments were run on a 2.67GHz Intel Core PC
with 4GB physical memory (2GB was allocated for the JVM),
running Windows 7.

\subsection{Results}

Table~\ref{tab:results} summarizes the number of detected
dependent tests and the time cost for each algorithm
in \ourtool.

\subsubsection{Detected Dependent Tests}
\label{sec:detectedtests}

%\todo{I rewrote the below paragraph.  Please review.}

\ourtool detected 29 human-written dependent tests (among which 27
dependent tests were previously unknown) and 1311
automatically-generated dependent tests.  A larger percentage (15\%) of
automatically-generated tests are dependent.  Developers' understanding of
the code, and their goals when writing the tests, help them build
well-structured tests that carefully initialize and destroy the shared
objects they may use.
By contrast,  most automated test generation tools are not ``state-aware'': the
generated tests often ``misuse'' APIs, such as not setting up
the environment correctly.  This misuse may
indicate that the tests are invalid; it may indicate weaknesses, poor
design, or fragility of the APIs; or it may indicate that the human-written
tests have failed to exercise some functionality.

The root cause of all the detected dependent tests is improper access to
static fields. The XML Security and Crystal developers use more
static fields in the test code,
so there are relatively more dependent tests detected in them.
%This concurs with our findings in Section~\ref{sec:studyfindings}.
%\edit{Can we say anything
%  about the relative frequency of the three?}

The randomized algorithm is surprisingly effective in
detecting dependent tests. In our experiments, when run \trialnum times,
it identifies \textit{more} dependent tests and found all
dependent tests identified by the other two algorithms.
For the human-written
test suites, the randomized algorithm detects 2 more dependent
tests in \jt. These two tests only
manifest when a sequence of three tests are run in a specified,
non-default order. Both exhaustive and \dependenceaware{} $k$-bounded
algorithms fail to detect these two tests, because
they cannot scale to $k$=3 for 
\jt. Related, the randomized algorithm
detects more dependent
tests in the automatically-generated test suites,
because both the exhaustive and \dependenceaware{} $k$-bounded
failed to scale to $k$=2 for all automatically-generated test suites.

The \dependenceaware{} bounded algorithm found the same
number of dependent tests as the exhaustive bounded algorithm, except
that it missed one dependent test in XML Security's
automatically-generated test suite.
The dependent test was missed because \ourtool
did not track static field access in the \CodeIn{java.security} package
of the JDK, and Javari did not provide annotations for APIs
in that package.


\subsubsection{Performance of \ourtool}
\label{sec:performance}

The time cost of the randomized algorithm 
is proportional to the run time of the suite and the number of runs.
Overall, the time cost is acceptable for practical use.
For example, the randomized algorithm took around 1.5 hours
to finish 1000 runs,  for \jt's human-written test
suite (\jodatimetests tests).
 
The time cost of running the exhaustive $k$-bounded algorithm
is prohibitive. The JVM initialization time is the main cost.
The exhaustive algorithm failed to
scale to one human-written test suite and all four automatically-generated
test suites when $k$=2, and failed to scale to all test suites
when $k$=3. The primary reason is the large
number of possible test permutations. 
For example, there are 15,011,750 size-2 permutations
for \jt's human-written test suite (\jodatimetests tests),
which would take approximately 58 days to finish.
%For example, running all
%Joda-Time's \jodatimetests human-written
%unit tests when $k$=2 requires running 15,011,750 test pairs, taking
%approximately 

Table~\ref{tab:results} gives an estimated time cost for each
test suite that an algorithm failed to scale to. For each test suite,
we randomly chose 1000 permutations from all
test permutations, executed them, and measured the average time cost
per permutation. Then, we multiple
the average cost by the total number of permutations to estimate
the time cost.

The \dependenceaware{} $k$-bounded algorithm ran about
an order of magnitude faster
than the exhaustive $k$-bounded algorithm,  when $k$=2.
%On average,
%it took 3.3$\times$ and 1.1$\times$ less time to run all
%human-written tests and automatically-generated tests, respectively, when $k$=1.
%The \dependenceaware{} algorithm took an order of magnitude
%less time to run both human-written tests and automatically-generated tests,
%The speedups are largely determined by performance on Joda-Time.
The \dependenceaware{} algorithm helps most when there are relatively many
tests, each one of them relatively small.




\subsubsection{Comparison of Algorithms}
\label{sec:algcomparison}

We next discuss the tradeoffs between choosing different detection
algorithms in \ourtool. Although the randomized algorithm
detects the most dependent tests in our subject programs,
it has several limitations. First, there is no guarantee of how many
dependent tests the randomized algorithm can detect. A randomized
algorithm might even produce different results across different runs.
Second, there is no clear stopping criterion
for running the randomized algorithm in practice.
Thus, it can be hard for users
to know how many runs would be enough to find all dependent tests in a test suite.
Third, given an identified dependent test, users
need to inspect the tests executed before it and isolate a minimized
subsequence of tests (either
manually or using an assisting tool~\cite{Zeller:2002}) to understand the dependence root cause.

By contrast, both the exhaustive $k$-bounded and the depend\-ence-aware
$k$-bounded algorithms systematically search for dependent
tests of a given size and do not suffer from the above limitations.
However, the major limitation that prevents them being applied to a
large test suite is the time cost to
explore all possible test permutations.


%\emph{three} tests to manifest. While these are easy to reproduce, we
%did not check that our tool finds them, because the time needed to
%run our naive algorithm on Joda-Time with $k=3$ is measured in months.


%\enlargethispage{5pt}

\begin{table}
\centering
\setlength{\tabcolsep}{0.25\tabcolsep}
\begin{tabular}{|l|l|}
%\toprule
\hline
\textbf{Label} & \textbf{Technique Description} \\
\hline
T1 & Randomized ordering \\
T3 & Prioritize on coverage of statements \\
T4 & Prioritize on coverage of statements not yet covered\\
T5 & Prioritize on coverage of methods\\
T7 & Prioritize on coverage of functions not yet covered \\
%\bottomrule
\hline
%\textbf{Total}& &  & &  \\ 
%\hline
\end{tabular}
\caption{Five test prioritization techniques used
to assess the impact of dependent tests. These five
techniques are introduced in Table 1
of~\cite{Elbaum:2000:PTC:347324.348910}. (We use
the same labels as in~\cite{Elbaum:2000:PTC:347324.348910}. We did not
implement the other 9 test prioritization techniques
introduced in~\cite{Elbaum:2000:PTC:347324.348910}, since
they require a fault history that is not
available for our subject programs.)
}
\tinyrelax
\label{tab:testprio}
\end{table}



\begin{table}
\centering
% \setlength{\tabcolsep}{1.25\tabcolsep}
\begin{tabular}{|l|c|c|c|c|c|}
%\toprule
\hline
\textbf{Subject Program} & T1 & T3 & T4 & T5 & T7 \\
\hline
\jt& 0 & 0 & 1 & 0 & 0\\
XML Security& 0 & 0 & 0 & 0 & 0 \\
Crystal& 12 & 11 & 16 & 11 & 12 \\
Synoptic& 0 & 0 & 0 & 0 & 0 \\
%\bottomrule
\hline
\textbf{Total} & 12 & 11 & 17 & 11 & 12\\
\hline
%\textbf{Total}& &  & &  \\ 
%\hline
\end{tabular}
\caption{Differences in test results between original and prioritized
  human-written unit test suites.
Each cell shows the number of tests
that do not return the same results as they do when executed
in the default, unprioritized order.
}
\smallrelax
\label{tab:testprioresult}
\end{table}


%  LocalWords:  LOC Joda b609d7d66d T1 T3 T4 T7 unprioritized

\subsubsection{The Impact on Test Prioritization}
\label{sec:impact}

We implemented five test prioritization techniques~\cite{Elbaum:2000:PTC:347324.348910} (summarized in Table~\ref{tab:testprio}) and
applied them to the human-written test suites of our subject programs.


For each test prioritization algorithm, we counted the number
of dependent tests that return different results (pass or fail) in
the prioritized order than they do when executed in the
unprioritized order. Table~\ref{tab:testprioresult} summarizes
the results.

The dependent tests in our subject programs interfere with
\textit{all} five test prioritization techniques in
Table~\ref{tab:testprio}.
%\todo{we need to edit the sentence below, since a reviewer has
%strong opinion about the word "assume"}
This is because all these techniques
implicitly assume that there are no test dependences in
the input test suite. Violation of this assumption, as
happened in real-world test suites, can cause the prioritized suite to fail
even though the original suite passed.

We did not evaluate the effect of test dependence on metrics such as 
APFD~\cite{Rothermel:1999:TCP:519621.853398}; there is no point optimizing
such a metric at the cost of false positives or false negatives.


%One possible to remedy this problem is to 

%as  in the
%4 human-written test suites that can
%theoretically affect the results of these two techniques.
%For test selection, we manually checked whether
%there exists a subsequence of tests
%that do not return the same results
%when executed in the \textit{same} order
%as in the original test suite. 

%For test prioritization, we manually check whether
%there exists a possibly-\textit{reordered}
%subsequence of tests that do not return the
%same results as in the original test suite.

%As a result, 26 out of 29 dependent tests can
%%besides the synoptic test, and 2 jodatime tests
%affect the results of test selection,
%and all 29 dependent tests can affect the results
%of test prioritization.



\subsection{Discussion}
\label{sec:expdiscussion}

%\subsubsection{Developers' Reactions}

\noindent \textbf{Developers' Reactions to Dependent Tests.}
We sent the identified human-written dependent tests to the
subject program developers, asking for their feedback.

One dependent test in \jt was previously known
and had already been fixed. \jt's
developers confirmed the other new dependent
tests, and thought that they are due to unintended interactions
in the design of the library.
%
The Crystal developers confirmed that all dependent tests
found in Crystal were not intentional and happened because of dependence
through global variables. The developers considered the
dependent tests undesirable and opened a bug report for
this issue~\cite{crystalbugreport}.
%
The dependent test in Synoptic was previously known.
The developers merged two related tests to fix
the dependent test.
%
%After receiving our reported dependent tests in XML-Security,
The SIR~\cite{sir} maintainers confirmed our reported dependent
tests in XML-Security, and accepted our
suggested patch to fix them. They also highlighted the practice
that tests should \textit{always} ``stand alone''
without dependency on other tests, and characterized that as
``test engineering 101''. 
%They also accepted our suggested
%patch to fix the dependent tests.


%\noindent
%\setlength{\tabcolsep}{0.1\tabcolsep}
%\begin{tabular}{|l|c|}
%\hline
%Downstream Testing Techniques& \#Affected Dependent Tests\\
%\hline
%Test Selection & \\
%Test Prioritization& \\
%\hline
%\end{tabular}

\vspace{1mm}
\noindent \textbf{Threats to Validity}
There are several threats to the validity of our evaluation.
First, the \subjnum open-source
programs and their test suites may not be
representative enough. Thus, we cannot claim the results
can be generalized to an arbitrary program.
However, these are the first \subjnum subject programs
we tried, and the fact that we found dependent tests
in all of them is suggestive.
Second, in this evaluation, we focus specifically on
the {manifest dependence} between \textit{unit tests}.
We did not investigate possible test dependence that may arise
in other types of tests, such as integration tests.
Third, due to the computational complexity of the general dependent test
detection problem, we do not yet have
empirical data regarding \ourtool's recall and how many dependent
tests exist in a test suite. 
Fourth, we only assessed the
impact of dependent tests on five test prioritization
techniques.
%, and we only evaluated the impact of dependent
%tests on prioritizing unit tests.
Using other test prioritization techniques
might achieve different results. 

%\todo{this paper focuses on manifest test dependence, what about
%potential test dependence, as well as some other cases in the
%study}

\vspace{1mm}

\noindent \textbf{Experimental Conclusions}
We have four chief findings. \textbf{(1)}
Dependent tests do exist in practice, both in
human-written and automatically-generated test suites.
%can contain substantially more dependent tests than a human-written
%test suite.
\textbf{(2)} Like the dependent tests
studied in Section~\ref{sec:study}, the identified
dependent tests in our subject programs generally reveal weakness
in a test suite rather than defects in the tested code.
\textbf{(3)} Dependent tests can interfere with
test prioritization techniques and cause unexpected output.
\textbf{(4)} 
%\todo{need to figure out a proper english sentence to say like:
%The dependent test detection problem is inherently
%complex, a smarter algorithm may not achieve great results.}
The randomized algorithm is the most cost-effective in
detecting dependent tests, although it has no guarantee
of the number of dependent tests it can detect.
%Testers
%can use this simple yet effective algorithm in practice.

%\todo{Idea:  how about combining the randomized and dependence-aware
%  algorithms?  That is, generate random orderings that violate dependences;
%  or generate random orderings and discard ones that do not violate
%  dependences.  This probably isn't a great idea, but a reader might think
%  of it.}

%  LocalWords:  Joda dependences multi 67GHz 4GB 2GB tradeoffs subsequence


%
%\enlargethispage{5pt}

\begin{table}
\centering
\setlength{\tabcolsep}{0.25\tabcolsep}
\begin{tabular}{|l|l|}
%\toprule
\hline
\textbf{Label} & \textbf{Technique Description} \\
\hline
T1 & Randomized ordering \\
T3 & Prioritize on coverage of statements \\
T4 & Prioritize on coverage of statements not yet covered\\
T5 & Prioritize on coverage of methods\\
T7 & Prioritize on coverage of functions not yet covered \\
%\bottomrule
\hline
%\textbf{Total}& &  & &  \\ 
%\hline
\end{tabular}
\caption{Five test prioritization techniques used
to assess the impact of dependent tests. These five
techniques are introduced in Table 1
of~\cite{Elbaum:2000:PTC:347324.348910}. (We use
the same labels as in~\cite{Elbaum:2000:PTC:347324.348910}. We did not
implement the other 9 test prioritization techniques
introduced in~\cite{Elbaum:2000:PTC:347324.348910}, since
they require a fault history that is not
available for our subject programs.)
}
\tinyrelax
\label{tab:testprio}
\end{table}



\begin{table}
\centering
% \setlength{\tabcolsep}{1.25\tabcolsep}
\begin{tabular}{|l|c|c|c|c|c|}
%\toprule
\hline
\textbf{Subject Program} & T1 & T3 & T4 & T5 & T7 \\
\hline
\jt& 0 & 0 & 1 & 0 & 0\\
XML Security& 0 & 0 & 0 & 0 & 0 \\
Crystal& 12 & 11 & 16 & 11 & 12 \\
Synoptic& 0 & 0 & 0 & 0 & 0 \\
%\bottomrule
\hline
\textbf{Total} & 12 & 11 & 17 & 11 & 12\\
\hline
%\textbf{Total}& &  & &  \\ 
%\hline
\end{tabular}
\caption{Differences in test results between original and prioritized
  human-written unit test suites.
Each cell shows the number of tests
that do not return the same results as they do when executed
in the default, unprioritized order.
}
\smallrelax
\label{tab:testprioresult}
\end{table}


%  LocalWords:  LOC Joda b609d7d66d T1 T3 T4 T7 unprioritized

\subsubsection{The Impact on Test Prioritization}
\label{sec:impact}

We implemented five test prioritization techniques~\cite{Elbaum:2000:PTC:347324.348910} (summarized in Table~\ref{tab:testprio}) and
applied them to the human-written test suites of our subject programs.


For each test prioritization algorithm, we counted the number
of dependent tests that return different results (pass or fail) in
the prioritized order than they do when executed in the
unprioritized order. Table~\ref{tab:testprioresult} summarizes
the results.

The dependent tests in our subject programs interfere with
\textit{all} five test prioritization techniques in
Table~\ref{tab:testprio}.
%\todo{we need to edit the sentence below, since a reviewer has
%strong opinion about the word "assume"}
This is because all these techniques
implicitly assume that there are no test dependences in
the input test suite. Violation of this assumption, as
happened in real-world test suites, can cause the prioritized suite to fail
even though the original suite passed.

We did not evaluate the effect of test dependence on metrics such as 
APFD~\cite{Rothermel:1999:TCP:519621.853398}; there is no point optimizing
such a metric at the cost of false positives or false negatives.


%One possible to remedy this problem is to 

%as  in the
%4 human-written test suites that can
%theoretically affect the results of these two techniques.
%For test selection, we manually checked whether
%there exists a subsequence of tests
%that do not return the same results
%when executed in the \textit{same} order
%as in the original test suite. 

%For test prioritization, we manually check whether
%there exists a possibly-\textit{reordered}
%subsequence of tests that do not return the
%same results as in the original test suite.

%As a result, 26 out of 29 dependent tests can
%%besides the synoptic test, and 2 jodatime tests
%affect the results of test selection,
%and all 29 dependent tests can affect the results
%of test prioritization.



%\section{Disucssion and Future Work}
\label{sec:discussion}

%In the introduction we posed several questions why test dependence
%may have received little attention despite the ease of constructing
%concrete but contrived examples. 
%contributions suggest answers, to differing degrees, to these questions:


Although we have studied and detected a substantive set of
real-world dependent tests, from both human-written test suites and
automatically generated test suites, a broader
investigation of the impact of dependent tests,
how to eliminate and prevent dependent tests
is beyond the scope of this paper. 
We next discuss a set of open questions addressing this and
other possible concerns resulting from dependent teests.
Exploring answers to these open questions comprises
our future work.

\vspace{1mm}

\noindent \textbf{{Impact of dependent tests.}}
Dependent tests can mask faults and lead to
spurious bug reports; they can also 
compromise the application of
testing techniques such as test selection,
prioritization, and parallelization, since
most current techniques just assume independence and
make no statement about what happens when this
assumption is not true. However, for both perspectives,
the extent of impact is unknown. This question
merits more comphrensive empirical studies.  


On the other hand, the question of how existing
testing techniques should
handle test dependence is also open.
For downstream testing techniques such as test
selection, prioritization, and parallelization,
one straightforward way to amend this situation
might be to augment such techniques to respect a
defined partial order among tests. And this partial order
can be derived from knowledge about dependent tests,
or be detected by our \ourtool tool.
%Like contrived examples of test
%dependence itself, it is easy to produce simple examples where
%downstream techniques produce incorrect output when applied to dependent
%tests.
%under the assumption that the input tests have no dependences.
%However, 



\vspace{1mm}

\noindent \textbf{{Eliminating dependent tests.}}
As found in our study (Section~\ref{sec:study}),
developers often ignore dependent tests due to
the lack of proper tool support. For some
dependent tests that get eliminated,
the practice of eliminating them
remains mostly manual and ad hoc --- software developers
usually manually hardcode test
execution orders in a configuration file or
simply merge tests, when a dependent test is reported. 
A more flexible and robust methodology for
dependent test elimination should be developed.

On the other hand, dependent tests in
an automatically-generated test suite can be 
more challenging to eliminate.
As suggested by our experiments, dependent tests are
more prevalent in automatically-generated test suites
than in human-written test suites.
Further, this problem is even exacerbated by the fact that
almost all automated test generation
techniques we are aware of produce tests
that are hard to read for humans, are undocumented, and their intent
cannot easily be gleaned from naming conventions and other aids
developers normally use. Therefore, it requires more effort
from developers to identify the root cause of dependence
and then remove the dependence. While there is some work to alleviate
this problem~\cite{fraseretal:ISSTA:2011}, the question
of eliminating  automatically-generated dependent tests
still remains open.


%As discussed in our experiments, it appears that test
%dependence in automatically generated test suites is 
%even more troublesome than in human-written suites. 


\vspace{1mm}


\noindent \textbf{{Preventing dependent tests.}}
Detecting dependent tests is not obvious in most
cases. Thus, a natural question is how could
software developers prevent dependent tests when
writing testing code.

One possible way is encouraging developing to
use advanced testing frameworks that support test dependence,
so that developers can explicitly specify test
dependence when writing tests.
However, using different testing frameworks may
bring up the compatibility issue to the existing tests.

Stylized coding patterns can also be useful. Developers
should be encouraged to write tests ``defensively'' by
specifying necessary test execution pre-conditions and
using less (or properly mocking) global variables or shared resources. 
There is already some work aiming at automating this
process to prevent the potential
for dependences by refactoring programs to use
less global state~\cite{wlokaetal:FSE:2009}. 

%We conjecture that if and when this happens in practice, it is
%hard to notice in part because current techniques~\cite{} do not surface the necessary information.

%defensive programming? explicitly state testing oracles?

%A better understanding of the 
%frequency and scope of consequences from test dependence should be
%developed.  

%Of particular concern is the
%masking of program faults because, unlike weaknesses
%in test suites or spurious bug reports, masking
%faults could be costly to find by other methods or to leave in the program.
%Tools that surface test dependences may help researchers
%and practitioners study and deal with dependences more effectively.





%\vspace{1mm}

%\noindent \textbf{\textit{What about other cases of test dependences?}}



%Several of our examples identified situations in which test dependence
%masked faults in the underlying program. 
%In another example, developers wasted time tracking down a non-existent
%fault because of a spurious report that was due to an undocumented
%test dependence. Even though these
%are real and reproducible examples,

%It is not possible to make any
%general claims about the frequency nor the significance of the
%repercussions of test dependence.  At the same time, it seems unlikely
%that these are the \emph{only} software systems where test dependence
%causes problems.


%\vspace{1mm}


%This is a more subtle question, 
%because the answer depends not only on the tools being used but also on the
%perceptions and insights of the developers.  If tools always run
%tests in the same context, and if developers never consider the possibility
%of test dependence, then it is unlikely that dependence will be observed.
%Masking of faults in the underlying program is a good illustration of this.

 

%\medskip


%Our 
%prototype tool shows that even our approximate algorithm
%can reveal large numbers of important dependences. Faster and more
%precise approaches are plausible, especially as more understanding
%of test dependence ``in the field'' is acquired.

%avoiding test dependences, to removing or documenting test dependences that are found, etc.


%Given a technique that detects dependence, what further
%  actions can and should be taken?


%also be supported by testimony from practitioners. With regard to the
%relevance of test dependence, the following questions seem to be of
%particular interest:
%\begin{enumerate}
%  \item Does test dependence occur often enough, and is its impact
%  critical enough to make further enquiry worthwhile?
%  \item 
%\end{enumerate}

%The frequency and consequences of test dependence in our work
%so far seem to justify further investigation.  We are especially
%concerned about whether test dependence is masking 

%We strongly believe that both the frequency and the
%impact of test dependence merit investigation of the phenom\-e\-non. The
%fact that within our small example set we found a number of masked
%faults that directly impacted the users of the software, and the
%observation that any form of test dependence prevents the successful
%use of many second-order testing techniques are strong
%indicators that this phenomenon deserves as much attention as any
%other technique that can improve the bug finding strength of testing.

%Thinking about the higher-level causes of test dependence leads to a large number of different questions,
%many of which relate not only to technical issues, but also to the social
%and human environment in which software is created. Exploring this
%domain, while very difficult to do well, is likely to yield the best
%explanations for the creation of test dependence.

%How actionable the knowledge of test dependence is, ironically,
%depends. At this early stage, we suggest inspection of code, tests and
%specifications to understand what causes the dependence, and where the
%fault lies. As our examples show, dependence can point to faults in
%the program, in the tests, or to insufficient documentation and
%specification. We hope and expect that it will be possible to
%determine or exclude some of these causes automatically with further
%analyses of code and tests. 
%There are also several
%projects that either augment or replace JUnit with the express goal to
%declare test dependence
%explicitly\footnote{\url{https://code.google.com/p/depunit/},\\
%\url{https://code.google.com/p/junitum/},\\ \url{http://testng.org}}, but not to detect it.




%
%
%The key take-aways from this paper are that  prior work and existing
%tool broadly assume test independence,
%%is broadly assumed by prior work and by existing tools 
%and that there is
%at least incipient empirical evidence that this assumption can lead to
%unexpected and likely negative repercussions.  
%%We present an initial
%formalization of test dependence that embodies both the execution
%order of a test suite and also the environment in which tests in a
%suite are executed. % the formalization allows us to prove that the
%problem of determining if there are dependences among the tests in a
%test suite is NP-complete.  
%Substantive examples of test dependences
%in the field, along with descriptions of the consequences of these
%dependences, argue the potential practicality of further
%investigations of test dependence.  
%Initial algorithms designed with the
%NP-completeness of the problem in mind, along with an initial tool, allow
%us to take initial steps towards practical applications, as well as to check
%the validity of examples of test dependence that we had previously identified
%in \emph{ad hoc} ways.\todo{JW}{This is all a bit too initial...}  A set of open questions related to test dependence 
%provides a partial, surely incomplete, roadmap for further work in the area.

%\todo{JW}{I think this section should be about
%\begin{enumerate}
%  \item What prompted our research
%  \item Which research questions did we answer, and which research 
%  questions remain open.
%  \item Summary what the reader should take away from the paper.
%\end{enumerate}
%The current version falls short of this in several ways. 
%\begin{enumerate}
%  \item It doesn't address ``the big picture'' at all.
%  \item It focuses on minor technical details \emph{and} phrases
%  minor technical issues as research problems in a way that assumes
%  answers to interesting questions we didn't even ask (comments in the
%  text below elaborate on that)
%  \item It focuses too much on our ``results'', which is wrong,
%  considering how little actual data we have.
%\end{enumerate}
%}

%More concretely, in this paper, we examine the importance of test dependence in theory and practice. 
%While the
%research literature usually assumes tests to be independent, or evades
%the issue entirely, our interest was piqued when we found a number of
%test dependences in real-world software.
%This led us to explore in more depth if this is an issue in a broader
%range of software systems, and whether test dependence causes real
%problems. 
%


%We found that the human-written test suites of many open-source libraries contain
% dependent tests, that test suites automatically generated with
% Randoop contain even larger numbers of dependent tests, and that in
% both cases these tests cause various
%kinds of problems, from preventing test prioritization to actually
%hiding real faults in the programs. However, we must emphasize that
%our exploration should not be construed as a complete
%experiment that provides conclusive evidence. Rather we inspected programs that we were familiar with.
%Hence, before the conclusions drawn from the examples in
%Section~\ref{sec:examples} can be generalized,
%a proper controlled experiment should be carried out with a broader
%range of projects, and with proper control for confounding factors
%such as project type and programmer expertise, which most likely have
%a profound impact on test depenendence. Nonetheless, we believe that the anecdotal evidence we
%presented in this paper makes clear that this is a worthwhile research
%endeavour.

%As a main contribution of this paper we identify test dependence as
%an important research subject that so far has been mostly ignored by
%the software testing community.

%Before summarizing our work and contributions, we outline a set of
%open research topics that could further improve the way we identify
%and manage test dependences: 
%\begin{itemize}
%
%\item Our initial algorithms for detecting test dependences 
%leave room for improvement in efficiency.  Conventional optimizations
%should apply \todo{JW}{I don't know what this means. We proved that
%the problem is NP-complete the full algorithm is exponential.
%``Optimizations'' in the case can only mean approximations, right?}, as should incremental algorithms and/or on-line versions of
%the algorithms --- for when test cases are added to a suite, among other
%potential performance improvements.\todo{JW}{What's the point here?}
%
%\item \todo{JW}{This is a theory paper. I think this is completely
%irrelevant} Our initial tool for detecting test dependences 
%leaves room for improvements in applicability (what ``forms'' of test
%suites/programs do we handle, such as JUnit, etc.?), in user
%interface, etc.
%
%\item We have only scratched the surface of the interaction
%between test dependence and downstream testing tools like selection,
%prioritization, and parallelization.  \todo{DN}{We have claimed earlier
%we will show that dependence can cause these approaches to fail.
%Where in the paper do we do that, and how do we do that?  I believe it's
%pretty obvious, but we'll need to be careful about doing it.} \todo{KM}{I think
%we even have real life examples (from Mike), etc. for this, however I might be
%mistaken. Generating theoretical example is obvious, but I agree that we lack
%that example at the moment.} How should these interactions be handled? For
%example, should a ``test dependence manager'' ensure that suites have no dependences before a downstream tool is invoked or should those tools check that their output test sub-suite has no dependences?
%
%\item \todo{JW}{This is an interesting question, as it to some extent
%asks what is actionable about dependencies. However, I think this
%should be discussed elsewhere (in the motivation, examples?), because
%this isn't really a research question or anything. The answer to this
%seems pretty obvious to me, and is what is described in this
%paragraph.} What should a development/test team do when a test dependence is identified?
%It could indicate that there is a problem in the tests themselves, in which case
%they could be fixed.  It could indicate that there is a problem in the program
%being tested, in which case it could be fixed.  It could indicate that the
%dependence is necessary, in which case the test could perhaps be merged together
%to ensure that the dependence is respected by the test framework and downstream
%tools.
%
%\item \todo{JW}{What?} We argue additional work exploring the interrelationship between
%individual unit tests should be performed, and show initial examples
%that test dependencies must be considered when executing a test suite,
%fixing a regression error, and generating new tests.
%
%\item This leads to our next recommendation for further software testing
%research. Some work in this area, discussed previously, has already
%attracted attentions. Nevertheless, we strongly beleive that more
%work exploring open questions on how to integrate \textit{test dependence}
%to the testing process is necessary to be understood.
%
%\todo{JW}{The following two paragraphs are too narrowly focused on
%test dependences being a bad thing that needs to be avoided. I would
%at the very least at the meta question asking when, where, how often
%dependences are bad, intentional, necessary, practical etc.}
%
%First, how to eliminate existing dependent tests (or retrofitting
%dependent tests into tests without inter-dependencies)? A critical
%part in retrofitting existing dependent tests is to identify test
%code that may affect the behavior of our tests, and then make the
%affected tests more ``robust'' to such affecting code. 
%\todo{DN}{Something especially about initialization code for tests?}
%
%\todo{JW}{This is too vague, both the problem description and the
%solution.}
%
%
%Second, how to prevent new dependent tests being produced?
%Programmers should be encouraged
%to encapsulate common test execution environment setting up and destroying
%code into separate pre- and post-conditions, to ensure each
%test is executed in a desirable environment and also probably cleared up
%the environment after its execution. One promising way to achieve
%this is to extend Design-by-Contract~\cite{Leitner:2007} to human-written unit
%tests. For an automated tool, it may employ
%capture-and-replay~\cite{Elbaum:2006} techniques to save the probable environment when a test is created,
%and then recover the needed environment when executing a
%test.\todo{JW}{This sounds weird to me. How would that work.}
%
%\item Finally, what would be the impact of dependent tests to the whole software testing
%process?  Most existing research in the fields of regression selection
%and prioritization has an implicit assumption on test dependence.
%.\textbf{unfinished}.. \todo{JW}{This could be interesting, but I
%doubt there is any general conclusion we can draw here.}
%
%\end{itemize}


%Our formalism provides a precise definition of manifest test dependence,
%allows reasoning about test dependence, and enables the proof that
%detecting manifest test
%dependence in test suites is NP-complete.  


\section{Related Work}
\label{sec:related}

%Denoting a group of test cases as a ``suite of test programs'' began around the
%mid-1970's~\cite[p.~217]{brown:CSUR:1974}; similar terms include
%``testcase dataset''~\cite{milleretal:ICRS:1975} and ``scenario,''
%which an IEEE Standard defines as ``groups of test cases;
%synonyms are script, set, or suite''~\cite[p.~10]{IEEE:829-1998}.

Treating test suites explicitly as \emph{mathematical sets} of tests dates at least
to Howden~\cite[p.~554]{howden:ToC:1975} and remains common in the literature.
The execution order of tests in a suite is usually not considered:
%or informally, suggesting that the potential of executing a given test
%in different contexts is immaterial to those results: 
that is, test independence is assumed. We next discuss some
existing definitions of test dependence, techniques that
assume test dependence, and tools that support specifying
test dependence.


\subsection{Test Dependence}

Definitions in the testing literature are generally clear that the
conditions under which a test is executed may affect its result. 
The
importance of context in testing has been explored in some depth in
some domains including databases~\cite{Gray:1994:QGB:191843.191886,Chays:2000:FTD:347324.348954,
kapfhammeretal:FSE:2003}, with results about test
generation, test adequacy criteria, etc., and mobile
applications~\cite{Wang:2007:AGC}.
For the database domain, Kapfhammer and Soffa formally
define independent test suites and distinguish them from
other suites that ``can capture more of an application's
interaction with a database while requiring the constant monitoring of
database state and the potentially frequent re-computations of test
adequacy''~\cite[p.~101]{kapfhammeretal:FSE:2003}.
By contrast, our definition differs from that of Kapfhammer
and Soffa by considering
test results rather than program and database states
(which may not be visible to users).
%Considering only manifest test dependences allows
%us to more easily situate this research in the empirical domain (Section~\ref{sec:formaldiscussion}).

The IEEE Standard for Software and System Test
Documentation (829-1998) \S 11.2.7, ``Intercase
Dependencies,'' says in its entirety: ``List the identifiers of
test cases that must be executed prior to this test
case. Summarize
the nature of the dependences''~\cite{IEEE:829-1998}.  The succeeding version of this
standard (829-2008) adds a single sentence: ``If
test cases are documented (in a tool or otherwise) in the order in
which they need to be executed, the Intercase Dependencies for most or
all of the cases may not be needed''~\cite{IEEE:829-2008}.


%In addition to the work by Kapfhammer and
%Soffa~\cite{kapfhammeretal:FSE:2003},
%there are a handful of categorical references that
%acknowledge that tests can
%be dependent based on context, suggesting 
%ways to document or find situations where the independence
%assumption fails to hold.  


%McGregor and Korson discuss interaction tests that
%are intended to identify ``two methods that may directly or indirectly
%cause each other to produce incorrect results'' and suggest constructing such
%interaction tests by identifying the values shared via parameter passing
%between methods
% that two or more test cases share~\cite[p~.69]{mcgregoretal:CACM:1994}.

Bergelson and Exman characterize a form of test dependence informally:
given two tests that each pass, the composite
execution of these tests may still
fail~\cite[p.~38]{bergelsonetal:EEE:2006}.  That is, if 
\suite{t_1} executed by itself passes and \suite{t_2} executed by itself passes,
executing the sequence \suite{t_1, t_2} in the same context may fail.
However, they do not provide any empirical evidence of
test dependence nor any detection algorithms.

Some practitioners acknowledge test dependence as a possible, albeit low probability, event:
\begin{quote}
Unit testing \dots  
requires that we test the unit in isolation. That is, we
want to be able to say, \emph{to a very high degree of confidence} [emphasis added], that
any actual results obtained from the execution of test cases are
purely the result of the unit under test. The introduction of
other units may color our results~\cite{unit-test-def}.
\end{quote}
They further note that other tests, as well as stubs and drivers,
may ``interfere with the straightforward
execution of one or more test cases.''


Compared with these informal definitions,
we formalize test dependence, and provide empirical evidence
to show that test dependence does arise in practice, and could
have costly repercussions.
%They give an informal definition of what it means for the execution of a
%test to influence the outcome of another test.  We define
%this precisely, and we also define manifest test dependence in terms
%of execution environments
%and test execution order rather than in terms of code use.

%Other definitions of test dependence are primarily considered
%to be \textit{syntactic} dependences between program units, for example
%methods calling other methods, and classes using other classes~\cite{bergelsonetal:EEE:2006,briandetal:SEKE:2002}. 
%\emph{Syntactic} dependence here means that a unit \code{A} cannot be
%compiled and executed without unit \code{B} being present. If we test
%such a unit \code{A} without convincing ourselves first that \code{B}
%is correct, a test failure for \code{A} is harder to interpret,
%because it could just as well indicate a fault in \code{B}.
%Zhang and Ryder extend this notion to \emph{semantic} dependences,
%which is closer to our approach~\cite{zhangetal:TR:2006}. 
%They use a notion of
%``test outcome'' to determine whether or not syntactically dependent
%classes or methods can influence each others results, and consider
%only those that can to be semantically dependent.
%They give an informal definition of what it means for the execution of a
%test to influence the outcome of another test.  We define
%this precisely, and we also define manifest test dependence in terms
%of execution environments
%and test execution order rather than in terms of code use.


\subsection{Techniques Assuming Test Independence}

The assumption of test independence lies at the heart of most,
if not all, techniques for automated regression test selection,
test case prioritization, test generation, coverage-based
fault localization, etc. 


Test prioritization seeks to reorder a test suite to detect
software defects more quickly. 
Early work in test
prioritization~\cite{Wong:1997:SER:851010.856115,Rothermel:1999:TCP:519621.853398}
laid the foundation for the most commonly used problem definition:
consider the set of all permutations of a test suite and find the best
award value for an objective function over that
set~\cite{Elbaum:2000:PTC:347324.348910}.  The most common objective
functions favor permutations where more faults in the underlying
program  are found with running fewer tests.
Test independence is often explicitly asserted as a
requirement for most test selection and prioritization work (e.g.,~\cite[p.~1500]{Rummel:2005:TPR:1066677.1067016}).
%For some test selection and prioritization work,
%test independence is even explicitly asserted as a requirement.
%For example, Rummel et al.\ states in
%A number of studies carefully evaluation various prioritization techniques
%empirically~\cite[\emph{et
%alia}]{Rothermel:1999:TCP:519621.853398,Do:2010:ETC:1907658.1908088}. 
Evaluations of selection and prioritization techniques
~\cite[\emph{et alia}]{Rothermel:1999:TCP:519621.853398,Do:2010:ETC:1907658.1908088}
are based in part on the test independence
assumption as well as the assumption that the set of faults in the underlying
program is known beforehand; the possibility that test dependence may unmask additional faults in the program is not studied.

%\begin{quote}
%A test suite contains a tuple of tests \suite{T_1 $\ldots$ T_R} that execute in a specified order.  We require that each test is
%independent so that there are no test execution ordering dependencies.  This requirement enables our prioritization algorithm to
%re-order the tests in any sequence that maximizes the suite's
%ability to isolate defects.  The assumption of test dependence
%is acceptable because the JUnit test execution framework
%provides \code{setUp} and \code{tearDown} methods that execute before
%and after a test case and can be used to clear application
%state.
%\end{quote}

Most automated test generation
techniques~\cite{PachecoLET2007, Wang:2007:AGC,
ZhangSBE2011} do not take test dependence
into consideration. As shown in our experiments (Section~\ref{sec:evaluation}),
a large number of tests generated by Randoop are dependent.
We speculate that these dependences arise because automated
test generation techniques generally create new tests
based on the program state after executing the previous test,
for the sake of test diversity and efficiency. 
To the best of our knowledge, only JCrasher~\cite{Csallner:2004}
provides a mode to clear the environment changes caused
by a previous test. Such functionality helps eliminate
potential test dependence, but may make generated
tests less behaviorally-diverse --- as they cannot be constructed
on top of previous tests. Exploring how to
incorporate test dependence into the design of automated
test generator is our future work.

Coverage-based fault localization techniques~\cite{Jones:2002:VTI}
often treat a test suite as a collection of test cases
whose result is \textit{independent} of the order of their
execution. They can also be impacted by test dependence.
In a recent evaluation of several coverage-based fault locators,
 Steimann et al.\ found fault locators' accuracy has been significantly
 affected by tests failed due to the violation of the test
 independence assumption~\cite{Steimann:2013}. 
 %For example, if a test depends on a static field whose value is set by
 %previous test cases. 
 Compared to our work, Steimann et al.'s
 work focuses on identifying possible threats to validity
 in evaluating coverage-based fault localization, and does
 not present any formalism, study, or detection algorithms
 for dependent tests.


%define a test suite as a
%collection of test cases whose result is \textit{independent}
%of the order of their execution~\cite{Steimann:2013}.

As shown in Sections~\ref{sec:study} and~\ref{sec:evaluation},
the test independence assumption often does not hold for either
human-written or automatically-generated tests. Thus, techniques
that rely on this assumption may need to be reformulated.

\subsection{Tools Supporting Test Dependence}
\label{sec:supporting}

Testing frameworks provide mechanisms
for developers to define the context for tests.
JUnit, for example, provides means to
automatically execute setup and clean-up tasks
(\code{setUp()} and \code{tearDown()} in JUnit
3.x, and methods annotated with \code{@Before} and \code{@After} in
JUnit 4.x). Ensuring that these mechanisms are used properly, however, is
beyond the scope of any framework, although the latest release of JUnit
(version 4.11)
supports executing tests in lexicographic order by test method name~\cite{junitordering}.


Only a few tools explicitly consider test dependence, by
allowing developers to annotate dependent tests and
provide supporting mechanisms to ensure that the test execution framework
respects those annotations.  DepUnit~\cite{depunit}
allows developers to define soft and hard dependences. Soft dependences control
test ordering, while hard dependences in addition control whether specific tests are
run at all.  TestNG~\cite{testng} 
allows dependence annotations and supports a variety of execution policies
that respect these dependences
such as sequential execution
in a single thread, execution of a single test class per thread, etc.\
What distinguishes our work from these approaches is that, while they allow dependences
to be made explicit and respected during execution, they do not help developers
\emph{identify} dependences.  A tool that finds dependences
(Section~\ref{sec:impl}) could co-exist
with such frameworks by generating annotations for them.

Our previous work~\cite{DBLP:conf/sigsoft/MusluSW11} proposed
an algorithm to find bugs by executing each unit
test in isolation. With a different focus,
this work investigates the validity of the test independence assumption
rather than finding new bugs,
and presents four new results.
Further, as indicated by our study and experiments, most dependent
tests reveal weakness in the test code rather than bugs in the program. Thus,
using test dependence may not achieve a high return in finding bugs.

%  LocalWords:  Howden Kapfhammer Soffa dependences Intercase Bergelson
%  LocalWords:  Exman JCrasher Steimann setUp tearDown DepUnit TestNG


\tinysqueeze

%\vspace{-1mm}
\section{Conclusion and Future Work}
\label{sec:questions}


Test independence is broadly assumed but rarely addressed, and
test dependence has largely been ignored in previous
research on software testing. 
To understand dependent tests, we described one of the first studies on
real-world dependent tests. We showed that 
test dependence \textit{does} arise in practice, and could 
have non-trivial repercussions. We also
formalized the dependent test detection
problem. To detect dependent tests, we designed
and implemented three algorithms to identify manifest test dependence
in a test suite. Our experiments revealed
dependent tests in every subject program
we studied, from both human-written and automatically-generated test
suites. The revealed dependent tests interfere with
five existing test prioritization techniques.
Our tool is publicly available at
\url{https://testisolation.googlecode.com/}.
% \todo{XXX}.

Our findings are of utility to practitioners and researchers.
Both can learn that test dependence is a real problem that should not be
ignored any longer, because it leads to false positive and false negative
test results.
Practitioners can adjust their practice based on what code patterns most
often lead to test dependence, and they can use our tool to 
find dependent tests.
Researchers are posed important but challenging new problems, such as how
to adapt testing methodologies to account for dependent tests how to detect
and correct all dependent tests.

%\enlargethispage{5pt}

As future work, we plan to study
the impact of dependent tests on other
downstream testing techniques, such as test selection and
test parallelization.
We also plan to develop
a general methodology to eliminate dependent tests.
Another future direction is investigating how to
prevent dependent tests.
%
One possible way is encouraging developers to
use advanced testing frameworks that support test dependence~\cite{testng},
so that developers can explicitly specify test
dependence when writing tests.
%
Stylized coding patterns may also be useful. Developers
should be encouraged to write tests ``defensively'' by
specifying necessary test execution pre-conditions and
using less (or properly mocking) global variables or shared resources. 

%There is already some work aiming at automating this
%process to prevent the potential
%for dependences by refactoring programs to use
%less global state~\cite{wlokaetal:FSE:2009}. 

%This question also applies to automated test generators.
%While there is some work to alleviate
%this problem~\cite{vmvm, RobinsonEPAL2011,fraseretal:ISSTA:2011}, the question
%of removing automatically-generated dependent tests
%still remains open.

\begin{comment}
Our future work should focus on the following directions:

\vspace{1mm}

\noindent \textbf{{Investigating impact of dependent tests
on other downstream testing techniques.}}
We have shown that dependent tests can compromise the application of
five existing test prioritization techniques.
%assumption is not true. However,
We plan to conduct more comprehensive empirical studies to
measure the impact of dependent tests to other
downstream techniques, such as test selection,
test parallelization, test factoring, experimental
debugging techniques, and mutation analysis.
We are also interested in how to enhance these
testing techniques to support test dependence.
%
%Another open question is how
%testing techniques should handle test dependence.
One straightforward way 
might be to augment such techniques to respect a
defined partial order among tests. This partial order
can be derived from knowledge about dependent tests,
or can be detected by our \ourtool tool.
%Like contrived examples of test
%dependence itself, it is easy to produce simple examples where
%downstream techniques produce incorrect output when applied to dependent
%tests.
%under the assumption that the input tests have no dependences.
%However, 



\vspace{1mm}

\noindent \textbf{{Eliminating dependent tests.}}
As reflected in our study (Section~\ref{sec:study}),
the practice of eliminating dependent tests
remains mostly manual and ad hoc --- software developers
usually manually hardcode test
execution orders in a configuration file or
simply merge or remove tests.
A more flexible and robust methodology for
dependent test elimination should be developed.
This question also applies to automated test generators.
While there is some work to alleviate
this problem~\cite{vmvm, RobinsonEPAL2011,fraseretal:ISSTA:2011}, the question
of removing automatically-generated dependent tests
still remains open.

%\todo{Do not forget to check words: subset, and subsequence through the paper.  Use them properly.}

%On the other hand, 
%almost all automated test generation
%techniques we are aware of produce tests
%that are hard to read for humans, are undocumented, and their intent
%cannot easily be gleaned from naming conventions and other aids
%developers normally use. Therefore, it requires more effort
%from developers to identify the root cause of dependence
%and then remove the dependence. While there is some work to alleviate
%this problem~\cite{fraseretal:ISSTA:2011}, the question
%of eliminating  automatically-generated dependent tests
%still remains open.


%As discussed in our experiments, it appears that test
%dependence in automatically generated test suites is 
%even more troublesome than in human-written suites. 


\vspace{1mm}


\noindent \textbf{{Preventing dependent tests.}}
%Detecting dependent tests is not obvious in most
%cases. Thus, a natural question is how could
%software developers prevent dependent tests when
%writing testing code.
One possible way is encouraging developers to
use advanced testing frameworks that support test dependence~\cite{testng},
so that developers can explicitly specify test
dependence when writing tests.
%However, using different testing frameworks may
%bring up the backward-compatibility issue to the existing tests.

Stylized coding patterns can also be useful. Developers
should be encouraged to write tests ``defensively'' by
specifying necessary test execution pre-conditions and
using less (or properly mocking) global variables or shared resources. 
There is already some work aiming at automating this
process to prevent the potential
for dependences by refactoring programs to use
less global state~\cite{wlokaetal:FSE:2009}. 

\end{comment}

%The source code of our tool implementation is publicly
%available at: \url{http://testisolation.googlecode.com}.


\balance

%\begin{comment}
\tinysqueeze
\section{Acknowledgments}
 Bilge Soran participated in the project
that led to the initial result.  
Reid Holmes and Laura Inozemtseva identified a \jodatime dependence.  Cheng Zhang suggested exploring software issue tracking systems
to study dependent tests. Yuriy Brun, Colin Gordon, Mark Grechanik, Adam Porter, Michal
Young, Reid Holmes, and anonymous reviewers provided insightful comments on a draft.
This work was supported in part by NSF grants
CCF-1016701 and CCF-0963757. 
%\end{comment}

\bibliographystyle{abbrv}
\bibliography{references}


%\subsection{Practical Considerations}
%\label{sec:practical}
%In principle, there is no \emph{ground truth} for the order of test
execution.
Therefore, we assert that the
\emph{programmer-defined} execution order, and consequently the test
results from executing the test suite in that order, are the ground
truth for our experiments.
%would naturally serves 
%the ``truth'' for our definitions of test dependence, and records
%the results from that execution order as intended results (line 2, Figure~\ref{fig:dtalgorithm}).

When a dependent test is identified, programmers may wish to know
a minimal list of other tests on which the identified test depends. 
Given an execution sequence that manifests the dependence, Delta
Debugging (also implemented in our tool) can
be used to return a shortest subsequence 
that still manifests the dependence~\cite{Zeller:2002}. 
%to minimize the recorded
%test list before the dependent test was executed.
%\todo{SZ}{is it clear? or need more explanation?}


In practice, another possible way to help detect potential dependent tests is
to leverage programmers' domain knowledge or employ some program analyses
to identify a subset of tests that are likely to contain dependent tests,
and run the algorithm only on that subset instead of the whole suite.

\todo{sz}{need a summary sentence here for the whole section 5.}

%\todo{DN}{I'm on the side of removing DD if reasonable, and coming back
%to the idea later, maybe in a ``practical considerations'' section/subsection}
%In addition, the algorithm employs Delta debugging~\cite{Zeller:2002}
%to minimize the test set that are executed before a test in
%an execution (lines 7--8). Together with the minimized
%dependent test set, the test revealing with different behaviors
%are added to the output (line 9).


%\todo{JW}{We should mention that we used the tool to find/verify the
%examples. We should also mention that isolation corresponds to $k=1$
%and we did pair-wise (corresponding to $k=2$).}
%\todo{JW}{
%For reverser execution, as far as I remember, we didn't use it to for
%the actual examples we have. But we might claim that it is useful for
%identifying particular kinds of deps. But it would be better if we had
%an example for that.}



%\section{Unused text snippets}
%

\begin{itemize}

\item \todo{KM}{I kind of understand what this paragraph is saying.
However the many minor mistakes in the writing make it very hard to
follow up.} Patterns of dependent tests.
In many cases, there is a \textit{N -- 1} \todo{KM}{The first time I read this,
I read as N minus 1, which is the incorrect way. Maybe write down as ``N to 1''}
dependence relationship, in which $N$-th \todo{KM}{Is ``th'' really needed?} distinct tests depends\todo{KM}{This should be either ``tests depend on'' (more likely) or ``test depends on'', but I couldn't decide} on the same test, which probably is used to set up the environment. For
such cases, the 1 depend test \todo{KM}{``1 dependent test'' or ``first
dependent test''} should be moved to the common \CodeIn{setUp}
\todo{KM}{Consistency: consider @Before} method.
Less frequently, there is a \textit{1 -- N} dependence relationship
\todo{KM}{Have we ever seen this, or is this purely theoretical?}, in which one
test depends on $N$ tests to set up its testing environment.  In one subject, a newly-added test changes the shared variable state of an existing test. Although the newly-added test is executed after the existing test and reveals the same behavior when executing in isolation, the existing test exhibit different behavior if it is executed after the newly-added test.


\item 
\todo{JW}{If we really want to discuss this, this should be connected
to the theory section, as all this follows from theory. A ``finding''
might be that these differences actually matter in practice. And I
don't think we checked that.}
Different techniques have their own strength in detecting dependent tests. We
have investigate three methods (i.e., executing in isolation, executing in a
reversed order\todo{KM}{Did we (do we) really do this?}, and executing in
k-permutation) to identify dependent tests, and found each method complements others. There exist certain tests that can only be found by one method
but missed by the other two.  Executing tests in isolation found more dependent tests
than executing in a reversed order, and executing every $k$-permutation is
infeasible in practice due to the exponentially large number of possible combinations.

\end{itemize}


\end{document}
% vim:wrap:wm=8:bs=2:expandtab:ts=4:tw=70:


%  LocalWords:  Soran Cheng Grechanik Michal CCF '14

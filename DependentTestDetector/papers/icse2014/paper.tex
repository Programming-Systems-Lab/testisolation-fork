\documentclass[letterpaper]{sig-alternate}
\usepackage{graphicx}
\usepackage[draft]{fixme}
\usepackage{url}
\usepackage{amsfonts}
\usepackage{amssymb}
\usepackage{amsmath}
\usepackage{algorithmic}
\usepackage{booktabs}
\usepackage{listings}
\usepackage[T1]{fontenc}
\usepackage{lmodern}
\usepackage{color}
\usepackage[draft]{hyperref}
\usepackage{xspace}
\usepackage{subfigure}
\usepackage{bold-extra}

\usepackage{color}
\input{macros.tex}
%\newcommand{\CodeIn}[1]{{\small\texttt{#1}}}

%\newcommand{\todo}[1]{{\color{red}\bfseries [[#1]]}}
\newcommand{\todo}[1]{{\color{red}\bfseries [[{#1}]]}}
\newcommand{\edit}[1]{{\color{blue}\bfseries [[{#1}]]}}
%\newcommand{\CodeIn}[1]{{\small\texttt{#1}}}
\newcommand{\code}[1]{{\small\texttt{#1}}}
 % Add line between figure and text
\makeatletter
\def\topfigrule{\kern3\p@ \hrule \kern -3.4\p@} % the \hrule is .4pt high
\def\botfigrule{\kern-3\p@ \hrule \kern 2.6\p@} % the \hrule is .4pt high
\def\dblfigrule{\kern3\p@ \hrule \kern -3.4\p@} % the \hrule is .4pt high
\makeatother
 % If there is a line, you can get away with reducing the separation between
 % figures and text.  Don't do this without the line, though.
\addtolength{\textfloatsep}{-.5\textfloatsep}
\addtolength{\dbltextfloatsep}{-.5\dbltextfloatsep}
\addtolength{\floatsep}{-.5\floatsep}
\addtolength{\dblfloatsep}{-.5\dblfloatsep}

\newtheorem{definition}{Definition}
\newtheorem{theorem}{Theorem}
\newtheorem{corollary}{Corollary}
%\newtheorem{proof}{Proof}
\newcommand{\pass}{\ensuremath{\mathit{PASS}}\xspace}
\newcommand{\fail}{\ensuremath{\mathit{FAIL}}\xspace}
\newcommand{\dtnum}{{{104}}\xspace}
\newcommand{\repnum}{{{5}}\xspace}
\newcommand{\ourtool}{DTDetector\xspace}

% commands for formalization
\newcommand{\suites}[0]{\ensuremath{\mathcal{S}\xspace}}
\newcommand{\environs}[0]{\ensuremath{\mathcal{E}\xspace}}
\newcommand{\manifest}[1]{\ensuremath{\prec_{#1}}}
\newcommand{\suite}[1]{\ensuremath{ \langle  #1 \rangle }}
\newcommand{\env}[0]{\ensuremath{\mathbf{E}}\xspace}
\newcommand{\exec}[2]{\ensuremath{\varepsilon(#1,#2)}}
\newcommand{\result}[2]{\ensuremath{R(#1|#2)}}

\newcommand{\question}[2]{\medskip\noindent\hspace{\parindent}\begin{minipage}{0.9\columnwidth}\textit{#1}
\textsc{#2}
\end{minipage}
\medskip}


%commands must be the last package imported
% This package provides the \todo{Name}{Comment} command
%\usepackage{commands}
%remove syntax highlighting from Java code
\lstset{basicstyle=\ttfamily,tabsize=2,keywordstyle=\ttfamily,stringstyle=\ttfamily,commentstyle=\ttfamily,
captionpos=b,numberstyle=\small\ttfamily,numbersep=1ex,
keywordstyle=\color{red}}

%\renewcommand{\todo}[2]{}

\author{
Author names\\ 
\affaddr{Department of Computer Science \& Engineering}\\ 
\affaddr{University of Washington, Seattle, USA} \\ 
\email{\{emails\}@cs.washington.edu}
}

\title{Understanding and Detecting Dependent Tests}
\subtitle{\todo{Is this title good after reading the paper structure?}
\todo{One issue is that the term ``dependent tests'' may not be known to
  readers; putting jargon in the title is not a good way to communicate the
  message nor to draw people into the paper.  Can we think of words that
  will convey the ideas more effectively?  Something about ``conflicting
  tests'', or ``conflicts within a test suite'', or the like??}}

\begin{document}
\maketitle

\begin{abstract}
Test dependence arises when executing a test in different environments
causes it to return different results. It has been broadly assumed
that tests in a suite are inherently independent. This paper
investigates the validity of this assumption, and presents four results. 

First, we describe a study of \todo{xx}
real-world dependent tests from \todo{xx} bug repositories
to show that test dependence arises in practice.
Our study shows that test dependence can have potentially costly
repercussions such as masking program faults and leading
to spurious bug reports, and can be hard to identify
unless explicitly searched for.

Second, we formally define test dependence in terms of
test suites as ordered sequences of tests along with explicit
environments in which these tests are executed. We use this
formalization to formulate the problem
of detecting dependent tests, and prove that a useful special
case is NP-complete. 

Third, guided by the study, we propose two algorithms to detect
dependent tests in a test suite. 
%Our algorithms use both static and
%dynamic program analyses to quickly \todo{the goal
%of algorithms here.}

Fourth, we implement our dependent test detection algorithms
in a prototype tool, and apply it to \todo{xx} real-world programs.
Our tool revealed a large number of unknown dependent tests, 
indicating that on average\todo{xx}\% of the human-written tests are
dependent and \todo{xx}\% of the automatically-generated tests
are dependent.

\end{abstract}


%\category{D.2.5}{Testing and Debugging}
%\keywords{Software testing; test dependence; test selection; test prioritization}


\section{Introduction}

%\todo{Is ``dependent test'' a term that can be applied to one test in
%  isolation?  Or is a pair of tests dependent on one another if the result
%  of one of them changes if the other one is run first?  The paper doesn't
%  make this clear anywhere in sections 1 and 2, and this leads to some
%  confusion for the reader.}

%Informally, a \emph{dependent test} produces different test
%results when executed in different environments. 
Consider a test suite containing two tests \code{A}
and \code{B}, where running \code{A} and then \code{B} leads
to \code{A} passing, while running \code{B} and then
\code{A} leads to \code{A} failing. We call \code{A}
an \textit{order-dependent} test (in the context of this test suite), since its result depends on
whether it runs after \code{B} or not.



In a test suite, all the test cases should be independent:
no test should affect any other test's result, and
running the tests in any order should produce the same test results.
%Practitioners are well aware of test dependence:  coding
%guidelines~\cite{unit-test-def,Massol:2003} and
%standards~\cite{IEEE:829-1998,IEEE:829-2008} say to avoid or document it,
%and tools support those goals~\cite{junitordering,depunit,testng, easymock, randomjunit}.
%%\todo{Cite a mocking framework}.
%Researchers are also aware of test
%dependence~\cite{Csallner:2004, Steimann:2013, Gray:1994:QGB:191843.191886,Chays:2000:FTD:347324.348954,kapfhammeretal:FSE:2003,Wang:2007:AGC, Samimi:2013:DM}.
%%\todo{Cite research about creating mocks.}
%Nonetheless, much testing research and practice
%assumes test independence;
%this includes techniques for test selection~\cite{harroldetal:OOPSLA:2001,RenCR2006},
%test prioritization~\cite{Elbaum:2000:PTC:347324.348910},
%and test parallelization~\cite{Misailovic:2007}.
%% These are not offected:
%%, test factoring~\cite{Saff:2005}, and test carving~\cite{Elbaum:2006}.
%theoretical results, algorithms, and tools may behave unexpectedly
%in the presence of dependent tests.
%A test suite that contains dependent tests affects the applications
%of these techniques.
%These techniques produce incorrect results if run on a test suite that contains dependent tests. 
%\todo{The claim of the above sentence is too strong, and may irritate readers, given we do not have strong evidence.
%I would tone down it as: dependent tests can
%affect the results of these techniques. Further, the ``correctness''
%of a test selection/prioritization technique is decided by whether
%a test has been selected or not, rather than whether the test should
%maintain the same resutls or not.}
The assumption of of test independence 
is important so that tests behave consistently
as designed. In addition, 
many techniques assume test independence, including test
prioritization~\cite{Elbaum:2000:PTC:347324.348910, Kim:2002:HTP:581339.581357, Rummel:2005:TPR:1066677.1067016, Srivastava:2002:EPT:566172.566187, Jiang:2009:ART},
test selection~\cite{harroldetal:OOPSLA:2001, Orso:2004:SRT,
Briand:2009:ART, Zhang:2012:RMT, Nanda:2011:RTP, hsu09may},
test execution~\cite{Kim:2013:OUT, Misailovic:2007, SPLAT},
test factoring~\cite{Saff:2005, Wu:2010:LRV}, test carving~\cite{Elbaum:2006},
and experimental debugging techniques~\cite{Zeller:2002,
Steimann:2013, Zhang:2013:IMF}.
%often implicitly assume no test dependences in
%a test suite. 
\todo{check the following sentencee, people may ask
is the above paper list representative enough? be aware.}
\todo{I think it would be more compelling for this section to only cite
  papers that do not mention but implicitly assume test independence.  The
  current writing sounds like you might be cherry-picking:  why this set of
  papers?  Are the ratios representative?  But if you just give a list (as
  long as possible) of papers that don't assume it, then that makes the
  point you want to, and no one will misinterpret the list as trying to be
  representative.}
However, this critical assumption is
rarely questioned, investigated, or even mentioned:
in the above paper list,
4 papers explicitly assumed test independence~\cite{Rummel:2005:TPR:1066677.1067016, Orso:2004:SRT, harroldetal:OOPSLA:2001, Steimann:2013},
14 papers did not mention but implicitly
assumed test independence~\cite{Elbaum:2000:PTC:347324.348910,
Jiang:2009:ART, Srivastava:2002:EPT:566172.566187,
Zhang:2012:RMT, Misailovic:2007, Elbaum:2006, Saff:2005,
Zeller:2002, Briand:2009:ART, Wu:2010:LRV, Kim:2013:OUT,
Zhang:2013:IMF, Nanda:2011:RTP, hsu09may 
},
1 paper considered violation of the test independence
assumption as a threat to validity~\cite{SPLAT},
and 1 paper acknowledged the potential dependences
between tests~\cite{Kim:2002:HTP:581339.581357}. 
Anecdotally, researchers have told us that test dependence
is not a significant concern.
We wish to investigate the validity of this unverified conventional wisdom,
in order to understand whether test dependence arises in practice, 
the repercussions of dependent tests, and how to 
detect dependent tests.

%is generally ignored.
%Is this acceptable, because test dependence does
%not arise in practice?
%Is it because even when test dependence arises, there are few
%negative repercussions?
%Is it because no one has studied this problem or thought to examine it?
%Is it because the problem is important but is too hard to analyze or understand?

\subsection{Manifest Test Dependence}

%To explore these questions, 
This paper focuses on test
dependence that manifests as a
difference in test result (i.e., passing or failing) as determined by the testing oracle.
We adopt the results of the default
%% True, but a distraction.
% , usually implicit,
order of execution of a test suite as the
expected results; these are the results that a developer sees when running
the suite in the standard way. A test is dependent when there exists a possibly
reordered subsequence of the original test suite, in which
the test's result (determined by its existing testing
oracles) differs from its expected result in the
original test suite.
%
That is, manifest test dependence
%we focus on a \emph{manifest} perspective of test dependence,
requires a concrete order of the test suite that
produces {different} results than expected.  
%
%



This paper uses \textit{dependent test} as a shorthand for
\textit{manifest order-dependent test}
unless otherwise noted.
A single test may consist of setup and teardown
code, multiple statements, and multiple assertions
distributed through the test.



%\subsection{Causes of Dependent Tests}

\subsection{Causes and Repercussions}
%\todo{I merged the two original subsections into the following one. looks better?}
%\todo{also some text editing below}
%\subsection{Repercussions of Dependent Tests}
\label{sec:intro-repercussions}

%\todo{Fold this section into Section~\ref{sec:intro-repercussions}??}

Test dependence results from interactions with other tests,
as reflected in the execution environment.
Tests may make \textit{implicit} assumptions about their
execution environment -- values of global variables,
contents of files, etc. A dependent test
manifests when another alters the execution
environment in a way that invalidates those assumptions.

%A test has the potential to yield
%different test results when executed in different environments
%--- global variables with different values, differences in the file system, etc.

Why does this happen?
%As
%suggested by the principle of unit testing~\cite{Greiler:2013:SAT, Massol:2003},
Each test ought to initialize (or mock) the execution environment
and/or any resources it will use.
Likewise, after test execution, it should reset the
execution environment and external resources
to avoid affecting other tests' execution.
However, developers sometimes
%are as likely to
make mistakes when writing tests as when they are writing other code.
Even though frameworks such as
JUnit provide ways to set up the environment for a test execution and clean
up the environment afterward,
they cannot ensure that it is done
properly. This means that tests, like other code,
will have unintended and unexpected behaviors in some cases.
%And
%as programs increase in complexity, so may tests, which may
%increase the frequency of such problems in tests, which may
%in turn increase the frequency of test dependence.
%\edit{check the grammar of the above sentence}



%\todo{Where does this paragraph belong?  Here?}
%This principle is adopted and confirmed by many
%real-world developers (Sections~\ref{sec:study} and~\ref{sec:expdiscussion}).
%When it needs to interact with the execution environment,
%it should mock or carefully resetting external resources
%
%Ideally, each test should not depend on its environment, because it
%initializes any resources it will use 
%Likewise, the test should not modify its environment, because of mocks or
%resetting resources after test execution. 
%




% Our study of \dtnum real-world, confirmed dependent
% tests 
% % from \repnum software issue tracking systems
% (Section~\ref{sec:study}) identified two 
Here are three consequences of the fact that a dependent
test gives different results depending on when it is executed
during testing.

\textbf{(1)}
Dependent tests can
\emph{mask faults in a program}. Specifically, executing a test suite in the
default order does not expose the fault, whereas
executing the same test suite in a different order does. 
% We found a
One bug~\cite{clibug} in the Apache CLI library~\cite{cli}
was masked by two dependent tests
for 3 years (Section~\ref{sec:repercussion}).

\textbf{(2)}
Test dependences can lead to \emph{spurious bug reports}.
When a dependent test fails, it usually represents
%\edit{change to use the word of "represents", good? using
%"reveals" may sound like dependent test is good to improve code quality}
a weakness in the test
suite (such as failure to perform proper initialization) rather than a bug
in the program. 
When a test should pass but
fails after reordering due to the dependence,
people who are not aware of the dependence can get confused
and might report bugs.
% about the failing test,
%even though this is exactly the intended behavior.
%Programmers made these errors even though frameworks such as
%JUnit provide ways to set up the environment for a test execution and clean
%up the environment afterward.
%
As an example, the Eclipse developers
investigated a bug report~\cite{eclipsebug} in SWT for
more than a month before realizing that the 
bug report was invalid and was caused by test dependences
(i.e., a test should pass, but it failed when a user
ran tests in a different order).
%were intentional,
%allowing them to close the bug report without a change to the system.
%


%Second, guided by the findings of our study, we design two algorithms
%to detect manifest dependent tests. By applying our algorithms
%to \todo{xx} open-source programs and their test suites, we 
%found a large number of unknown dependent tests, .

\textbf{(3)}
Dependent tests can \textit{interfere with downstream testing
techniques} that change a test suite and thereby change a test's execution environment.
Examples of such techniques include
test selection techniques (that identify a subset of
the input test suite to run during
regression testing)~\cite{harroldetal:OOPSLA:2001, Orso:2004:SRT,
Briand:2009:ART, Zhang:2012:RMT, Nanda:2011:RTP, hsu09may},
test prioritization techniques (that reorder the
input to discover defects sooner)~\cite{Elbaum:2000:PTC:347324.348910, Kim:2002:HTP:581339.581357, Rummel:2005:TPR:1066677.1067016, Srivastava:2002:EPT:566172.566187, Jiang:2009:ART},
test parallelization techniques (that schedule the input tests for execution across multiple
CPUs)~\cite{Misailovic:2007}, test execution techniques~\cite{Kim:2013:OUT},
test factoring~\cite{Saff:2005, Wu:2010:LRV} and test carving~\cite{Elbaum:2006} (which
convert large system tests into smaller unit tests),
%test generation (which re-executes suites as it builds them up)~\cite{PachecoE2005,RobinsonEPAL2011},
experimental debugging techniques (such as Delta
Debugging~\cite{Zeller:2002, Steimann:2013, Zhang:2013:IMF} and mutation
analysis~\cite{Zhang:2012:RMT, Schuler:2009:EMT, Zhang:2013:FMT},
which run a set of tests repeatedly), etc. 
Most of these downstream testing techniques implicitly assume that
there are no test dependences in the input test suite. Violation of
this assumption, as we show happens in practice, can cause unexpected
output. %\todo{change erroneous to different?} 
As an example, test prioritization may produce a reordered sequence
of tests that do not
return the same results as they do when executed in
the default order. Section~\ref{sec:impact}
provides empirical evidence to show that
dependent tests do affect the output of five test prioritization
techniques.


\subsection{Contributions}
\label{sec:contributions}

This paper addresses and questions
conventional wisdom about the test independence assumption. 
This paper makes the following contributions:

\begin{itemize}

  \item \textbf{Study.} We describe a study of \dtnum real-world
  dependent tests from \repnum software issue tracking
  systems to characterize dependent tests that
  arise in practice.  Test dependence can have
  potentially non-trivial repercussions and can be hard to identify
  (Section~\ref{sec:study}).

\item \textbf{Formalization.} We formalize test dependence
  in terms of test suites as ordered sequences of tests and explicit execution
  environments for test suites.  The formalization enables reasoning about test dependence
  as well as a proof that finding manifest dependent tests is an NP-complete
  problem (Section~\ref{sec:formalism}).

  \item \textbf{Algorithms.} We present three algorithms
  to detect dependent tests:
  one randomized, one exhaustive bounded, and one that prunes the search
  space using dynamic analyses.
  All three algorithms are \emph{sound} but \emph{incomplete}:
  every dependent test they identify is real, but the algorithms
  do not guarantee to find all dependent tests (Section~\ref{sec:detecting}). 
  %\edit{check above when the algorithm section is written}

  \item \textbf{Evaluation.} We implemented our algorithms in a prototype
  tool, called \ourtool (Section~\ref{sec:impl}).
  \ourtool detected 27 previously-unknown dependent tests in human-written
  unit tests in \subjnum real-world subject programs.
  % (and even more in automatically-generated tests).
  The developers confirmed all of these as
  undesired (Section~\ref{sec:evaluation}).

  %\item \textbf{Assessment.} 
  \item \textbf{Impact Assessment.} We implemented five test prioritization
  techniques and evaluated them on \subjnum subject programs
  that contain dependent tests. The results show that all
  five test prioritization techniques are affected by dependent tests
  (Section~\ref{sec:evaluation}).

  % \textit{every} subject program we studied, from both  and automatically-generated
  % unit tests (Section~\ref{sec:evaluation}).
  %been discovered before, showing that on average \todo{xx}\% of the human-written
  %unit tests are dependent and \todo{xx}\% of the automatically-generated
  %unit tests are dependent
  %Finally, we discuss a set of open questions and other possible impacts of dependent
  %tests in Section~\ref{sec:discussion}.
\end{itemize}


%  LocalWords:  Kapfhammer Soffa subsequence SWT CLI NUM dependences

%  LocalWords:  teardown


%\section{Dependent Tests in Practice}
\section{Real-World Dependent Tests}
\label{sec:study}

To investigate whether dependent tests can reach beyond theory
and appear in real-world programs, this section presents an empirical
study of concrete examples of test dependence found in
well-known open source programs. 


\subsection{Sources and Study Methodology}

We chose five well-known, public-accessible software issue
tracking systems to examine: Apache~\cite{apachebug},
Eclipse~\cite{eclipsebug}, JBoss~\cite{jbossbug},
Hibernate~\cite{hibernatebug}, and Codehaus~\cite{codehausbug}.
Each issue tracking system serves tens of projects, and
holds thousands of bug reports, feature requests, improvement
suggestions, etc.

For each issue tracking system, we searched for a set of keywords
(``dependent test'', ``test dependence'', ``test execution order'',
etc.\todo{for concreteness and reproducibility, list every search term}), and manually examined the matched results. For each match, we read the
description of the issue report, the discussions between reporters
and developers, and the fixing patches (if available). This information
helped us understand whether the report is about test dependence
--- a test manifesting different behaviors under different
test execution orders. Each dependent test candidate was examined by
at least two people and the whole process consisted of several
rounds of (re-)study, cross checking, and efforts to reproduce. We ignored reports
that are described vaguely, and excluded tests whose results are
affected by non-determinism (e.g., multi-threading).
In total, we have spent more than 6 person-months to collect and analyze
the dependent tests. 


\subsection{Findings}

\begin{table*}[t]
\vspace{1mm}
\centering
\small{
\setlength{\tabcolsep}{.60\tabcolsep}
\begin{tabular}{|l||l|l|l|l||l|l|l||c|c|c||c|c|c|}
\hline
%1&2&3&4&5&6&7&8&9&10&11&12&13&14\\
\textbf{Issue}&\multicolumn{4}{|c||}{\textbf{Identified Dependent Tests}}&\multicolumn{3}{|c||}{\textbf{Dependence Manifestation}}&\multicolumn{3}{|c||}{\textbf{Dependence Resolution}}&\multicolumn{3}{|c|}{\textbf{Dependence Root Cause}}\\
\cline{2-5}\cline{9-14}
\textbf{Tracking} &Total&\multicolumn{3}{|c||}{Severity}&\multicolumn{3}{|c||}{\#Involved Tests}&
Average&\multicolumn{2}{|c||}{Patch Location}&Static&File & Execution\\
\cline{3-8}\cline{10-11}
\textbf{System}&Number&Major&Minor&Trivial&1&2&>=3&Days&Code&Test&Variable&System&Env\\
\hline
Apache&2&3&4&5&6&7&8&9&10&11&12&13&14\\
\hline
Eclipse&2&3&4&5&6&7&8&9&10&11&12&13&14\\
\hline
JBoss&2&3&4&5&6&7&8&9&10&11&12&13&14\\
\hline
Hibernate&2&3&4&5&6&7&8&9&10&11&12&13&14\\
\hline
Codehuas&2&3&4&5&6&7&8&9&10&11&12&13&14\\
\hline
\hline
\textbf{Total} &2&3&4&5&6&7&8&9&10&11&12&13&14\\
\hline
\end{tabular}
}
\vspace{-2mm}
\caption{{\label{tab:studyresults} Summary of dependent tests
found in 5 issue tracking systems. Column ``Total Number'' shows
the total number of all identified dependent tests. Column ``Severity''
classifies dependent tests based on its impact to the program,
as decided by developers in the report. Column ``\#Invovled Tests''
classifies dependent tests by the the number of tests needed
to manifest the dependence. Column ``Average Days'' shows the
average days needed by developers to resolve a dependent test.
Column ``Patch Location'' shows how developers fix a
dependent tests. Column ``Code'' shows the number of dependent
tests fixed by modifying program code, and column ``Test'' shows
the number or dependent tests fixed by modifying test code.
Column ``Dependence Root Cause'' classifies dependent tests by
its root causes, including inappropriate accessing shared static
variables, file systems, and other execution environment (e.g.,
language, locale, or other system properties).
}
}
\end{table*}


Table~\ref{tab:studyresults} summarizes the dependent tests.


\subsubsection{Characteristics}

\todo{We may be able to reduce the length of this section.}

We summarize three characteristics of dependent tests:
manifestation, root cause, and developers' reaction.

\vspace{1mm}
\noindent \textbf{{Manifestation: at least 82\% of the dependent
tests in the study can be manifested by no more than 2 tests.}}
A dependent test is manifested if we produce a test suite that is a 
subset of the original suite, such that the test fails.
We measure the size of the subsuite.
If the test fails when run in isolation, the number of tests to manifest
the dependent test is 1.
If the test fails when run after one other test (often, the subsuite is
running these two tests in the opposite order as the full original test
suite), then the number of tests to manifest the dependent test is 2.

% In theory,
% given a $n$-sized test suite, dependent test can occur in any
% length of permutations. However, among \dtnum collected tests,
% 86 (82\%) of them can be manifested by running no more than
% 2 tests. 

\todo{Discuss the ``unknown'' column.  What happened?  Could we not
  reproduce it at all?  Why not?}

\vspace{1mm}
\noindent \textbf{{Root cause: at least 52\% of the dependent tests
in the study arise because of the improper access to shared static
variables.}} Among \dtnum dependent tests, 58 (52\%) of them
arise due to inappropriate access to
shared static variables; and 13 (12\%) of them arise
due to inappropriate access to file systems or other
execution environment. 

\vspace{1mm}
\noindent \textbf{{Developers' reactions: dependent tests
often indicate flaws in the test code, and developers usually
ignore dependent tests due to the lack of tool support.}}
In some cases, they are intentional, developers are aware
of them and document them, but in other cases they are
inadvertent. Among \dtnum collected dependent tests,
97 (93\%) of them were treated as major or minor problems,
but only 34 (32\%) of them got fixed by developers. Among
the 34 fixed dependent tests, only 7 fixes are
on the program code, while the other fixes are on the
tested code. This indicates that dependent tests usually
reveal potential flaws in the test code rather than the test code.
Based on developers' discussion, we found that although
developers admitted that such test dependence should be removed,
they often leave the dependence unresolved, sometimes by merging
two tests or adding explanatory documentation.
\todo{It seems to me that merging two tests does resolve the test
  dependence, in the sense that there is no longer any order in which the
  new tests can be run such that they fail.  Why do you call this leaving
  the dependence unresolved?  Also, there are so many items in the
  ``unfixed'' column that I think you should subdivide it further, for
  example into ``merged tests'', ``documentation'', and ``unfixed'' (for
  those the developers really did nothing).  Are there other possibilities?}
The primary
reason is that the current JUnit testing framework does not
support to explicitly specify test dependence in the test code.
\todo{Does it support specifing an order in which to run all tests?}


%Test dependence can cause problems, not only
%when test suites are reordered, but even when they are
%executed in the intended order.



\subsubsection{Repercussions of Dependent Tests}
\label{sec:repercussion}

\begin{table}
\centering
\setlength{\tabcolsep}{0.45\tabcolsep}
\begin{tabular}{|c||c|c|}
%\toprule
\hline
\textbf{Issue Tracking System} & \textbf{False Alarm} & \textbf{Missed Alarm} \\
\hline
Apache &24 & 2 \\
\hline
Eclipse & 59 & 0 \\
\hline
JBoss& 6 & 0 \\
\hline
Hibernate & 3 & 0 \\
\hline
Codehaus & 2 & 0 \\
\hline
\hline
\textbf{Total}  & 94 & 2 \\
\hline
\end{tabular}
\caption{
Manifestation of the \dtnum dependent tests.
}
\label{tab:reper}
\end{table}

%  LocalWords:  JBoss Codehaus


We found three major consequences of dependent tests (Table~\ref{tab:reper}).
%  We describe each category below and give concrete examples. 

\vspace{1mm}

\noindent \textbf{Poor Test Construction.} Most identified
dependent tests fall into this category. The test dependences
arise due to incorrect initialization of program state by one
or more tests, and reveal flaws in the test suite itself
rather than the tested code. Typically, one test initializes
a global variable or the execution environment; and another
test does not perform any initialization, but
relies on the program state after the first test's execution.
%global variable that is a part of the environment, but the test does
%not properly initialize it.  In the second case, a test should but
%does not call
%an initialization function before later invocations to a complex library.
Such dependence in the test code is often masked because
the initializing test always executes before other tests in the
default execution order. The dependent tests are revealed
until the initializing test is reordered to execute
\textit{after} other tests. 

%the default test execution order includes tests that initialize the library.  The defect is
%inconsequential until and unless the flawed test is reordered, either manually or by
%a downstream tool, to execute before any other initializing test.

\vspace{1mm}

\noindent \textbf{Spurious Bug Reports}
Sometimes developers introduce dependent tests intentionally because it is
easier, more efficient, or more convenient~\cite{kapfhammeretal:FSE:2003, whittakeretal:2012}.
%DB-testing}.
Even though the developers are aware of these dependences
when they create tests, this knowledge can get lost, 
and other people who are not aware of these dependences can get confused 
when they run a subset of the test suite that manifests the
dependences.

As a result, software users or maintainers
might report bugs about the failing tests, even though this
is exactly the expected behavior. 
If the dependence is not documented clearly and
correctly, it can take a considerable amount of time to work out that
these reported failures are spurious.
%Or worse, the developers may try
%to fix a bug that is not there.
For example,
in September 2003, a user filed a
bug report in SWT~\cite{swt}\footnote{\url{https://bugs.eclipse.org/bugs/show_bug.cgi?id=43500}},
stating that tests were failing unexpectedly
if she runs any other test before \texttt{TestDisplay} --- 
a test suite creates a new \code{Display} object and tests it.
However, this bug report was spurious and was
caused by undocumented test dependence.
Its root cause is quite simple: in SWT, only one global \texttt{Display}
object is allowed; the tests that reporters try to run
create, but do not dispose of a \code{Display} object, while
the tests in \code{TestDisplay} attempt to create
a new \code{Display} object, which fails, as one
is already created. This is the desired behavior,
and points to a potential problem in the test suite rather
than the code.

\todo{What is the relationship between the categories?  It seems that the
  above example belongs in ``poor test construction'' as well as ``spurious
  bug reports''.  In fact, doesn't \emph{every} dependent test belong in
  ``poor test construction''?  Why or why not?}


\vspace{1mm}

\noindent \textbf{Masking Faults}. In rare cases,
dependent tests can hide a fault in the
program, \emph{exactly} when the test suite is executed in its default
order. Masking occurs when a test case $t$ \emph{should}
reveal a fault, but tests executed before $t$ in a test suite always
generate environments in which $t$ does not reveal the fault.


\begin{figure}[t]
\noindent \textbf{\small{Fault-related code in CLI:}}
\vspace{-2mm}
\begin{CodeOut}
\begin{alltt} 
public final class OptionBuilder \{
  \textbf{private static String argName;}
  private static void reset() \{
    ...
    \textbf{argName = "arg";}
    ...
  \}
  public static Option create(String opt) \{
    ...
    \textbf{OptionBuilder.reset();}
    ...
  \}
\}
\end{alltt}
\end{CodeOut}
\textbf{\small{Two dependent tests that mask faults:}}
\vspace{-2mm}
\begin{CodeOut}
\begin{alltt} 
  public void test13666() \{
    ...
    OptionBuilder.create("");
    ...
  \}
  public void testOptionWithoutShortFormat2() \{
    assertEquals(argName, "arg");
  \}
\end{alltt}
\end{CodeOut}
\vspace*{-13pt}
\caption{Simplified code and two dependent tests 
that mask faults in CLI~\cite{cli}. \todo{not finished,
need explanation here}
}
\label{fig:option_builder}
\end{figure}


We only found two dependent tests in
the Apache CLI library~\cite{cli} for this category.
In CLI, two test cases 
\code{Bugs\-Test.test13666} and \code{Help\-For\-mat\-ter\-Test.test\-Op\-tion\-With\-out\-Short\-For\-mat2}
fail when run in isolation,
but both pass when run in the default order.

Figure~\ref{fig:option_builder} shows the simplified code and
tests. Both dependent tests can reveal this fault,  but
the default order of test execution makes both tests pass
accidentally. Such dependent tests
have a non-trivial impact in practice.
This fault is reported in the bug
database several times,\footnote{\url{https://issues.apache.org/jira/browse/CLI-26} \url{https://
issues.apache.org/jira/browse/CLI-186} \url{https://issues.apache.org/jira/browse/
CLI-187}} starting on March 13, 2004 (CLI-26). The report was marked as resolved
\emph{three years} later on March 15, 2007 when developers
realized the test dependence.

%, but is then reopened as CLI-186 on
%July 31, 2009. About one month later, the bug is duplicated as
%CLI-187, and the actual fix happens one 
%year later on June 19, 2010, about six years after the bug was first reported (and four years
%total on the open-issue list).


%On this report, one of the developers commented:
%\begin{quote}
%I reproduced the issue, it requires a dedicated test case since it is tied to the initialization 
%of a static field in OptionBuilder.
%\end{quote}

%Despite the realization that a dedicated test is required, no such
%test was ever created.

%\paragraph{Eclipse SWT: Causing Spurious Bug Reports}


\subsubsection{Implications}

We summarize the main implications of our findings.

\noindent \textbf{{Dependent tests exist in practice, but
they can be hard to identify unless explicitly searched for.}}
None of the dependent tests we studied is identified by
running the existing test suite in the default order. Often,
the test dependence is not reported until and unless the
test suite is reordered, either manually by a user or
a maintainer or by a downstream tool. This indicates that
specialized tools to detect such dependence are needed.

\vspace{1mm}
\noindent \textbf{Dependent test detection techniques
can bound the search space to a small number of tests.}
In theory, it is necessary
to exhaustively execute all $n!$ permutations of a $n$-sized
test suite to detect all dependent tests. This is
not feasible for realistic $n$.  Our study demonstrates that
most dependent tests can be manifested by executing
no more than 2 tests.  Exhaustively executing all 
short subsequences of a test suite is tractable.

\vspace{1mm}
\noindent \textbf{Dependent test detection techniques
should focus on analyzing static variable accesses.}
More than half of the studied dependent tests are caused
by static variable access. Thus, focusing on analyzing
static variables can be a cost-effective way for a
tool design. \todo{need to re-write the above}

%dependent tests.
%can bound Tools


%\vspace{1mm}
%\noindent \textbf{Dependent test fixing tool
%Test dependence reveals flaws in the test code.}
%This indicates that a potential dependent test fixing tool should target
%the test code


\subsection{Threats to validity}

Our findings apply in the context of our study and methodology and may not
apply to arbitrary programs.
The applications we studied are widely used and have comprehensive test suites.
However, a limitation is that they are all written in 
Java and have JUnit test suites.  

We accepted the developers' judgment regarding which tests are dependent
and the severity of each test.  We did not intentionally ignore
any test dependence in the issue tracking system.
However, a limitation is that the developers might have made a mistake,
might not have marked a test dependence in a way we found it, and may not
have found all the dependent tests in those projects. 

%We believe the dependent tests in our study
%provide a representative sample in these software applications.

% In our study, we do not emphasize any quantitative characteristic
% results, and most of our findings are consistent across
% the examined dependent tests.

%  LocalWords:  JBoss Codehaus reproducibility multi dependences SWT CLI
%  LocalWords:  TestDisplay test13666 subsequences



\section{Formalizing Test Dependence}
\label{sec:formalism}

%\todo{need a transition sentence here.}
%A standard textbook states that ``[a] test case includes not only input data but
%also any relevant \emph{execution conditions}
%\dots''~\cite[p.~152, emphasis added]{pezze-young:2007}.   

The result of a test not only depends on
its input data but also its \emph{execution conditions}.
To characterize the relevant execution conditions, 
our formalism represents the notions of
(a) the order in which test cases are executed and (b) the environment in which a test suite is executed.  


\subsection{Definitions}
\label{sec:definitions}

We express test dependences through the results of executing
\emph{ordered} sequences of tests in a given \emph{environment}.


\begin{definition}[Environment]
An \emph{environment} \env for the execution of a test
consists of all values of global variables, files,
operating
system services, etc. that
can be accessed by the test and program code exercised by the test
case.
\end{definition}

We use $\env_0$ to represent the initial environment, such
as a fresh JVM initialized by frameworks like JUnit
before executing any test.


\begin{definition}[Test]

A test is a sequence of executable program statements, and an oracle
--- a boolean predicate that
decides whether a test passes or fails.
\end{definition}

%Simplifying from Staats
%et al.~\cite{staatsetal:ICSE:2011}, and without loss of generality,
%we consider an oracle to be a boolean predicate over tests and environments.

%While oracles in practice, and specifications in theory, play an
%important role in testing, we do not incorporate them in our
%formalism, because explicit specifications often do not exist, and for
%our purposes the oracle judgement, rather than its full definition, is sufficient.

\begin{definition}[Test Suite]
A test suite\/ $T$ is an $n$-tuple (i.e., ordered sequence) of tests
\suite{t_1, t_2, \dots, t_n}.

%When it is clear which test suite we are talking about, or the details
%of the suite are not important, we use $T$ to denote the entire test
%suite $(t_1, \dots, t_n)$.
\end{definition}

\begin{definition}[Test Execution]
Let\/ \alltests\ be the set of all possible
tests and\/ \environs\ the set of all possible
environments.
The function\/ ${f}: \alltests \times \environs \rightarrow
\environs$ is called test
execution. $f$ maps the execution of a test\/ $ t \in
\alltests$ 
in an environment\/ $\env \in \environs$ to a new (potentially updated)
environment\/ $\env' \in \environs$.

For the execution of a test suite\/ $T = \suite{t_1, t_2, \dots, t_n}$
we use the shorthand\/
$\exec{T}{\env}$ for $\exec{t_n}{\exec{t_{n-1}}{\dots \exec{t_1}
{\env} \dots }}$.
\end{definition}

\begin{definition}[Test Result]
The result of a test $t$ executed in an environment\/ $\env$,
denoted\/ \result{t}{\env}, is defined by the test's oracle
and is either \pass or \fail.

The result of a test suite\/ \suite{t_1,\dots,t_n}, executed in an
environment\/ \env, denoted\/ \result{\suite{t_1,\dots,t_n}}{\env} is a
sequence of results\/ \suite{o_1,\dots,o_n} with $o_i \in \{\pass,\fail\}$.

%For test outcomes of sequences where all individual outcomes are
%either \pass or \fail, we use the notation $(\pass^*)$ and $(\fail^*)$,
%respectively.

For example, $\result{\suite{t_1, t_2}}{\env_1} = \suite{\fail, \pass}$ represents that 
$t_1$ fails given the environment\/ $\env_1$, and\/ $t_2$ passes.
\end{definition}



%\begin{definition}[Potential Test Dependence] \label{def:dependency}
%Given a test suite\/ $T$,
%a test\/ $t_l \in T$ is \emph{potentially dependent} on test\/ $t_k
%\in T$, if and only if\/
%$\exists \env : \result{T}{\env} = \suite{o_1,\dots, o_n} \wedge
%\result{\suite{t_k,t_l}}{\env} = \suite{o_k, o_l} \wedge
%\result{t_l}{\env} = \neg o_l$.
%We write\/ $t_k \prec t_l$ when\/ $t_l$ is potentially dependent on\/ $t_k$.
%\end{definition}

%\begin{definition}[Potential Test Dependence] \label{def:manifest}
%Given a test suite\/ $T$, two tests\/ $t_i, t_j \in T$
%are \textit{potentially} dependent, if\/ $\exists {S \subseteq T}: t_i, t_j \in S \wedge
%$\exec{S, t_i, t_j}{\env}$ \neq $\exec{S, t_j, t_i}{\env_0}$.
%$We write\/ $t_i \prec t_j$ for potential dependence.\footnote{$S \subseteq T$ means that $S$ is a subsequence of
%$T$.}
%\end{definition}

%\todo{check the above potential test dependence definition.}
%Two tests are potentially dependent when they are accessing the
%same memory location or external file systems during execution.
%Thus, executing them in different orders can lead to different envrionments. However,
%such environment differences may not be checked and then manifest
%through the test code.

\todo{SZ: I am not satisfied with the following definition. In
particular, the definition of S1, S2}

\begin{definition}[Manifest Dependent Test] \label{def:manifest}
Given a test suite\/ $T$, a test $t \in T$ is a
manifest dependent test if $\exists {S_1 \subseteq T - t}$
and ${S_2 \subseteq T - t}$: \result{t}{f(S_1, \env_0)} $\neq$
\result{t}{f(S_2, \env_0)}.
\footnote{$S \subseteq T - t$ means that $S$ is a subsequence of $T$
\todo{need to emphasize S is an ordered tuple}
and $t \notin S$.}
\end{definition}

%We refine this definition of dependence to require a concrete environment guaranteed
%to \emph{manifest} a dependence:

%\begin{definition}[Manifest Test Dependence] \label{def:manifest}
%Given a test suite\/ $T$, two tests\/ $t_i, t_j \in T$,
%the dependence\/ $t_i \prec t_j$ \emph{manifests} in a given
%environment\/
%$\env$ if\/ $\exists {S \subseteq T}: t_i, t_j \in S \wedge
%\result{T}{\env}
%= \suite{o_1, \dots, o_n} \wedge \result{S}{\env} =
%\suite{\dots,o_i,o_j} \wedge \result{t_j}{\env} = \neg o_j$. We
%write\/ $t_i \manifest{\env} t_j$ for manifest dependence.
%%\footnote{$S \subseteq T$ means that $S$ is a subsequence of $T$.}
%\end{definition}

%Note that the dependent tests $t_i$\/ and $t_j$ do not have to be
%adjacent in the original test suite, but that they must be adjacent in
%the shortest test suite that manifests the dependence.

%The intuition behind manifest dependences is that in practice we
%do not construct arbitrary environments to execute tests in. Rather,
%we use the natural environment $\env_0$ provided by frameworks such as JUnit,

The definition of manifest dependent test (for short, dependent test)
focuses on environment modified by reordering test execution.
A dependent test manifests only
if there is a sequence of tests $S^*$ whose execution
$\exec{S^*}{\env_0}$  produces the
environment $\env'$ that will make a test exhibit a
different result than in its default execution order.


\subsection{The Dependent Test Detection Problem}

From a practical perspective, techniques that affect the ordering of
test suites must respect dependences. Otherwise, their results might
not be interpreted correctly. Detecting
dependences in existing test suites, though appears straightforward,
is actually a non-trivial problem.
In the following, we first give a precise definition of the problem of
detecting dependent tests, and then prove that this problem
is NP-complete. 


\begin{definition}[Dependent Test Detection Problem]
Given a set suite\/ $T = \suite{t_1, \dots, t_n}$ and an initial environment\/
$\env_0$, for a given test\/ $t \in T$, is $t$ a dependent test?
%there a test suite\/ $S
%\subseteq T$ that manifests a test dependence involving\/ $t_i$? 
\end{definition}

We prove that this problem is NP-hard by reducing the NP-complete Exact Cover problem
to the Dependent Test Detection
problem~\cite{karp:NP:1972}. 
Then we provide a linear time algorithm to verify any answer to the
question.
%Then we sketch an exponential
%time algorithm that can solve the problem.
Together these two parts prove the the Dependent Test Detection Problem is NP-complete.

\begin{theorem}
The problem of determining a dependent test in
a test suite is NP-hard.
\end{theorem}

\begin{proof}
%We prove this claim by reducing Exact Cover to Dependent Test
%Detection.
In the Exact Cover problem,
we are given a set $X$ = \{$x_1, x_2, x_3, \dots, x_m$\} and a collection $S$ of subsets of $X$.
The goal is to identify a sub-collection $S^*$ of $S$ such that \textit{each}
element in $X$ is contained in \textit{exactly} one subset in $S^*$.  

Assume a set $V = \{v_1, v_2, v_3, \dots, v_m\}$ of variables,
and a set $S = \{S_1, S_2, \dots, S_n\}$ with $S_i \subseteq V$ for $ 1\leq i
\leq n$. 

We now construct a tested program $P$, and a test suite
$T = \suite{t_1, t_2, \dots t_n , t_{n+1}}$ as follows:

\begin{itemize}

\item $P$ consists of $m$ global variables 
$v_1, v_2,\dots, v_m$, each with initial value 1.

\item 
For $1 \le i \le n$, $t_i$ is constructed as follows:
for $1 \le j \le m$, if $x_j \in S_i$, then adding a
single assignment statement \CodeIn{$v_j$ = $v_j$ - 1} to $t_i$.

$t_{n+1}$ consists only of the oracle
\CodeIn{assert($v_1$ != 0 || $v_2$ != 0 \dots || $v_m$ !=0)}.

\end{itemize}

In the above construction, the tests $t_i$ for $1 \le i \le n$ 
will always pass. The only
test that may fail and thus exhibit different behavior is $t_{n+1}$, which 
\emph{only} fails when each variable $v_i$ appears exactly
once in a test case.

For the given test $t_{n+1}$, if we can
find a sequence \suite{t_{i_1}, t_{i_2},\dots, t_{i_j}}
that makes $t_{n+1}$ fail, the subsets $S^*$ corresponding
to each $t_{i_j}$ are an exact cover of $V$.
\end{proof}

In practice, the structure of the proof directly translates to the
structure of test suites. $t_{n+1}$ is the dependent test, $S$ is
defined by the tests that write variables used by $t_{n+1}$, and every
exact cover of $S$ represents an independent shortest test suite that
is a manifest dependency of $t_{n+1}$.

To complete the proof that Dependent Test Detection is NP-complete, we
provide an algorithm to verify solutions to the problem, that is
linear in the size of the test suite.
Given a test suite $T$, a test $t \in T$ and a sequence
$S \subseteq T$ that manifests a dependency on $t$, we first execute $T$, then $S$, and
compare the result for $t$ in both executions. 
If the results differ the solution is correct, if they do not differ,
the solution is rejected.
Since in the worst case we have to execute $2n$ tests, the complexity
of this algorithm is linear. 



\subsection{Discussion}
\label{sec:formaldiscussion}

This formalism aims to lay a foundation for reasoning about
test dependence in a precise way. It only considers
deterministic tests, and excludes tests whose results
might be affected non-determinism such as thread scheduling.
For the sake of simplicity, our formalism excludes the case that a test 
can depend on itself. Further, our study of real-world dependent
tests indicate that self-dependent tests occur very rare
in practice.


%Third, the manifest test dependence problem is NP-complete;
%although that is daunting (but less so than undecidability),
%approximate algorithms can be defined for large classes of NP-complete
%problems.  


%The potential for test dependence arises from the test structure and
%the oracle:
%%from the test results: 
%what
%global state do the tests read and write, and does that global state contribute
%to the computed result evaluated by the oracle?  
%At the same time, the \emph{potential} for a test dependence
%is realized only if 
%the values derived from the context \emph{actually} affect the test results.
%affect program state that
%is checked by the oracle can a dependence on the environment affect
%the outcome of a test.
%This potential \emph{manifests} when test results differ between
%executions in different environments.

%The fact that manifestation of test dependence depends on both test
%structures and test results means that dependences can silently propagate
%through sequences of tests before they become apparent.

%The abstract examples above, and the concrete examples presented 
%in Section~\ref{sec:examples} share some common features that
%ultimately lead to dependences. 
%%All applications and libraries we studied 
%They rely on global variables to some extent, and the
%tests that check behavior that depends on these variables usually
%assume these variables to be in some state. This state is typically
%defined by the default execution order of the test suite, and rarely
%established explicitly before each test.

% vim:wrap:wm=8:bs=2:expandtab:ts=4:tw=70:



\section{Detecting Dependent Tests}
\label{sec:detecting}

\newcommand{\smalltrialnum}{10\xspace}
\newcommand{\mediumtrialnum}{100\xspace}
\newcommand{\trialnum}{1000\xspace}

\newcommand{\testlist}[0]{\ensuremath{T^k_i}}
\newcommand{\executeTestsInOrder}[1]{\result{#1}{\env_0}}

An exhaustive search would execute all $n!$
permutations of the test suite to detect dependent tests.
However, this is not feasible for realistic $n$.
Since the general form of the dependent test detection problem is
NP-complete, we do not expect to find an efficient algorithm for it.

To approximate the exact solution, this section
presents three approximate algorithms that find a \textit{subset} of
all dependent tests.
%In this section we present two algorithms to detect dependent
%tests. 
Section~\ref{sec:randomized} describes a randomized algorithm
that repeatedly executes all the tests of a suite in random order.
Section~\ref{sec:basic} describes an exhaustive bounded algorithm that
executes all possible sequences of $k$ tests for a bounding parameter $k$.
Section~\ref{sec:advalgorithm} describes a dependence-aware $k-$bounded algorithm.
The dependence-aware algorithm dynamically collects the fields that each test
reads or writes, and uses such collected information to reduce the search space.
All algorithms are \textit{sound} but \textit{incomplete}:
every dependent test they find is real, but they do not
any guarantee to find every dependent test.

\subsection{Randomized Algorithm}
\label{sec:randomized}

\begin{figure}[t]
\textbf{Input}: a test suite $\mathit{T}$\\
\textbf{Output}: a set of dependent tests $\mathit{dependentTests}$\\
\vspace{-5mm}
\begin{algorithmic}[1]
\STATE $\mathit{dependentTests}$ $\leftarrow$ $\emptyset$
\STATE $\mathit{expectedResults}$ $\leftarrow$ $\result{T}{\env_0}$
\FOR{each $\mathit{ts}$ in getPossibleExecOrder($\mathit{T}$)}
\STATE $\mathit{execResults}$ $\leftarrow$ $\result{ts}{\env_0}$
\FOR{each test $\mathit{t}$ in $\mathit{ts}$}
\IF{$\mathit{execResults}$[$\mathit{t}$] $\neq$ $\mathit{expectedResults}$[$\mathit{t}$]}
\STATE $\mathit{dependentTests}$ $\leftarrow$ $\mathit{dependentTests}$ $\cup$ $\mathit{t}$
\ENDIF
\ENDFOR
\ENDFOR
\RETURN $\mathit{dependentTests}$
%\ENDWHILE
\end{algorithmic}

% getPossibleExecOrder($T$, $k$): returns a set of test suites, each of size
% $\le k$; each suite is composed of tests selected from $T$ without replacement.\\

\vspace{-3mm}
\caption {The algorithm to detect dependent tests.
The getPossibleExecOrder function is instantiated
by different algorithms in Figures~\ref{fig:randalgorithm},
~\ref{fig:exhaustivealgorithm}, and~\ref{fig:impralg}.
}
\label{fig:basealgorithm}
\end{figure}


\begin{figure}[t]
getPossibleExecOrder($T$):\\
\vspace{-5mm}
\begin{algorithmic}[1]
\FOR{$i$ in 1..$\mathit{numtrials}$}
\STATE \textbf{yield} shuffle($T$)
\ENDFOR
\end{algorithmic}

\vspace{-3mm}
\caption {The randomized algorithm to detect dependent tests.
It uses the algorithm of Figure~\ref{fig:basealgorithm}, re-defining
the getPossibleExecOrder function.
Our experiments use $\mathit{numtrials} = \smalltrialnum,
\mediumtrialnum, \trialnum$.}
\label{fig:randalgorithm}
\end{figure}


Figure~\ref{fig:randalgorithm} shows the algorithm.
Given a test suite $T = \suite{t_1, t_2, \ldots, t_n}$, this algorithm
first executes $T$ with its default order
to obtain the \emph{expected result} of each test (line 2).
Then, it randomizes the original
test execution order (line 3), and then executes each test
again to observe its result (line 4). The algorithm checks
whether the result of any test differs from the
expected result (lines 5--8). 

In our implementation, this algorithm repeatedly
shuffles the test suite, and executes each test until
no more dependent tests are identified within a
pre-defined number of iterations (default: \smalltrialnum iterations).
Our experiments use $\smalltrialnum,
\mediumtrialnum, \trialnum$ as the iteration numbers.

%\todo{How did we choose 10 iterations?  It might be good to try with some
%  different number such as 20 and report that there was no difference.
%  More generally, when choosing an arbitrary number, it adds credibility to
%  either give a justification for the number or to do an experiment to show
%  that the number is a reasonable choice.}


\subsection{Exhaustive Bounded Algorithm}
\label{sec:basic}

\begin{figure}[t]
\textbf{Auxiliary methods}:\\
kPermutations($T$, $k$): returns all $k$-permutations of $T$; that is, all
sequences of $k$ elements of $T$ without repetition

\medskip

getPossibleExecOrder($T$):\\
\vspace{-5mm}
\begin{algorithmic}[1]
\RETURN kPermutations($T$, $k$)
\end{algorithmic}

\vspace{-3mm}
\caption {The exhaustive $k$-bounded algorithm to detect dependent tests.
It uses the algorithm of Figure~\ref{fig:basealgorithm}, re-defining the
getPossibleExecOrder function.
%Our experiments use $k=1$ and $k=2$ to bound the length of
%test execution. 
} 
\label{fig:exhaustivealgorithm}
\end{figure}


%To detect all possible dependent tests, 

This algorithm uses the findings of our study
(Section~\ref{sec:study})
that most dependent tests can be found by running only short
subsequences of test suites. For example,
in our study, \pertange of the real-world dependent tests
can be found by running no more than 2 tests together.
Instead of executing all permutations of the
whole test suite, our algorithm (Figure~\ref{fig:exhaustivealgorithm})
executes all $k$-permutations for a bounding
parameter $k$.
By doing so, the algorithm reduces
the number of permutations to execute
to $O(n^k)$, which for small $k$ is tractable. 


Given a test suite $T = \suite{t_1, t_2, \ldots, t_n}$, our algorithm
first executes $T$ with its default order
to obtain the \emph{expected result} of each test (line 2). 
It then executes every $k$-permutations of tests,
and checks whether the result of any test differs
from the expected result (lines 3--10). Finally, the algorithm returns the set
of all tests $t_i \in T$
that exhibit different results.





\subsection{Dependence-Aware Bounded Algorithm}
\label{sec:advalgorithm}

\begin{figure}[t]
\textbf{Auxiliary methods}:\\
recordFieldAccess($ts$): returns fields accessed (i.e., read and written) by each test in an ordered sequence of tests $ts$. \\
computeLastWrites($ts$, $reads$, $write$): for each field read by each test in $ts$, identify the last test in $ts$ that writes to the field.\\

\vspace{-2mm}

getPossibleExecOrder($T$):\\
\vspace{-5mm}
\begin{algorithmic}[1]
\STATE $\langle reads, writes\rangle$ $\leftarrow$ recordFieldAccess($T$)\\
\COMMENT{$\mathit{reads}$ and $\mathit{writes}$'s data type:
  Map$\langle$Test, Set$\langle$Field$\rangle$$\rangle$
% which maps a test to a set of fields.
}
\STATE $\mathit{lastWrites}$ $\leftarrow$ computeLastWrites($T$, $\mathit{reads}$, $\mathit{writes}$)
\\\COMMENT{$\mathit{lastWrites}$'s data type: Map$\langle$Test, Map$\langle$Field, Test$\rangle$$\rangle$, which maps a test to fields it reads and
the corresponding last test that writes each field.}
\STATE $\mathit{result}$ $\leftarrow$ $\emptyset$
\FOR{each $\mathit{ts}$ in kPermutations($\mathit{T}$, $\mathit{k}$)}
\STATE $\mathit{lastWriteTS}$ $\leftarrow$ computeLastWrites($ts$, $\mathit{reads}$, $\mathit{writes}$)
%\vspace{-3mm}
\IF{$\exists$ Test $\mathit{t} \in ts$ such that $\mathit{lastWrite}$[$t$] $\neq$ $\mathit{lastWriteTS}$[$t$]  }
\STATE $\mathit{result}$ $\leftarrow$ $\mathit{result} \cup \mathit{ts}$
\ENDIF
\ENDFOR
\RETURN $\mathit{result}$
\end{algorithmic}

\vspace{-3mm}
\caption {The dependence-aware $k$-bounded algorithm to detect dependent tests.
It uses the algorithm of Figure~\ref{fig:basealgorithm}, re-defining the
getPossibleExecOrder function.
%Our experiments uses $k=1$, \todo{experiment setting, how
%large k it can scale to.}. 
} 
\label{fig:impralg}
\end{figure}

\begin{figure}
\centering

\strut \hspace{-25mm} Initially: \code{x = y = 1;}

\subfigure{
\begin{minipage}{.43\columnwidth}
\code{Test1 \\  if(x == 0) \{ y = 0;\}\\ }
\end{minipage}
}
\subfigure{
\begin{minipage}{.43\columnwidth}
\code{Test2\\ assert y == 1;}
\vspace{0.9em}
\end{minipage}
}
\subfigure{
\begin{minipage}{.43\columnwidth}
\code{Test3\\ x = 0;}

\vspace{0.5em}
\end{minipage}
}
\subfigure{
\begin{minipage}{.43\columnwidth}
\code{Test4 \\  if(x == 0) \{ y = 1;\}\\ }
\end{minipage}
}

\strut \hspace{-3mm} Field read and write information by each test when executing
in the order of Test1, Test2, Test3, and Test4.
\vspace{1mm}

\begin{tabular}{|c|l|l|l|l|}
%\toprule
\hline
\textbf{Global Fields } & \textbf{Test1} & \textbf{Test2} & \textbf{Test3}& \textbf{Test4}\\
\hline
\code{x} & Read & & Write& Read\\
\hline
\code{y} & & Read & &Write \\
\hline
\end{tabular}

\vspace{4mm}

Number of permutations that need to be executed.
\setlength{\tabcolsep}{1.3\tabcolsep}
\begin{tabular}{|c|c|c|}
%\toprule
\hline
\textbf{$k$ value} & \textbf{Exhaustive } &
\textbf{Dependence-Aware }  \\
\hline
1 & 4 & 1\\
\hline
2 & 12 & 5\\
\hline
3 & 24 & 18 \\
\hline
4 & 24 & 20 \\
\hline
\hline
\textbf{Total} & 64 & 44 \\
\hline
\end{tabular}


\Caption{Example tests to illustrate the dependence-aware
$k$-bounded dependent test detection algorithm (Figure~\ref{fig:impralg}).}
\label{fig:rwexample}
\end{figure}

%The exhaustive $k-$bounded algorithm cannot scale to a realistic test suite.
%For example, it would take months to 
%execute all 3-permutations
%in JodaTime's test suite (3875 tests, Table~\ref{tab:subjects}).

The dependence-aware $k$-bounded algorithm
detects the same number of dependent tests
as the exhaustive $k$-bounded algorithm (when using the same $k$),
but it uses dynamic analyses to improve efficiency.
It avoids executing test
permutations that are guaranteed to have the
same results as being executed in the default order.
The intuition of our algorithm is that,
given a test permutation, for \textit{each}
test, if \textit{all global fields} it reads
are written by the \textit{same} test as
in the default execution order, all tests in
the permutation are guaranteed to
have the same results and the permutation can be safely ignored. 

The algorithm in Figure~\ref{fig:impralg} redefines
the getPossibleExecOrder function in Figure~\ref{fig:basealgorithm}.
The redefined getPossibleExecOrder function first records
fields that each test reads and writes (line 1).
Then, for each field read by a test,
the function identifies the last test that writes
that field (line 3).
When generating all possible test permutations
up to length $k$, the algorithm checks whether
\textit{all} fields \textit{each} test (in the generated permutation)
may read are written by the same tests as in the
default order (lines 5--8). If so, all tests in the permutation
must produce the same results, and the algorithm discards
this permutation without executing it. Otherwise,
the function adds the generated permutation to the result
set (line 7), and runs the algorithm in Figure~\ref{fig:basealgorithm}
to identify dependent tests. %among it.
Due to space limit, we omit the proof of the correctness
of the dependence-aware $k$-bounded algorithm. Interested
readers can refer to~\cite{proof-dependence-aware} for details.

The dependence-aware $k$-bounded algorithm has two major benefits.
First, it clusters tests by the fields they
read and write. Thus, only tests reading or writing
the same global field(s), rather than \textit{all} tests
in a suite, are treated as potentially dependent.
Second, even for tests reading or writing the same global
field(s), some permutations can still be discarded
without executing by checking the tests performing the last write.

%in which every test must
%reveal the same result to reduce the total number
%of permutations which need to be explored.

To illustrate the second benefit, Figure~\ref{fig:rwexample}
shows an example
consisting of 2 global fields and 4 tests. In the default execution
order, Test 1 reads the \textit{initial} value of \code{x};
Test2 reads the \textit{initial} value of \code{y}, and Test4
reads the value of \code{x} written by Test3.
Based on such recorded information, when creating permutations with $k=1$,
our algorithm determines that only
Test4 needs to be executed, since Test4, if executed in isolation,
will read the \textit{initial} value of \code{x} rather than the \code{x}
value written by Test3. Similarly, when creating permutations with $k=2$,
only 5 out of 12 permutations ($\langle$Test1, Test4$\rangle$,
$\langle$Test2, Test4$\rangle$, $\langle$Test4, Test1$\rangle$,
$\langle$Test4, Test2$\rangle$, and $\langle$Test4, Test3$\rangle$) need to be executed.
Due to space limit, we omit test permutations when
$k=3$ and $k=4$, and list the number of permutations that need
to execute in Figure~\ref{fig:rwexample}.

%\todo{illustrate why simply group tests by read/write is incorrect.}


%\todo{give an intuition of why it can improve scalability, since
%not every test will overlap with each other.}



%In case a test $t_2$ does not potentially \todo{?} depend on a previously executed test $t_1$, this means 
%that all permutations of tests in which $t_2$ follows $t_1$ do not need to be considered for potential \todo{?} dependency
%given that all tests on which $t_1$ potentially \todo{?} depends on are present as a prefix of that permutation, 
%in the same order that the test suite was initially executed, thus reducing the number of permutations which
%need to be explored.


\begin{comment}

\edit{I add a simple proof below, which I am not happy with}

We next proof the correctness of the dependence-aware
$k$-bounded algorithm.

\begin{theorem}
All tests in a test permutation discarded by the getPossibleExecOrder
function (Figure~\ref{fig:impralg}) are guaranteed to produce the same results as in its default order.
\end{theorem}

\begin{proof}
We prove this theorem by contradiction. Given a test
suite $T$ and a test permutation $P$ of $T$, let's assume
permutation $P$ is discarded by the getPossibleExecOrder function
since all fields read by each test in $P$ are written
by the same test as in $T$.


Suppose there exists a test $t \in P$ produces a different result
than in $T$. Test $t$ must read a different value
for at least one shared field. Without loss of generality,
let's suppose $f$ is a shared field for which $t$ reads
a different value. Based on our assumption, 
$f$ is written by the same test, denoted as $t'$, as in $T$.
Since the $f$'s value is different in $P$, $t'$
must read different value for other shared fields.
By induction, there must exist a test $t'' \in P$ that 
reads a different \textit{initialization} value
for at least one shared field. This contradicts
the assumption that tests are executed in the same environment.
\edit{the above proof is not clear, need to
think about how to make it shorter and clearer.}
\end{proof}
\end{comment}


\section{Tool Implementation}
\label{sec:impl}


We implemented our three dependent test detection algorithms in
a prototype tool, called \ourtool. \ourtool
supports JUnit 3.x/4.x tests. %, and is fully-automated.

\ourtool launches a fresh JVM when
executing a test permutation, ensuring there is no interaction between
different runs. When comparing the observed result of
a test in a permutation with its default result,
\ourtool considers two JUnit test results to be the same when the
tests either both pass, or exhibit exactly the same exception
(from the same line of code) or assertion violation.

%For the exhaustive $k$-bounded dependent test detection algorithm
%(Section~\ref{sec:basic}), \ourtool takes a test suite
%and the bounding parameter $k$ as inputs, 
%exhaustively executes every $k$-tuple
%of tests, and compares execution results to identify possible
%dependence tests.
 

To implement the dependence-aware $k$-bounded algorithm (Section~\ref{sec:advalgorithm}),
\ourtool uses ASM~\cite{asm} to perform load-time bytecode
instrumentation. Specifically, \ourtool inserts code to monitor each
static field access (including read and write). File accesses are
also monitored through installing a Java \code{SecurityManager} which provides 
read / write level information on files. Static field and file access are merged to produces a trace file for a \ourtool-instrumented test suite.

Mutable and immutable objects must be treated differently, as even a ``read'' access to a static
field will provide access to all the heap reachable from the object stored in that field. Such accesses must be treated as a read followed by a write. This issue seriously impacts the effectiveness of the dependency-aware algorithm. Classes known to only have immutable instances were provided as input to \ourtool. The instrumented Javari JDK provided by the Checker framework \cite{??}, classes marked as immutable by developer comments, and manual inspection of widely used classes (e.g., \code{java.awt.Color}, \code{java.util.Locale}) are the source of this information.

In practice many objects are not mutated even though this is not prohibited by the Java language (e.g., primitive arrays being used as constants). The algorithm was also instructed to ignore such fields, determined by manually inspecting fields which followed the Java naming convention of all-capital letters for constants, and fields commonly referenced by test code. 

In order to make the problem tractable, it was assumed that objects reachable from the static fields do not alias one another. This is important since in presence of aliasing, writing to one static field may effect others. This was not an issue in the subject programs we explored.

The other assumption made was that the JDK does not contain state itself, as the JDK was not instrumented. This is not entirely true and was an issue in \todo{explain this issue}. However, we believe this shortcoming is not fundamental and can be dealt with by  instrumenting static method calls to certain packages in the JDK.


%\ourtool also instruments
%calls to the \CodeIn{clone()} method and reflection
%methods, to record the objects they may create.
%\edit{a few sentences about security manager, and file accesses here}


%\edit{There are several sources of unsoundness in \ourtool's
%implementation. First, ...say the tradeoff in implementation here,
%such as dynamic class loading, native methods, etc....
%Despite these limitations, \ourtool works well in practice
%and identifies many real-world dependent tests (Section~\ref{sec:evaluation}).
%5}

%In addition, \ourtool implements the Delta debugging
%algorithm~\cite{Zeller:2002} to
%return the shortest sequence of tests that manifest a dependent test.
%This is useful to understand
%a manifested dependent test in a shuffled test suite.

The source code of \ourtool is available at:\\ \url{http://testisolation.googlecode.com}.

% vim:wrap:wm=8:bs=2:expandtab:ts=4:tw=70:

%\todo{one improvement space: distinguish different values}

%\todo{I prefer to talk about file systems, optimizations
%that use user-provided annotations, the prefix of a permutation,
%and use multiple executions to
%guide selection in the implementation section.}

%  LocalWords:  ASM tradeoff


\section{Empirical Evaluation}
\label{sec:evaluation}



\newcommand{\jodatimetests}{3875\xspace}
\newcommand{\xmlsecuritytests}{108\xspace}
\newcommand{\crystaltests}{75\xspace}
\newcommand{\synoptictests}{118\xspace}
\newcommand{\totaltests}{4176\xspace}

\newcommand{\jodatimeautotests}{2639\xspace}
\newcommand{\xmlsecurityautotests}{665\xspace}
\newcommand{\crystalautotests}{3198\xspace}
\newcommand{\synopticautotests}{2467\xspace}
\newcommand{\totalautotests}{8969\xspace}


\begin{table}
\centering
\setlength{\tabcolsep}{0.4\tabcolsep}
\begin{tabular}{|l|l|c|c|l|}
%\toprule
\hline
\textbf{Program} & \textbf{LOC} & \textbf{\#Tests} & \textbf{\#Auto Tests} & \textbf{Revision}
\\
\hline
%JFreechart & 92253 & \jfreecharttests & \jfreechartautotests& 1.0.15\\
%if we add JFreechart, need to change the total num, and num of subject program
%\midrule
\jt & 27183 & \jodatimetests
% 3875 is retrieved by running mvn test on the related revision
& \jodatimeautotests&  b609d7d66d\\
XML Security & 18302 & \xmlsecuritytests & \xmlsecurityautotests& version 1.0.4 \\ 
Crystal & 4676 & \crystaltests & \crystalautotests& trunk version\\
Synoptic & 28872 & \synoptictests & \synopticautotests&  trunk version\\ 
%\bottomrule
\hline
%\textbf{Total}& &  & &  \\ 
%\hline
\end{tabular}
\caption{Subject programs used in our evaluation.
Column ``\#Tests'' shows the number of human-written
unit tests. Column
``\#Auto Tests'' shows the number of 
unit tests generated by Randoop~\cite{PachecoLET2007}.
}
\label{tab:subjects}
\end{table}

%  LocalWords:  LOC Joda b609d7d66d

\newcommand{\unknown}{N/A\xspace}
\newcommand{\infy}{$\infty$\xspace}

\begin{table*}
\centering
\setlength{\tabcolsep}{0.12\tabcolsep}
\begin{tabular}{|l|c|C|C|C|c|c|c|c|c|c|c|c|c|c|c|c|}
%\toprule
\hline
\textbf{Subject} & & \multicolumn{7}{|c|}{\textbf{\#Detected Dependent Tests}} & \multicolumn{7}{|c|}{\textbf{Analysis Cost (second)}}\\
%\midrule
\cline{3-16}
\textbf{Programs} & \textbf{\#Tests} & \multicolumn{3}{|c|}{\textbf{Randomized}} & \multicolumn{2}{|c|}{\textbf{Exhaustive }} & \multicolumn{2}{|c|}{\textbf{Dependence-Aware}} & \multicolumn{3}{|c|}{\textbf{Randomized}} & \multicolumn{2}{|c|}{\textbf{Exhaustive }} & \multicolumn{2}{|c|}{\textbf{Dependence-Aware}} \\
%\cline{3-8}\cline{10-15}
& & \smalltrialnum & \mediumtrialnum & \trialnum& \; $k$=1 & $k$=2 & \quad $k$=1 \;\; \quad & $k$=2 & \smalltrialnum & \mediumtrialnum & \trialnum & \; $k$=1 & $k$=2 &  \quad $k$=1 \quad \quad & $k$=2  \\
\hline
%\bottomrule
\multicolumn{16}{|l|}{ }\\
\multicolumn{16}{|l|}{\textbf{Human-written unit tests} }\\
\hline
JodaTime & \jodatimetests & 1 & 1 & 6 & 2 &\unknown&& &   57 & 528 & 5538 &1265& \infy & &   \\
XML Security& \xmlsecuritytests & 1 & 4 & 4 &4 &4 & 4 & 4  &65 & 594 & 5977 & 106 &  11927 & 93 & 3322  \\
Crystal & \crystaltests & 18 & 18 & 18 &17&18&  & &14& 131 & 1304 & 166 & 7323 &   & \\
Synoptic & \synoptictests & 1 &1  & 1 & 0 &1 & &&  7 & 67 & 760& 25 & 3372&  &  \\
\hline
\textbf{Total} & \totaltests & 21&24&29& 23 &\unknown&  & &  143 & 1320 & 13579 &1562& \infy &   &  \\
\hline
\multicolumn{16}{|l|}{ }\\
\multicolumn{16}{|l|}{\textbf{Automatically-generated unit tests} }\\
\hline
JodaTime & \jodatimeautotests & 586 &815& 966 & 534 & \unknown&& & 131  & 1139 & 9000 & 448 & \infy & &  \\
XML Security& \xmlsecurityautotests& 167 & 171 & 171 & 129 &&  &  & 50 & 430 & 4174 & 133 & \infy & & \\
Crystal & \crystalautotests & 159 & 162 & 164 & 55 & \unknown& & & 103 & 949& 9436  & 2477 & \infy & & \\
Synoptic & \synopticautotests & 3 & 7 & 10 &2& \unknown& &  &81& 770  & 6311 & 454 & \infy & & \\
\hline
\textbf{Total} & \totalautotests &915&1155& 1311 & 720& & & &365 &3288 & 28921 & 3512 & \infy & & \\
\hline
\end{tabular}
\caption{Experimental results. Column ``\#Tests'' shows the total number
of tests, taken from Table~\ref{tab:subjects}. Column ``\#Detected Dependent Tests''
shows the number of detected dependent tests in each subject program.
Columns ``Randomized'', ``Exhaustive'' and ``Dependence-Aware'' show the results
of applying the randomized algorithm, the exhaustive $k$-bounded algorithm and the dependence-aware
$k$-bounded algorithms, respectively. 
When evaluating the randomized algorithm, we use $numtrials$ =
$\smalltrialnum, \mediumtrialnum, \trialnum$ in the algorithm (Figure~\ref{fig:randalgorithm}).
Column ``Analysis Costs (second)''
shows the time cost (in seconds) of each algorithm under
different settings.
}
\label{tab:results}
\end{table*}


%We evaluated two aspects of \ourtool's
%effectiveness, answering the following
%research questions:
Our evaluation answers the following research questions:

\vspace{-1mm}

\begin{enumerate}
\item How many dependent tests can each detection
algorithm detect in
real-world programs (Section~\ref{sec:detectedtests})?

\item How long does each algorithm in \ourtool take to detect dependent
tests (Section~\ref{sec:performance})?

\item Which algorithm is the most cost-effective one in detecting
dependent tests (Section~\ref{sec:algcomparison})?
%\item How does \ourtool's effectiveness compare to an alternative
%approach based on test execution order randomization
%(Section~\ref{sec:random})?
\end{enumerate}

\subsection{Subject Programs}


Table~\ref{tab:subjects} lists the programs and
tests used in our evaluation.

JodaTime~\cite{jodatime} is an open source
date and time library. It is a mature project that
has been under active development
for more than eight years. XML Security~\cite{xmlsecurity}
is a component library implementing XML signature and encryption
standards. XML Security is included in
the SIR repository, and has been used widely
as a subject program in the software testing community.
Crystal~\cite{crystal} is a tool that
pro-actively examines developers' code and
precisely identifies and reports on textual,
compilation, and behavioral conflicts.
Synoptic~\cite{synoptic} is a tool to mine a finite state
machine model representation of a system from logs.

Each subject program has a human-written JUnit test suite.
In addition, for each subject program, we use
Randoop~\cite{PachecoLET2007}, a state-of-the-art automated
test generation tool, to create a suite of 5,000 tests.
Randoop automatically drops textually-redundant tests 
and outputs a subset of the generated tests as
shown in Table~\ref{tab:subjects}.


\subsection{Evaluation Procedure}

We evaluate each algorithm 
on both the human-written test suite 
and the automatically-generated test suite
for each subject program in Table~\ref{tab:subjects}.

%We run the three algorithms proposed
%in Section~\ref{sec:detecting} on both
%human-written and automatically-generated test suites
%of each subject program.

We run the randomzied algorithm \smalltrialnum, \mediumtrialnum,
and \trialnum times on each test suite, and record
the total number of detected dependent tests and time cost
for each setting. The choice of \trialnum times is based
on the a practical guideline of using randomized algorithm
in software engineering, as summarized in~\cite{Arcuri:2011}.
%
For the exhausitive $k$-bounded algorithm
and the dependence-aware $k$-bounded algorithm,
we use isolated execution ($k = 1$), and
pairwise execution ($k = 2$). The choice of $k$ is
based on the results of our empirical
study (Section~\ref{sec:study}) that a small $k$
can find most realistic test dependences.

\todo{say a few sentences about the manual part, such as
listing the immutable fields. say the approximate manual time used here.}

Each output dependent test is examined manually to make
sure the test dependence is not caused by non-deterministic
factors, such as multi-threading.

Our experiments were run on a 2.67GHz Intel Core PC
with 4GB physical memory (2GB was allocated for the JVM),
running Windows 7.

\subsection{Results}

Table~\ref{tab:results} summarizes the number of detected
dependent tests and the time cost by each algorithm
in \ourtool.

\subsubsection{Detected Dependent Tests}
\label{sec:detectedtests}

A total number of 29 dependent tests are detected
from all human-written test suites, and 1311
dependent tests are detected from all automatically-generated
test suites. The percentage of dependent tests
in automatically-generated test suites is significantly
higher than in human-written test suites. We identified
two possible reasons: first, developers usually have deep
knowledge the intended purpose of a program when writing tests
for it. Such expertise helps them to build relatively well-structured and coherent
tests, e.g., tests that carefully initialize and destroy the
shared objects they may use. Second, 
 it is challenging for automated test generation tools to understand
 a program's intended purpose as well as how
specific parts of the program depend on the environment.
Thus, automated tools often do not explicitly generate code that sets up the
environment correctly.  In addition, to cover more
program states, automated tools often incrementally
builds tests on top of the program state after executing
other tests. This could further make a test's result depend on another.


%If, at the same time, other tests are
%generated that as a side e?ect create the needed environment, test dependence ensues



The randomized algorithm is surpringly effective in
detecting dependent tests. After running \trialnum times,
it identifies \textit{more} dependent tests than the other
two algorithms can find. For human-written
tests, the randomized algorithm detects 2 more dependent
tests in the JodaTime program. These two tests only
manifest when a sequence of three tests are run in a specified,
non-default order. Both exhaustive and dependence-aware $k$-bounded
algorithm fail to detect these two tests, because
they can not scale to $k$=3 for the
JodaTime program. Related, the randomized algorithm
detects significantly more dependent
tests in the automatically-generated test suites, for
two reasons. First,
both the exhaustive and dependence $k$-bounded
failed to scale to $k$=2 for 3 automatically-generated test suites;
and second, some detected dependent tests requires executing more than 2 tests
to manifest.

%Compared to the randomized algorithm, On the other hand,

\subsubsection{Performance of \ourtool}
\label{sec:performance}

The time cost of the randomized algorithm 
varies across different subject programs, and
is proportional to number of runs. Overall, the
time cost is acceptable for practical use.
For example, the randomized algorithm took around 1.5 hours
to finish 1000 runs,  for the largest human-written test
suite (\jodatimetests tests in JodaTime).
 
The time cost of running the exhaustive $k$-bounded algorithm
is prohibitive. The exhaustive algorithm failed to
scale to one human-written test suite and three automatically-generated
test suites when $k$=2, and failed to scale to all test suites
when $k$=3. The primary reason is due to the extremely large
number of all possible tuples. For example, running all JodaTime's \jodatimetests human-written
unit tests when $k$=2 requires running 15,011,750 test pairs, costing
approximately 1400 hours (58 days). 

The dependence-aware $k$-bounded algorithm helps improve
the efficiency of \todo{more results here.}

Table~\ref{tab:results} gives an estimated time cost for each
test suite that an algorithm failed to scale to. For each test suite,
we randomly chose 1000 samples from all possible
test tuples, executed them, and measured the time cost. Then,
we calcuated the average time cost of each sample, and multiples
the average cost with the number of all samples.



\subsubsection{Comparison of Algorithms}
\label{sec:algcomparison}

We next discuss the tradeoffs of choosing different detection
algorithms. Although the randomized algorithm
detects the most dependent tests in our subject programs,
it has several shortcomings that must be considered
in practice. First, there is no guarantee of
the randomized algorithm's results. A randomized
algorithm might even produce different results across runs,
and makes reproducing a dependent test harder.
Second, more importantly, there is no stop criteria
of using the randomized algorithm in shuffling and
running the test suite. Thus, it is hard for users
to know how many runs would be enough for a test suite.
Third, when a dependent test is identified, users
may need to inspect every test that is executed
before the dependent test, and isolate  a minimized
subsequence of tests (either
manually or using an assisting tool) to understand the dependence root cause.

By contrast, both exhaustive $k$-bounded and dependence-aware
$k$-bounded algorithms do not suffer from the above shortcomings.
One major factor that may prevent them being applied to a
large test suite might be the time cost for systematically
exploring all possible test tuples.


%\emph{three} tests to manifest. While these are easy to reproduce, we
%did not check that our tool finds them, because the time needed to
%run our naive algorithm on JodaTime with $k=3$ is measured in months.



\subsection{Discussion}
\label{sec:expdiscussion}

\subsubsection{Developers' Reactions}

We sent the identified human-written dependent tests to the
subject program developers, asking for their feedback.

One dependent test in JodaTime was previoulsy-known,
and had already been fixed. JodaTime's
developers confirmed the other two new dependent
tests, and thought that they is due to interactions
that are not intended in the design of the library.

The Crystal developers confirmed that all dependent tests
found in Crystal were not intentional and mostly likely
happended because they were not aware of the potential dependency
caused by global variables. The developers treat the
dependencies as undesirable and opened a bug report for
this report issue\footnote{\url{https://code.google.com/p/crystalvc/issues/ detail?id=57}}.

The Synoptic developers merged two related tests to fix
the dependent tests.

%After receiving our reported dependent tests in XML-Security,
The SIR~\cite{sir} maintainers confirmed our reported dependent
tests in XML-Security. They highlighted the practice
that tests should \textit{always} ``stand alone''
without dependency on other tests, and treated that as
``test engineering 101''. They also accepted our suggested
patch to fix the dependent tests.

\subsubsection{Threats to Validity}

There are several threats to validity in our evaluation.
First, the \subjnum open-source
programs and their test suites may not be
representative enough. Thus, we can not claim the results
can be generalized to an arbitrary program.
Second, in this evaluation, we focus specifically on
the {manifest dependence} between \textit{unit tests}.
We did not investigate possible test dependence that may arise
in other types of tests, such as integration tests
or system tests.
Third, due to the computational complexity of the general dependent test
detection problem, we do not yet have
empirical data of how many dependent
tests exist in a test suite and how many percentages of dependence tests
\ourtool can catch.  Giving a reasonable estimation is one of our future work.

%\todo{this paper focuses on manifest test dependence, what about
%potential test dependence, as well as some other cases in the
%study}

\subsubsection{Experimental Conclusions}


We have three chief findings: \textbf{(1)}
Dependent tests do exist in real-world test suites.
They may not be obvious to be identified
unless been explicitly searched for.
An automatically-generated test suite can contain
substantially more dependent tests than a human-written
test suite.
\textbf{(2)} Like the dependent tests
studied in Section~\ref{sec:study}, the identified
dependent tests in our subject programs reveal weakness
in a test suite rather than defect in the tested code.
And \textbf{(3)} In terms
of the number of detected dependent tests
and the time cost, the randomized algorithm is the
most cost-effective one.
%However, it does not
%have any guarantee in 



\section{Discussion and Future Work}
\label{sec:discussion}

%In the introduction we posed several questions why test dependence
%may have received little attention despite the ease of constructing
%concrete but contrived examples. 
%contributions suggest answers, to differing degrees, to these questions:


We have detected and studied a substantive set of
real-world dependent tests, from both human-written test suites and
automatically generated test suites. This motivates
the need for a broader
investigation of the impact of dependent tests,
how to eliminate and prevent dependent tests.
%is beyond the scope of this paper. 
%We next discuss a set of open questions addressing this and
%other possible concerns resulting from dependent teests.
%Exploring answers to these open questions comprises
%our future work.

\vspace{1mm}

\noindent \textbf{{Impact of dependent tests.}}
Dependent tests can mask faults and lead to
spurious bug reports; they can also 
compromise the application of
testing techniques such as test selection,
prioritization, and parallelization, since
most current techniques just assume independence and
make no statement about what happens when this
assumption is not true. However,
more comprehensive empirical studies should measure  
the extent of this impact.


Another open question is how should
testing techniques such as test
selection, prioritization, and parallelization
handle test dependence.
One straightforward way 
might be to augment such techniques to respect a
defined partial order among tests. This partial order
can be derived from knowledge about dependent tests,
or be detected by our \ourtool tool.
%Like contrived examples of test
%dependence itself, it is easy to produce simple examples where
%downstream techniques produce incorrect output when applied to dependent
%tests.
%under the assumption that the input tests have no dependences.
%However, 



\vspace{1mm}

\noindent \textbf{{Eliminating dependent tests.}}
As found in our study (Section~\ref{sec:study}),
developers sometimes intentionally introduce dependent tests,
and do not fix many dependent reported tests.
For some dependent tests that get eliminated,
the practice of eliminating them
remains mostly manual and ad hoc --- software developers
usually manually hardcode test
execution orders in a configuration file or
simply merge or remove tests, when a dependent test is reported. 
A more flexible and robust methodology for
dependent test elimination should be developed.

On the other hand, dependent tests in
an automatically-generated test suite can be 
more challenging to eliminate.
As suggested by our experiments, dependent tests are
more prevalent in automatically-generated test suites
than in human-written test suites.
Further, this problem is even exacerbated by the fact that
almost all automated test generation
techniques we are aware of produce tests
that are hard to read for humans, are undocumented, and their intent
cannot easily be gleaned from naming conventions and other aids
developers normally use. Therefore, it requires more effort
from developers to identify the root cause of dependence
and then remove the dependence. While there is some work to alleviate
this problem~\cite{fraseretal:ISSTA:2011}, the question
of eliminating  automatically-generated dependent tests
still remains open.


%As discussed in our experiments, it appears that test
%dependence in automatically generated test suites is 
%even more troublesome than in human-written suites. 


\vspace{1mm}


\noindent \textbf{{Preventing dependent tests.}}
Detecting dependent tests is not obvious in most
cases. Thus, a natural question is how could
software developers prevent dependent tests when
writing testing code.

One possible way is encouraging developing to
use advanced testing frameworks that support test dependence,
so that developers can explicitly specify test
dependence when writing tests.
However, using different testing frameworks may
bring up the compatibility issue to the existing tests.

Stylized coding patterns can also be useful. Developers
should be encouraged to write tests ``defensively'' by
specifying necessary test execution pre-conditions and
using less (or properly mocking) global variables or shared resources. 
There is already some work aiming at automating this
process to prevent the potential
for dependences by refactoring programs to use
less global state~\cite{wlokaetal:FSE:2009}. 

%We conjecture that if and when this happens in practice, it is
%hard to notice in part because current techniques~\cite{} do not surface the necessary information.

%defensive programming? explicitly state testing oracles?

%A better understanding of the 
%frequency and scope of consequences from test dependence should be
%developed.  

%Of particular concern is the
%masking of program faults because, unlike weaknesses
%in test suites or spurious bug reports, masking
%faults could be costly to find by other methods or to leave in the program.
%Tools that surface test dependences may help researchers
%and practitioners study and deal with dependences more effectively.





%\vspace{1mm}

%\noindent \textbf{\textit{What about other cases of test dependences?}}



%Several of our examples identified situations in which test dependence
%masked faults in the underlying program. 
%In another example, developers wasted time tracking down a non-existent
%fault because of a spurious report that was due to an undocumented
%test dependence. Even though these
%are real and reproducible examples,

%It is not possible to make any
%general claims about the frequency nor the significance of the
%repercussions of test dependence.  At the same time, it seems unlikely
%that these are the \emph{only} software systems where test dependence
%causes problems.


%\vspace{1mm}


%This is a more subtle question, 
%because the answer depends not only on the tools being used but also on the
%perceptions and insights of the developers.  If tools always run
%tests in the same context, and if developers never consider the possibility
%of test dependence, then it is unlikely that dependence will be observed.
%Masking of faults in the underlying program is a good illustration of this.

 

%\medskip


%Our 
%prototype tool shows that even our approximate algorithm
%can reveal large numbers of important dependences. Faster and more
%precise approaches are plausible, especially as more understanding
%of test dependence ``in the field'' is acquired.

%avoiding test dependences, to removing or documenting test dependences that are found, etc.


%Given a technique that detects dependence, what further
%  actions can and should be taken?


%also be supported by testimony from practitioners. With regard to the
%relevance of test dependence, the following questions seem to be of
%particular interest:
%\begin{enumerate}
%  \item Does test dependence occur often enough, and is its impact
%  critical enough to make further enquiry worthwhile?
%  \item 
%\end{enumerate}

%The frequency and consequences of test dependence in our work
%so far seem to justify further investigation.  We are especially
%concerned about whether test dependence is masking 

%We strongly believe that both the frequency and the
%impact of test dependence merit investigation of the phenom\-e\-non. The
%fact that within our small example set we found a number of masked
%faults that directly impacted the users of the software, and the
%observation that any form of test dependence prevents the successful
%use of many second-order testing techniques are strong
%indicators that this phenomenon deserves as much attention as any
%other technique that can improve the bug finding strength of testing.

%Thinking about the higher-level causes of test dependence leads to a large number of different questions,
%many of which relate not only to technical issues, but also to the social
%and human environment in which software is created. Exploring this
%domain, while very difficult to do well, is likely to yield the best
%explanations for the creation of test dependence.

%How actionable the knowledge of test dependence is, ironically,
%depends. At this early stage, we suggest inspection of code, tests and
%specifications to understand what causes the dependence, and where the
%fault lies. As our examples show, dependence can point to faults in
%the program, in the tests, or to insufficient documentation and
%specification. We hope and expect that it will be possible to
%determine or exclude some of these causes automatically with further
%analyses of code and tests. 
%There are also several
%projects that either augment or replace JUnit with the express goal to
%declare test dependence
%explicitly\footnote{\url{https://code.google.com/p/depunit/},\\
%\url{https://code.google.com/p/junitum/},\\ \url{http://testng.org}}, but not to detect it.




%
%
%The key take-aways from this paper are that  prior work and existing
%tool broadly assume test independence,
%%is broadly assumed by prior work and by existing tools 
%and that there is
%at least incipient empirical evidence that this assumption can lead to
%unexpected and likely negative repercussions.  
%%We present an initial
%formalization of test dependence that embodies both the execution
%order of a test suite and also the environment in which tests in a
%suite are executed. % the formalization allows us to prove that the
%problem of determining if there are dependences among the tests in a
%test suite is NP-complete.  
%Substantive examples of test dependences
%in the field, along with descriptions of the consequences of these
%dependences, argue the potential practicality of further
%investigations of test dependence.  
%Initial algorithms designed with the
%NP-completeness of the problem in mind, along with an initial tool, allow
%us to take initial steps towards practical applications, as well as to check
%the validity of examples of test dependence that we had previously identified
%in \emph{ad hoc} ways.\todo{JW}{This is all a bit too initial...}  A set of open questions related to test dependence 
%provides a partial, surely incomplete, roadmap for further work in the area.

%\todo{JW}{I think this section should be about
%\begin{enumerate}
%  \item What prompted our research
%  \item Which research questions did we answer, and which research 
%  questions remain open.
%  \item Summary what the reader should take away from the paper.
%\end{enumerate}
%The current version falls short of this in several ways. 
%\begin{enumerate}
%  \item It doesn't address ``the big picture'' at all.
%  \item It focuses on minor technical details \emph{and} phrases
%  minor technical issues as research problems in a way that assumes
%  answers to interesting questions we didn't even ask (comments in the
%  text below elaborate on that)
%  \item It focuses too much on our ``results'', which is wrong,
%  considering how little actual data we have.
%\end{enumerate}
%}

%More concretely, in this paper, we examine the importance of test dependence in theory and practice. 
%While the
%research literature usually assumes tests to be independent, or evades
%the issue entirely, our interest was piqued when we found a number of
%test dependences in real-world software.
%This led us to explore in more depth if this is an issue in a broader
%range of software systems, and whether test dependence causes real
%problems. 
%


%We found that the human-written test suites of many open-source libraries contain
% dependent tests, that test suites automatically generated with
% Randoop contain even larger numbers of dependent tests, and that in
% both cases these tests cause various
%kinds of problems, from preventing test prioritization to actually
%hiding real faults in the programs. However, we must emphasize that
%our exploration should not be construed as a complete
%experiment that provides conclusive evidence. Rather we inspected programs that we were familiar with.
%Hence, before the conclusions drawn from the examples in
%Section~\ref{sec:examples} can be generalized,
%a proper controlled experiment should be carried out with a broader
%range of projects, and with proper control for confounding factors
%such as project type and programmer expertise, which most likely have
%a profound impact on test depenendence. Nonetheless, we believe that the anecdotal evidence we
%presented in this paper makes clear that this is a worthwhile research
%endeavour.

%As a main contribution of this paper we identify test dependence as
%an important research subject that so far has been mostly ignored by
%the software testing community.

%Before summarizing our work and contributions, we outline a set of
%open research topics that could further improve the way we identify
%and manage test dependences: 
%\begin{itemize}
%
%\item Our initial algorithms for detecting test dependences 
%leave room for improvement in efficiency.  Conventional optimizations
%should apply \todo{JW}{I don't know what this means. We proved that
%the problem is NP-complete the full algorithm is exponential.
%``Optimizations'' in the case can only mean approximations, right?}, as should incremental algorithms and/or on-line versions of
%the algorithms --- for when test cases are added to a suite, among other
%potential performance improvements.\todo{JW}{What's the point here?}
%
%\item \todo{JW}{This is a theory paper. I think this is completely
%irrelevant} Our initial tool for detecting test dependences 
%leaves room for improvements in applicability (what ``forms'' of test
%suites/programs do we handle, such as JUnit, etc.?), in user
%interface, etc.
%
%\item We have only scratched the surface of the interaction
%between test dependence and downstream testing tools like selection,
%prioritization, and parallelization.  \todo{DN}{We have claimed earlier
%we will show that dependence can cause these approaches to fail.
%Where in the paper do we do that, and how do we do that?  I believe it's
%pretty obvious, but we'll need to be careful about doing it.} \todo{KM}{I think
%we even have real life examples (from Mike), etc. for this, however I might be
%mistaken. Generating theoretical example is obvious, but I agree that we lack
%that example at the moment.} How should these interactions be handled? For
%example, should a ``test dependence manager'' ensure that suites have no dependences before a downstream tool is invoked or should those tools check that their output test sub-suite has no dependences?
%
%\item \todo{JW}{This is an interesting question, as it to some extent
%asks what is actionable about dependencies. However, I think this
%should be discussed elsewhere (in the motivation, examples?), because
%this isn't really a research question or anything. The answer to this
%seems pretty obvious to me, and is what is described in this
%paragraph.} What should a development/test team do when a test dependence is identified?
%It could indicate that there is a problem in the tests themselves, in which case
%they could be fixed.  It could indicate that there is a problem in the program
%being tested, in which case it could be fixed.  It could indicate that the
%dependence is necessary, in which case the test could perhaps be merged together
%to ensure that the dependence is respected by the test framework and downstream
%tools.
%
%\item \todo{JW}{What?} We argue additional work exploring the interrelationship between
%individual unit tests should be performed, and show initial examples
%that test dependencies must be considered when executing a test suite,
%fixing a regression error, and generating new tests.
%
%\item This leads to our next recommendation for further software testing
%research. Some work in this area, discussed previously, has already
%attracted attentions. Nevertheless, we strongly beleive that more
%work exploring open questions on how to integrate \textit{test dependence}
%to the testing process is necessary to be understood.
%
%\todo{JW}{The following two paragraphs are too narrowly focused on
%test dependences being a bad thing that needs to be avoided. I would
%at the very least at the meta question asking when, where, how often
%dependences are bad, intentional, necessary, practical etc.}
%
%First, how to eliminate existing dependent tests (or retrofitting
%dependent tests into tests without inter-dependencies)? A critical
%part in retrofitting existing dependent tests is to identify test
%code that may affect the behavior of our tests, and then make the
%affected tests more ``robust'' to such affecting code. 
%\todo{DN}{Something especially about initialization code for tests?}
%
%\todo{JW}{This is too vague, both the problem description and the
%solution.}
%
%
%Second, how to prevent new dependent tests being produced?
%Programmers should be encouraged
%to encapsulate common test execution environment setting up and destroying
%code into separate pre- and post-conditions, to ensure each
%test is executed in a desirable environment and also probably cleared up
%the environment after its execution. One promising way to achieve
%this is to extend Design-by-Contract~\cite{Leitner:2007} to human-written unit
%tests. For an automated tool, it may employ
%capture-and-replay~\cite{Elbaum:2006} techniques to save the probable environment when a test is created,
%and then recover the needed environment when executing a
%test.\todo{JW}{This sounds weird to me. How would that work.}
%
%\item Finally, what would be the impact of dependent tests to the whole software testing
%process?  Most existing research in the fields of regression selection
%and prioritization has an implicit assumption on test dependence.
%.\textbf{unfinished}.. \todo{JW}{This could be interesting, but I
%doubt there is any general conclusion we can draw here.}
%
%\end{itemize}


%Our formalism provides a precise definition of manifest test dependence,
%allows reasoning about test dependence, and enables the proof that
%detecting manifest test
%dependence in test suites is NP-complete.  

%  LocalWords:  hoc hardcode pre dependences


\section{Related Work}
\label{sec:related}

%\enlargethispage{5pt}

%Denoting a group of test cases as a ``suite of test programs'' began around the
%mid-1970's~\cite[p.~217]{brown:CSUR:1974}; similar terms include
%``testcase dataset''~\cite{milleretal:ICRS:1975} and ``scenario,''
%which an IEEE Standard defines as ``groups of test cases;
%synonyms are script, set, or suite''~\cite[p.~10]{IEEE:829-1998}.

Treating test suites explicitly as \emph{mathematical sets} of tests dates at least
to Howden~\cite[p.~554]{howden:ToC:1975} and remains common in the literature.
The execution order of tests in a suite is usually not considered:
%or informally, suggesting that the potential of executing a given test
%in different contexts is immaterial to those results: 
that is, test independence is assumed. Nonetheless,
some research has considered it. We next discuss some
existing definitions of test dependence, techniques that
assume test dependence, and tools that support specifying
test dependence.

\subsection{Test Dependence}

Definitions in the testing literature are generally clear that the
conditions under which a test is executed may affect its result. 
The
importance of context in testing has been explored 
in databases~\cite{Gray:1994:QGB:191843.191886,Chays:2000:FTD:347324.348954,
kapfhammeretal:FSE:2003}, with results about test
generation, test adequacy criteria, etc., and mobile
applications~\cite{Wang:2007:AGC}.
For the database domain, Kapfhammer and Soffa formally
define independent test suites and distinguish them from
other suites that ``can capture more of an application's
interaction with a database while requiring the constant monitoring of
database state and the potentially frequent re-computations of test
adequacy''~\cite[p.~101]{kapfhammeretal:FSE:2003}.
By contrast, our definition differs from that of Kapfhammer
and Soffa by considering
test results rather than program and database states
(which may not affect the test results).
%Considering only manifest test dependences allows
%us to more easily situate this research in the empirical domain (Section~\ref{sec:formaldiscussion}).

The IEEE Standard for Software and System Test
Documentation (829-1998) \S 11.2.7, ``Intercase
Dependencies,'' says in its entirety: ``List the identifiers of
test cases that must be executed prior to this test
case. Summarize
the nature of the dependences''~\cite{IEEE:829-1998}.  The succeeding version of this
standard (829-2008) adds a single sentence: ``If
test cases are documented (in a tool or otherwise) in the order in
which they need to be executed, the Intercase Dependencies for most or
all of the cases may not be needed''~\cite{IEEE:829-2008}.


%In addition to the work by Kapfhammer and
%Soffa~\cite{kapfhammeretal:FSE:2003},
%there are a handful of categorical references that
%acknowledge that tests can
%be dependent based on context, suggesting 
%ways to document or find situations where the independence
%assumption fails to hold.  


%McGregor and Korson discuss interaction tests that
%are intended to identify ``two methods that may directly or indirectly
%cause each other to produce incorrect results'' and suggest constructing such
%interaction tests by identifying the values shared via parameter passing
%between methods
% that two or more test cases share~\cite[p~.69]{mcgregoretal:CACM:1994}.

Bergelson and Exman characterize a form of test dependence informally:
given two tests that each pass, the composite
execution of these tests may still
fail~\cite[p.~38]{bergelsonetal:EEE:2006}.
%% Cut for space
% That is, if 
% $t_1$ executed by itself passes and $t_2$ executed by itself passes,
% executing the sequence \suite{t_1, t_2} in the same context may fail.
However, they do not provide any empirical evidence of
test dependence nor any detection algorithms.

The C2 wiki acknowledges test dependence as undesirable~\cite{unit-test-def}:
%a possible, albeit low probability, event:
\tinysqueeze
\begin{quote}
Unit testing \dots  
requires that we test the unit in isolation. That is, we
want to be able to say, \emph{to a very high degree of confidence} [emphasis added], that
any actual results obtained from the execution of test cases are
purely the result of the unit under test. The introduction of
other units may color our results.
\end{quote}
\tinysqueeze
They further note that other tests, as well as stubs and drivers,
may ``interfere with the straightforward
execution of one or more test cases.''


Compared with these informal definitions,
we formalize test dependence and characterize 
test dependence in practice.
%, and could
%have costly repercussions.
%They give an informal definition of what it means for the execution of a
%test to influence the outcome of another test.  We define
%this precisely, and we also define manifest test dependence in terms
%of execution environments
%and test execution order rather than in terms of code use.

%Other definitions of test dependence are primarily considered
%to be \textit{syntactic} dependences between program units, for example
%methods calling other methods, and classes using other classes~\cite{bergelsonetal:EEE:2006,briandetal:SEKE:2002}. 
%\emph{Syntactic} dependence here means that a unit \code{A} cannot be
%compiled and executed without unit \code{B} being present. If we test
%such a unit \code{A} without convincing ourselves first that \code{B}
%is correct, a test failure for \code{A} is harder to interpret,
%because it could just as well indicate a fault in \code{B}.
%Zhang and Ryder extend this notion to \emph{semantic} dependences,
%which is closer to our approach~\cite{zhangetal:TR:2006}. 
%They use a notion of
%``test outcome'' to determine whether or not syntactically dependent
%classes or methods can influence each others results, and consider
%only those that can to be semantically dependent.
%They give an informal definition of what it means for the execution of a
%test to influence the outcome of another test.  We define
%this precisely, and we also define manifest test dependence in terms
%of execution environments
%and test execution order rather than in terms of code use.


%\tinysqueeze
\subsection{Techniques Assuming Test Independence}

The assumption of test independence lies at the heart of most
techniques for automated regression test selection~\cite{harroldetal:OOPSLA:2001, Orso:2004:SRT,
Briand:2009:ART, Zhang:2012:RMT, Nanda:2011:RTP},
test case prioritization~\cite{Elbaum:2000:PTC:347324.348910, Kim:2002:HTP:581339.581357, Rummel:2005:TPR:1066677.1067016, Srivastava:2002:EPT:566172.566187, Jiang:2009:ART}, 
coverage-based fault localization~\cite{Steimann:2013, Zhang:2013:IMF, Jones:2002:VTI}, etc. 


Test prioritization seeks to reorder a test suite to detect
software defects more quickly. 
Early work in test
prioritization~\cite{Wong:1997:SER:851010.856115,Rothermel:1999:TCP:519621.853398}
laid the foundation for the most commonly used problem definition:
consider the set of all permutations of a test suite and find the best
award value for an objective function over that
set~\cite{Elbaum:2000:PTC:347324.348910}.  The most common objective
functions favor permutations where higher code coverage
is achieved and more faults in the underlying
program  are found with running fewer tests.
Test independence is
%often explicitly asserted as
a requirement for most test selection and prioritization work (e.g.,~\cite[p.~1500]{Rummel:2005:TPR:1066677.1067016}).
%For some test selection and prioritization work,
%test independence is even explicitly asserted as a requirement.
%For example, Rummel et al.\ states in
%A number of studies carefully evaluation various prioritization techniques
%empirically~\cite[\emph{et
%alia}]{Rothermel:1999:TCP:519621.853398,Do:2010:ETC:1907658.1908088}. 
Evaluations of selection and prioritization techniques~\cite[\emph{et alia}]{Rothermel:1999:TCP:519621.853398,Do:2010:ETC:1907658.1908088}
are based in part on the test independence
assumption as well as the assumption that the set of faults in the underlying
program is known beforehand; the possibility that test dependence may
interfere with these techniques is not studied.
%unmask additional faults in the program is not studied.

%\begin{quote}
%A test suite contains a tuple of tests \suite{T_1 $\ldots$ T_R} that execute in a specified order.  We require that each test is
%independent so that there are no test execution ordering dependencies.  This requirement enables our prioritization algorithm to
%re-order the tests in any sequence that maximizes the suite's
%ability to isolate defects.  The assumption of test dependence
%is acceptable because the JUnit test execution framework
%provides \code{setUp} and \code{tearDown} methods that execute before
%and after a test case and can be used to clear application
%state.
%\end{quote}


Coverage-based fault localization techniques~\cite{Jones:2002:VTI}
often treat a test suite as a collection of test cases
whose result is \textit{independent} of the order of their
execution. They can also be impacted by test dependence.
In a recent evaluation of several coverage-based fault locators,
 Steimann et al.\ found that fault locators' accuracy is
 affected by tests that fail due to violation of the test
 independence assumption~\cite{Steimann:2013}. 
 %For example, if a test depends on a static field whose value is set by
 %previous test cases. 
 Compared to our work, Steimann et al.'s
 work focuses on identifying possible threats to validity
 in evaluating coverage-based fault locators, and does
 not present any formalism, study, or detection algorithms
 for dependent tests.

Test independence is different than determinism.
%
Non-determinism does not imply dependence:  a program may execute
non-deterministically, but its tests may deterministically succeed.
Further, a test may non-deterministically pass/fail without being
affected by any other test, including its own previous executions.
%
Determinism does not imply independence:  a program may have no sources of
nondeterminism, but two of its tests can be dependent.
%
The testing community sometimes mentions determinism (such as
multithreading) and execution environment (such as library
versions), without considering test dependence~\cite{Orso:2004:SRT}.
%
A stronger assumption than determinism is the Controlled Regression
Testing Assumption (CRTA)~\cite{Rothermel:1996:ART}.  It forbids porting to another system,
nondeterminism, time-dependencies, and interactions with the external
environment.  It also forbids test dependence, though the authors did not
mention test dependence explicitly.  The authors state that CRTA is ``not
necessarily impossible'' to employ.  We have a practical focus on the
often-overlooked issue of test dependence.

As shown in Sections~\ref{sec:study} and~\ref{sec:evaluation},
the test independence assumption often does not hold for either
human-written or automatically-generated tests; and the dependent
tests identified in our subject programs interfere with
existing test prioritization techniques. Thus, techniques
that rely on this assumption may need to be reformulated.

\begin{comment}
Most automated test generation
techniques~\cite{PachecoLET2007, Wang:2007:AGC,
ZhangSBE2011} do not take test dependence
into consideration. As shown in our experiments
(Section~\ref{sec:evaluation}) and previous work~\cite{RobinsonEPAL2011},
a large number of tests generated by Randoop are dependent.
We speculate that these dependences arise because automated
test generators generally create new tests
based on the program state after executing the previous test,
for the sake of test diversity and efficiency. 
When Randoop generates a nondeterministic test, it can disable the test but
leave it in the suite where it is executed in order to prevent other tests
that are dependent on it from beginning to fail~\cite{RobinsonEPAL2011}.
%I feel test generation should not belong here
%introduce 130X overhead
%may make generated
%tests less behaviorally-diverse --- as they cannot be constructed
%on top of previous tests.
Exploring how to incorporate test dependence into the design of an automated
test generator is future work.
\end{comment}

%define a test suite as a
%collection of test cases whose result is \textit{independent}
%of the order of their execution~\cite{Steimann:2013}.


\subsection{Tools Supporting Test Dependence}
\label{sec:supporting}

%\todo{This section can probably be shortened.}

Testing frameworks provide mechanisms
for developers to define the context for tests.
%JUnit, for example, provides means to
%automatically execute setup and clean-up tasks
%(\code{setUp()} and \code{tearDown()} in JUnit
%3.x, and annotations \code{@Before} and \code{@After} in
%JUnit 4.x). 
JUnit 4.11 supports
executing tests in lexicographic order by test method name~\cite{junitordering}.
%However, ensuring that these mechanisms are used properly is
%beyond the scope and capability of any framework. 
%Further, our empirical study and
%experimental results indicate that programmers often do not
%use them properly and introduce dependent tests. 
%
%Only a few tools explicitly 
%allow developers to annotate dependent tests and
%provide supporting mechanisms to ensure that the test execution framework
%respects those annotations. 
DepUnit~\cite{depunit} allows developers to define
dependences between two unit tests.
%Soft dependences control test ordering, while hard dependences in addition control whether specific tests are run at all.  
TestNG~\cite{testng} 
allows dependence annotations and supports a variety of execution policies
that respect these dependences.
% such as sequential execution
%in a single thread, execution of a single test class per thread, etc.\
What distinguishes our work from these testing frameworks is that, while they allow dependences
to be made explicit and respected during execution, they do not help developers
\emph{identify} dependences.  
%A tool that finds dependences
%(Section~\ref{sec:impl}) could co-exist
%with such frameworks by generating annotations for them.

Haidry and Miller~\cite{10.1109/TSE.2012.26} proposed a set of
test prioritization techniques that consider
test dependence.  
Their work assumes that dependencies between tests are
known, and improves existing test prioritization techniques
to make them produce a test ordering that preserves the test dependencies.
%Their work
%assumes that dependencies between tests are known (and are represented as
%partial orderings, such as that one test should be executed before another)
%without providing any empirical evidence of whether dependent tests
%exist in practice.
%\todo{Can/should we say that they did not motivate that their techniques
%  are needed in practice, but we have provided evidence of their value?}
By contrast, our work formally defines test dependence,
studies the characteristics of real-world test dependence,
shows how to detect dependent tests,
and empirically evaluates whether dependent tests exist in real-world
programs and
their impact on test prioritization techniques.


Our previous work~\cite{DBLP:conf/sigsoft/MusluSW11} proposed
an algorithm to find bugs by executing each unit
test in isolation. With a different focus,
this work investigates the validity of the test independence assumption
rather than finding new bugs,
and it presents five new results.
Further, as indicated by our study and experiments, most dependent
tests reveal weakness in the test code rather than bugs in the program. Thus,
using test dependence may not achieve a high return in bug finding.

A simple way to eliminate test dependence is
starting a new process or otherwise completely re-initializing the environment (variables,
heap, files, etc.)\ before executing each test;
JCrasher~\cite{Csallner:2004} does this, as do
some SIR applications~\cite{sir} and
some database or GUI testing tools~\cite{kapfhammeretal:FSE:2003,
Chays:2000:FTD:347324.348954, Gray:1994:QGB:191843.191886
}.
%\todo{Check the math.}
However, such an approach is computationally expensive:
Table~\ref{tab:results} shows that executing each test in a
separate JVM introduces 10--138$\times$ slowdown (compare the 
``Exhaustive $k=1$'' column to the ``Rev'' column).

% (let ((ratios (list (/ 1265 57 .1) (/ 106 65 .1) (/ 106 14 .1) (/ 166 14 .1) (/ 25 7 .1) (/ 133 50 .1) (/ 2477 103 .1) (/ 454 81 .1)))) (list (apply #'min ratios) (apply #'max ratios)))


%Such an overhead 
%may not be acceptable in practice.

%JCrasher~\cite{Csallner:2004}
%provides a mode to clear the environment changes caused
%by a previous test before generating a new one. Such functionality may happen to eliminate
%potential test dependence, but requires a tester to
%run each generated test in a separate JVM\@.
%Executing each test in a separate JVM is prohibitively expensive
%and might be useful when executing
%a test is much more expensive than re-initializing the environment.
%However, 

%  LocalWords:  Howden Kapfhammer Soffa dependences Intercase Bergelson C2
%  LocalWords:  Exman JCrasher Steimann setUp tearDown DepUnit TestNG
%%  LocalWords:  locators nondeterminism multithreading nondeterministic
%  LocalWords:  Haidry


\section{Conclusion}
\label{sec:questions}

Test independence is broadly assumed but rarely addressed, and
test dependence has largely been ignored in previous
research on software testing. 
To understand dependent tests, we described one of the first studies on
real-world dependent tests. We showed that 
test dependence \textit{does} arise in practice, and could have
non-trivial negative repercussions. We also
formalized the the dependent test detection
problem. To detect dependent tests, we designed
and implemented three algorithms to identify manifest test dependence
in a test suite. Our empirical evaluation revealed
previously-unknown dependent tests in every subject program
we studied, from both human-written and automatically-generated test
suites.

\begin{comment}
The results of this paper offer initial answers to the four questions we posed
in the introduction. First, test dependence \textit{does}
arise in practice, from both human-written test suites and automatically-generated
test suites. Second, dependent tests can have
non-trivial negative repercussions. Third, we speculate that this
problem is not thoroughly examined before due to the
lack of empirical evidence, a rigorous problem definition,
and detection algorithms; alternatively people may simply believe that well-designed
test suites do not have this problem.
Fourth,
detecting manifest dependent tests is a NP-complete problem,
for which efficient algorithms are unlikely to exist. However,
in practice, a simple algorithm like the randomized algorithm
described in this paper can work fairly well.
\end{comment}

%The source code of our tool implementation is publicly
%available at: \url{http://testisolation.googlecode.com}.



\subsection*{Acknowledgments} Bilge Soran was a participant in the project
that led to the initial result.  Yuriy Brun and Colin Gordon provided advice about
the formal notation.  Reid Holmes and Laura Inozemtseva identified the initial \jodatime dependence.  Cheng Zhang suggested exploring software issue tracking systems
to study dependent tests. Mark Grechanik, Adam Porter, Michal
Young, and Reid Holmes provided timely and insightful comments on a draft.
This work was supported in part by NSF grants
CCF-1016701 and CCF-0963757

\bibliographystyle{plain}
\bibliography{references}


%\subsection{Practical Considerations}
%\label{sec:practical}
%In principle, there is no \emph{ground truth} for the order of test
execution.
Therefore, we assert that the
\emph{programmer-defined} execution order, and consequently the test
results from executing the test suite in that order, are the ground
truth for our experiments.
%would naturally serves 
%the ``truth'' for our definitions of test dependence, and records
%the results from that execution order as intended results (line 2, Figure~\ref{fig:dtalgorithm}).

When a dependent test is identified, programmers may wish to know
a minimal list of other tests on which the identified test depends. 
Given an execution sequence that manifests the dependence, Delta
Debugging (also implemented in our tool) can
be used to return a shortest subsequence 
that still manifests the dependence~\cite{Zeller:2002}. 
%to minimize the recorded
%test list before the dependent test was executed.
%\todo{SZ}{is it clear? or need more explanation?}


In practice, another possible way to help detect potential dependent tests is
to leverage programmers' domain knowledge or employ some program analyses
to identify a subset of tests that are likely to contain dependent tests,
and run the algorithm only on that subset instead of the whole suite.

\todo{sz}{need a summary sentence here for the whole section 5.}

%\todo{DN}{I'm on the side of removing DD if reasonable, and coming back
%to the idea later, maybe in a ``practical considerations'' section/subsection}
%In addition, the algorithm employs Delta debugging~\cite{Zeller:2002}
%to minimize the test set that are executed before a test in
%an execution (lines 7--8). Together with the minimized
%dependent test set, the test revealing with different behaviors
%are added to the output (line 9).


%\todo{JW}{We should mention that we used the tool to find/verify the
%examples. We should also mention that isolation corresponds to $k=1$
%and we did pair-wise (corresponding to $k=2$).}
%\todo{JW}{
%For reverser execution, as far as I remember, we didn't use it to for
%the actual examples we have. But we might claim that it is useful for
%identifying particular kinds of deps. But it would be better if we had
%an example for that.}



%\section{Unused text snippets}
%

\begin{itemize}

\item \todo{KM}{I kind of understand what this paragraph is saying.
However the many minor mistakes in the writing make it very hard to
follow up.} Patterns of dependent tests.
In many cases, there is a \textit{N -- 1} \todo{KM}{The first time I read this,
I read as N minus 1, which is the incorrect way. Maybe write down as ``N to 1''}
dependence relationship, in which $N$-th \todo{KM}{Is ``th'' really needed?} distinct tests depends\todo{KM}{This should be either ``tests depend on'' (more likely) or ``test depends on'', but I couldn't decide} on the same test, which probably is used to set up the environment. For
such cases, the 1 depend test \todo{KM}{``1 dependent test'' or ``first
dependent test''} should be moved to the common \CodeIn{setUp}
\todo{KM}{Consistency: consider @Before} method.
Less frequently, there is a \textit{1 -- N} dependence relationship
\todo{KM}{Have we ever seen this, or is this purely theoretical?}, in which one
test depends on $N$ tests to set up its testing environment.  In one subject, a newly-added test changes the shared variable state of an existing test. Although the newly-added test is executed after the existing test and reveals the same behavior when executing in isolation, the existing test exhibit different behavior if it is executed after the newly-added test.


\item 
\todo{JW}{If we really want to discuss this, this should be connected
to the theory section, as all this follows from theory. A ``finding''
might be that these differences actually matter in practice. And I
don't think we checked that.}
Different techniques have their own strength in detecting dependent tests. We
have investigate three methods (i.e., executing in isolation, executing in a
reversed order\todo{KM}{Did we (do we) really do this?}, and executing in
k-permutation) to identify dependent tests, and found each method complements others. There exist certain tests that can only be found by one method
but missed by the other two.  Executing tests in isolation found more dependent tests
than executing in a reversed order, and executing every $k$-permutation is
infeasible in practice due to the exponentially large number of possible combinations.

\end{itemize}


\end{document}
% vim:wrap:wm=8:bs=2:expandtab:ts=4:tw=70:


%  LocalWords:  Soran Cheng Grechanik Michal CCF

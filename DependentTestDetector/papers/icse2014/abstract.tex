\begin{abstract}
%{\color{red}
%\noindent Dear friendly pre-reviewer: 
%
%\noindent This is a reasonably complete draft that we plan to submit
%to FSE 2012 on Friday March 16.  There are a few specifics we'd love
%for you to look at, but you're of course free to comment on other
%stuff (including typos, grammar, etc.); the conclusion is probably
%the part we've worked on the least so far, if you need to do something
%more lightly.  (OK, the abstract has been worked on even less!)  Among
%the specifics we'd like to hear about are: (a) The related work is
%reasonable extensive, but isn't done.  If you have any suggestions of
%work we've missed, or work that would help others put our work in
%context, or about the ``feel'' of what it says, let us know. (b) The
%core content is even more important; do we have something here?  Are
%you interested?  If not, why not?
%
%\noindent Thanks in advance for anything you point out!
%
%\noindent David, Jochen, Kivanc and Sai
%}
Test dependence arises when executing a test in different environments
causes it to return different results.  In this paper, we
show through a set of substantive real-world examples that test dependence arises in practice. 
We also show that test dependence can have potentially costly
repercussions such as masking program faults, and can be hard to identify
unless explicitly searched for: We found a dependence
that only manifests when a sequence of three tests are run in a specified, non-default order.
%.  For example, we identify several situations where test dependence masks
%program faults: in these situations, running the test suite in the default order does not expose a fault
%but running the suite in a different order does.  
%We also argue that existing
%tools rarely ``surface'' test dependences in a direct way, making it harder for developers
%to observe them.

We formally define test dependence in terms of test suites as ordered
sequences of tests along with explicit environments in which these tests are
executed. We use this formalization to formulate the concrete problem
of detecting dependence in test suites, prove that a useful special
case is NP-complete, and propose an initial algorithm that
approximates solutions to this problem.

%To a lesser degree, we describe how two trends in software testing may interact
%with test dependence: one, downstream testing tools such as selection, prioritization,
%and parallelization are increasingly common, and may assume that the
%suites they take as input have no test dependences; and, two, automated test
%generation tools are becoming more common, and we provide some initial
%evidence that test dependence appears to be orders of magnitude more common
%in automatically-generated test suites than in manually-produced suites.
%

%WE show that, in practice,
%test dependence does occur sometimes and does have grave consequences.
%We further argue, that given the increasing importance of second-order
%testing techniques, such as prioritization and parallelization, which
%are directly affected by test dependence, this topic deserves
%attention from the research community. As a first step, 
%
%
%\todo{KM}{The abstract was only about the ``theory'' side of our paper, so I
%tried to start writing one more paragraph about the ``manifestation'' side. It
%is not perfect (and not complete) but I still think that something like this
%needed in the abstract.} 
%In the second half of the paper, we explain the dependences in the manual
%written and auto-generated test suites of known and popular open source
%software. We present an initial exploration over six software and report 75
%(resp. 1975) dependent tests in manual written (resp. auto-ge\-ne\-rated) test
%suites.

\end{abstract}

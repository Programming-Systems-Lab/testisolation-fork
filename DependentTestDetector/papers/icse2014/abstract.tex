\begin{abstract}
Test dependence arises when executing a test in different environments
causes it to return different results. \todo{need a transition sentence here} 
To understand and detect dependent tests, this paper presents four results. 

First, we formally define test dependence in terms of
test suites as ordered sequences of tests along with explicit
environments in which these tests are executed. We use this
formalization to formulate the problem
of detecting dependent tests, and prove that a useful special
case is NP-complete. 

Second, we describe a study of \todo{NUM}
real-world dependent tests from \todo{NUM} bug repositories
to show that test dependence arises in practice.
We also show that test dependence can have potentially costly
repercussions such as masking program faults and leading
to spurious bug reports, and can be hard to identify
unless explicitly searched for.

Third, guided by the study, we propose a set of algorithms to detect
dependent tests. Our algorithms use both static and
dynamic program analyses to quickly \todo{the goal
of algorithms here.}

Fourth, we implement our dependent test detection algorithms
in a prototype tool, and apply it to \todo{NUM} real-world programs.
Our tool revealed a large number of dependent tests, suggesting
that \todo{the implication.}

\end{abstract}

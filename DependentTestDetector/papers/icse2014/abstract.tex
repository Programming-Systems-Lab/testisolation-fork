\begin{abstract}

In a test suite, all the test cases should be independent:  no test
should affect any other test's result, and running the tests in any order
should produce the same test results.
Test independence is important so that tests behave consistently as intended by the developers.
In addition, techniques such as test %selection and
prioritization assume that the tests in a suite are independent,
but they do not justify that assumption.
Test dependence is a little-studied phenomenon.
This paper presents five results related to test dependence.

First, we characterize the test dependence that arises in practice.
We studied \dtnum real-world dependent tests from \repnum
issue tracking systems.
Our study
%  identifies common characteristics of dependent tests.  It 
shows that test dependence can be hard for programmers to identify.
It also shows that test dependence can cause
non-trivial consequences, such as masking
program faults and leading to spurious bug reports.

Second, we formally define test dependence in terms of
test suites as ordered sequences of tests along with explicit
environments in which these tests are executed.
We formulate the problem
of detecting dependent tests and prove that a useful special
case is NP-complete. 

Third, guided by the study of real-world dependent tests, we
propose and compare three algorithms to detect
dependent tests in a test suite. 

Fourth, we applied our dependent test detection algorithms
to \subjnum real-world programs and found
dependent tests 
in each human-written
and automatically-generated test suite.

Fifth, we empirically assessed the impact of
dependent tests on five test prioritization techniques
and found that dependent tests affect the output 
of \textit{all} five techniques.

%the follwoing claim might be too strong
%Our tool revealed a large number of previously-unknown dependent tests.
%In our study, on average \todo{xx}\% of the human-written tests are
%dependent and \todo{xx}\% of the automatically-generated tests
%are dependent.

\end{abstract}

\label{dummy-label-for-etags}

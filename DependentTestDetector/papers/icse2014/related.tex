\section{Related Work}
\label{sec:related}

%\todo{missing related work, JCrasher~\cite{Csallner:2004}, the ISSTA'13 paper~\cite{Steimann:2013}}

%Much literature on testing omits details about the structure of a
%test suite; this is reasonable and unsurprising because particulars
%about test suites are often immaterial to the results or discussions.
Denoting a group of test cases as a ``suite of test programs'' began around the
mid-1970's~\cite[p.~217]{brown:CSUR:1974}; similar terms include
``testcase dataset''~\cite{milleretal:ICRS:1975} and ``scenario,''
which an IEEE Standard defines as ``groups of test cases;
synonyms are script, set, or suite''~\cite[p.~10]{IEEE:829-1998}.
Treating test suites explicitly as \emph{mathematical sets} of tests dates at least
to Howden~\cite[p.~554]{howden:ToC:1975} and remains common in the literature.
The execution order of tests in a suite is usually addressed implicitly
or informally, suggesting that the potential of executing a given test
in different contexts is immaterial to those results: that is, test
independence is assumed. 

%When a test suite is a set, the test suite passes if and only if all
%the test cases in it pass. 
%Combined with the unordered property of sets this suggests that the
%context in which each test is executed must have no
%effect on its result regardless of the ordering of execution.

%example~\cite[\emph{et alia}]{eldredetal:1978,harroldetal:OOPSLA:2001,staatsetal:ICSE:2011}.  

%\todo{xx}

\subsection{Test Dependence}

In addition to the work by Kapfhammer and
Soffa~\cite{kapfhammeretal:FSE:2003},
there are a handful of categorical references that
acknowledge that tests can
be dependent based on context, suggesting 
ways to document or find situations where the independence
assumption fails to hold.  

The IEEE Standard for Software and System Test
Documentation (829-1998) \S 11.2.7, ``Intercase
Dependencies,'' says in its entirety: ``List the identifiers of
test cases that must be executed prior to this test
case. Summarize
the nature of the dependences''~\cite{IEEE:829-1998}.  The succeeding version of this
standard (829-2008) adds a single sentence: ``If
test cases are documented (in a tool or otherwise) in the order in
which they need to be executed, the Intercase Dependencies for most or
all of the cases may not be needed''~\cite{IEEE:829-2008}.

McGregor and Korson discuss interaction tests that
are intended to identify ``two methods that may directly or indirectly
cause each other to produce incorrect results'' and suggest constructing such
interaction tests by identifying the values shared via parameter passing
between methods
 that two or more test cases share~\cite[p~.69]{mcgregoretal:CACM:1994}.

Bergelson and Exman characterize a form of test dependence
explicitly: given two tests that each pass, the composite
execution of these tests may still
fail~\cite[p.~38]{bergelsonetal:EEE:2006}.  That is, if 
\suite{t_1} executed by itself passes and \suite{t_2} executed by itself passes,
executing the sequence \suite{t_1, t_2} in the same context may fail.

Some practitioners acknowledge test dependence as a possible, albeit low probability, event:
\begin{quote}
Unit testing \dots  
requires that we test the unit in isolation. That is, we
want to be able to say, \emph{to a very high degree of confidence\/} [emphasis added], that
any actual results obtained from the execution of test cases are
purely the result of the unit under test. The introduction of
other units may color our results~\cite{unit-test-def}.
\end{quote}
They further note that other tests, as well as stubs and drivers, are
other units that may ``interfere with the straightforward
execution of one or more test cases.''

Other definitions of test dependence are primarily considered
to be \textit{syntactic} dependences between program units, for example
methods calling other methods, and classes using other classes~\cite{bergelsonetal:EEE:2006,briandetal:SEKE:2002}. 
\emph{Syntactic} dependence here means that a unit \code{A} cannot be
compiled and executed without unit \code{B} being present. If we test
such a unit \code{A} without convincing ourselves first that \code{B}
is correct, a test failure for \code{A} is harder to interpret,
because it could just as well indicate a fault in \code{B}.
Zhang and Ryder extend this notion to \emph{semantic} dependences,
which is closer to our approach~\cite{zhangetal:TR:2006}. 
They use a notion of
``test outcome'' to determine whether or not syntactically dependent
classes or methods can influence each others results, and consider
only those that can to be semantically dependent.
They give an informal definition of what it means for the execution of a
test to influence the outcome of another test.  We define
this precisely, and we also define manifest test dependence in terms
of execution environments
and test execution order rather than in terms of code use.

%Testing frameworks
%%, and 
%%IBM's Rational tool family, and Microsoft's MSDN---
%provide mechanisms
%to help developers define the context for tests more effectively.
%JUnit\footnote{\url{http://junit.org}},
%for example, provides means to
%automatically execute methods (\code{setUp()} and \code{tearDown()} in JUnit
%3.x, and methods annotated with \code{@Before} and \code{@After} in
%JUnit 4.x) to help handle common setup and clean-up tasks between
%tests. Ensuring that these mechanisms are used properly, however, is
%beyond the scope of any framework.
%\todo{DN}{I stole some of the above paragraph from the intro.  We will have
%to accommodate that by removing one and forward/backward references or such.}

%\todo{JW}{During my first read, I'm not sure why this is here and how
%it relates to the previous discussion}\todo{DN}{You're right, I'm taking it
% out.  And both
%of these todos.}


%Characterizing test suites as ``collections'' of test cases is
%increasingly common, especially in descriptions of tools.  For example,
%the Javadoc for the JUnit \code{TestSuite} class includes: ``A TestSuite is a Composite
%of Tests. It runs a collection of test cases.''
%Sun's JUnit Primer says: ``A TestSuite contains a collection of
%tests\dots'', and similar
%terms are found in literature from IBM for the Rational tool family, from Microsoft's MSDN, etc.

%\todo{DN}{Removed the footnote and comments on fail/fail composing to pass.  I
%was wrong and I don't think it matters or is worth confusing things to cover
%it at this point in the paper.}
%\footnote{They
%also assert that if $T_{1}$ and $T_{2}$ each fail in isolation, that their composite
%execution will fail.  This is a variant of the test independence assumption,
%which we also believe is/show is false.  DN: Any examples of fail/fail to pass?
%\todo{KM}{David, I don't believe fail/fail can lead to pass with composition.
%The reason is the following: \\
%Let $\Gamma$ be the initial environment. \\
%T1 == fail implies that R(T1, $\Gamma$) = fail \\
%T2 == fail implies that R(T2, $\Gamma$) = fail \\
%So, whatever order you choose to run T1 and T2, you will at least get one
%failure (the first test that is run) because we know that: \\
%R(\{T1, T2\}, $\Gamma$) = \{fail, *\} and \\
%R(\{T2, T1\}, $\Gamma$) = \{fail, *\}.}}

\subsection{Techniques Assuming Test Independence}

The assumption of test independence lies at the heart of most,
if not all, techniques for automated regression test selection,
test case prioritization, test generation, testing-based
fault localization, etc. 


Test prioritization seeks to reorder a test suite to detect
software defects more quickly. However,
the reordering is perhaps the most common way to change the
execution environment of a test. Early work in test
prioritization~\cite{Wong:1997:SER:851010.856115,Rothermel:1999:TCP:519621.853398}
laid the foundation for the most commonly used problem definition:
consider the set of all permutations of a test suite and find the best
award value for an objective function over that
set~\cite{Elbaum:2000:PTC:347324.348910}.  The most common objective
functions favor permutations where more faults in the underlying
program  are found with running fewer tests.
%determine the number of tests from the permutations that need
%to be run to detect a set of faults in the underlying program.
A number of results carefully study various prioritization algorithms
empirically, most by Rothermel and colleagues,
spanning over a decade~\cite[\emph{et
alia}]{Rothermel:1999:TCP:519621.853398,Do:2010:ETC:1907658.1908088}.  These evaluations are based in part on the assumption that the set of faults in the underlying program is known beforehand; the possibility that test dependence may unmask additional faults in the program is not studied.
% \todo{DN}{I can't get the et alia above to look right -- where does that semicolon come from?  I'd like it to be [2,9, et alia] or such.
% In a pinch, just include the two numbers.}


Echelon defines a
heuristic that exploits both a mapping between tests and
executed program paths and also a binary differencing between two
program versions to select a subset of tests intended to quickly
identify program faults~\cite{Srivastava:2002:EPT:566172.566187}.
Test dependence is not considered in the approach.



Test independence is explicitly asserted as a requirement for
prioritization by Rummel et al.:
\begin{quote}
A test suite contains a tuple of tests \suite{T_1 $\ldots$ T_R} that execute in a specified order.  We require that each test is
independent so that there are no test execution ordering dependencies.  This requirement enables our prioritization algorithm to
re-order the tests in any sequence that maximizes the suite's
ability to isolate defects.  The assumption of test dependence
is acceptable because the JUnit test execution framework
provides \code{setUp} and \code{tearDown} methods that execute before
and after a test case and can be used to clear application state~\cite[p.~1500]{Rummel:2005:TPR:1066677.1067016}.
\end{quote}

Automated test generation techniques~\cite{} are also affected
by test dependence. As we shown in our study (Section~\ref{}),
a large number of tests generated by Randoop are dependent.
It appears that test dependence has not been taken into
consideration in the design of most test generators.
We speculate that these dependences arise because automatic
test generation techniques generally create new tests
based on the program state after executing the previous test,
for the sake of test diversity and efficiency. 

Testing-based fault localization techniques are also impacted
by test dependence. Steimann et al. define a test suite as a
collection of test cases whose result is \textit{independent}
of the order of their execution~\cite{Steimann:2013}.
In their evaluation of several coverage-based fault locators, they found
fault locators' accuracy has been significantly affected
by tests failed due to the violation of the test independence assumption.
For example, if a test depends on a static field whose value is
set by previous test cases.  


\todo{tune down the sentence a bit, must mention the
degree of impact is unknown currently.}
For all above common testing techniques, if the test independence
assumption does not hold, then techniques that rely on
the test independence assumption may need to be reformulated.

\subsection{Tools Supporting Test Dependence}

Only a few tools explicitly considers test dependence, by
allowing developers to annotate dependent tests and
provide supporting mechanisms to ensure that the test execution framework
respects those annotations.  DepUnit~\cite{depunit}
allows developers to define soft and hard dependences. Soft dependences control
test ordering, while hard dependences in addition control whether specific tests are
run at all.  TestNG~\cite{testng} is a testing framework intended to improve upon JUnit,
and allows dependence annotations and supports a variety of execution policies such as sequential execution
in a single thread, execution of a single test class per thread, etc.\
that respect these dependences.
What distinguishes our work from these approaches is that, while they allow dependences
to be made explicit and respected during execution, they do not help developers
\emph{identify} dependences.  A tool that finds dependences (Section~\ref{sec:tool}) could co-exist
with such frameworks by generating annotations for them.


In the literature of automated test generation research,
to the best of our knowledge, only JCrasher~\cite{Csallner:2004}
provides a mode to clear the environment changes caused
by a previous test. Such functionality helps eliminate
potential test dependence, but may make generated
tests less behaviorally-diverse --- as they cannot be constructed
on top of previous tests. Exploring how to
incoporate test dependence into the design of automated
test generator is one of our future work.






%
%when re-executing it in the replay environment. that is the same
%related point as the test factoring work like david's note above.
%%also, 
%I suspect that Elbaum may point that out if he is reviewing this
%paper. I experienced that once, even getting reviews requesting to
%cite a much less related paper of his.



%\subsection*{Test Suites as Collections}


%and it is even the
%characterization of test suites used in Wikipedia (which also contradicts the IEEE Standard by stating that
%``[c]ollections of test cases are sometimes incorrectly termed a test plan, a test script, or even a test scenario.'')~\cite{wiki:test-suites}.
%It has been difficult---and seems unnecessary for our results---to identify precisely when ``collections''
%came into use for test suites.  It seems likely that this arose from the now-common
%terminology for containers that group multiple elements together: sets, lists,
%hash tables, arrays, etc.

%\subsection{Syntactic and Semantic Test Dependencies}


%Santelices et al.\ define a formal model of how changes might interact
%at the source code level and present a technique for detecting such
%interactions that arise at
%run-time~\cite{Santelices:2010:PDR:1828417.1828487}.  In contrast to
%our approach, they identify changes that interact rather than tests
%that depend upon each other.

%There are results that identify dependences that
%may surface, for example, when there is aggressive testing of parameter
%settings by a single test case; one (of many) approaches uses
%bounded model checking to vary the parameters~\cite{Sullivan:2004:SAB:1013886.1007531}.



%\todo{DN}{Young says: ``If you are referring to the work I am familiar with, I think what has
%been treated in some depth is combinations of parameter settings in a
%single test case.  That's a sort of dependence as well (e.g., the
%parameters might enable a couple of features that interact in nasty
%ways), but it doesn't involve variations in the ordering of operations
%or test cases.''  I can't find anything on this, but I'm probably searching
%wrong.  If something finds a good citation or two, please include it and write
%something here.  If not, delete this entirely.}

%Another kind of dependence helps address
%the testing of configurable software, which can be combinatorial with respect to the
%set of configurable options~\cite{Cohen97theaetg,Cohen:2003}.  

%In
%practice, there are often far fewer
%configurations that are used and thus should be tested.  This
%often structures the configuration space in a way that allows
%potential dependences to be explored more
%efficiently, as a test suite that binds multiple configuration option values
%can test all configurations that share those settings.
%The dependences considered in this approach are not between tests, but rather within
%the configuration option space.


%Another set of results searches for
%efficient (small) test suites that aggressively exercise potentially unexpected
%interactions---dependences---among the components of the
%program~\cite{Cohen97theaetg,Cohen:2003}.  These approaches are used for configurable software where
%in principle there are a combinatorial set of instances to test based on variations of the
%configuration options.  The objective is to temper the combinatorial blow-up by identifying
%components sharing some configurable option values; for example, 
%if a block of code is included in such system instances, a test that exercises that block
%can be used as a proxy for interactions between the shared options.
%
%These approaches tend to use covering arrays as a way to determine which of
%the dependencies among components are/are not exercised.  In the case of configurable
%software, there may be a combinatorial number of interactions to be considered.
%These approaches focus on generating a set of effective tests based on program interactions,
%rather than our focus on identifying dependent tests based on ordering, environment, and the
%results of the tests.
%
%
%In principle it could be combinatorial, and the idea of covering
%arrays is that most of the interactions that matter involve just a
%small number of choices.  If there are things that break only for a
%single setting of parameters A, B, C, D, E, F, G, H, we're hosed $A!-(B but
%if something breaks whenever B has value 1 and G has value 2, then
%something like covering arrays has a chance.  



%
%The research we propose is basic research that impacts most
%aspects of testing, ranging from (automatic) test generation through
%regression testing, test case selection and ending with
%considerations on the right test granularity. 
%Similar to the distinguished paper of Staats et
%al.~\cite{staatsetal:ICSE:2011}, we propose to give a rigorous
%foundation to an important aspect of software testing that is
%present but rarely examined in detail in current research.

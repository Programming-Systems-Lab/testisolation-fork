\section{Disucssion and Future Work}
\label{sec:discussion}

%In the introduction we posed several questions why test dependence
%may have received little attention despite the ease of constructing
%concrete but contrived examples. 
%contributions suggest answers, to differing degrees, to these questions:


Although we have studied and detected a substantive set of
real-world dependent tests, from both human-written test suites and
automatically generated test suites, a broader
investigation of the impact of dependent tests,
how to eliminate and prevent dependent tests
is beyond the scope of this paper. 
We next discuss a set of open questions addressing this and
other possible concerns resulting from dependent teests.
Exploring answers to these open questions comprises
our future work.

\vspace{1mm}

\noindent \textbf{{Impact of dependent tests.}}
Dependent tests can mask faults and lead to
spurious bug reports; they can also 
compromise the application of
testing techniques such as test selection,
prioritization, and parallelization, since
most current techniques just assume independence and
make no statement about what happens when this
assumption is not true. However, for both perspectives,
the extent of impact is unknown. This question
merits more comphrensive empirical studies.  


On the other hand, the question of how existing
testing techniques should
handle test dependence is also open.
For downstream testing techniques such as test
selection, prioritization, and parallelization,
one straightforward way to amend this situation
might be to augment such techniques to respect a
defined partial order among tests. And this partial order
can be derived from knowledge about dependent tests,
or be detected by our \ourtool tool.
%Like contrived examples of test
%dependence itself, it is easy to produce simple examples where
%downstream techniques produce incorrect output when applied to dependent
%tests.
%under the assumption that the input tests have no dependences.
%However, 



\vspace{1mm}

\noindent \textbf{{Eliminating dependent tests.}}
As found in our study (Section~\ref{sec:study}),
developers often ignore dependent tests due to
the lack of proper tool support. For some
dependent tests that get eliminated,
the practice of eliminating dependent tests
remains mostly ad hoc --- software developers
often manually specify test
execution orders (in a configuration file) or
simply merge tests, when a dependent test is reported. 
A systematic and more robust methodology for
dependent test elimination should be developed.

On the other hand, dependent tests in
an automatically-generated test suite can be 
more challenging to eliminate.
As suggested by our experiments, dependent tests are
more prevalent in automatically-generated test suites
than in human-written test suites.
Further, This is even exacerbated by the fact that
almost all automated test generation
techniques we are aware of produce tests
that are hard to read for humans, are undocumented, and their intent
cannot easily be gleaned from naming conventions and other aids
developers normally use. Therefore, it makes more difficult
to identify the root cause of dependence and then remove
the dependence. While there is some work to alleviate
this problem~\cite{fraseretal:ISSTA:2011}, the question
of eliminating  automatically-generated dependent tests
still remains open.


%As discussed in our experiments, it appears that test
%dependence in automatically generated test suites is 
%even more troublesome than in human-written suites. 


\vspace{1mm}


\noindent \textbf{{Preventing dependent tests.}}
Dependent tests, if occur, may have non-trivial consequences.
A natural question is how could software developers
prevent dependent tests at the beginning.

One possible way is asking developers to
explicitly specify test dependence
when writing tests in a testing framework that support
test dependence (see Section~\ref{sec:supporting} for details).
However, doing so may bring up the compatibility issue
for the existing tests.

Stylized coding patterns can also be used. Developers
should be encouraged to write tests ``defensively'' by
specifying necessary test execution pre-conditions and
properly mocking global variables or shared resources. 
There is already some work aiming at automating this
process to prevent the potential
for dependences by refactoring programs to use
less global state~\cite{wlokaetal:FSE:2009}. 

%We conjecture that if and when this happens in practice, it is
%hard to notice in part because current techniques~\cite{} do not surface the necessary information.

%defensive programming? explicitly state testing oracles?

%A better understanding of the 
%frequency and scope of consequences from test dependence should be
%developed.  

%Of particular concern is the
%masking of program faults because, unlike weaknesses
%in test suites or spurious bug reports, masking
%faults could be costly to find by other methods or to leave in the program.
%Tools that surface test dependences may help researchers
%and practitioners study and deal with dependences more effectively.





%\vspace{1mm}

%\noindent \textbf{\textit{What about other cases of test dependences?}}



%Several of our examples identified situations in which test dependence
%masked faults in the underlying program. 
%In another example, developers wasted time tracking down a non-existent
%fault because of a spurious report that was due to an undocumented
%test dependence. Even though these
%are real and reproducible examples,

%It is not possible to make any
%general claims about the frequency nor the significance of the
%repercussions of test dependence.  At the same time, it seems unlikely
%that these are the \emph{only} software systems where test dependence
%causes problems.


%\vspace{1mm}


%This is a more subtle question, 
%because the answer depends not only on the tools being used but also on the
%perceptions and insights of the developers.  If tools always run
%tests in the same context, and if developers never consider the possibility
%of test dependence, then it is unlikely that dependence will be observed.
%Masking of faults in the underlying program is a good illustration of this.

 

%\medskip


%Our 
%prototype tool shows that even our approximate algorithm
%can reveal large numbers of important dependences. Faster and more
%precise approaches are plausible, especially as more understanding
%of test dependence ``in the field'' is acquired.

%avoiding test dependences, to removing or documenting test dependences that are found, etc.


%Given a technique that detects dependence, what further
%  actions can and should be taken?


%also be supported by testimony from practitioners. With regard to the
%relevance of test dependence, the following questions seem to be of
%particular interest:
%\begin{enumerate}
%  \item Does test dependence occur often enough, and is its impact
%  critical enough to make further enquiry worthwhile?
%  \item 
%\end{enumerate}

%The frequency and consequences of test dependence in our work
%so far seem to justify further investigation.  We are especially
%concerned about whether test dependence is masking 

%We strongly believe that both the frequency and the
%impact of test dependence merit investigation of the phenom\-e\-non. The
%fact that within our small example set we found a number of masked
%faults that directly impacted the users of the software, and the
%observation that any form of test dependence prevents the successful
%use of many second-order testing techniques are strong
%indicators that this phenomenon deserves as much attention as any
%other technique that can improve the bug finding strength of testing.

%Thinking about the higher-level causes of test dependence leads to a large number of different questions,
%many of which relate not only to technical issues, but also to the social
%and human environment in which software is created. Exploring this
%domain, while very difficult to do well, is likely to yield the best
%explanations for the creation of test dependence.

%How actionable the knowledge of test dependence is, ironically,
%depends. At this early stage, we suggest inspection of code, tests and
%specifications to understand what causes the dependence, and where the
%fault lies. As our examples show, dependence can point to faults in
%the program, in the tests, or to insufficient documentation and
%specification. We hope and expect that it will be possible to
%determine or exclude some of these causes automatically with further
%analyses of code and tests. 
%There are also several
%projects that either augment or replace JUnit with the express goal to
%declare test dependence
%explicitly\footnote{\url{https://code.google.com/p/depunit/},\\
%\url{https://code.google.com/p/junitum/},\\ \url{http://testng.org}}, but not to detect it.




%
%
%The key take-aways from this paper are that  prior work and existing
%tool broadly assume test independence,
%%is broadly assumed by prior work and by existing tools 
%and that there is
%at least incipient empirical evidence that this assumption can lead to
%unexpected and likely negative repercussions.  
%%We present an initial
%formalization of test dependence that embodies both the execution
%order of a test suite and also the environment in which tests in a
%suite are executed. % the formalization allows us to prove that the
%problem of determining if there are dependences among the tests in a
%test suite is NP-complete.  
%Substantive examples of test dependences
%in the field, along with descriptions of the consequences of these
%dependences, argue the potential practicality of further
%investigations of test dependence.  
%Initial algorithms designed with the
%NP-completeness of the problem in mind, along with an initial tool, allow
%us to take initial steps towards practical applications, as well as to check
%the validity of examples of test dependence that we had previously identified
%in \emph{ad hoc} ways.\todo{JW}{This is all a bit too initial...}  A set of open questions related to test dependence 
%provides a partial, surely incomplete, roadmap for further work in the area.

%\todo{JW}{I think this section should be about
%\begin{enumerate}
%  \item What prompted our research
%  \item Which research questions did we answer, and which research 
%  questions remain open.
%  \item Summary what the reader should take away from the paper.
%\end{enumerate}
%The current version falls short of this in several ways. 
%\begin{enumerate}
%  \item It doesn't address ``the big picture'' at all.
%  \item It focuses on minor technical details \emph{and} phrases
%  minor technical issues as research problems in a way that assumes
%  answers to interesting questions we didn't even ask (comments in the
%  text below elaborate on that)
%  \item It focuses too much on our ``results'', which is wrong,
%  considering how little actual data we have.
%\end{enumerate}
%}

%More concretely, in this paper, we examine the importance of test dependence in theory and practice. 
%While the
%research literature usually assumes tests to be independent, or evades
%the issue entirely, our interest was piqued when we found a number of
%test dependences in real-world software.
%This led us to explore in more depth if this is an issue in a broader
%range of software systems, and whether test dependence causes real
%problems. 
%


%We found that the human-written test suites of many open-source libraries contain
% dependent tests, that test suites automatically generated with
% Randoop contain even larger numbers of dependent tests, and that in
% both cases these tests cause various
%kinds of problems, from preventing test prioritization to actually
%hiding real faults in the programs. However, we must emphasize that
%our exploration should not be construed as a complete
%experiment that provides conclusive evidence. Rather we inspected programs that we were familiar with.
%Hence, before the conclusions drawn from the examples in
%Section~\ref{sec:examples} can be generalized,
%a proper controlled experiment should be carried out with a broader
%range of projects, and with proper control for confounding factors
%such as project type and programmer expertise, which most likely have
%a profound impact on test depenendence. Nonetheless, we believe that the anecdotal evidence we
%presented in this paper makes clear that this is a worthwhile research
%endeavour.

%As a main contribution of this paper we identify test dependence as
%an important research subject that so far has been mostly ignored by
%the software testing community.

%Before summarizing our work and contributions, we outline a set of
%open research topics that could further improve the way we identify
%and manage test dependences: 
%\begin{itemize}
%
%\item Our initial algorithms for detecting test dependences 
%leave room for improvement in efficiency.  Conventional optimizations
%should apply \todo{JW}{I don't know what this means. We proved that
%the problem is NP-complete the full algorithm is exponential.
%``Optimizations'' in the case can only mean approximations, right?}, as should incremental algorithms and/or on-line versions of
%the algorithms --- for when test cases are added to a suite, among other
%potential performance improvements.\todo{JW}{What's the point here?}
%
%\item \todo{JW}{This is a theory paper. I think this is completely
%irrelevant} Our initial tool for detecting test dependences 
%leaves room for improvements in applicability (what ``forms'' of test
%suites/programs do we handle, such as JUnit, etc.?), in user
%interface, etc.
%
%\item We have only scratched the surface of the interaction
%between test dependence and downstream testing tools like selection,
%prioritization, and parallelization.  \todo{DN}{We have claimed earlier
%we will show that dependence can cause these approaches to fail.
%Where in the paper do we do that, and how do we do that?  I believe it's
%pretty obvious, but we'll need to be careful about doing it.} \todo{KM}{I think
%we even have real life examples (from Mike), etc. for this, however I might be
%mistaken. Generating theoretical example is obvious, but I agree that we lack
%that example at the moment.} How should these interactions be handled? For
%example, should a ``test dependence manager'' ensure that suites have no dependences before a downstream tool is invoked or should those tools check that their output test sub-suite has no dependences?
%
%\item \todo{JW}{This is an interesting question, as it to some extent
%asks what is actionable about dependencies. However, I think this
%should be discussed elsewhere (in the motivation, examples?), because
%this isn't really a research question or anything. The answer to this
%seems pretty obvious to me, and is what is described in this
%paragraph.} What should a development/test team do when a test dependence is identified?
%It could indicate that there is a problem in the tests themselves, in which case
%they could be fixed.  It could indicate that there is a problem in the program
%being tested, in which case it could be fixed.  It could indicate that the
%dependence is necessary, in which case the test could perhaps be merged together
%to ensure that the dependence is respected by the test framework and downstream
%tools.
%
%\item \todo{JW}{What?} We argue additional work exploring the interrelationship between
%individual unit tests should be performed, and show initial examples
%that test dependencies must be considered when executing a test suite,
%fixing a regression error, and generating new tests.
%
%\item This leads to our next recommendation for further software testing
%research. Some work in this area, discussed previously, has already
%attracted attentions. Nevertheless, we strongly beleive that more
%work exploring open questions on how to integrate \textit{test dependence}
%to the testing process is necessary to be understood.
%
%\todo{JW}{The following two paragraphs are too narrowly focused on
%test dependences being a bad thing that needs to be avoided. I would
%at the very least at the meta question asking when, where, how often
%dependences are bad, intentional, necessary, practical etc.}
%
%First, how to eliminate existing dependent tests (or retrofitting
%dependent tests into tests without inter-dependencies)? A critical
%part in retrofitting existing dependent tests is to identify test
%code that may affect the behavior of our tests, and then make the
%affected tests more ``robust'' to such affecting code. 
%\todo{DN}{Something especially about initialization code for tests?}
%
%\todo{JW}{This is too vague, both the problem description and the
%solution.}
%
%
%Second, how to prevent new dependent tests being produced?
%Programmers should be encouraged
%to encapsulate common test execution environment setting up and destroying
%code into separate pre- and post-conditions, to ensure each
%test is executed in a desirable environment and also probably cleared up
%the environment after its execution. One promising way to achieve
%this is to extend Design-by-Contract~\cite{Leitner:2007} to human-written unit
%tests. For an automated tool, it may employ
%capture-and-replay~\cite{Elbaum:2006} techniques to save the probable environment when a test is created,
%and then recover the needed environment when executing a
%test.\todo{JW}{This sounds weird to me. How would that work.}
%
%\item Finally, what would be the impact of dependent tests to the whole software testing
%process?  Most existing research in the fields of regression selection
%and prioritization has an implicit assumption on test dependence.
%.\textbf{unfinished}.. \todo{JW}{This could be interesting, but I
%doubt there is any general conclusion we can draw here.}
%
%\end{itemize}


%Our formalism provides a precise definition of manifest test dependence,
%allows reasoning about test dependence, and enables the proof that
%detecting manifest test
%dependence in test suites is NP-complete.  

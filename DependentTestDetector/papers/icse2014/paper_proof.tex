\title{Correctness Proof of the Dependence-Aware $k$-Bounded Algorithm}

\documentclass[12pt]{article}

\begin{document}
%\maketitle


\noindent \textbf{Correctness Proof of the Dependence-Aware $k$-Bounded Algorithm}\\


\noindent \textbf{Theorem} All tests in a test permutation discarded by the getPossibleExecOrder
function in the dependence-aware $k$-bounded algorithm are guaranteed to produce the same results as in its default order.\\



\noindent \textbf{Proof.} 

When $k=$1, the algorithm discards tests that never read or write a global field. A test that does not access any global field will never be affected by other tests, and thus is guaranteed to produce the same result in any execution environment.

When $k\ge$2, We prove this theorem by contradiction. Given a test suite $T$ and a test permutation $P$ of $T$, let's assume  $P$ is discarded by the
getPossibleExecOrder function. Based on the definition of the algorithm,
all global fields read by each test in $t \in P$ are \textit{not} written by the tests
executed before $t$ in $P$. Further, the initial value of each shared field is the
same across different executions.


Suppose there exists a test $t \in P$ produces a different result than in $T$.
Test $t$ must read a different value for at least one shared field. Without loss
of generality, let's suppose $f$ is the shared field read by $t$ with a different
value. There are two cases about $f$ and $t$:

\begin{enumerate}
\item $f$ is not written by any tests executed before $t$ in $P$. So that, $t$ must
read a different initial value of $f$. This contradicts with our assumption
that each global field has the same initial value across executions.

\item $f$ is written (with a different value) by another test $t'$
executed before $t$ in $P$. There are two cases about $t'$:

\begin{enumerate}
\item $t'$ writes $f$ when executed in isolation. This contradicts with our
assumption that every global fields $t$ reads is not written all tests
(including $t'$) when executed in isolation.

\item $t'$ does not write $f$ when executed in isolation. In this case, there must exist another test $t''$ $\in P$ executed before $t'$
in $P$, whose execution changes the execution path of $t'$
and causes $t'$ to read $f$. In other words, $t''$'s execution interferes with $t'$ , so that (1) $t''$ reads a different initial value for $f$
(this contracts with our assumption), or (2) there must be another global field $f'$ which $t'$ reads and $t''$ writes.
\end{enumerate}

\end{enumerate}

Based on case 2(b), since $P$ contains a finite number of tests, there must exist a test $\hat{t} \in P$ that has no tests 
executed before it and writes to the shared 
fields it reads, but $\hat{t}$ still reads a
 different value for at least one shared field. The only possibility is that $\hat{t}$ reads a different initial value
for at least one shared field. This contradicts the assumption that tests are
executed in the same environment with the initial values of all shared 
fields.

\end{document}
\section{Empirical Evaluation}
\label{sec:evaluation}

To show the effectiveness of our proposed
dependent test detection algorithms, we conducted
an evaluation on \todo{XXX} open-source programs (Table~\ref{tab:subjects}).
In our evaluation, we seek to answer the following research questions:

\begin{itemize}
\item \textbf{RQ1:} How effectively do our algorithms detect
dependent tests?
\item \textbf{RQ2:} How do the proposed static and dynamic program analyses
in the improved algorithm increase the algorithm efficiency?
\end{itemize}

\todo{check the research questions above? any more?}

\subsection{Subject Programs}


\begin{table}
\centering
\setlength{\tabcolsep}{0.4\tabcolsep}
\begin{tabular}{|l|l|c|c|l|}
%\toprule
\hline
\textbf{Program} & \textbf{LOC} & \textbf{\#Tests} & \textbf{\#Auto Tests} & \textbf{Revision}
\\
\hline
%\midrule
JodaTime & 27183 & \jodatimetests
% 3875 is retrieved by running mvn test on the related revision
& \jodatimeautotests&  b609d7d66d\\
XML Security & 18302 & \xmlsecuritytests & \xmlsecurityautotests& version 1.0.4 \\ 
Crystal & 4676 & \crystaltests & \crystalautotests& trunk version\\
Synoptic & 28872 & \synoptictests & \synopticautotests&  trunk version\\ 
%\bottomrule
\hline
%\textbf{Total}& &  & &  \\ 
%\hline
\end{tabular}
\caption{Subject programs used in our evaluation.
Column ``\#Tests'' shows the number of human-written
unit tests associated with each program. Column
``\#Auto Tests'' shows the number of automatically-generated
unit tests for each program, by Randoop~\cite{PachecoLET2007}.
}
\label{tab:subjects}
\end{table}

\newcommand{\unknown}{N/A\xspace}
\newcommand{\ignore}{---\xspace}
\newcommand{\infy}{$\infty$\xspace}

\begin{table*}
\centering
\setlength{\tabcolsep}{0.12\tabcolsep}
\begin{tabular}{|l|c|c|C|C|C|c|c|c|c|c|c|c|c|c|c|c|c|c|}
%\toprule
\hline
\textbf{Subject} & \textbf{\#} & \multicolumn{8}{c|}{\textbf{\# Detected Dependent Tests}} & \multicolumn{8}{c|}{\textbf{Analysis Cost (seconds)}}\\
%\midrule
\cline{3-18}
\textbf{Programs} & \textbf{Tests} & \textbf{Rev} & \multicolumn{3}{c|}{\textbf{Randomized}} & \multicolumn{2}{c|}{\textbf{Exhaustive }} & \multicolumn{2}{c|}{\textbf{Dep-Aware}} & \textbf{Rev}& \multicolumn{3}{|c|}{\textbf{Randomized}} & \multicolumn{2}{c|}{\textbf{Exhaustive }} & \multicolumn{2}{c|}{\textbf{Dep-Aware}} \\
%\cline{3-8}\cline{10-15}
& & & \smalltrialnum & \mediumtrialnum & \trialnum& \; $k$=1 & $k$=2 & \quad $k$=1 \;\; \quad & $k$=2 && \smalltrialnum & \mediumtrialnum & \trialnum & \; $k$=1 & $k$=2 &  \quad $k$=1 \quad \quad & $k$=2  \\
\hline
%\bottomrule
\multicolumn{18}{|l|}{ }\\
\multicolumn{18}{|l|}{\textbf{Human-written unit tests} }\\
\hline
%JFreechart & \jfreecharttests & 6 & 8 & 8 & 0 & $\ge$0 * & 0 & $\ge$0 * &  66  & 625 & 6097 & 694 & 2$\times$$10^6$ *  &310  &  1$\times$$10^6$ *\\
%the data of jfreechart is above, MUST update the total column
\jt & \jodatimetests & 2 & 1 & 1 & 6 & 2 & $\ge$2 * & 2& $\ge$2 * & 18&   57 & 528 & 5538 &1265& 4$\times$$10^6$ * & 291 & 5$\times$$10^5$ *  \\
XML Security& \xmlsecuritytests & 0 & 1 & 4 & 4 &4 &4 & 4 & 4  & 18&65 & 594 & 5977 & 106 &  11927 & 93 & 3322  \\
Crystal & \crystaltests & 18 & 18 & 18 & 18 &17&18& 17 & 18 & 3 &14& 131 & 1304 & 166 & 7323 & 95  & 4155 \\
Synoptic & \synoptictests & 1 & 1 &1  & 1 & 0 &1 & 0 & 1 & 2 &  7 & 67 & 760& 25 & 3372& 24 & 1797 \\
\hline
\textbf{Total} & \totaltests & 21 & 21&24&\textbf{29}& 23 & $\ge$24 & 23 & $\ge$25 &41&  143 & 1320 & 13579 &1562&  4$\times$$10^6$ *& 503  & 5$\times$$10^5$ *\\
\hline
\multicolumn{18}{|l|}{ }\\
\multicolumn{18}{|l|}{\textbf{Automatically-generated unit tests} }\\
\hline
%JFreechart & \jfreechartautotests& \ignore & \ignore & \ignore & \ignore & \ignore & \ignore & \ignore & \ignore & \ignore & \ignore & \ignore & \ignore & \ignore &  \ignore \\
\jt & \jodatimeautotests &\ignore & \ignore & \ignore & \ignore & \ignore & \ignore & \ignore & \ignore & \ignore & \ignore & \ignore & \ignore & \ignore & \ignore & \ignore &  \ignore \\
XML Security& \xmlsecurityautotests&138& 167 & 171 & 171 & 129 & $\ge$129 * & 128  & $\ge$128 *   & 7& 50 & 430 & 4174 & 133 & 1$\times$$10^5$ * & 128 & 5$\times$$10^4$ * \\
Crystal & \crystalautotests & 75 & 159 & 162 & 164 & 55 & $\ge$55 * & 55 & $\ge$55 *  & 22 & 103 & 949& 9436  & 2477 & 8$\times$$10^6$ *& 2297 & 1$\times$$10^6$ * \\
Synoptic & \synopticautotests &3 & 3 & 7 & 10 &2& $\ge$2 * & 2 & $\ge$2 *   & 13 &81& 770  & 6311 & 454 & 1$\times$$10^6$ *& 454 & 2$\times$$10^4$ * \\
\hline
\textbf{Total} & \totalautotests &216 &329 &340 & \textbf{345} & 186 & $\ge$186  & 185 & $\ge$185  &42&234&2149& 19921& 3064 & 1$\times$$10^7$ *& 2879& 1$\times$$10^6$ * \\
\hline
\end{tabular}
\caption{Experimental results.  Column ``\# Tests'' shows the total number
of tests, taken from Table~\ref{tab:subjects}. Column ``\# Detected Dependent Tests''
shows the number of detected dependent tests in each test suite.
% Columns ``Rev'', ``Randomized'', ``Exhaustive'' and ``Dep-Aware'' show the results
% of applying the reversal algorithm, randomized algorithm, exhaustive $k$-bounded algorithm, and the \dependenceaware{}, respectively.
%$k$-bounded algorithm, respectively. 
When evaluating the randomized algorithm, we used $\mathit{numtrials}$ =
$\smalltrialnum$, $\mediumtrialnum$, and $\trialnum$ (Figure~\ref{fig:randalgorithm}).
%``\unknown'' means the technique does not scale to the test
%suite (i.e., requiring more than 1 day to execute all test permutations),
%and thus the exact number of dependent tests is unknown.
``\ignore'' means the test suite is not evaluated due to its non-determinism.
%Column ``Analysis Cost''
%shows the time cost of each algorithm.
An asterisk (*) means the algorithm did not finish
within 1 day:
the number of dependent tests is those discovered before timing out, and 
the time estimation methodology is described in Section~\ref{sec:performance}.
\tinyrelax
}
\label{tab:results}
\end{table*}

%  LocalWords:  Joda numtrials


Table~\ref{tab:subjects} summarizes the programs used in our evaluation.

JodaTime~\cite{jodatime} is an open source
date and time library intended to improve upon the weaknesses of the
date and time facilities provided by the standard JDK.
%written to enhance the capabilities provided by the standard
%JDK such as allowing multiple calendar systems. 
It is a mature project that has been under active development
for more than eight years.

XML Security~\cite{xmlsecurity}
is a component library implementing XML signature and encryption
standards. Each released
version of XML Security has a human-written JUnit test suite that
achieves fairly high statement coverage.


Crystal~\cite{crystal} is a tool that
pro-actively examines developers' code and precisely identifies and reports on textual, compilation, and behavioral conflicts.

Beanutils~\cite{beanutils}
is a library that provides services for collections of
Java beans. 

\todo{say why choose these subject programs}

\subsection{Evaluation Procedure}

\todo{including manual tests and auto-generated tests}

We used the prototype to detect dependent
tests in the subject programs in Table~\ref{tab:subjects} using isolated execution ($k = 1$)
and pairwise execution ($k = 2$).

While we believe that most test dependences can be found with small
$k$. This is in part because the set of dependent tests that can be
found with a bound $k$ is always a subset of the set of dependent
tests that can be found with any bound $k' > k$. Additionally, our
empirical study (Section~\ref{sec:study}) indicates that small $k$
find many dependences. 

%, while larger $k$ do not. However, in principle
%it is conceivable
%that any number of chain dependences with chains longer than any tried $k$ exist
%in all the libraries we analyzed.


For each subject program, we also used Randoop~\cite{PachecoLET2007}, a state-of-the-art automated
test generation tool, to generate 5,000 tests
for each program, and then drop what it considers to be redundant
tests. Then, we applied our prototype to detect dependent tests in these automatically-generated
test suites.

\todo{add more about how to identify dependent tests in the tool's output}

%The discussion of the examples in this section is distinguished by
%the problems caused by test dependence (\emph{Kind}): when faults are masked because
%tests make incorrect assumptions about the global environment (Section~\ref{sec:mask}); 
%when tests do not
%respect required initialization protocols (Section~\ref{sec:examples:initialization}); and when
%undocumented test dependence leads to spurious bug reports (Section~\ref{sec:spurious}).
%We also describe dependent tests in an automatically-generated test
%suite (Section~\ref{sec:autogen}).


\subsection{Results}

\todo{show the basic results here, i.e., the num
of dependent tests found. Its consequence is discussed
below.}

\subsubsection{RQ1: Algorithm Effectiveness}

All the dependent tests reported in Figure~\ref{fig:example-summary},
except for two dependent tests in JodaTime and the dependences in SWT, 
can already be found by isolated execution.
 
During manual bug diagnosis in JodaTime, we identified two test dependences that require
\emph{three} tests to manifest. While these are easy to reproduce, we
did not check that our tool finds them, because the time needed to
run our naive algorithm on JodaTime with $k=3$ is measured in months.

\todo{say this concurs with the study findings}

\subsubsection{RQ2: Improvement from Program Analyses}

\subsection{Dependence in Human-written Tests}

\todo{The following text needs to be re-organized.
Be consistent to the categories used in the study section}

Most of the test dependences we found in human-written
test suites arise due to incorrect initialization
of program state by one or more tests. Such test
dependence often suggests that a test, or a test suite, 
has been constructed poorly in some dimension. 
While test dependences that mask faults
correspond to a defect in the program source,
these dependences correspond to defects in the test code.

There are two common patterns where incorrect
initialization leads to test dependence.
The first example is probably the most common. 
The tested program code relies on a
global variable that is a part of the environment, but the test does
not properly initialize it.  In the second case, a test should but
does not call
an initialization function before later invocations to a complex library.
This flaw in the test code is masked because the default test suite execution
order includes other tests that initialize the library.  The defect is
inconsequential until and unless the flawed test is reordered, either manually or by
a downstream tool, to execute before any other initializing test.

We next show several examples to illustrate test dependence in human-written
test suites.

\newcommand{\jodatime}{JodaTime\xspace}
%\paragraph{\jodatime: Complex interactions that mask faults}
%\label{sec:jodatime}
%\newcommand{\periodType}{\texttt{Period\-Type}}
\newcommand{\durationFieldType}{\texttt{Duration\-Field\-Type}}
\newcommand{\forFields}{\texttt{for\-Fields}}


\jodatime\ uses intricate caching mechanisms that are high\-ly complex
and coupled.  All dependences we found are complex,
in two cases even requiring a
specific ordering of \emph{three} tests to manifest.

In a simple dependence, \jodatime{} caches \periodType{} objects, which 
% Caching is done by using a
% global \texttt{HashMap} that holds the \periodType{}s that are created by
% \forFields{} method. 
contain an array of
\durationFieldType{}s (e.g., week, month). 
The order of \durationFieldType{}s in the array is an
important of the data representation, and 
two \periodType{}s with the same \durationFieldType{}s in a different
order are not equal internally in \jodatime, even though they are equal
to \jodatime clients.
%However, this implementation detail should be transparent to the user: As long
%as two \periodType{} objects have the same \durationFieldType{}s, they should represent
%the same period. 
To make this internal detail transparent to users of \jodatime, 
new \periodType{}s are normalized before they are cached. However, a fault in the code 
% checking for existing objects
makes it possible to insert non-normalized \periodType{}s into the
cache, leading to cache misses when searching for correctly normalized
\periodType{}s.
%This is even acknowledged by the developers: \forFields{}
%method first creates a \periodType{} using method argument \texttt{types} (the
%ordering provided by the user) and checks whether the cache already contains
%this ordering. However, if this fails, then another \periodType{} with
%normalized ordering (\texttt{checkedType}) is created at the end of the method 
%and the cache is rechecked as shown in Figure~\ref{fig:jodatime_forFields}.


%\begin{figure}
% \lstset{language=Java,numbers=left}
%%\lstset{language=Java}
%\begin{lstlisting}
%PeriodType forFields
%  (DurationFieldType[] types) {
%  ...
%  PeriodType input =
%    new PeriodType(null, types, null);
%  ...
%  // recheck cache in case
%  // initial array order was wrong
%  PeriodType check = ...
%  PeriodType checkedType = cache.get(check);
%  if (checkedType != null) {
%    cache.put(input, checkedType);
%    return checkedType;
%  }
%  cache.put(input, type);
%  return type;
%}
%\end{lstlisting}
%\caption{Fault related code from \code{Pe\-ri\-od\-Type.for\-Fields}
%\newline (rev. 3937d82f6670e5a30b2809b13cb6d05a7e606037)}
%\label{fig:jodatime_forFields}
%\end{figure}
%
%
%In spite to the developers' extra check, a small mistake in
%Figure~\ref{fig:jodatime_forFields} creates a bug in the program. Note that the
%developers are putting the \texttt{input} variable (the ordering provided by
%the user) as the `key' of the cache.
%As a result, if \forFields{} method is called with a wrongly ordered
%\texttt{types} parameter first, then a \periodType{} with wrong ordering will be
%added to the cache. Later, calling the same method with correct ordering of
%the same content causes a cache miss.

A test that checks for correct normalization when
caching objects
%The developers even have a very simple test case that checks for this.
%\texttt{Test\-Period\-Type.test\-For\-Fields4} method creates two
%\durationFieldType{}s: first with wrong ordering and the second with correct
%ordering, both representing the same period. The test creates the corresponding
%\periodType{}s by calling \forFields{} method with these \durationFieldType{}s.
%Finally, the objects retrieved from \forFields{} method are asserted for
%equality. 
fails in isolation but passes when the entire test suite
executes in the default order; this happens
because a prior test creates the expected
\periodType{}, and thus it is already in the cache for the
later test.
This behavior has been reported as a bug and has been fixed by the
developers.
%never fails during the development due to a simple dependency with the previous
%test in the same class. \texttt{test\-For\-Fields3} method creates the same
%\periodType{} content --- which will be used in the next test --- with the
%correct ordering for some other purpose. As a result, this \periodType{} is added with
%the correct ordering to the cache, which guarantees correct retrieval for both
%the correct and wrong ordering in the next test.


%The test case that would catch the bug was written the same revision it is
%introduced. However, since the developers never ran that test in isolation the
%bug lived for more than six years. During this time period, there have been 773
%commits for the project. Even the buggy file (\periodType{}) is
%changed for nine times and the buggy method (\forFields{}) is changed
%once. Finally, the bug gets reported by a
%user\footnote{\url{http://sourceforge.net/mailarchive/message.php?msg_id=28501345}}
%in December 06, 2011 and is fixed the same
%day\footnote{\url{https://github.com/JodaOrg/joda-time/compare/b609d7d66d...d6791cb5f9}}.
%During the same commit, the developers also removed the dependency for
%the related test by creating a unique
%\periodType{} for that test. The actual fix contains changing two
%variables:
%two instances of variable \texttt{input} (possibly wrong ordering) to variable
%\texttt{check} (to correct ordering) at the end in
%Figure~\ref{fig:jodatime_forFields}.

After inspecting the code, we reported the more complex dependence of
three tests to the developers of \jodatime. They confirmed the
phenomenon, but contended that it is due to interactions that are not
intended in the design of the library~\cite{jodatime}. In particular, one of the
methods, \code{DateTimeZone.setProvider()}, is only supposed to be
called a single time to initialize the library.  In practice, multiple
tests initialize the library, which leaves incorrect values in the cache
and causes other tests to fail under some execution orders.
%However, the tests call it more
%than once, which causes at least two cases to %break the caching
%%mechanism by leaving 
%leave incorrect values in the cache, causing the tests to fail.

\todo{here is a jodatime example commented out in the source, that dependence
masks faults. It is very verbose, should we discuss that in the paper.}

%\subsubsection{Poor Test Construction}\label{sec:examples:initialization}

%Based on our interaction with the \jodatime developers, this last
%dependence does not
%mask a fault in the program.  
%
%In contrast to the previous section
%where the dependences led to defects in the program source, this section concerns defects
%in the test source.

%In some sense, dependences that are due to missing initialization are
%the dual to dependences that mask faults.  Both reveal problems in source code.
%However, masked faults reside in the program source, while incorrect
%initialization is a fault that resides in the test suite.

%
%The following two examples show two common patterns where incorrect
%initialization leads to test dependence.
%The first example is probably the most common. 

%The second example employs a common pattern for complex
%libraries that requires a call to an initialization function before
%any other part of the library can be used.
%In both cases, other tests perform the required setup, and because
%they occur before the dependent tests in the normal execution order,
%no tests fail under normal circumstances.

\paragraph{Crystal: Global Variables Considered Harmful}
%Dependent tests in Crystal fall into the following three groups, which
%share the same root cause of global data dependence across multiple tests:
%
%\begin{itemize}
%
%\item 9 dependent tests come from the \CodeIn{DataSourceTest} class.
%In that class, a test method \CodeIn{testSetField} initializes a global variable \CodeIn{data}
%and other test methods read the value of the \CodeIn{data} variable.
%As a result, when a test using \CodeIn{data} is executed in isolation or executed
%before the \CodeIn{testSetField} method, a \CodeIn{NullPointerException} is
%thrown.
%
%\item 7 dependent tests come from class \CodeIn{LocalStateResultTest}.
%In that class, a test method \CodeIn{testLocalStateResult} initializes a global variable
%\CodeIn{localState} and other test methods use that variable. Therefore,
%7 tests using the \CodeIn{localState} variable exhibits
%a \CodeIn{NullPointerException} when they are executed before \CodeIn{testLocalStateResult}.
%\todo{KM}{I see no difference between the first item and this one. They all
%seem to happen due to one test initializing a global variable and the others
%reading the same global variable}
%
%\item 1 dependent test comes from class \CodeIn{ConflictDaemonTest}. This
%test uses a shared global variable which requires other tests in the
%same test class to initialize, \todo{KM}{Again, the same thing. I believe we
%should say that all dependencies are due to initialization of a global variable
%and then explain all of them as three sentences (instead of bullet pointing
%them as they were really different)}
%\end{itemize}



\paragraph{XML Security: Global Initialization}

%Are these test dependences realistic, or part of the modifications SIR
%made? by SZ: they are realistic, we use the original version without
%any modification from SIR people

XML Security\footnote{\url{http://projects.apache.org/projects/xml_security_java.html}}
is a component library implementing XML signature and encryption
standards. Each released
version of XML Security has a human-written JUnit test suite that
achieves fairly high statement coverage.

Four stable released versions (1.0.2, 1.0.4, 1.0.5d2, and 1.0.71) of XML Security
have been incorporated in the Soft\-ware-artifact Infrastructure Repository
(SIR).\footnote{\url{http://sir.unl.edu}}
We found that at least two out of the four versions contain dependent tests. Specifically, in versions 1.0.4 and 1.0.5d2, \code{test\_Y1}, \code{test\_Y2}, and \code{test\_Y3}
in class \code{ExclusiveC14NInterop} show dependent behavior.
Since the dependences are the same in both versions, in the further
discussion and in Figure~\ref{fig:example-summary}, we consider only
version 1.0.4.

For all three dependences, the cause of the dependence is the same: before any
method in the library can be used, the global initialization function 
\code{Init.init()} has to be called. Internally, it initializes
the static field that the code tested by the dependent tests rely
on.

Given that the error when executing the dependent tests clearly explains the
cause of the error, we speculate that developers either simply forgot to
initialize the tests properly, or expected that these tests would always execute
in the order defined
in the test suite.

%on a global variable. Take \code{test\_Y1} as an example. This test passes when being executed
%with other tests, but fails by throwing an \code{InvalidCanonicalizerException}
%when executed individually.
%The root cause of such behavior difference is that, in XML-security, the \code{Init.init()} method initializes
%the static field \code{Canonicalizer.canonicalizerHash}, and test \code{test\_Y1} needs to use
%that static field to create a \code{Canonicalizer} instance. 
%When executing this test in the programmer-fixed order, method \code{Init.init()} has been called by
%other tests executed before \code{test\_Y1}, so that test \code{test\_Y1} passes.
%However, without calling \code{Init.init()} first,
%\code{test\_Y1} fails to create the \code{Canonicalizer} instance.
%
%Based on the dumped error message in the \code{InvalidCano-\\nicalizerException}:
%
%\begin{quote}
%``You must initialize the xml-security library correctly before you use it.
%Call the static method ``org.apache.xml.security.Init.init()'' to do that before you use any functionality
%from that library''
%\end{quote}

%We speculate that programmers should realize this potential dependence, but they
%overlook to enforce \code{test\_Y1} to be executed in a desirable order. Instead,
%programmers may have put an implicit assumption that tests in a suite can be executed in isolation
%and miss to add the necessary preconditions for \code{test\_Y1}. 

% vim:wrap:wm=8:bs=2:expandtab:ts=4:tw=70:



\subsection{Dependence in Auto Generated Tests}
\label{sec:autogen}
\newcommand{\pub}{\texttt{Prop\-er\-ty\-Utils\-Bean}}
\newcommand{\fhm}{\texttt{Fast\-Hash\-Map}}
\newcommand{\cub}{\texttt{ConvertUtilsBean}}

Take the Beanutils program as an example, Randoop generates a test suite consisting of
2692 tests, in which \ourtool detected 299 dependent tests.

After a close inspection of the automatically-generated test code, we found
the primary reason for the dependencies is missing initialization 
(cf.~Sec.~\ref{sec:humantest}).
%is the unintended program state before test execution.
Specifically, 248 tests attempt to retrieve values from a cache before
anything has been added to the cache. This particular dependence could be
fixed by adding a single line of setup code to each test.
Most of the other dependencies could be fixed with similarly low effort, too.
However, this particular fix requires understanding of at least part
of the program semantics, which is a feat beyond the abilities of
current test generation tools.
%and fully automatically.

%implicitly assume a particular program state (
%the caching state) when executing in the generated order. In Beanutils,
%\pub{} --- which can be
%accessed as a singleton --- contains a global cache from \texttt{Class} to
%\fhm{}. These 248 tests retrieve the value for \cub{}
%from this cache and assert that the result is not \texttt{null}, without adding
%anything to the cache first. However, some other generated tests call
%\texttt{Prop\-er\-ty\-Utils.get\-Prop\-er\-ty\-Editor\-Class}, which adds
%a \{\cub{}, \fhm{}\} pair to the cache internally. As a result, the former tests
%fail when run in isolation (since the cache is empty and it returns null), however
%pass when run within the whole test suite. Reasons for the remaining 51 dependent
%tests are similar, which we omit here for brevity.


Given the high ratio of dependent tests in the automatically generated
test suite, we speculate more reasons.
%that the following two phenomena could be
%reasons for this.

First, developers usually have a good understanding
of the code under test. This knowledge helps them to
build well-structured and coherent test suites.
Automated tools, on the other hand, have no such knowledge. One
possible consequence of this is illustrated by the example: the
automated tool does not understand the cache protocol and thus does
not know that it must add values to the cache first. 

%unit
%tests, programmers tend to put logically-related code in the same unit test to test certain software functionality. By contrast, automated test generation tools are often not aware of the underlying program structure nor the test execution environment when creating new ones. In particular, random test
%generation tools like Randoop invokes tested methods
%with little guidance. Thus, a generated test is more likely to depend on the
%execution of others.

%are more likely to ``interleave''
%with each other, such as, invoking the same static method mutating program states.
%\todo{JW}{I do not understand this explanation at all. What has this
%interleaving to do with dependences?}

Second, it is often hard for automated tools to
generate code that sets up the execution environment correctly,
since doing so requires understanding which specific parts
of the environment have been affected.
%to understand that
%specific parts of the code depend on the environment, and thus may not
%explicitly . 
Without proper setup code, if, at the same time, 
other tests are generated that as a side effect
create the needed environment, test dependence arises.

%Second, test frameworks like JUnit offer constructs \code{@Before}
%(\code{@After}) to permit programmers to abstract common execution environment
%construction (de-construction) code for each unit test. Such mechanism prevents
%dependent tests exhibiting to some extent. However, to the best of our knowledge,
%most automated test generation tools do not leverage
%such mechanism to enforce generated unit tests to execute in an intended environment.

%\todo{JW}{While this is true, I'm not sure if this is the right plave
%to put it}
% \textbf{Kivanc has investigated this, but i can not find the email now.}
% \todo{KM:}{ This is not completely true. I just have the stack trace for the
% first dependency which seems to be due to a mistake in caching. However, I
% don't know if this is a bug, or any information about other dependencies.}




%We already showed some evidence that test dependence is not uncommon
%in human-written tests. Given the increasing importance of
%automatically generated tests, we also wanted to at least get a
%glimpse of what is happening in that area.
%As a very preliminary, and by no means exhaustive or conclusive
%investigation, we applied Randoop to all the projects for which the
%source was readily available (this excludes SWT).

%Why this strong division happens, and whether the differences between
%the programs can be used to derive guidelines for better testing is an
%interesting question left to future work.

%\todo{JW}{I no longer think the following paragraph is true}
%This is at once surprising and troublesome. It is surprising, because
%in our experience test dependence occurs either because it is too much
%hassle to write proper test setup code for every single test, or
%because developers are not aware that global state is relevant to the
%code that is being tested. The first point should not at all be
%relevant to automated techniques, as the effort of generating boiler
%plate code is negligible compared to the cost of figuring out useful
%parts of the code to test. The second aspect is fairly well amenable
%to static analysis. Thus overall, there is no reason why automated
%tools could not avoid test dependence altogether.



%\section{How does the theory relate to our examples}

\subsection{Experimental Discussion}

\subsubsection{Threats to Validity}

There are two major threats to validity in our evaluation.
First, the \todo{NUM} open-source
programs and their test suites may not be
representative enough. Thus, we can not claim the results
can be generalized to an arbitrary program.
Second, in this evaluation, we focus specifically on
dependence between unit tests.
\todo{only consider unit tests, no system tests.} 

\subsubsection{Conclusions}


We have XXX chief findings: \textbf{(1)}
\textbf{(2)}, \textbf{(3)} ...
The examples where dependence identified weaknesses in the tests themselves
are even less likely to be observed.  
Our prototype tool shows
the potential for revealing dependences, allowing
developers to observe them and make conscious decisions about how, or even whether,
to deal with the dependences. 

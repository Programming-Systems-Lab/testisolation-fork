\section{Empirical Evaluation}
\label{sec:examples}

To show the effectiveness of our proposed
dependent test detection algorithms, we conducted
an evaluation on \todo{XXX} open-source programs (Table~\ref{}).
In our evaluation, we seek to answer the following research questions:

\begin{itemize}
\item \textbf{RQ1:} How effectively do our algorithms detect
dependent tests?
\item \textbf{RQ2:} How do the proposed static and dynamic program analyses
in \todo{algorithm-name} improve the algorithm efficiency?
\end{itemize}

\subsection{Subject Programs}

\newcommand{\unknown}{N/A\xspace}
\newcommand{\infy}{$\infty$\xspace}

\begin{table*}
\centering
\setlength{\tabcolsep}{0.12\tabcolsep}
\begin{tabular}{|l|c|C|C|C|c|c|c|c|c|c|c|c|c|c|c|c|}
%\toprule
\hline
\textbf{Subject} & & \multicolumn{7}{|c|}{\textbf{\#Detected Dependent Tests}} & \multicolumn{7}{|c|}{\textbf{Analysis Cost (second)}}\\
%\midrule
\cline{3-16}
\textbf{Programs} & \textbf{\#Tests} & \multicolumn{3}{|c|}{\textbf{Randomized}} & \multicolumn{2}{|c|}{\textbf{Exhaustive }} & \multicolumn{2}{|c|}{\textbf{Dependence-Aware}} & \multicolumn{3}{|c|}{\textbf{Randomized}} & \multicolumn{2}{|c|}{\textbf{Exhaustive }} & \multicolumn{2}{|c|}{\textbf{Dependence-Aware}} \\
%\cline{3-8}\cline{10-15}
& & \smalltrialnum & \mediumtrialnum & \trialnum& \; $k$=1 & $k$=2 & \quad $k$=1 \;\; \quad & $k$=2 & \smalltrialnum & \mediumtrialnum & \trialnum & \; $k$=1 & $k$=2 &  \quad $k$=1 \quad \quad & $k$=2  \\
\hline
%\bottomrule
\multicolumn{16}{|l|}{ }\\
\multicolumn{16}{|l|}{\textbf{Human-written unit tests} }\\
\hline
JodaTime & \jodatimetests & 1 & 1 & 6 & 2 &\unknown&& &   57 & 528 & 5538 &1265& \infy & &   \\
XML Security& \xmlsecuritytests & 1 & 4 & 4 &4 &4 & 4 & 4  &65 & 594 & 5977 & 106 &  11927 & 93 & 3322  \\
Crystal & \crystaltests & 18 & 18 & 18 &17&18&  & &14& 131 & 1304 & 166 & 7323 &   & \\
Synoptic & \synoptictests & 1 &1  & 1 & 0 &1 & &&  7 & 67 & 760& 25 & 3372&  &  \\
\hline
\textbf{Total} & \totaltests & 21&24&29& 23 &\unknown&  & &  143 & 1320 & 13579 &1562& \infy &   &  \\
\hline
\multicolumn{16}{|l|}{ }\\
\multicolumn{16}{|l|}{\textbf{Automatically-generated unit tests} }\\
\hline
JodaTime & \jodatimeautotests & 586 &815& 966 & 534 & \unknown&& & 131  & 1139 & 9000 & 448 & \infy & &  \\
XML Security& \xmlsecurityautotests& 167 & 171 & 171 & 129 &&  &  & 50 & 430 & 4174 & 133 & \infy & & \\
Crystal & \crystalautotests & 159 & 162 & 164 & 55 & \unknown& & & 103 & 949& 9436  & 2477 & \infy & & \\
Synoptic & \synopticautotests & 3 & 7 & 10 &2& \unknown& &  &81& 770  & 6311 & 454 & \infy & & \\
\hline
\textbf{Total} & \totalautotests &915&1155& 1311 & 720& & & &365 &3288 & 28921 & 3512 & \infy & & \\
\hline
\end{tabular}
\caption{Experimental results. Column ``\#Tests'' shows the total number
of tests, taken from Table~\ref{tab:subjects}. Column ``\#Detected Dependent Tests''
shows the number of detected dependent tests in each subject program.
Columns ``Randomized'', ``Exhaustive'' and ``Dependence-Aware'' show the results
of applying the randomized algorithm, the exhaustive $k$-bounded algorithm and the dependence-aware
$k$-bounded algorithms, respectively. 
When evaluating the randomized algorithm, we use $numtrials$ =
$\smalltrialnum, \mediumtrialnum, \trialnum$ in the algorithm (Figure~\ref{fig:randalgorithm}).
Column ``Analysis Costs (second)''
shows the time cost (in seconds) of each algorithm under
different settings.
}
\label{tab:results}
\end{table*}


Figure~\ref{fig:example-summary}
summarizes the projects we studied and the results: The table
summarizes the number of tests in the suites produced by the
developers (\emph{MT}), the number of tests we generated automatically
with Randoop (\emph{AT}), and the corresponding numbers of dependent
tests in those test suites (\emph{MTD} and \emph{ATD}, respectively). 
The discussion of the examples in this section is distinguished by
the problems caused by test dependence (\emph{Kind}): when faults are masked because
tests make incorrect assumptions about the global environment (Section~\ref{sec:mask}); 
when tests do not
respect required initialization protocols (Section~\ref{sec:examples:initialization}); and when
undocumented test dependence leads to spurious bug reports (Section~\ref{sec:spurious}).
We also describe dependent tests in an automatically-generated test
suite (Section~\ref{sec:autogen}).
While this list---and associated set of examples---certainly is not exhaustive, it shows that there are
several classes of dependence-related problems that have practical
relevance.


\subsection{Evaluation Procedure}

\todo{including manual tests and auto-generated tests}

We used the prototype to verify the dependent tests reported by
users, developers, other researchers, and us, and to find new dependent
tests in the
example programs in Section~\ref{sec:examples} using isolated execution ($k = 1$)
and pairwise execution ($k = 2$).

All the dependent tests reported in Figure~\ref{fig:example-summary},
except for two dependent tests in JodaTime and the dependences in SWT, 
can already be found by isolated execution. Since we could not run the
test suite of SWT, we could not check these dependences with our tool.
During manual bug diagnosis in JodaTime, we identified two test dependences that require
\emph{three} tests to manifest. While these are easy to reproduce, we
did not check that our tool finds them, because the time needed to
run our naive algorithm on JodaTime with $k=3$ is measured in months.

While we believe that most test dependences can be found with small
$k$. This is in part because the set of dependent tests that can be
found with a bound $k$ is always a subset of the set of dependent
tests that can be found with any bound $k' > k$. Additionally, our
intuition and preliminary exploration seem to indicate that small $k$
find many dependences, while larger $k$ do not. However, in principle
it is conceivable
that any number of chain dependences with chains longer than any tried $k$ exist
in all the libraries we analyzed.

\subsection{Results}

\todo{show the basic results here, i.e., the num
of dependent tests found. Its consequence is discussed
below.}

\subsubsection{RQ1: Algorithm Effectiveness}

\subsubsection{RQ2: Improvement from Program Analyses}

\subsection{Consequences of Dependent Tests}

\todo{The following text is too verbose. Note be
consistent to the categories used in the study section}

\subsubsection{Masking Faults}\label{sec:mask}

\emph{Masking} is a particularly perplexing problem caused by
dependence.
The negative effect of masking is that it hides a fault in the
program, \emph{exactly} when the test suite is executed in its default
order. 
Masking occurs when a test case $t$ (a) \emph{should}
reveal a fault, (b) only does so when executed in a specific environment
$\env_R$, but (c) tests executed before $t$ in a test suite always
generate environments different from
$\env_R$.
%To express this more
More precisely and without loss of generality, assume any
environment with only a single variable. Then let $T =
\suite{t_1,\dots,t_n}$ be the test suite, and let $t_i, 1 < i \leq n$
be the test that should reveal the fault in environment $\env_R$. A
dependency $t_k \prec t_i, k < i$ masks the fault if
$\exec{\suite{t_1,\dots,t_{i-1}}}{\env_0} \neq \env_R$.

The following two examples illustrate masking in
practice.

\paragraph{CLI: A Long-Standing Bug}

\begin{figure}
% \lstset{language=Java,numbers=left}
%\lstset{language=Java}
\lstset{belowskip=0ex,escapechar={@},numbers=left,numberstyle=\small\ttfamily}
\begin{lstlisting}
public final class OptionBuilder {
  @\itshape\color{red}
  private static String argName;@
  
  private static void reset() {
    ...
    @\itshape\color{red}argName = "arg";@
    ...
  }
   
  public static Option create(String opt){
    Option option = 
      new Option(opt, description);
    ...
    option.setArgName(argName);
    @\itshape\color{red}OptionBuilder.reset();@
    return option;
  }
}
\end{lstlisting}
\caption{Fault-related code from \code{Option\-Build\-er.java}}
\label{fig:option_builder}
\end{figure}

A straightforward example of fault masking occurs in the Apache CLI
library~\cite{cli}.
Two test cases fail when run in isolation:
\code{test13666} and \code{test\-Op\-tion\-With\-out\-Short\-For\-mat2} in test
classes \code{Bugs\-Test} and \code{Help\-For\-mat\-ter\-Test},
respectively.

A detailed study of the code under test revealed that both 
tests fail due to the same hidden dependence. The fault is located in 
\code{OptionBuilder.java} and is caused by not initializing a global
variable early enough.
Figure~\ref{fig:option_builder} shows code that
illustrates the fault. 
%
By default,
\code{argName} is initialized to \code{null} (line 2), and only set to
its intended default value \code{"arg"} by the \code{create()} method
via calling \code{reset()} (line 15). 
Consequently, if clients of CLI do not explicitly initialize the value of
\code{argName}, the first option created will have \code{null} rather
than \code{"arg"} as its argument name.
%In CLI, there are two types of options: options with and without
%argument names. If an option without argument is created first,
%this fault will not lead to a failure, because the \code{null} value
%will be ignored. Consecutive calls to \code{create()} can rely on
%\code{reset()} to establish the desired default value.

Both dependent tests
% \code{test13666} and \code{test27635} (or \code
% {test\-Op\-tion\-With\-out\-Short\-For\-mat2}) 
can reveal this fault, since they create an option with 
the default argument as the first thing in their execution. However,
in the default order of test execution, 
%the test classes \code{BugTest} and \code{Help\-For\-mat\-ter\-Test} both
%contain other 
tests that create options with explicit arguments execute \emph{before} 
these dependent tests.
% \code{test13666}
% and \code{test27635} respectively. 
%Thus, when the tests in these classes are 
%executed in order, the tests executed before \code{test13666}
%and \code{test27635} call \code{create()} 
Thus, the tests that are executed before call \code{create()} at least once, which
sets the default \code{argName} value, thus masking the fault.


This fault is reported in the bug
database several times,\footnote{\url{https://issues.apache.org/jira/browse/CLI-26} \url{https://
issues.apache.org/jira/browse/CLI-186} \url{https://issues.apache.org/jira/browse/
CLI-187}} starting on March 13, 2004 (CLI-26). The report is marked as resolved
\emph{three years} later on March 15, 2007, but is then reopened as CLI-186 on
July 31, 2009. On this report, one of the developers commented:
\begin{quote}
I reproduced the issue, it requires a dedicated test case since it is tied to the initialization 
of a static field in OptionBuilder.
\end{quote}
Despite the realization that a dedicated test is required, no such
test was ever created.
About one month later, the bug is duplicated as CLI-187, and the
actual fix happens one 
year later on June 19, 2010, about six years after the bug was first reported (and four years
total on the open-issue list).

\newcommand{\jodatime}{JodaTime\xspace}
\paragraph{\jodatime: Complex interactions that mask faults}
\label{sec:jodatime}
\newcommand{\periodType}{\texttt{Period\-Type}}
\newcommand{\durationFieldType}{\texttt{Duration\-Field\-Type}}
\newcommand{\forFields}{\texttt{for\-Fields}}

\jodatime{}\footnote{\url{http://joda-time.sourceforge.net/}} is an open source
date and time library intended to improve upon the weaknesses of the
date and time facilities provided by the standard JDK.
%written to enhance the capabilities provided by the standard
%JDK such as allowing multiple calendar systems. 
It is a mature project that has been under active development
for more than eight years.

\jodatime\ uses intricate caching mechanisms that are high\-ly complex
and coupled.  All dependences we found are complex,
in two cases even requiring a
specific ordering of \emph{three} tests to manifest.

In a simple dependence, \jodatime{} caches \periodType{} objects, which 
% Caching is done by using a
% global \texttt{HashMap} that holds the \periodType{}s that are created by
% \forFields{} method. 
contain an array of
\durationFieldType{}s (e.g., week, month). 
The order of \durationFieldType{}s in the array is an
important of the data representation, and 
two \periodType{}s with the same \durationFieldType{}s in a different
order are not equal internally in \jodatime, even though they are equal
to \jodatime clients.
%However, this implementation detail should be transparent to the user: As long
%as two \periodType{} objects have the same \durationFieldType{}s, they should represent
%the same period. 
To make this internal detail transparent to users of \jodatime, 
new \periodType{}s are normalized before they are cached. However, a fault in the code 
% checking for existing objects
makes it possible to insert non-normalized \periodType{}s into the
cache, leading to cache misses when searching for correctly normalized
\periodType{}s.
%This is even acknowledged by the developers: \forFields{}
%method first creates a \periodType{} using method argument \texttt{types} (the
%ordering provided by the user) and checks whether the cache already contains
%this ordering. However, if this fails, then another \periodType{} with
%normalized ordering (\texttt{checkedType}) is created at the end of the method 
%and the cache is rechecked as shown in Figure~\ref{fig:jodatime_forFields}.


%\begin{figure}
% \lstset{language=Java,numbers=left}
%%\lstset{language=Java}
%\begin{lstlisting}
%PeriodType forFields
%  (DurationFieldType[] types) {
%  ...
%  PeriodType input =
%    new PeriodType(null, types, null);
%  ...
%  // recheck cache in case
%  // initial array order was wrong
%  PeriodType check = ...
%  PeriodType checkedType = cache.get(check);
%  if (checkedType != null) {
%    cache.put(input, checkedType);
%    return checkedType;
%  }
%  cache.put(input, type);
%  return type;
%}
%\end{lstlisting}
%\caption{Fault related code from \code{Pe\-ri\-od\-Type.for\-Fields}
%\newline (rev. 3937d82f6670e5a30b2809b13cb6d05a7e606037)}
%\label{fig:jodatime_forFields}
%\end{figure}
%
%
%In spite to the developers' extra check, a small mistake in
%Figure~\ref{fig:jodatime_forFields} creates a bug in the program. Note that the
%developers are putting the \texttt{input} variable (the ordering provided by
%the user) as the `key' of the cache.
%As a result, if \forFields{} method is called with a wrongly ordered
%\texttt{types} parameter first, then a \periodType{} with wrong ordering will be
%added to the cache. Later, calling the same method with correct ordering of
%the same content causes a cache miss.

A test that checks for correct normalization when
caching objects
%The developers even have a very simple test case that checks for this.
%\texttt{Test\-Period\-Type.test\-For\-Fields4} method creates two
%\durationFieldType{}s: first with wrong ordering and the second with correct
%ordering, both representing the same period. The test creates the corresponding
%\periodType{}s by calling \forFields{} method with these \durationFieldType{}s.
%Finally, the objects retrieved from \forFields{} method are asserted for
%equality. 
fails in isolation but passes when the entire test suite
executes in the default order; this happens
because a prior test creates the expected
\periodType{}, and thus it is already in the cache for the
later test.
This behavior has been reported as a bug and has been fixed by the
developers.
%never fails during the development due to a simple dependency with the previous
%test in the same class. \texttt{test\-For\-Fields3} method creates the same
%\periodType{} content --- which will be used in the next test --- with the
%correct ordering for some other purpose. As a result, this \periodType{} is added with
%the correct ordering to the cache, which guarantees correct retrieval for both
%the correct and wrong ordering in the next test.


%The test case that would catch the bug was written the same revision it is
%introduced. However, since the developers never ran that test in isolation the
%bug lived for more than six years. During this time period, there have been 773
%commits for the project. Even the buggy file (\periodType{}) is
%changed for nine times and the buggy method (\forFields{}) is changed
%once. Finally, the bug gets reported by a
%user\footnote{\url{http://sourceforge.net/mailarchive/message.php?msg_id=28501345}}
%in December 06, 2011 and is fixed the same
%day\footnote{\url{https://github.com/JodaOrg/joda-time/compare/b609d7d66d...d6791cb5f9}}.
%During the same commit, the developers also removed the dependency for
%the related test by creating a unique
%\periodType{} for that test. The actual fix contains changing two
%variables:
%two instances of variable \texttt{input} (possibly wrong ordering) to variable
%\texttt{check} (to correct ordering) at the end in
%Figure~\ref{fig:jodatime_forFields}.

After inspecting the code, we reported the more complex dependence of
three tests to the developers of \jodatime. They confirmed the
phenomenon, but contended that it is due to interactions that are not
intended in the design of the library~\cite{jodatime}. In particular, one of the
methods, \code{DateTimeZone.setProvider()}, is only supposed to be
called a single time to initialize the library.  In practice, multiple
tests initialize the library, which leaves incorrect values in the cache
and causes other tests to fail under some execution orders.
%However, the tests call it more
%than once, which causes at least two cases to %break the caching
%%mechanism by leaving 
%leave incorrect values in the cache, causing the tests to fail.


\subsubsection{Poor Test Construction}\label{sec:examples:initialization}

Based on our interaction with the \jodatime developers, this last
dependence does not
mask a fault in the program.  Instead, it represents a less severe consequence of test
dependence that suggest that a test, or a test suite, 
has been constructed poorly in some dimension.  While test dependences that mask faults
correspond to a defect
in the program source, these dependences correspond to defects in the test code.
%
%In contrast to the previous section
%where the dependences led to defects in the program source, this section concerns defects
%in the test source.

%In some sense, dependences that are due to missing initialization are
%the dual to dependences that mask faults.  Both reveal problems in source code.
%However, masked faults reside in the program source, while incorrect
%initialization is a fault that resides in the test suite.

The test dependences presented in this section arise due to incorrect initialization
of program state by one or more tests. In the first case,
%
%The following two examples show two common patterns where incorrect
%initialization leads to test dependence.
%The first example is probably the most common. 
tested program code relies on a
global variable that is a part of the environment, but the test does
not properly initialize it.  In the second case, a test should but
does not call
an initialization function before later invocations to a complex library.
This flaw in the test code is masked because the default test suite execution
order includes other tests that initialize the library.  The defect is
inconsequential until and unless the flawed test is reordered, either manually or by
a downstream tool, to execute before any other initializing test.

%The second example employs a common pattern for complex
%libraries that requires a call to an initialization function before
%any other part of the library can be used.
%In both cases, other tests perform the required setup, and because
%they occur before the dependent tests in the normal execution order,
%no tests fail under normal circumstances.

\paragraph{Crystal: Global Variables Considered Harmful}
Crystal~\cite{crystal} is a tool that
pro-actively examines developers' code and precisely identifies and reports on textual, compilation, and behavioral conflicts.

The latest release of Crystal contains 81 human-written unit tests. 
Of those, 75 are fully automated, and 18 exhibit
dependences.
All these dependencies are caused by incomplete initialization of the
environment when testing methods of three distinct classes
(\code{Data\-Source, Lo\-cal\-State\-Re\-sult, Con\-flict\-Daemon}).
In all cases, one test initializes the environment correctly, and all
other tests rely on that test executing first. 

A short conversation with the developers confirmed that this was not
intentional and most likely happened because the developers were not
aware of the potential dependency caused by the use of global
variables. Since we pointed out this problem, the developers treat the
dependencies as undesirable and opened a bug report to have this issue
resolved.\footnote{\url{https://code.google.com/p/crystalvc/issues/detail?id=57}}

%Dependent tests in Crystal fall into the following three groups, which
%share the same root cause of global data dependence across multiple tests:
%
%\begin{itemize}
%
%\item 9 dependent tests come from the \CodeIn{DataSourceTest} class.
%In that class, a test method \CodeIn{testSetField} initializes a global variable \CodeIn{data}
%and other test methods read the value of the \CodeIn{data} variable.
%As a result, when a test using \CodeIn{data} is executed in isolation or executed
%before the \CodeIn{testSetField} method, a \CodeIn{NullPointerException} is
%thrown.
%
%\item 7 dependent tests come from class \CodeIn{LocalStateResultTest}.
%In that class, a test method \CodeIn{testLocalStateResult} initializes a global variable
%\CodeIn{localState} and other test methods use that variable. Therefore,
%7 tests using the \CodeIn{localState} variable exhibits
%a \CodeIn{NullPointerException} when they are executed before \CodeIn{testLocalStateResult}.
%\todo{KM}{I see no difference between the first item and this one. They all
%seem to happen due to one test initializing a global variable and the others
%reading the same global variable}
%
%\item 1 dependent test comes from class \CodeIn{ConflictDaemonTest}. This
%test uses a shared global variable which requires other tests in the
%same test class to initialize, \todo{KM}{Again, the same thing. I believe we
%should say that all dependencies are due to initialization of a global variable
%and then explain all of them as three sentences (instead of bullet pointing
%them as they were really different)}
%\end{itemize}



\paragraph{XML Security: Global Initialization}

%Are these test dependences realistic, or part of the modifications SIR
%made? by SZ: they are realistic, we use the original version without
%any modification from SIR people


%on a global variable. Take \code{test\_Y1} as an example. This test passes when being executed
%with other tests, but fails by throwing an \code{InvalidCanonicalizerException}
%when executed individually.
%The root cause of such behavior difference is that, in XML-security, the \code{Init.init()} method initializes
%the static field \code{Canonicalizer.canonicalizerHash}, and test \code{test\_Y1} needs to use
%that static field to create a \code{Canonicalizer} instance. 
%When executing this test in the programmer-fixed order, method \code{Init.init()} has been called by
%other tests executed before \code{test\_Y1}, so that test \code{test\_Y1} passes.
%However, without calling \code{Init.init()} first,
%\code{test\_Y1} fails to create the \code{Canonicalizer} instance.
%
%Based on the dumped error message in the \code{InvalidCano-\\nicalizerException}:
%
%\begin{quote}
%``You must initialize the xml-security library correctly before you use it.
%Call the static method ``org.apache.xml.security.Init.init()'' to do that before you use any functionality
%from that library''
%\end{quote}

%We speculate that programmers should realize this potential dependence, but they
%overlook to enforce \code{test\_Y1} to be executed in a desirable order. Instead,
%programmers may have put an implicit assumption that tests in a suite can be executed in isolation
%and miss to add the necessary preconditions for \code{test\_Y1}. 

% vim:wrap:wm=8:bs=2:expandtab:ts=4:tw=70:


\subsubsection{Spurious Bug Reports and Bug Fixes}\label{sec:spurious}
Sometimes developers introduce dependent tests intentionally because it is
easier, more efficient or more convenient to write unit tests for some modules
in that way~\cite{kapfhammeretal:FSE:2003, whittakeretal:2012}.
%DB-testing}.
Even though the developers are aware of these instances
when they create them, this knowledge can get lost, 
and other people who are not aware of these dependences can get confused 
when they run a subset of the test suite that manifests the
dependences.

As a result, they
might report bugs backed by the failing tests, although this is exactly the expected
behavior. If the dependence is not documented clearly and
correctly, it can take a considerable amount of time to work out that
these reported failures are spurious. Or worse, the developers may try
to fix a bug that is not there.

\paragraph{Eclipse SWT: Causing Spurious Bug Reports}
\newcommand{\ite}{\texttt{Invalid\-Thread\-Access\-Exception}}

The Eclipse Standard Widget Toolkit
(SWT)\footnote{\url{http://eclipse.org/swt/}} is a cross-platform GUI
library developed within the Eclipse framework.
%\todo{KM}{Year here\ldots}.
%It has been developed to combine best parts of Sun's Abstract Window Toolkit (AWT)
%and Swing: native look and feel and native performance.
%
Due to the difficulty of obtaining source, compiling and running
test suites with the SWT project, we only examined some test cases
manually, after a bug report indicated test dependence. The
numbers reported in Figure~\ref{fig:example-summary} are the number
of tests we manually examined, and the number of dependencies we found
among those respectively.

As is common practice in GUI toolkits, SWT permits only one
\texttt{Display} object per thread. Attempting to create multiple
\texttt{Display}s in a single thread causes an \ite{}. 
%In other words each thread is responsible of disposing its \texttt{Display} after it is done with it. 
To permit the reuse of \texttt{Display}s, SWT provides two 
methods: \texttt{Display.getDefault} and \texttt{new Shell}. These
methods return the existing \texttt{Display} or create a new one if none exists.


In the test suite of SWT, all tests except those in the class \texttt{Test\_org\_eclipse\_swt\_widgets\_Display}
(\texttt{TestDisplay} for short) retrieve the current \texttt{Display} by using
one of the latter methods. On the other hand, all tests in
\texttt{TestDisplay} create their \texttt{Display} at the beginning of the test
and dispose of it at the end. 

%\texttt{TestDisplay.setup} contains the following
%comment:
%\begin{quote}
%There can only be one Display object per thread. If a second Display is created
%on the same thread, an InvalidThreadAccessException is thrown. 
% \\ Each test will create its own Display and must dispose of it before
% completing.
%\end{quote}


In September 2003, a user reported a
bug,\footnote{\url{https://bugs.eclipse.org/bugs/show_bug.cgi?id=43500}}
stating that tests throw an \ite{}
if she runs any other test before \texttt{DisplayTest}. 
The cause of this is simple: any other test creates, but does not
dispose of a \code{Display} object. Then the tests in
\code{TestDisplay} attempt to create a new object, which fails, as one
is already associated with the current thread.
Since this is the expected and desired behavior, the bug report is
spurious (except maybe it points to a problem in the test suite,
rather than the code).


%Let us examine the bug
%report: running any other test would create a \texttt{Display} (through one of
%the latter two methods) and would not dispose it. Thus, when these test
%complete, the main thread owns a \texttt{Display}. At this moment, when the same
%thread tries to run \texttt{DisplayTest} and thus tries to create another
%\texttt{Display}, an \ite{} is thrown. However, note that this is really the
%intended case when a thread attempts to create multiple \texttt{Display}s. In
%other words, this dependency leads to a spurious bug: there is a change in the
%test outcome when the order of tests are changed, however this does not
%correspond to a bug in the program. Nevertheless, understanding this dependency
%--- even though the comment on \texttt{DisplayTest.setup} existed --- takes
%about a month for the developers. One of the developers closes the bug with the
%following comment:
%\begin{quote}
%Turns out that the tests really are order-dependent - the Display tests must 
%be run first. It's not an SWT bug or anything, it's just the way the tests are 
%written, and I think it would be weird to code around it. \\
%\ldots I'm not going to make any code changes, but for the `fix' I have added a big 
%comment in the AllTests method saying that the Display tests must go first.
%\end{quote}
%We believe that the way this bug is handled shows that the
%dependencies between tests can lead to confusion even when there is no real bug. 


% This led to a spurious bug report. This is actually a good example,
% because it shows how hard it is to tell the difference between a bug
% and dependent test.
% So what do we fix? The "bug" or the tests?


\subsubsection{Dependence in Auto Generated Tests}
\label{sec:autogen}
\newcommand{\pub}{\texttt{Prop\-er\-ty\-Utils\-Bean}}
\newcommand{\fhm}{\texttt{Fast\-Hash\-Map}}
\newcommand{\cub}{\texttt{ConvertUtilsBean}}

As shown in Table~\ref{tab:results}, in most
projects, a large fraction of the remaining tests are dependent, while
in some projects, there are almost no dependent tests.

Take the Beanutils program as an example, Randoop generates a test suite consisting of
2692 unit tests for a recent release of Beanutils (version 1.8.3).
The prototype tool we outline in Section~\ref{sec:impl}
detected 299 dependent tests in this test
suite.

After a close inspection of the automatically generated test code, we found
the primary reason for the dependencies is missing initialization 
(cf.~Sec.~\ref{sec:examples:initialization}).
%is the unintended program state before test execution.
Specifically, 248 tests attempt to retrieve values from a cache before
anything has been added to the cache. This particular dependence could be
fixed by adding a single line of setup code to each test.
Most of the other dependencies could be fixed with similarly low effort, too.
However, this particular fix requires understanding of at least part
of the program semantics, which is a feat beyond the abilities of
current test generation tools.
%and fully automatically.

%implicitly assume a particular program state (
%the caching state) when executing in the generated order. In Beanutils,
%\pub{} --- which can be
%accessed as a singleton --- contains a global cache from \texttt{Class} to
%\fhm{}. These 248 tests retrieve the value for \cub{}
%from this cache and assert that the result is not \texttt{null}, without adding
%anything to the cache first. However, some other generated tests call
%\texttt{Prop\-er\-ty\-Utils.get\-Prop\-er\-ty\-Editor\-Class}, which adds
%a \{\cub{}, \fhm{}\} pair to the cache internally. As a result, the former tests
%fail when run in isolation (since the cache is empty and it returns null), however
%pass when run within the whole test suite. Reasons for the remaining 51 dependent
%tests are similar, which we omit here for brevity.


Given the high ratio of dependent tests in the automatically generated
test suite, we speculate that the following two phenomena could be
reasons for this.

First, developers usually know a lot about the intended purpose of a
program when they write tests for it. This knowledge helps them to
build well-structured and coherent test suites.
Automated tools, on the other hand, have no such knowledge. One
possible consequence of this is illustrated by the example: the
automated tool does not understand the cache protocol and thus does
not know that it must add values to the cache first. 

%unit
%tests, programmers tend to put logically-related code in the same unit test to test certain software functionality. By contrast, automated test generation tools are often not aware of the underlying program structure nor the test execution environment when creating new ones. In particular, random test
%generation tools like Randoop invokes tested methods
%with little guidance. Thus, a generated test is more likely to depend on the
%execution of others.

%are more likely to ``interleave''
%with each other, such as, invoking the same static method mutating program states.
%\todo{JW}{I do not understand this explanation at all. What has this
%interleaving to do with dependences?}

Second, it is often hard for automated tools to understand that
specific parts of the code depend on the environment, and thus may not
explicitly generate code that sets up the environment correctly. If,
at the same time, other tests are generated that as a side effect
create the needed environment, test dependence ensues.

%Second, test frameworks like JUnit offer constructs \code{@Before}
%(\code{@After}) to permit programmers to abstract common execution environment
%construction (de-construction) code for each unit test. Such mechanism prevents
%dependent tests exhibiting to some extent. However, to the best of our knowledge,
%most automated test generation tools do not leverage
%such mechanism to enforce generated unit tests to execute in an intended environment.

%\todo{JW}{While this is true, I'm not sure if this is the right plave
%to put it}
% \textbf{Kivanc has investigated this, but i can not find the email now.}
% \todo{KM:}{ This is not completely true. I just have the stack trace for the
% first dependency which seems to be due to a mistake in caching. However, I
% don't know if this is a bug, or any information about other dependencies.}



On the other hand, test dependence in automatically generated test suites is 
even more troublesome than in human-written suites. The reason for this is
that all automated test generation tools we are aware of produce tests
that are hard to read for humans, are undocumented, and their intent
cannot easily be gleaned from naming conventions and other aids
developers normally use. While there is some work to alleviate this problem, it
still remains difficult to determine whether a failed test points to a bug
in the program or a dependent test~\cite{fraseretal:ISSTA:2011}.

%We already showed some evidence that test dependence is not uncommon
%in human-written tests. Given the increasing importance of
%automatically generated tests, we also wanted to at least get a
%glimpse of what is happening in that area.
%As a very preliminary, and by no means exhaustive or conclusive
%investigation, we applied Randoop to all the projects for which the
%source was readily available (this excludes SWT).

%Why this strong division happens, and whether the differences between
%the programs can be used to derive guidelines for better testing is an
%interesting question left to future work.

%\todo{JW}{I no longer think the following paragraph is true}
%This is at once surprising and troublesome. It is surprising, because
%in our experience test dependence occurs either because it is too much
%hassle to write proper test setup code for every single test, or
%because developers are not aware that global state is relevant to the
%code that is being tested. The first point should not at all be
%relevant to automated techniques, as the effort of generating boiler
%plate code is negligible compared to the cost of figuring out useful
%parts of the code to test. The second aspect is fairly well amenable
%to static analysis. Thus overall, there is no reason why automated
%tools could not avoid test dependence altogether.



%\section{How does the theory relate to our examples}

\subsection{Experimental Discussion}

\subsubsection{Threats to Validity}

There are two major threats to validity in our evaluation.
First, the \todo{NUM} open-source
programs and their test suites may not be
representative enough. Thus, we can not claim the results
can be generalized to an arbitrary program.
Second, in this evaluation, we focus specifically on
dependence between unit tests.
\todo{only consider unit tests, no system tests.} 

\subsubsection{Conclusions}


We have XXX chief findings: \textbf{(1)}
\textbf{(2)}, \textbf{(3)} ...
The examples where dependence identified weaknesses in the tests themselves
are even less likely to be observed.  
Our prototype tool shows
the potential for revealing dependences, allowing
developers to observe them and make conscious decisions about how, or even whether,
to deal with the dependences. 

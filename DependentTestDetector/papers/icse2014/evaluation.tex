\section{Empirical Evaluation}
\label{sec:evaluation}

To show the effectiveness of our proposed
dependent test detection algorithms, we conducted
an evaluation on \todo{XXX} open-source programs (Table~\ref{tab:subjects}).
In our evaluation, we seek to answer the following research questions:

\begin{itemize}
\item \textbf{RQ1:} How effectively do our algorithms detect
dependent tests?
\item \textbf{RQ2:} How do the proposed static and dynamic program analyses
in the improved algorithm increase the algorithm efficiency?
\end{itemize}

\todo{check the research questions above? any more?}
\todo{should we add human-written tests v.s. automatically-generated tests?
I prefer not, since it does not lead to much insight.}

\subsection{Subject Programs}


\begin{table}
\centering
\setlength{\tabcolsep}{0.4\tabcolsep}
\begin{tabular}{|l|l|c|c|l|}
%\toprule
\hline
\textbf{Program} & \textbf{LOC} & \textbf{\#Tests} & \textbf{\#Auto Tests} & \textbf{Revision}
\\
\hline
%\midrule
JodaTime & 27183 & \jodatimetests
% 3875 is retrieved by running mvn test on the related revision
& \jodatimeautotests&  b609d7d66d\\
XML Security & 18302 & \xmlsecuritytests & \xmlsecurityautotests& version 1.0.4 \\ 
Crystal & 4676 & \crystaltests & \crystalautotests& trunk version\\
Synoptic & 28872 & \synoptictests & \synopticautotests&  trunk version\\ 
%\bottomrule
\hline
%\textbf{Total}& &  & &  \\ 
%\hline
\end{tabular}
\caption{Subject programs used in our evaluation.
Column ``\#Tests'' shows the number of human-written
unit tests associated with each program. Column
``\#Auto Tests'' shows the number of automatically-generated
unit tests for each program, by Randoop~\cite{PachecoLET2007}.
}
\label{tab:subjects}
\end{table}

\newcommand{\unknown}{N/A\xspace}
\newcommand{\ignore}{---\xspace}
\newcommand{\infy}{$\infty$\xspace}

\begin{table*}
\centering
\setlength{\tabcolsep}{0.12\tabcolsep}
\begin{tabular}{|l|c|c|C|C|C|c|c|c|c|c|c|c|c|c|c|c|c|c|}
%\toprule
\hline
\textbf{Subject} & \textbf{\#} & \multicolumn{8}{c|}{\textbf{\# Detected Dependent Tests}} & \multicolumn{8}{c|}{\textbf{Analysis Cost (seconds)}}\\
%\midrule
\cline{3-18}
\textbf{Programs} & \textbf{Tests} & \textbf{Rev} & \multicolumn{3}{c|}{\textbf{Randomized}} & \multicolumn{2}{c|}{\textbf{Exhaustive }} & \multicolumn{2}{c|}{\textbf{Dep-Aware}} & \textbf{Rev}& \multicolumn{3}{|c|}{\textbf{Randomized}} & \multicolumn{2}{c|}{\textbf{Exhaustive }} & \multicolumn{2}{c|}{\textbf{Dep-Aware}} \\
%\cline{3-8}\cline{10-15}
& & & \smalltrialnum & \mediumtrialnum & \trialnum& \; $k$=1 & $k$=2 & \quad $k$=1 \;\; \quad & $k$=2 && \smalltrialnum & \mediumtrialnum & \trialnum & \; $k$=1 & $k$=2 &  \quad $k$=1 \quad \quad & $k$=2  \\
\hline
%\bottomrule
\multicolumn{18}{|l|}{ }\\
\multicolumn{18}{|l|}{\textbf{Human-written unit tests} }\\
\hline
%JFreechart & \jfreecharttests & 6 & 8 & 8 & 0 & $\ge$0 * & 0 & $\ge$0 * &  66  & 625 & 6097 & 694 & 2$\times$$10^6$ *  &310  &  1$\times$$10^6$ *\\
%the data of jfreechart is above, MUST update the total column
\jt & \jodatimetests & 2 & 1 & 1 & 6 & 2 & $\ge$2 * & 2& $\ge$2 * & 18&   57 & 528 & 5538 &1265& 4$\times$$10^6$ * & 291 & 5$\times$$10^5$ *  \\
XML Security& \xmlsecuritytests & 0 & 1 & 4 & 4 &4 &4 & 4 & 4  & 18&65 & 594 & 5977 & 106 &  11927 & 93 & 3322  \\
Crystal & \crystaltests & 18 & 18 & 18 & 18 &17&18& 17 & 18 & 3 &14& 131 & 1304 & 166 & 7323 & 95  & 4155 \\
Synoptic & \synoptictests & 1 & 1 &1  & 1 & 0 &1 & 0 & 1 & 2 &  7 & 67 & 760& 25 & 3372& 24 & 1797 \\
\hline
\textbf{Total} & \totaltests & 21 & 21&24&\textbf{29}& 23 & $\ge$24 & 23 & $\ge$25 &41&  143 & 1320 & 13579 &1562&  4$\times$$10^6$ *& 503  & 5$\times$$10^5$ *\\
\hline
\multicolumn{18}{|l|}{ }\\
\multicolumn{18}{|l|}{\textbf{Automatically-generated unit tests} }\\
\hline
%JFreechart & \jfreechartautotests& \ignore & \ignore & \ignore & \ignore & \ignore & \ignore & \ignore & \ignore & \ignore & \ignore & \ignore & \ignore & \ignore &  \ignore \\
\jt & \jodatimeautotests &\ignore & \ignore & \ignore & \ignore & \ignore & \ignore & \ignore & \ignore & \ignore & \ignore & \ignore & \ignore & \ignore & \ignore & \ignore &  \ignore \\
XML Security& \xmlsecurityautotests&138& 167 & 171 & 171 & 129 & $\ge$129 * & 128  & $\ge$128 *   & 7& 50 & 430 & 4174 & 133 & 1$\times$$10^5$ * & 128 & 5$\times$$10^4$ * \\
Crystal & \crystalautotests & 75 & 159 & 162 & 164 & 55 & $\ge$55 * & 55 & $\ge$55 *  & 22 & 103 & 949& 9436  & 2477 & 8$\times$$10^6$ *& 2297 & 1$\times$$10^6$ * \\
Synoptic & \synopticautotests &3 & 3 & 7 & 10 &2& $\ge$2 * & 2 & $\ge$2 *   & 13 &81& 770  & 6311 & 454 & 1$\times$$10^6$ *& 454 & 2$\times$$10^4$ * \\
\hline
\textbf{Total} & \totalautotests &216 &329 &340 & \textbf{345} & 186 & $\ge$186  & 185 & $\ge$185  &42&234&2149& 19921& 3064 & 1$\times$$10^7$ *& 2879& 1$\times$$10^6$ * \\
\hline
\end{tabular}
\caption{Experimental results.  Column ``\# Tests'' shows the total number
of tests, taken from Table~\ref{tab:subjects}. Column ``\# Detected Dependent Tests''
shows the number of detected dependent tests in each test suite.
% Columns ``Rev'', ``Randomized'', ``Exhaustive'' and ``Dep-Aware'' show the results
% of applying the reversal algorithm, randomized algorithm, exhaustive $k$-bounded algorithm, and the \dependenceaware{}, respectively.
%$k$-bounded algorithm, respectively. 
When evaluating the randomized algorithm, we used $\mathit{numtrials}$ =
$\smalltrialnum$, $\mediumtrialnum$, and $\trialnum$ (Figure~\ref{fig:randalgorithm}).
%``\unknown'' means the technique does not scale to the test
%suite (i.e., requiring more than 1 day to execute all test permutations),
%and thus the exact number of dependent tests is unknown.
``\ignore'' means the test suite is not evaluated due to its non-determinism.
%Column ``Analysis Cost''
%shows the time cost of each algorithm.
An asterisk (*) means the algorithm did not finish
within 1 day:
the number of dependent tests is those discovered before timing out, and 
the time estimation methodology is described in Section~\ref{sec:performance}.
\tinyrelax
}
\label{tab:results}
\end{table*}

%  LocalWords:  Joda numtrials


Table~\ref{tab:subjects} summarizes the programs used in our evaluation.

JodaTime~\cite{jodatime} is an open source
date and time library intended to improve upon the weaknesses of the
date and time facilities provided by the standard JDK.
It is a mature project that has been under active development
for more than eight years.

XML Security~\cite{xmlsecurity}
is a component library implementing XML signature and encryption
standards. Each released
version of XML Security has a human-written JUnit test suite that
achieves fairly high statement coverage.


Crystal~\cite{crystal} is a tool that
pro-actively examines developers' code and precisely identifies and reports on textual, compilation, and behavioral conflicts.

\todo{Add Derby, and Synoptic}

\todo{say why choose these subject programs}

\subsection{Evaluation Procedure}

We use \ourtool to detect dependent
tests among the human-written test suite 
associated with each subject program in Table~\ref{tab:subjects}.

We also evaluate \ourtool on automatically-generated
test suites. For each subject program, we use
Randoop~\cite{PachecoLET2007}, a state-of-the-art automated
test generation tool, to generate 5,000 tests.
Randoop drops tests that it considers to be redundant,
and outputs a subset of the generated tests as
shown in Table~\ref{tab:subjects}.
Then, we applied \ourtool to detect dependent tests
in these automatically-generated test suites.

For both human-written and automatically-generated
test suites, we use isolated execution ($k = 1$)
and pairwise execution ($k = 2$), since our empirical
study (Section~\ref{sec:study}) indicates that  a small $k$
can find most test dependences. 
\todo{The experimental methodology might be different
for the improved algorithm, which can incoporate user
annoations, and is not restricted to k = 1, or 2}

Each output dependent test is examined manually to make
sure its different results are not caused by non-deterministic
factors, such as multi-threading.


\subsection{Results}

\todo{show the basic results here, i.e., the num
of dependent tests found. Its consequence is discussed
below.}

\subsubsection{RQ1: Algorithm Effectiveness}

\todo{How many} dependent tests reported in Table~\ref{fig:example-summary}
are found by isolated execution, and \todo{how many} dependent
tests reported in the same are found by pairwise execution.

\todo{does two algorithms find the same number of tests?}
 
\todo{say this concurs with the study findings}

Characteristics of the identified dependent tests are discussed
in Section~\ref{sec:humantest} and~\ref{sec:autotest} in details.

%During manual bug diagnosis in JodaTime, we identified two test dependences that require
%\emph{three} tests to manifest. While these are easy to reproduce, we
%did not check that our tool finds them, because the time needed to
%run our naive algorithm on JodaTime with $k=3$ is measured in months.

\subsubsection{RQ2: Improvement from Program Analyses}

\todo{How much improvement can we gain?}

\subsection{Dependence in Human-written Tests}
\label{sec:humantest}

\todo{The following text needs to be re-organized.
Be consistent to the categories used in the study section}

Most of the test dependences we found in human-written
test suites arise due to incorrect initialization
of program state by one or more tests. Such test
dependence often suggests defects in the
test code. 

There are two common patterns where incorrect
initialization leads to test dependence.
The first pattern is probably the most common. 
The test code relies on a
global variable, but the test itself does
not properly initialize it.  In the second pattern,
a test should but does not call
an initialization function before later invocations to a complex library.
This flaw in the test code is masked because the default test suite execution
order includes other tests that initialize the library.  The defect is
inconsequential until and unless the flawed test is reordered, either manually or by
a downstream tool, to execute before any other initializing test.
We next example dependent tests for these two patterns.



%
\ourtool identified 18 dependent tests in Crystal.
All these dependent tests
are caused by incomplete initialization of the
environment when testing methods of three distinct classes
(\code{Data\-Source, Lo\-cal\-State\-Re\-sult, Con\-flict\-Daemon}).
In all cases, one test initializes the environment correctly, and all
other tests rely on that test executing first. 
Crystal developers confirmed that such test dependence was not
intentional and most likely happened because they were not
aware of the potential dependency caused by the use of global
variables. Since we pointed out this problem, the developers treat the
dependencies as undesirable and opened a bug report to have this issue
resolved.\footnote{\url{https://code.google.com/p/crystalvc/issues/detail?id=57}}


The XML Security version used in our evaluation is included
in the Soft\-ware-artifact
Infrastructure Repository (SIR)~\cite{sir}. \ourtool identified
3 dependent tests (\code{test\_Y1}, \code{test\_Y2}, and \code{test\_Y3}
in class \code{ExclusiveC14NInterop}), which share the same root
cause of the dependence: before any
method in the library can be used, the global initialization function 
\code{Init.init()} has to be called but is not called. The
\code{Init.init()} method initializes the static field that
each dependent test relies on.

For dependent tests in a human-written test suite, we speculate that
developers either simply forgot to
initialize the tests properly, or expected that these tests would always execute
in the order defined
in the test suite.



\subsection{Dependence in Auto Generated Tests}
\label{sec:autotest}

Dependent tests in an automatically-generated test suite
are more common.  As shown in Table~\ref{tab:results}, in most
projects, a large fraction of the remaining tests are dependent.

\newcommand{\pub}{\texttt{Prop\-er\-ty\-Utils\-Bean}}
\newcommand{\fhm}{\texttt{Fast\-Hash\-Map}}
\newcommand{\cub}{\texttt{ConvertUtilsBean}}

Take the Beanutils program as an example, Randoop generates a test suite consisting of
2692 tests, in which \ourtool detected 299 dependent tests.

After a close inspection of the automatically-generated test code, we found
the primary reason for the dependencies is missing initialization 
(cf.~Sec.~\ref{sec:humantest}).
%is the unintended program state before test execution.
Specifically, 248 tests attempt to retrieve values from a cache before
anything has been added to the cache. This particular dependence could be
fixed by adding a single line of setup code to each test.
Most of the other dependencies could be fixed with similarly low effort, too.
However, this particular fix requires understanding of at least part
of the program semantics, which is a feat beyond the abilities of
current test generation tools.
%and fully automatically.

%implicitly assume a particular program state (
%the caching state) when executing in the generated order. In Beanutils,
%\pub{} --- which can be
%accessed as a singleton --- contains a global cache from \texttt{Class} to
%\fhm{}. These 248 tests retrieve the value for \cub{}
%from this cache and assert that the result is not \texttt{null}, without adding
%anything to the cache first. However, some other generated tests call
%\texttt{Prop\-er\-ty\-Utils.get\-Prop\-er\-ty\-Editor\-Class}, which adds
%a \{\cub{}, \fhm{}\} pair to the cache internally. As a result, the former tests
%fail when run in isolation (since the cache is empty and it returns null), however
%pass when run within the whole test suite. Reasons for the remaining 51 dependent
%tests are similar, which we omit here for brevity.


Given the high ratio of dependent tests in the automatically generated
test suite, we speculate more reasons.
%that the following two phenomena could be
%reasons for this.

First, developers usually have a good understanding
of the code under test. This knowledge helps them to
build well-structured and coherent test suites.
Automated tools, on the other hand, have no such knowledge. One
possible consequence of this is illustrated by the example: the
automated tool does not understand the cache protocol and thus does
not know that it must add values to the cache first. 

%unit
%tests, programmers tend to put logically-related code in the same unit test to test certain software functionality. By contrast, automated test generation tools are often not aware of the underlying program structure nor the test execution environment when creating new ones. In particular, random test
%generation tools like Randoop invokes tested methods
%with little guidance. Thus, a generated test is more likely to depend on the
%execution of others.

%are more likely to ``interleave''
%with each other, such as, invoking the same static method mutating program states.
%\todo{JW}{I do not understand this explanation at all. What has this
%interleaving to do with dependences?}

Second, it is often hard for automated tools to
generate code that sets up the execution environment correctly,
since doing so requires understanding which specific parts
of the environment have been affected.
%to understand that
%specific parts of the code depend on the environment, and thus may not
%explicitly . 
Without proper setup code, if, at the same time, 
other tests are generated that as a side effect
create the needed environment, test dependence arises.

%Second, test frameworks like JUnit offer constructs \code{@Before}
%(\code{@After}) to permit programmers to abstract common execution environment
%construction (de-construction) code for each unit test. Such mechanism prevents
%dependent tests exhibiting to some extent. However, to the best of our knowledge,
%most automated test generation tools do not leverage
%such mechanism to enforce generated unit tests to execute in an intended environment.

%\todo{JW}{While this is true, I'm not sure if this is the right plave
%to put it}
% \textbf{Kivanc has investigated this, but i can not find the email now.}
% \todo{KM:}{ This is not completely true. I just have the stack trace for the
% first dependency which seems to be due to a mistake in caching. However, I
% don't know if this is a bug, or any information about other dependencies.}




%We already showed some evidence that test dependence is not uncommon
%in human-written tests. Given the increasing importance of
%automatically generated tests, we also wanted to at least get a
%glimpse of what is happening in that area.
%As a very preliminary, and by no means exhaustive or conclusive
%investigation, we applied Randoop to all the projects for which the
%source was readily available (this excludes SWT).

%Why this strong division happens, and whether the differences between
%the programs can be used to derive guidelines for better testing is an
%interesting question left to future work.

%\todo{JW}{I no longer think the following paragraph is true}
%This is at once surprising and troublesome. It is surprising, because
%in our experience test dependence occurs either because it is too much
%hassle to write proper test setup code for every single test, or
%because developers are not aware that global state is relevant to the
%code that is being tested. The first point should not at all be
%relevant to automated techniques, as the effort of generating boiler
%plate code is negligible compared to the cost of figuring out useful
%parts of the code to test. The second aspect is fairly well amenable
%to static analysis. Thus overall, there is no reason why automated
%tools could not avoid test dependence altogether.




\subsection{Discussion}

\subsubsection{Threats to Validity}

There are two major threats to validity in our evaluation.
First, the \todo{NUM} open-source
programs and their test suites may not be
representative enough. Thus, we can not claim the results
can be generalized to an arbitrary program.
Second, in this evaluation, we focus specifically on
dependence between unit tests.
\todo{only consider unit tests, no system tests.} 
Third, due to the computational complexity of the general dependent test
detection problem, it is difficult to know precisely how many dependent
tests exist in a test suite. Thus, we do not
yet have empirical data that shows how many percentages of dependence tests
\ourtool can catch.  Giving a reasonable estimation is one of our future work.

\subsubsection{Experimental Conclusions}


We have XXX chief findings: \textbf{(1)}
\textbf{(2)}, \textbf{(3)} ...
The examples where dependence identified weaknesses in the tests themselves
are even less likely to be observed.  
Our prototype tool shows
the potential for revealing dependences, allowing
developers to observe them and make conscious decisions about how, or even whether,
to deal with the dependences. 

\section{Empirical Evaluation}
\label{sec:evaluation}


\newcommand{\jt}{Joda-Time\xspace}

\newcommand{\jfreecharttests}{2234\xspace}%change the total num
\newcommand{\jodatimetests}{3875\xspace}
\newcommand{\xmlsecuritytests}{108\xspace}
\newcommand{\crystaltests}{75\xspace}
\newcommand{\synoptictests}{118\xspace}
\newcommand{\totaltests}{4176\xspace}

\newcommand{\jfreechartautotests}{2946\xspace}
\newcommand{\jodatimeautotests}{2639\xspace}
\newcommand{\xmlsecurityautotests}{665\xspace}
\newcommand{\crystalautotests}{3198\xspace}
\newcommand{\synopticautotests}{2467\xspace}
\newcommand{\totalautotests}{8969\xspace}


\begin{table}
\centering
\setlength{\tabcolsep}{0.4\tabcolsep}
\begin{tabular}{|l|l|c|c|l|}
%\toprule
\hline
\textbf{Program} & \textbf{LOC} & \textbf{\#Tests} & \textbf{\#Auto Tests} & \textbf{Revision}
\\
\hline
%\midrule
JodaTime & 27183 & \jodatimetests
% 3875 is retrieved by running mvn test on the related revision
& \jodatimeautotests&  b609d7d66d\\
XML Security & 18302 & \xmlsecuritytests & \xmlsecurityautotests& version 1.0.4 \\ 
Crystal & 4676 & \crystaltests & \crystalautotests& trunk version\\
Synoptic & 28872 & \synoptictests & \synopticautotests&  trunk version\\ 
%\bottomrule
\hline
%\textbf{Total}& &  & &  \\ 
%\hline
\end{tabular}
\caption{Subject programs used in our evaluation.
Column ``\#Tests'' shows the number of human-written
unit tests associated with each program. Column
``\#Auto Tests'' shows the number of automatically-generated
unit tests for each program, by Randoop~\cite{PachecoLET2007}.
}
\label{tab:subjects}
\end{table}

\newcommand{\unknown}{N/A\xspace}
\newcommand{\ignore}{---\xspace}
\newcommand{\infy}{$\infty$\xspace}

\begin{table*}
\centering
\setlength{\tabcolsep}{0.12\tabcolsep}
\begin{tabular}{|l|c|c|C|C|C|c|c|c|c|c|c|c|c|c|c|c|c|c|}
%\toprule
\hline
\textbf{Subject} & \textbf{\#} & \multicolumn{8}{c|}{\textbf{\# Detected Dependent Tests}} & \multicolumn{8}{c|}{\textbf{Analysis Cost (seconds)}}\\
%\midrule
\cline{3-18}
\textbf{Programs} & \textbf{Tests} & \textbf{Rev} & \multicolumn{3}{c|}{\textbf{Randomized}} & \multicolumn{2}{c|}{\textbf{Exhaustive }} & \multicolumn{2}{c|}{\textbf{Dep-Aware}} & \textbf{Rev}& \multicolumn{3}{|c|}{\textbf{Randomized}} & \multicolumn{2}{c|}{\textbf{Exhaustive }} & \multicolumn{2}{c|}{\textbf{Dep-Aware}} \\
%\cline{3-8}\cline{10-15}
& & & \smalltrialnum & \mediumtrialnum & \trialnum& \; $k$=1 & $k$=2 & \quad $k$=1 \;\; \quad & $k$=2 && \smalltrialnum & \mediumtrialnum & \trialnum & \; $k$=1 & $k$=2 &  \quad $k$=1 \quad \quad & $k$=2  \\
\hline
%\bottomrule
\multicolumn{18}{|l|}{ }\\
\multicolumn{18}{|l|}{\textbf{Human-written unit tests} }\\
\hline
%JFreechart & \jfreecharttests & 6 & 8 & 8 & 0 & $\ge$0 * & 0 & $\ge$0 * &  66  & 625 & 6097 & 694 & 2$\times$$10^6$ *  &310  &  1$\times$$10^6$ *\\
%the data of jfreechart is above, MUST update the total column
\jt & \jodatimetests & 2 & 1 & 1 & 6 & 2 & $\ge$2 * & 2& $\ge$2 * & 18&   57 & 528 & 5538 &1265& 4$\times$$10^6$ * & 291 & 5$\times$$10^5$ *  \\
XML Security& \xmlsecuritytests & 0 & 1 & 4 & 4 &4 &4 & 4 & 4  & 18&65 & 594 & 5977 & 106 &  11927 & 93 & 3322  \\
Crystal & \crystaltests & 18 & 18 & 18 & 18 &17&18& 17 & 18 & 3 &14& 131 & 1304 & 166 & 7323 & 95  & 4155 \\
Synoptic & \synoptictests & 1 & 1 &1  & 1 & 0 &1 & 0 & 1 & 2 &  7 & 67 & 760& 25 & 3372& 24 & 1797 \\
\hline
\textbf{Total} & \totaltests & 21 & 21&24&\textbf{29}& 23 & $\ge$24 & 23 & $\ge$25 &41&  143 & 1320 & 13579 &1562&  4$\times$$10^6$ *& 503  & 5$\times$$10^5$ *\\
\hline
\multicolumn{18}{|l|}{ }\\
\multicolumn{18}{|l|}{\textbf{Automatically-generated unit tests} }\\
\hline
%JFreechart & \jfreechartautotests& \ignore & \ignore & \ignore & \ignore & \ignore & \ignore & \ignore & \ignore & \ignore & \ignore & \ignore & \ignore & \ignore &  \ignore \\
\jt & \jodatimeautotests &\ignore & \ignore & \ignore & \ignore & \ignore & \ignore & \ignore & \ignore & \ignore & \ignore & \ignore & \ignore & \ignore & \ignore & \ignore &  \ignore \\
XML Security& \xmlsecurityautotests&138& 167 & 171 & 171 & 129 & $\ge$129 * & 128  & $\ge$128 *   & 7& 50 & 430 & 4174 & 133 & 1$\times$$10^5$ * & 128 & 5$\times$$10^4$ * \\
Crystal & \crystalautotests & 75 & 159 & 162 & 164 & 55 & $\ge$55 * & 55 & $\ge$55 *  & 22 & 103 & 949& 9436  & 2477 & 8$\times$$10^6$ *& 2297 & 1$\times$$10^6$ * \\
Synoptic & \synopticautotests &3 & 3 & 7 & 10 &2& $\ge$2 * & 2 & $\ge$2 *   & 13 &81& 770  & 6311 & 454 & 1$\times$$10^6$ *& 454 & 2$\times$$10^4$ * \\
\hline
\textbf{Total} & \totalautotests &216 &329 &340 & \textbf{345} & 186 & $\ge$186  & 185 & $\ge$185  &42&234&2149& 19921& 3064 & 1$\times$$10^7$ *& 2879& 1$\times$$10^6$ * \\
\hline
\end{tabular}
\caption{Experimental results.  Column ``\# Tests'' shows the total number
of tests, taken from Table~\ref{tab:subjects}. Column ``\# Detected Dependent Tests''
shows the number of detected dependent tests in each test suite.
% Columns ``Rev'', ``Randomized'', ``Exhaustive'' and ``Dep-Aware'' show the results
% of applying the reversal algorithm, randomized algorithm, exhaustive $k$-bounded algorithm, and the \dependenceaware{}, respectively.
%$k$-bounded algorithm, respectively. 
When evaluating the randomized algorithm, we used $\mathit{numtrials}$ =
$\smalltrialnum$, $\mediumtrialnum$, and $\trialnum$ (Figure~\ref{fig:randalgorithm}).
%``\unknown'' means the technique does not scale to the test
%suite (i.e., requiring more than 1 day to execute all test permutations),
%and thus the exact number of dependent tests is unknown.
``\ignore'' means the test suite is not evaluated due to its non-determinism.
%Column ``Analysis Cost''
%shows the time cost of each algorithm.
An asterisk (*) means the algorithm did not finish
within 1 day:
the number of dependent tests is those discovered before timing out, and 
the time estimation methodology is described in Section~\ref{sec:performance}.
\tinyrelax
}
\label{tab:results}
\end{table*}

%  LocalWords:  Joda numtrials


%We evaluated two aspects of \ourtool's
%effectiveness, answering the following
%research questions:
Our evaluation answers the following research questions:

\vspace{-1mm}

\todo{It's inconsistent that item 1 uses ``each detection
algorithm'' and item 2 uses ``each algorithm in \ourtool''.  Is there a
difference?  If not, use a single consistent term to refer to the concept.}

\begin{enumerate}
\vspace{-1mm}
\item How many dependent tests can each detection
algorithm detect in
real-world programs (Section~\ref{sec:detectedtests})?

\item How long does each algorithm in \ourtool take to detect dependent
tests (Section~\ref{sec:performance})?

\item Can dependent tests interfere with downstream testing techniques
such as test prioritization (Section~\ref{sec:impact})?

\end{enumerate}

\subsection{Subject Programs}


Table~\ref{tab:subjects} lists the programs and
tests used in our evaluation. We used these subject
programs because\todo{I don't see how this is relevant to our work at all.
  Why bother to mention it?  ``they have been developed for
a considerable amount of time (3--10 years) and''} each
of them includes a well-written unit test suite.

\jt~\cite{jodatime} is a mature open source
date and time library.  XML Security~\cite{xmlsecurity}
is a component library implementing XML signature and encryption
standards. 
\todo{I would add citations to the Crystal and Synoptic papers, too.}
Crystal~\cite{crystal} is a tool that
pro-actively examines developers' code and
identifies textual, compilation, and behavioral conflicts.
Synoptic~\cite{synoptic} is a tool to mine a finite state
machine model representation of a system from logs.
All of the subject programs' test suites are designed to be executed in
a single JVM, rather than requiring separate processes per test case~\cite{vmvm}.

Given the increasing importance of automated test generation
tools~\cite{PachecoLET2007, ZhangSBE2011, Csallner:2004, fraseretal:ISSTA:2011},
we also want to investigate dependent tests in automatically-generated
test suites. For each subject program, we used
Randoop~\cite{PachecoLET2007}, an automated
test generation tool, to create a suite of 5,000 tests.
Randoop discards redundant tests~\cite[\S III.E]{RobinsonEPAL2011};
Table~\ref{tab:subjects} shows how many tests it output.

We discarded the automatically-generated test suite of
\jt, since many tests in it are non-deterministic ---
they depend on the current time.


\subsection{Evaluation Procedure}

We evaluated each algorithm 
on both the human-written test suite 
and the automatically-generated test suite
of each subject program in Table~\ref{tab:subjects}.

%We run the three algorithms proposed
%in Section~\ref{sec:detecting} on both
%human-written and automatically-generated test suites
%of each subject program.

We ran the randomized algorithm \smalltrialnum, \mediumtrialnum,
and \trialnum times on each test suite, and recorded
the total number of detected dependent tests and time cost
for each setting. The choice of \trialnum times is based
on a practical guideline for using randomized algorithms
in software engineering, as summarized in~\cite{Arcuri:2011}.
%
For the exhaustive $k$-bounded algorithm
and the \dependenceaware{} $k$-bounded algorithm,
we use isolated execution ($k = 1$) and
pairwise execution ($k = 2$). The choice of $k$ is
based on the results of our empirical
study (Section~\ref{sec:study}) that a small $k$
can find most realistic dependent tests.

We provided \ourtool with a list of 39 ``dependence-free'' fields
for the 4 subject programs. This manual step required
about 30 minutes in total.

%\edit{say a few sentences here about the manual part, such as
%the approximate manual time cost in
%listing the immutable fields.}

We examined each output dependent test manually to make
sure the test dependence is not caused by non-deter\-min\-istic
factors, such as multi-threading.

Our experiments were run on a 2.67GHz Intel Core PC
with 4GB physical memory (2GB was allocated for the JVM),
running Windows 7.

\subsection{Results}

Table~\ref{tab:results} summarizes the number of detected
dependent tests and the time cost for each algorithm
in \ourtool.

\subsubsection{Detected Dependent Tests}
\label{sec:detectedtests}

%\todo{I rewrote the below paragraph.  Please review.}

\ourtool detected 29 human-written dependent tests (among which 27
dependent tests were previously unknown) and 1311
automatically-generated dependent tests.  A larger percentage (15\% vs.\ \todo{XXX}\%) of
automatically-generated tests are dependent.  Developers' understanding of
the code, and their goals when writing the tests, help them build
well-structured tests that carefully initialize and destroy the shared
objects they may use.
By contrast,  most automated test generation tools are not ``state-aware'': the
generated tests often ``misuse'' APIs, such as not setting up
the environment correctly.  This misuse may
indicate that the tests are invalid; it may indicate weaknesses, poor
design, or fragility of the APIs; or it may indicate that the human-written
tests have failed to exercise some functionality.

%Among the 29 human-written dependent tests, 23 tests
%can be detected by running in isolation, 2 tests
%require 2 tests to manifest the dependence, and 4
%tests require 3 tests to manifest the dependence.

The root cause of all the detected dependent tests is improper access to
static fields. The XML Security and Crystal developers use more
static fields in the test code,
so those projects have relatively more dependent tests.
%This concurs with our findings in Section~\ref{sec:studyfindings}.
%\edit{Can we say anything
%  about the relative frequency of the three?}

The randomized algorithm is surprisingly effective in
detecting dependent tests. In our experiments, when run \trialnum times,
it found every
dependent test identified by the other algorithms, plus 
4 more human-written dependent
tests in \jt. These 4 tests only
manifest when a sequence of three tests is run in a specified,
non-default order. Both exhaustive and \dependenceaware{} $k$-bounded
algorithms fail to detect these tests, because
they cannot scale to $k$=3 for 
\jt.  The randomized algorithm also
detects more dependent
tests in the automatically-generated test suites.

The \dependenceaware{} bounded algorithm found the same
number of dependent tests as the exhaustive bounded algorithm\todo{Write
  Javari annotations for the \CodeIn{java.security} package, and re-run the
  experiment so that all of the rest of this paragraph can be cut from the
  camera-ready version of the paper.}, except
that it missed one dependent test in XML Security's
automatically-generated test suite.
The dependent test was missed because \ourtool
did not track static field accesses in the \CodeIn{java.security} package
of the JDK, and Javari did not provide annotations for APIs
in that package.


\subsubsection{Performance of \ourtool}
\label{sec:performance}

The time cost of the reversal algorithm is very low, and
the time cost of the randomized algorithm 
is proportional to the run time of the suite and the number of runs.
Overall, the time cost is acceptable for practical use.
For example, the randomized algorithm took around 1.5 hours
to finish 1000 runs,  for \jt's human-written test
suite (\jodatimetests tests).
 
The time cost of running the exhaustive $k$-bounded algorithm
is prohibitive. The JVM initialization time is the main cost.
The exhaustive algorithm failed to
scale to one human-written test suite and all four automatically-generated
test suites when $k$=2, and failed to scale to all test suites
when $k$=3. The primary reason is the large
number of possible test permutations. 
For example, there are 15,011,750 size-2 permutations
for \jt's human-written test suite (\jodatimetests tests),
which would take approximately 58 days to execute.
%For example, running all
%Joda-Time's \jodatimetests human-written
%unit tests when $k$=2 requires running 15,011,750 test pairs, taking
%approximately 

Table~\ref{tab:results} gives an estimated time cost for each
test suite that an algorithm failed to scale to. For each test suite,
we randomly chose 1000 permutations from all
test permutations, executed them, and measured the average time cost
per permutation. Then, we multiplied
the average cost by the total number of permutations to estimate
the time cost.

The \dependenceaware{} $k$-bounded algorithm ran about
an order of magnitude faster
than the exhaustive $k$-bounded algorithm,  when $k$=2.
%On average,
%it took 3.3$\times$ and 1.1$\times$ less time to run all
%human-written tests and automatically-generated tests, respectively, when $k$=1.
%The \dependenceaware{} algorithm took an order of magnitude
%less time to run both human-written tests and automatically-generated tests,
%The speedups are largely determined by performance on Joda-Time.
The \dependenceaware{} algorithm helps most when there are relatively many
tests, each one of them relatively small.



%\emph{three} tests to manifest. While these are easy to reproduce, we
%did not check that our tool finds them, because the time needed to
%run our naive algorithm on Joda-Time with $k=3$ is measured in months.


%\enlargethispage{5pt}

\begin{table}
\centering
\setlength{\tabcolsep}{0.25\tabcolsep}
\begin{tabular}{|l|l|}
%\toprule
\hline
\textbf{Label} & \textbf{Technique Description} \\
\hline
T1 & Randomized ordering \\
T3 & Prioritize on coverage of statements \\
T4 & Prioritize on coverage of statements not yet covered\\
T5 & Prioritize on coverage of methods\\
T7 & Prioritize on coverage of functions not yet covered \\
%\bottomrule
\hline
%\textbf{Total}& &  & &  \\ 
%\hline
\end{tabular}
\caption{Five test prioritization techniques used
to assess the impact of dependent tests. These five
techniques are introduced in Table 1
of~\cite{Elbaum:2000:PTC:347324.348910}. (We use
the same labels as in~\cite{Elbaum:2000:PTC:347324.348910}. We did not
implement the other 9 test prioritization techniques
introduced in~\cite{Elbaum:2000:PTC:347324.348910}, since
they require a fault history that is not
available for our subject programs.)
}
\tinyrelax
\label{tab:testprio}
\end{table}



\begin{table}
\centering
% \setlength{\tabcolsep}{1.25\tabcolsep}
\begin{tabular}{|l|c|c|c|c|c|}
%\toprule
\hline
\textbf{Subject Program} & T1 & T3 & T4 & T5 & T7 \\
\hline
\jt& 0 & 0 & 1 & 0 & 0\\
XML Security& 0 & 0 & 0 & 0 & 0 \\
Crystal& 12 & 11 & 16 & 11 & 12 \\
Synoptic& 0 & 0 & 0 & 0 & 0 \\
%\bottomrule
\hline
\textbf{Total} & 12 & 11 & 17 & 11 & 12\\
\hline
%\textbf{Total}& &  & &  \\ 
%\hline
\end{tabular}
\caption{Differences in test results between original and prioritized
  human-written unit test suites.
Each cell shows the number of tests
that do not return the same results as they do when executed
in the default, unprioritized order.
}
\smallrelax
\label{tab:testprioresult}
\end{table}


%  LocalWords:  LOC Joda b609d7d66d T1 T3 T4 T7 unprioritized

\subsubsection{The Impact on Test Prioritization}
\label{sec:impact}

We implemented five test prioritization techniques~\cite{Elbaum:2000:PTC:347324.348910} (summarized in Table~\ref{tab:testprio}) and
applied them to the human-written test suites of our subject programs.


For each test prioritization algorithm, we counted the number
of dependent tests that return different results (pass or fail) in
the prioritized order than they do when executed in the
unprioritized order. Table~\ref{tab:testprioresult} summarizes
the results.

The dependent tests in our subject programs interfere with
\textit{all} five test prioritization techniques in
Table~\ref{tab:testprio}.
%\todo{we need to edit the sentence below, since a reviewer has
%strong opinion about the word "assume"}
This is because all these techniques
implicitly assume that there are no test dependences in
the input test suite. Violation of this assumption, as
happened in real-world test suites, can cause the prioritized suite to fail
even though the original suite passed.

We did not evaluate the effect of test dependence on metrics such as 
APFD~\cite{Rothermel:1999:TCP:519621.853398}; there is no point optimizing
such a metric at the cost of false positives or false negatives.


%One possible to remedy this problem is to 

%as  in the
%4 human-written test suites that can
%theoretically affect the results of these two techniques.
%For test selection, we manually checked whether
%there exists a subsequence of tests
%that do not return the same results
%when executed in the \textit{same} order
%as in the original test suite. 

%For test prioritization, we manually check whether
%there exists a possibly-\textit{reordered}
%subsequence of tests that do not return the
%same results as in the original test suite.

%As a result, 26 out of 29 dependent tests can
%%besides the synoptic test, and 2 jodatime tests
%affect the results of test selection,
%and all 29 dependent tests can affect the results
%of test prioritization.



\subsection{Discussion}
\label{sec:expdiscussion}

%\subsubsection{Developers' Reactions}

\noindent \textbf{Developers' Reactions to Dependent Tests.}
We sent the identified human-written dependent tests to the
subject program developers, asking for their feedback.

One dependent test in \jt was previously known
and had already been fixed. \jt's
developers confirmed the other new dependent
tests, and thought that they are due to unintended interactions
in the library.
%
The Crystal developers confirmed that all dependent tests
found in Crystal were unintentional and happened because of dependence
through global variables. The developers considered the
dependent tests undesirable and opened a bug report for
this issue~\cite{crystalbugreport}.\todo{Did it get fixed?  Can you ask
  them to do so, so that you can report that?}
%
The dependent test in Synoptic was previously known.
The developers merged two related tests to fix
the dependent test.
%
%After receiving our reported dependent tests in XML-Security,
The SIR~\cite{sir} maintainers confirmed our reported dependent
tests in XML-Security, and accepted our
suggested patch to fix them. They also highlighted the practice
that tests should \textit{always} ``stand alone''
without dependency on other tests, and characterized that as
``test engineering 101''. 
%They also accepted our suggested
%patch to fix the dependent tests.


%\noindent
%\setlength{\tabcolsep}{0.1\tabcolsep}
%\begin{tabular}{|l|c|}
%\hline
%Downstream Testing Techniques& \# Affected Dependent Tests\\
%\hline
%Test Selection & \\
%Test Prioritization& \\
%\hline
%\end{tabular}

\vspace{1mm}
\noindent \textbf{Threats to Validity}
There are several threats to the validity of our evaluation.
First, the \subjnum open-source
programs and their test suites may not be
representative enough. 
%Thus, we cannot claim the results
%can be generalized to an arbitrary program.
However, these are the first \subjnum subject programs
we tried, and the fact that we found dependent tests
in all of them is suggestive.
Second, in this evaluation, we focus specifically on
the {manifest dependence} between \textit{unit tests}.
We did not investigate possible test dependence that may arise
in other types of tests, such as integration tests.
Third, due to the computational complexity of the general dependent test
detection problem, we do not yet have
empirical data regarding \ourtool's recall and how many dependent
tests exist in a test suite. 
Fourth, we only assessed the
impact of dependent tests on five test prioritization
techniques.
%, and we only evaluated the impact of dependent
%tests on prioritizing unit tests.
Using other test prioritization techniques
might achieve different results. 

%\todo{this paper focuses on manifest test dependence, what about
%potential test dependence, as well as some other cases in the
%study}

\vspace{1mm}

\noindent \textbf{Experimental Conclusions}
We have four chief findings. \textbf{(1)}
Dependent tests do exist in practice, both in
human-written and automatically-generated test suites.
%can contain substantially more dependent tests than a human-written
%test suite.
\textbf{(2)} These dependent tests reveal weakness
in a test suite rather than defects in the tested code.
\textbf{(3)} Dependent tests can interfere with
test prioritization techniques and cause unexpected test failures.
\textbf{(4)} 
%\todo{need to figure out a proper english sentence to say like:
%The dependent test detection problem is inherently
%complex, a smarter algorithm may not achieve great results.}
The randomized algorithm is the most effective in
detecting dependent tests.
%, but it has no guarantee
%of the number of dependent tests it can detect.
%Testers
%can use this simple yet effective algorithm in practice.

%\todo{Idea:  how about combining the randomized and dependence-aware
%  algorithms?  That is, generate random orderings that violate dependences;
%  or generate random orderings and discard ones that do not violate
%  dependences.  This probably isn't a great idea, but a reader might think
%  of it.}

%  LocalWords:  Joda dependences multi 67GHz 4GB 2GB tradeoffs subsequence
%%  LocalWords:  istic 67GHz 4GB 2GB java

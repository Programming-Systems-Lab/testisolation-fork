
XML Security\footnote{\url{http://projects.apache.org/projects/xml_security_java.html}}
is a component library implementing XML signature and encryption
standards. Each released
version of XML Security has a human-written JUnit test suite that
achieves fairly high statement coverage.

Four stable released versions (1.0.2, 1.0.4, 1.0.5d2, and 1.0.71) of XML Security
have been incorporated in the Soft\-ware-artifact Infrastructure Repository
(SIR).\footnote{\url{http://sir.unl.edu}}
We found that at least two out of the four versions contain dependent tests. Specifically, in versions 1.0.4 and 1.0.5d2, \code{test\_Y1}, \code{test\_Y2}, and \code{test\_Y3}
in class \code{ExclusiveC14NInterop} show dependent behavior.
Since the dependences are the same in both versions, in the further
discussion and in Figure~\ref{fig:example-summary}, we consider only
version 1.0.4.

For all three dependences, the cause of the dependence is the same: before any
method in the library can be used, the global initialization function 
\code{Init.init()} has to be called. Internally, it initializes
the static field that the code tested by the dependent tests rely
on.

Given that the error when executing the dependent tests clearly explains the
cause of the error, we speculate that developers either simply forgot to
initialize the tests properly, or expected that these tests would always execute
in the order defined
in the test suite.

%on a global variable. Take \code{test\_Y1} as an example. This test passes when being executed
%with other tests, but fails by throwing an \code{InvalidCanonicalizerException}
%when executed individually.
%The root cause of such behavior difference is that, in XML-security, the \code{Init.init()} method initializes
%the static field \code{Canonicalizer.canonicalizerHash}, and test \code{test\_Y1} needs to use
%that static field to create a \code{Canonicalizer} instance. 
%When executing this test in the programmer-fixed order, method \code{Init.init()} has been called by
%other tests executed before \code{test\_Y1}, so that test \code{test\_Y1} passes.
%However, without calling \code{Init.init()} first,
%\code{test\_Y1} fails to create the \code{Canonicalizer} instance.
%
%Based on the dumped error message in the \code{InvalidCano-\\nicalizerException}:
%
%\begin{quote}
%``You must initialize the xml-security library correctly before you use it.
%Call the static method ``org.apache.xml.security.Init.init()'' to do that before you use any functionality
%from that library''
%\end{quote}

%We speculate that programmers should realize this potential dependence, but they
%overlook to enforce \code{test\_Y1} to be executed in a desirable order. Instead,
%programmers may have put an implicit assumption that tests in a suite can be executed in isolation
%and miss to add the necessary preconditions for \code{test\_Y1}. 

% vim:wrap:wm=8:bs=2:expandtab:ts=4:tw=70:



%on a global variable. Take \code{test\_Y1} as an example. This test passes when being executed
%with other tests, but fails by throwing an \code{InvalidCanonicalizerException}
%when executed individually.
%The root cause of such behavior difference is that, in XML-security, the \code{Init.init()} method initializes
%the static field \code{Canonicalizer.canonicalizerHash}, and test \code{test\_Y1} needs to use
%that static field to create a \code{Canonicalizer} instance. 
%When executing this test in the programmer-fixed order, method \code{Init.init()} has been called by
%other tests executed before \code{test\_Y1}, so that test \code{test\_Y1} passes.
%However, without calling \code{Init.init()} first,
%\code{test\_Y1} fails to create the \code{Canonicalizer} instance.
%
%Based on the dumped error message in the \code{InvalidCano-\\nicalizerException}:
%
%\begin{quote}
%``You must initialize the xml-security library correctly before you use it.
%Call the static method ``org.apache.xml.security.Init.init()'' to do that before you use any functionality
%from that library''
%\end{quote}

%We speculate that programmers should realize this potential dependence, but they
%overlook to enforce \code{test\_Y1} to be executed in a desirable order. Instead,
%programmers may have put an implicit assumption that tests in a suite can be executed in isolation
%and miss to add the necessary preconditions for \code{test\_Y1}. 

% vim:wrap:wm=8:bs=2:expandtab:ts=4:tw=70:

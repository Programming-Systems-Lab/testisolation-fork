\newcommand{\masking}{Masking Faults}
\newcommand{\initializaiton}{Test Structure} %ialization}
\newcommand{\spurious}{Spurious Bug}

\begin{table*}
\centering
\setlength{\tabcolsep}{0.3\tabcolsep}
\begin{tabular}{|l||c|c|c||c|c|}
%\toprule
\hline
\textbf{Subject} & \multicolumn{3}{|c||}{\textbf{Detected Dependent Tests}} & \multicolumn{2}{|c|}{\textbf{Analysis Cost (second)}}\\
\cline{2-6}
\textbf{Programs} & \textbf{\#Dependent Tests} & \textbf{\#Dependent Auto Tests} &
\textbf{Kind} &
\textbf{Basic Algorithm} & \textbf{Improved Algorithm}
\\
%\midrule
\hline
JodaTime & 3875 
% 3875 is retrieved by running mvn test on the related revision
& 3 & \masking{} & 2663 & 711  \\
Crystal & 75 & 18 &\initializaiton{} & 2542 & 20 \\
XML Security & 108 & 3 & \initializaiton{}& 2947 & 925  \\ %, 1.0.5d2\\
Derby & &  & & & \\ 
%\bottomrule
\hline
\end{tabular}
\caption{Experimental results. Column ``Kind'' refers to the kind
of problem associated with the dependency. 
\todo{The subject program list is out-of-date. donot look at the table content, just check the table format.}
\todo{The kind column above may not make much sense. since the
test we detect cannot lead to spurious bug report, and are unlikely
to reveal bugs. They can only reveal tes flaws.}
\todo{Or we should replace "kind" to "root cause".}
%automatically-generated test suites are described in Section~\ref{sec:autogen}.
%\todo{KM}{I think that the last sentence is not necessary}}
}
\label{tab:results}
\end{table*}

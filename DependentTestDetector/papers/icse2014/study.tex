%\section{Dependent Tests in Practice}
\section{Understanding Dependent Tests: Characteristic Study}
\label{sec:study}

To investigate whether dependent tests can reach beyond theory
and appear in real-world programs, this section presents an empirical
study of concrete examples of test dependence found in
well-known open source programs. 


\subsection{Sources and Study Methodology}

We chose five well-known, public-accessible software issue
tracking systems to examine: Apache~\cite{apachebug},
Eclipse~\cite{eclipsebug}, JBoss~\cite{jbossbug},
Hibernate~\cite{hibernatebug}, and Codehuas~\cite{codehuasbug}.
Each issue tracking system serves for tens of project, and
keeps thousands of bug reports, feature requests, improvement
suggestions, etc.

For each issue tracking system, we searched for a set of keywords
(``dependent test'', ``test dependence'', ``test execution order'',
etc), and manually examined the matched results. For each match, we read the
description of the issue report, the discussions between reporters
and developers, and the fixing patches (if available). This information
helps us understand whether the report is about test dependence
 --- a test manifesting different behaviors under different
test execution orders. Each dependent test is examined by
at least two people and the whole process consists of serveral
rounds of (re-)study and cross checking. We ignored reports
that are described vaguely, and excluded tests whose results are
affected by non-determinism (e.g., multi-threading).
In total, we have spent more than half year to collect and analyze all dependent
tests. 


\subsection{Findings}

\begin{table*}[t]
\vspace{1mm}
\centering
\small{
\setlength{\tabcolsep}{.15\tabcolsep}
\begin{tabular}{|c||c|c|c|c||c|c|c|c|c||c|c|c|c|c||c|c|c|c|}
\hline
%1&2&3&4&5&6&7&8&9&10&11&12&13&14\\
\textbf{Issue}&\multicolumn{4}{|c||}{\textbf{Dependent Tests}}&\multicolumn{5}{|c||}{\textbf{\# Involved Tests for}}&\multicolumn{5}{|c||}{\textbf{Resolution}}&\multicolumn{4}{|c|}{\textbf{Root Cause}}\\
\cline{2-5}\cline{11-19}
\textbf{Tracking} &Total&\multicolumn{3}{|c||}{Severity}&\multicolumn{5}{|c||}{\textbf{Manifestation}}&
&\multicolumn{4}{|c||}{Patch Location}&Static&File & Data-& Unknown\\
\cline{3-10}\cline{12-15}
\textbf{System}&Number&Major&Minor&Trivial& Self &1 test&2 tests&3 tests & Unknown&Days&Code&Test&Doc&Unfixed&Variable&System& base &\\
\hline
Apache&26&22&3&1&0&5&18&1&2&93&5&20&0&1&9&3&8 &6\\
\hline
Eclipse&59&0&59&0&0&0&49&1&9&48&1&8&49&1&49&0&0 &10\\
\hline
JBoss&6&6&0&0&0&0&3&0&3&44&0&2&0&4&1&0& 0 & 5\\
\hline
Hibernate&3&1&1&1&0&0&3&0&0&6&0&1&0&2&0&0& 2 & 1\\
\hline
Codehaus&2&2&0&0&1&1&0&0&0&3&0&1&0&1&0&1&0 &1\\
\hline
\hline
\textbf{Total} & \dtnum &31&63&2&1&6&73&2&\unum&194&6&32&49&9&\svnum&4&10&23\\
\hline
\end{tabular}
}
\vspace{-2mm}
\caption{{\label{tab:studyresults}
Real-world dependent tests.
%Column ``Total Number'' shows the total number of identified dependent tests.
Column ``Severity'' is the developers' assessment of the importance of the
test dependence.
Column ``\# Involved Tests for Manifestation'' is the number of tests needed
to manifest the dependence. Column ``Self'' shows the number of
tests that depend on themselves. Column ``Days'' is the
average days taken by developers to resolve a dependent test.
Column ``Patch Location'' shows how developers resolved the dependent tests:
by modifying program code, by modifying test code, by adding
code comments, or not fixed.
%In column ``Dependence Root Cause'', ``other'' execution environment
%differences include language, locale, and databases.
}
}
%\todo{Split ``unfixed'' into ``documented'' and ``unfixed''}
\end{table*}

%  LocalWords:  JBoss Codehaus


Table~\ref{tab:studyresults} summarizes all collected dependent tests.


\subsubsection{Characteristics}

We summarize characteristics of dependent tests in three
aspects: manifestation, root cause, and developers' reaction.

\vspace{1mm}
\noindent \textbf{{Manifestation: at least 82\% of the dependent
tests in the study can be manifested by no more than 2 tests.}} In theory,
given a $n$-sized test suite, dependent test can occur in any
length of permutations. However, among \dtnum collected tests,
86 (82\%) of them can be manifested by running no more than
2 tests. 

\vspace{1mm}
\noindent \textbf{{Root cause: at least 52\% of the dependent tests
in the study arise because of the inproper access to shared static
variables.}} Among \dtnum dependent tests, 58 (52\%) of them
arise due to inappriopriate access to
shared static varialbes; and 13 (12\%) of them arise
due to inappropriate access to file systems or other
execution environment. 

\vspace{1mm}
\noindent \textbf{{Developers' reactions: dependent tests
often indicate flaws in the test code, and developers usually
ignore dependent tests due to the lack of tool support.}}
In some cases, they are intentional, developers are aware
of them and document them, but in other cases they are
inadvertent. Among \dtnum collected dependent tests,
97 (93\%) of them were treated as major or minor problems,
but only 34 (32\%) of them got fixed by developers. Among
the 34 fixed dependent tests, only 7 fixes are
on the program code, while the other fixes are on the
tested code. This indicates that dependent tests usually
reveal potential flaws in the test code rather than the test code.
Based on developers' discussion, we found that although
developers admitted that such test dependence should be removed,
they often leave the dependence unresolved, sometimes by merging
two tests or adding explaintory documentation. The primary
reason is that the current JUnit testing framework does not
support to explicitly specify test dependence in the test code.


%Test dependence can cause problems, not only
%when test suites are reordered, but even when they are
%executed in the intended order.



\subsubsection{Repercussions of Dependent Tests}
\label{sec:repercussion}

\begin{table}
\centering
\setlength{\tabcolsep}{0.15\tabcolsep}
\begin{tabular}{|c||c|c|}
%\toprule
\hline
\textbf{Issue Tracking} & \textbf{Revealing Weakness} & \textbf{Masking Faults} \\
\textbf{System } & \textbf{in a Test Suite} & \textbf{in a Program} \\
\hline
Apache &24 & 2 \\
\hline
Eclipse & 59 & 0 \\
\hline
JBoss& 6 & 0 \\
\hline
Hibernate & 3 & 0 \\
\hline
Codehaus & 2 & 0 \\
\hline
\hline
\textbf{Total}  & 94 & 2 \\
\hline
\end{tabular}
\caption{
Classification of the collected dependent tests
based on its repercussions.
}
\label{tab:reper}
\end{table}


The dependent tests identified in our study fall into 3
categories (Table~\ref{tab:reper}). We describe each category below and give
concrete examples. 
%classifies
%the collected tests into these 3 categories.

\vspace{1mm}

\noindent \textbf{Poor Test Construction.} Most identified
dependent tests fall into this category. The test dependences
arise due to incorrect initialization of program state by one
or more tests, and reveal flaws in the test suite itself
rather than the tested code. Typically, one test initializes
a global variable or the execution envrionment; and another
test does not perform any initialization, but
relies on the program state after the first test's execution.
%global variable that is a part of the environment, but the test does
%not properly initialize it.  In the second case, a test should but
%does not call
%an initialization function before later invocations to a complex library.
Such dependence in the test code is often masked because
the initializing test always executes before other tests in the
default execution order. The dependent tests are revealed
until the initializing test is reordered to execute
\textit{after} other tests. 

%the default test execution order includes tests that initialize the library.  The defect is
%inconsequential until and unless the flawed test is reordered, either manually or by
%a downstream tool, to execute before any other initializing test.

\vspace{1mm}

\noindent \textbf{Spurious Bug Reports}
Sometimes developers introduce dependent tests intentionally because it is
easier, more efficient or more convenient to write unit tests for some modules
in that way~\cite{kapfhammeretal:FSE:2003, whittakeretal:2012}.
%DB-testing}.
Even though the developers are aware of these instances
when they create them, this knowledge can get lost, 
and other people who are not aware of these dependences can get confused 
when they run a subset of the test suite that manifests the
dependences.

As a result, software users or maintainers
might report bugs backed by the failing tests, although this
is exactly the expected behavior. 
If the dependence is not documented clearly and
correctly, it can take a considerable amount of time to work out that
these reported failures are spurious.
%Or worse, the developers may try
%to fix a bug that is not there.
For example,
in September 2003, a user fired a
bug report in SWT~\cite{swt}\footnote{\url{https://bugs.eclipse.org/bugs/show_bug.cgi?id=43500}},
stating that tests were failing unexpectedly
if she runs any other test before \texttt{TestDisplay} --- 
a test suite creating a new \code{Display} object and testing its
implementation. However, this bug report was spurious and was
caused by undocumented test dependence.
Its root cause is quite simple: in SWT, only one global \texttt{Display}
object is allowed; the tests that reporters try to run
create, but do not dispose of a \code{Display} object, while
the tests in \code{TestDisplay} attempt to create
a new \code{Display} object, which fails, as one
is already created. This is the desired behavior,
and points to a potential problem in the test suite rather
than the code.

\vspace{1mm}

\noindent \textbf{Masking Faults}. In rare cases,
dependent tests can hide a fault in the
program, \emph{exactly} when the test suite is executed in its default
order. Masking occurs when a test case $t$ (a) \emph{should}
reveal a fault, (b) only does so when executed in a specific environment
$\env_R$, but (c) tests executed before $t$ in a test suite always
generate environments different from
$\env_R$.


\begin{figure}[t]
%\noindent \textbf{\small{Fault-related code in CLI:}}
%\vspace{-2mm}
\begin{CodeOut}
\begin{alltt} 
public final class OptionBuilder \{
  \textbf{private static String argName = null;}
  private static void reset() \{
    ...
    \textbf{argName = "arg";}
    ...
  \}
\}
\end{alltt}
\end{CodeOut}
\vspace*{-15pt}
\caption{Simplified fault-related code in CLI~\cite{cli} (revision 661513).
The fault was masked by two dependent tests for over 3 years.
}
\label{fig:option_builder}
\end{figure}

%  LocalWords:  OptionBuilder argName arg CLI


We only found two dependent tests in
the Apache CLI library~\cite{cli} for this category.
In CLI, two test cases fail when run in isolation:
\code{test13666} and \code{test\-Op\-tion\-With\-out\-Short\-For\-mat2} in test
classes \code{Bugs\-Test} and \code{Help\-For\-mat\-ter\-Test},
respectively, but both pass when running in the default order.

Figure~\ref{fig:option_builder} shows the simplified code and
tests. Both dependent tests can reveal this fault,  but
the default order of test execution makes both tests pass
accidently. Such dependent tests
make non-trivial impact in practice.
This fault is reported in the bug
database several times,\footnote{\url{https://issues.apache.org/jira/browse/CLI-26} \url{https://
issues.apache.org/jira/browse/CLI-186} \url{https://issues.apache.org/jira/browse/
CLI-187}} starting on March 13, 2004 (CLI-26). The report is marked as resolved
\emph{three years} later on March 15, 2007 until developers
realized the test dependence.

%, but is then reopened as CLI-186 on
%July 31, 2009. About one month later, the bug is duplicated as
%CLI-187, and the actual fix happens one 
%year later on June 19, 2010, about six years after the bug was first reported (and four years
%total on the open-issue list).


%On this report, one of the developers commented:
%\begin{quote}
%I reproduced the issue, it requires a dedicated test case since it is tied to the initialization 
%of a static field in OptionBuilder.
%\end{quote}

%Despite the realization that a dedicated test is required, no such
%test was ever created.

%\paragraph{Eclipse SWT: Causing Spurious Bug Reports}


\subsubsection{Implications}

We summarize the main implications of our findings.

\noindent \textbf{{Dependent tests exists in practice, but
they can be hard to identify unless explicitly searched for.}}
None of the dependent tests we studied is identified by
running the existing test suite in the default order. Often,
the test dependence is reported until and unless the
test suite is recordered, either manually by a user or
a maintainer or by a downstream tool. This indicates that
specialized tools to detect such dependence are needed.

\vspace{1mm}
\noindent \textbf{Dependent test detection techniques
can bound the search space to a small number of tests.}
In theory, given a $n$-sized test suite, a technique need
to exhaustively execute all $n!$ permutations of the
test suite to detect dependent tests. This is
not feasible for realistic $n$. As indicated by our study,
most of the dependent tests can be manifested by executing
no more than 3 tests. This finding suggests that
many realistic dependent tests might be found by
running just short subsequences of test suites
with a bound $k$ on the length of subsequences. That
effectively bounds the execution time to $O(n^k)$,
which for small $k$ is tractable.

\vspace{1mm}
\noindent \textbf{Dependent test detection techniques
should focus on analyzing static variable accesses.}
More than half of the studied dependent tests are caused
by static variable access. Thus, focusing on analyzing
static variables can be a cost-effective way for a
tool design. \todo{need to re-write the above}

%dependent tests.
%can bound Tools


%\vspace{1mm}
%\noindent \textbf{Dependent test fixing tool
%Test dependence reveals flaws in the test code.}
%This indicates that a potential dependent test fixing tool should target
%the test code


\subsection{Caveats}

Our findings need to be taken with the methodology in mind. The
issue tracking systems in our study cover representative and important
software applications with comprehensive test suites; and
all studied dependent tests are from well-tested
applications. Nevertheless, all applications we studied
are written in Java, and the tests are in the form of JUnit.
Thus, we do not claim our findings can be extended to
arbitrary programs.

The dependent tests in our study are collected from public-accessible
issue tracking systems. We have followed the decisions made by
developers to determine a dependent test, and its severity;
and did not intentionally ignore
any test dependence in the issue tracking system. However,
it is almost certain our methodology will miss
some dependent tests in the tracking system. This is
because our search keyword list is not complete. Further,
some dependent tests may never
be identified, or reported to the developers. Unfortunately,
there is no conceivable way to study these unreported
dependent tests. 

%We believe the dependent tests in our study
%provide a representative sample in these software applications.

In our study, we do not emphasize any quantitative characteristic
results, and most of our findings are consistent across
the examined dependent tests.

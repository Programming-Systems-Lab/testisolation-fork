%\section{Dependent Tests in Practice}
\section{Understanding Dependent Tests: Characteristic Study}
\label{sec:study}

To investigate whether dependent tests can reach beyond theory
and appear in real-world programs, this section presents an empirical
study of concrete examples of test dependence found in
well-known open source programs. 


\subsection{Sources and Study Methodology}

We chose five well-known, public-accessible software issue
tracking systems to examine: Apache~\cite{apachebug},
Eclipse~\cite{eclipsebug}, JBoss~\cite{jbossbug},
Hibernate~\cite{hibernatebug}, and Codehuas~\cite{codehuasbug}.
Each issue tracking system serves for tens of project, and
keeps thousands of bug reports, feature requests, improvement
suggestions, etc.

For each issue tracking system, we searched for a set of keywords
(``dependent test'', ``test dependence'', ``test execution order'',
etc), and manually examined the matched results. For each match, we read the
description of the issue report, the discussions between reporters
and developers, and the fixing patches (if available). This information
helps us understand whether the report is about test dependence
-- a test manifesting different behaviors under different execution
environment. Each likely dependent test is examined by
at least two people and the whole process consists of serveral
rounds of (re-)study and cross checking. We ignored reports
that are described vaguely, and excluded tests whose results are
affected by non-determinism (e.g., multi-threading).
In total, we have spent more than half year to collect and analyze all dependent
tests. 


\subsection{Findings}

\begin{table*}[t]
\vspace{1mm}
\centering
\small{
\setlength{\tabcolsep}{.15\tabcolsep}
\begin{tabular}{|c||c|c|c|c||c|c|c|c|c||c|c|c|c|c||c|c|c|c|}
\hline
%1&2&3&4&5&6&7&8&9&10&11&12&13&14\\
\textbf{Issue}&\multicolumn{4}{|c||}{\textbf{Dependent Tests}}&\multicolumn{5}{|c||}{\textbf{\# Involved Tests for}}&\multicolumn{5}{|c||}{\textbf{Resolution}}&\multicolumn{4}{|c|}{\textbf{Root Cause}}\\
\cline{2-5}\cline{11-19}
\textbf{Tracking} &Total&\multicolumn{3}{|c||}{Severity}&\multicolumn{5}{|c||}{\textbf{Manifestation}}&
&\multicolumn{4}{|c||}{Patch Location}&Static&File & Data-& Unknown\\
\cline{3-10}\cline{12-15}
\textbf{System}&Number&Major&Minor&Trivial& Self &1 test&2 tests&3 tests & Unknown&Days&Code&Test&Doc&Unfixed&Variable&System& base &\\
\hline
Apache&26&22&3&1&0&5&18&1&2&93&5&20&0&1&9&3&8 &6\\
\hline
Eclipse&59&0&59&0&0&0&49&1&9&48&1&8&49&1&49&0&0 &10\\
\hline
JBoss&6&6&0&0&0&0&3&0&3&44&0&2&0&4&1&0& 0 & 5\\
\hline
Hibernate&3&1&1&1&0&0&3&0&0&6&0&1&0&2&0&0& 2 & 1\\
\hline
Codehaus&2&2&0&0&1&1&0&0&0&3&0&1&0&1&0&1&0 &1\\
\hline
\hline
\textbf{Total} & \dtnum &31&63&2&1&6&73&2&\unum&194&6&32&49&9&\svnum&4&10&23\\
\hline
\end{tabular}
}
\vspace{-2mm}
\caption{{\label{tab:studyresults}
Real-world dependent tests.
%Column ``Total Number'' shows the total number of identified dependent tests.
Column ``Severity'' is the developers' assessment of the importance of the
test dependence.
Column ``\# Involved Tests for Manifestation'' is the number of tests needed
to manifest the dependence. Column ``Self'' shows the number of
tests that depend on themselves. Column ``Days'' is the
average days taken by developers to resolve a dependent test.
Column ``Patch Location'' shows how developers resolved the dependent tests:
by modifying program code, by modifying test code, by adding
code comments, or not fixed.
%In column ``Dependence Root Cause'', ``other'' execution environment
%differences include language, locale, and databases.
}
}
%\todo{Split ``unfixed'' into ``documented'' and ``unfixed''}
\end{table*}

%  LocalWords:  JBoss Codehaus


Table~\ref{tab:studyresults} summarizes all studied dependent tests.


\subsubsection{Characteristics}

We characterize dependent tests in three

\noindent \textbf{{Dependent tests exists in practice.}}
Developers treat it as a problem that should be fixed.

\vspace{1mm}
\noindent \textbf{{At least 82\% of the studied dependent
tests can be manifested by no more than 2 tests.}} In theory,
it may enumerate every possible combination.

\vspace{1mm}
\noindent \textbf{{Developers usually ignore dependent
tests due to lack of tool support.}}

\vspace{1mm}
\noindent \textbf{{At least 52\% of the dependent tests arise
because of accessing shared static variables.}}

Finding on dependence patterns

Findings on dependence manifestation

Findings on dependence resolution

Findings on Dependence avoidance



\subsubsection{Repercussions of Dependent Tests}

Most dependent tests identified in our study fall into a small
number of categories\todo{how many, be specific}. We describe each category below and give
concrete examples. 

\vspace{1mm}

\noindent \textbf{Poor Test Construction.}
The test dependences in this category arise due to incorrect initialization
of program state by one or more tests. In the first case,
tested program code relies on a
global variable that is a part of the environment, but the test does
not properly initialize it.  In the second case, a test should but
does not call
an initialization function before later invocations to a complex library.
This flaw in the test code is masked because the default test suite execution
order includes other tests that initialize the library.  The defect is
inconsequential until and unless the flawed test is reordered, either manually or by
a downstream tool, to execute before any other initializing test.

\vspace{1mm}

\noindent \textbf{Spurious Bug Reports}
Sometimes developers introduce dependent tests intentionally because it is
easier, more efficient or more convenient to write unit tests for some modules
in that way~\cite{kapfhammeretal:FSE:2003, whittakeretal:2012}.
%DB-testing}.
Even though the developers are aware of these instances
when they create them, this knowledge can get lost, 
and other people who are not aware of these dependences can get confused 
when they run a subset of the test suite that manifests the
dependences.

As a result, software users or maintainers
might report bugs backed by the failing tests, although this
is exactly the expected behavior. 
If the dependence is not documented clearly and
correctly, it can take a considerable amount of time to work out that
these reported failures are spurious.
%Or worse, the developers may try
%to fix a bug that is not there.
Consider a dependent test in SWT~\cite{swt},
in September 2003, a user reported a
bug,\footnote{\url{https://bugs.eclipse.org/bugs/show_bug.cgi?id=43500}}
stating that tests were failing unexpectedly
if she runs any other test before \texttt{TestDisplay} --
a test suite creates a new \code{Display} object and tests its
implementation. However, this bug report was spurious and was
caused by undocumented test dependence.
Its root cause is quite simple: in SWT, only one global \texttt{Display}
object is allowed; the tests that reporters try to run
creates, but does not dispose of a \code{Display} object, while
the tests in \code{TestDisplay} attempt to create
a new \code{Display} object, which fails, as one
is already created. This is the expected and desired behavior,
and points to a potential problem in the test suite rather
than the code.

\vspace{1mm}

\noindent \textbf{Masking Faults}. In rare cases,
dependent tests can hide a fault in the
program, \emph{exactly} when the test suite is executed in its default
order. Masking occurs when a test case $t$ (a) \emph{should}
reveal a fault, (b) only does so when executed in a specific environment
$\env_R$, but (c) tests executed before $t$ in a test suite always
generate environments different from
$\env_R$.


\begin{figure}[t]
%\noindent \textbf{\small{Fault-related code in CLI:}}
%\vspace{-2mm}
\begin{CodeOut}
\begin{alltt} 
public final class OptionBuilder \{
  \textbf{private static String argName = null;}
  private static void reset() \{
    ...
    \textbf{argName = "arg";}
    ...
  \}
\}
\end{alltt}
\end{CodeOut}
\vspace*{-15pt}
\caption{Simplified fault-related code in CLI~\cite{cli} (revision 661513).
The fault was masked by two dependent tests for over 3 years.
}
\label{fig:option_builder}
\end{figure}

%  LocalWords:  OptionBuilder argName arg CLI


We only found two dependent tests in
the Apache CLI library~\cite{cli}.
In CLI, two test cases fail when run in isolation:
\code{test13666} and \code{test\-Op\-tion\-With\-out\-Short\-For\-mat2} in test
classes \code{Bugs\-Test} and \code{Help\-For\-mat\-ter\-Test},
respectively, but both pass when running in the default order.

Figure~\ref{fig:option_builder} shows the simplified code and
tests. Both dependent tests can reveal this fault,  but
the default order of test execution makes both tests pass
accidently and then mask the faults. Such dependent tests
make non-trivial impact in practice.
This fault is reported in the bug
database several times,\footnote{\url{https://issues.apache.org/jira/browse/CLI-26} \url{https://
issues.apache.org/jira/browse/CLI-186} \url{https://issues.apache.org/jira/browse/
CLI-187}} starting on March 13, 2004 (CLI-26). The report is marked as resolved
\emph{three years} later on March 15, 2007, but is then reopened as CLI-186 on
July 31, 2009. About one month later, the bug is duplicated as
CLI-187, and the actual fix happens one 
year later on June 19, 2010, about six years after the bug was first reported (and four years
total on the open-issue list).


%On this report, one of the developers commented:
%\begin{quote}
%I reproduced the issue, it requires a dedicated test case since it is tied to the initialization 
%of a static field in OptionBuilder.
%\end{quote}

%Despite the realization that a dedicated test is required, no such
%test was ever created.

%\paragraph{Eclipse SWT: Causing Spurious Bug Reports}


\subsubsection{Implications}

\vspace{1mm}
\noindent \textbf{Need to explicit search for, tool support}

\vspace{1mm}
\noindent \textbf{Tools
should focus on static variables}
In some cases, they are intentional, developers are aware
of them and document them, but in other cases they are
inadvertent. Test dependence can cause problems, not only
when test suites are reordered, but even when they are
executed in the intended order.

\vspace{1mm}
\noindent \textbf{Bounding the length
of search}

\vspace{1mm}
\noindent \textbf{Most dependent tests are not easy to identify unless explicitly searched for.}

\subsection{Caveats}

Our findings need to be taken with the methodology in mind. The
bug repositories in our study cover representative and important
software applications with comprehensive test suites; and
all studied dependent tests are from well-tested
applications. Nevertheless, most applications we studied
are written in Java, and the tests are in the form of JUnit.
Thus, we do not claim our findings can be extended to
arbitrary programs.

The dependent tests in our study are collected from bug
repositories. We have followed the decisions made by
developers to classify the category, severity, and
\todo{other characteristics} of a dependent test,
and did not intentionally ignore
any test dependence in bug repository. However,
it is entirely possible, if not almost certainly,
that some dependent tests may never
be identified, or reported to the developers. Unfortunately,
there is no conceivable way to study these unreported
dependent tests. We believe the dependent tests in our study
provide a representative sample in these software applications.

In our study, we do not emphasize any quantitative characteristic
results, and most of our findings are consistent across
the examined dependent tests.

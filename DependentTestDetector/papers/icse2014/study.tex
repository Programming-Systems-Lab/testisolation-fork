%\section{Dependent Tests in Practice}
\section{Understanding Dependent Tests: Characteristic Study}
\label{sec:study}

To investigate whether dependent tests can reach beyond theory
and appear in real-world programs, this section presents an empirical
study of concrete examples of test dependence found in
well-known open source programs. 


\subsection{Sources and Study Methodology}

We chose five well-known, public-accessible software issue
tracking systems to examine: Apache~\cite{apachebug},
Eclipse~\cite{eclipsebug}, JBoss~\cite{jbossbug},
Hibernate~\cite{hibernatebug}, and Codehuas~\cite{codehuasbug}.
Each issue tracking system serves for tens of project, and
keeps thousands of bug reports, feature requests, improvement
suggestions, etc.

For each issue tracking system, we searched for a set of keywords
(``dependent test'', ``test dependence'', ``test execution order'',
etc), and manually examined the matched results. For each match, we read the
description of the issue report, the discussions between reporters
and developers, and the fixing patches (if available). This information
indicated to us whether the report is about test dependence
-- a test manifesting different behaviors under different execution
environment. Each likely dependent test is examined by
at least two people and the whole process consists of serveral
rounds of (re-)study and cross checking. In total,
we have spent more than half year to collect and inspect all dependent
tests. 


\subsection{Findings}

\begin{table*}[t]
\vspace{1mm}
\centering
\small{
\setlength{\tabcolsep}{.60\tabcolsep}
\begin{tabular}{|l||l|l|l|l||l|l|l||c|c|c||c|c|c|}
\hline
%1&2&3&4&5&6&7&8&9&10&11&12&13&14\\
\textbf{Issue}&\multicolumn{4}{|c||}{\textbf{Identified Dependent Tests}}&\multicolumn{3}{|c||}{\textbf{Dependence Manifestation}}&\multicolumn{3}{|c||}{\textbf{Dependence Resolution}}&\multicolumn{3}{|c|}{\textbf{Dependence Root Cause}}\\
\cline{2-5}\cline{9-14}
\textbf{Tracking} &Total&\multicolumn{3}{|c||}{Severity}&\multicolumn{3}{|c||}{\#Involved Tests}&
Average&\multicolumn{2}{|c||}{Patch Location}&Static&File & Execution\\
\cline{3-8}\cline{10-11}
\textbf{System}&Number&Major&Minor&Trivial&1&2&>=3&Days&Code&Test&Variable&System&Env\\
\hline
Apache&2&3&4&5&6&7&8&9&10&11&12&13&14\\
\hline
Eclipse&2&3&4&5&6&7&8&9&10&11&12&13&14\\
\hline
JBoss&2&3&4&5&6&7&8&9&10&11&12&13&14\\
\hline
Hibernate&2&3&4&5&6&7&8&9&10&11&12&13&14\\
\hline
Codehuas&2&3&4&5&6&7&8&9&10&11&12&13&14\\
\hline
\hline
\textbf{Total} &2&3&4&5&6&7&8&9&10&11&12&13&14\\
\hline
\end{tabular}
}
\vspace{-2mm}
\caption{{\label{tab:studyresults} Summary of dependent tests
found in 5 issue tracking systems. Column ``Total Number'' shows
the total number of all identified dependent tests. Column ``Severity''
classifies dependent tests based on its impact to the program,
as decided by developers in the report. Column ``\#Invovled Tests''
classifies dependent tests by the the number of tests needed
to manifest the dependence. Column ``Average Days'' shows the
average days needed by developers to resolve a dependent test.
Column ``Patch Location'' shows how developers fix a
dependent tests. Column ``Code'' shows the number of dependent
tests fixed by modifying program code, and column ``Test'' shows
the number or dependent tests fixed by modifying test code.
Column ``Dependence Root Cause'' classifies dependent tests by
its root causes, including inappropriate accessing shared static
variables, file systems, and other execution environment (e.g.,
language, locale, or other system properties).
}
}
\end{table*}


Table~\ref{tab:studyresults} summarizes all studied dependent tests.


\subsubsection{Characteristics}

Finding on dependence patterns

Findings on dependence manifestation

Findings on dependence resolution

Findings on Dependence avoidance



\subsubsection{Categories}

Most dependent tests identified in our study fall into a small
number of categories\todo{how many, be specific}. We describe each category below and give
concrete examples. 

\vspace{1mm}

\noindent \textbf{Masking Faults}
\emph{Masking} is a particularly perplexing problem caused by
dependence.
The negative effect of masking is that it hides a fault in the
program, \emph{exactly} when the test suite is executed in its default
order. 
Masking occurs when a test case $t$ (a) \emph{should}
reveal a fault, (b) only does so when executed in a specific environment
$\env_R$, but (c) tests executed before $t$ in a test suite always
generate environments different from
$\env_R$.
%To express this more
More precisely and without loss of generality, assume any
environment with only a single variable. Then let $T =
\suite{t_1,\dots,t_n}$ be the test suite, and let $t_i, 1 < i \leq n$
be the test that should reveal the fault in environment $\env_R$. A
dependency $t_k \prec t_i, k < i$ masks the fault if
$\exec{\suite{t_1,\dots,t_{i-1}}}{\env_0} \neq \env_R$.

\vspace{1mm}

\noindent \textbf{Spurious Bug Reports and Bug Fixes}
Sometimes developers introduce dependent tests intentionally because it is
easier, more efficient or more convenient to write unit tests for some modules
in that way~\cite{kapfhammeretal:FSE:2003, whittakeretal:2012}.
%DB-testing}.
Even though the developers are aware of these instances
when they create them, this knowledge can get lost, 
and other people who are not aware of these dependences can get confused 
when they run a subset of the test suite that manifests the
dependences.

As a result, they
might report bugs backed by the failing tests, although this is exactly the expected
behavior. If the dependence is not documented clearly and
correctly, it can take a considerable amount of time to work out that
these reported failures are spurious. Or worse, the developers may try
to fix a bug that is not there.


\vspace{1mm}

\noindent \textbf{Poor Test Construction.}
The test dependences in this category arise due to incorrect initialization
of program state by one or more tests. In the first case,
tested program code relies on a
global variable that is a part of the environment, but the test does
not properly initialize it.  In the second case, a test should but
does not call
an initialization function before later invocations to a complex library.
This flaw in the test code is masked because the default test suite execution
order includes other tests that initialize the library.  The defect is
inconsequential until and unless the flawed test is reordered, either manually or by
a downstream tool, to execute before any other initializing test.


%\subsubsection{Examples}
%\label{sec:examples}

%The discussion of the examples in this section is distinguished by
%the problems caused by test dependence (\emph{Kind}): when faults are masked because
%tests make incorrect assumptions about the global environment (Section~\ref{sec:mask}); 
%when tests do not
%respect required initialization protocols (Section~\ref{sec:examples:initialization}); and when
%undocumented test dependence leads to spurious bug reports (Section~\ref{sec:spurious}).
%We also describe dependent tests in an automatically-generated test
%suite (Section~\ref{sec:autogen}).


\todo{text below is taken from the original draft, too verbose, should be cut short}

We next give examples for each of the above categories.

\paragraph{CLI: A Long-Standing Bug}

\begin{figure}
% \lstset{language=Java,numbers=left}
%\lstset{language=Java}
\lstset{belowskip=0ex,escapechar={@},numbers=left,numberstyle=\small\ttfamily}
\begin{lstlisting}
public final class OptionBuilder {
  @\itshape\color{red}
  private static String argName;@
  
  private static void reset() {
    ...
    @\itshape\color{red}argName = "arg";@
    ...
  }
   
  public static Option create(String opt){
    Option option = 
      new Option(opt, description);
    ...
    option.setArgName(argName);
    @\itshape\color{red}OptionBuilder.reset();@
    return option;
  }
}
\end{lstlisting}
\caption{Fault-related code from \code{Option\-Build\-er.java}}
\label{fig:option_builder}
\end{figure}

A straightforward example of fault masking occurs in the Apache CLI
library~\cite{cli}.
Two test cases fail when run in isolation:
\code{test13666} and \code{test\-Op\-tion\-With\-out\-Short\-For\-mat2} in test
classes \code{Bugs\-Test} and \code{Help\-For\-mat\-ter\-Test},
respectively.

A detailed study of the code under test revealed that both 
tests fail due to the same hidden dependence. The fault is located in 
\code{OptionBuilder.java} and is caused by not initializing a global
variable early enough.
Figure~\ref{fig:option_builder} shows code that
illustrates the fault. 
%
By default,
\code{argName} is initialized to \code{null} (line 2), and only set to
its intended default value \code{"arg"} by the \code{create()} method
via calling \code{reset()} (line 15). 
Consequently, if clients of CLI do not explicitly initialize the value of
\code{argName}, the first option created will have \code{null} rather
than \code{"arg"} as its argument name.
%In CLI, there are two types of options: options with and without
%argument names. If an option without argument is created first,
%this fault will not lead to a failure, because the \code{null} value
%will be ignored. Consecutive calls to \code{create()} can rely on
%\code{reset()} to establish the desired default value.

Both dependent tests
% \code{test13666} and \code{test27635} (or \code
% {test\-Op\-tion\-With\-out\-Short\-For\-mat2}) 
can reveal this fault, since they create an option with 
the default argument as the first thing in their execution. However,
in the default order of test execution, 
%the test classes \code{BugTest} and \code{Help\-For\-mat\-ter\-Test} both
%contain other 
tests that create options with explicit arguments execute \emph{before} 
these dependent tests.
% \code{test13666}
% and \code{test27635} respectively. 
%Thus, when the tests in these classes are 
%executed in order, the tests executed before \code{test13666}
%and \code{test27635} call \code{create()} 
Thus, the tests that are executed before call \code{create()} at least once, which
sets the default \code{argName} value, thus masking the fault.


This fault is reported in the bug
database several times,\footnote{\url{https://issues.apache.org/jira/browse/CLI-26} \url{https://
issues.apache.org/jira/browse/CLI-186} \url{https://issues.apache.org/jira/browse/
CLI-187}} starting on March 13, 2004 (CLI-26). The report is marked as resolved
\emph{three years} later on March 15, 2007, but is then reopened as CLI-186 on
July 31, 2009. On this report, one of the developers commented:
\begin{quote}
I reproduced the issue, it requires a dedicated test case since it is tied to the initialization 
of a static field in OptionBuilder.
\end{quote}
Despite the realization that a dedicated test is required, no such
test was ever created.
About one month later, the bug is duplicated as CLI-187, and the
actual fix happens one 
year later on June 19, 2010, about six years after the bug was first reported (and four years
total on the open-issue list).


\paragraph{Eclipse SWT: Causing Spurious Bug Reports}
\newcommand{\ite}{\texttt{Invalid\-Thread\-Access\-Exception}}

The Eclipse Standard Widget Toolkit
(SWT)\footnote{\url{http://eclipse.org/swt/}} is a cross-platform GUI
library developed within the Eclipse framework.
%\todo{KM}{Year here\ldots}.
%It has been developed to combine best parts of Sun's Abstract Window Toolkit (AWT)
%and Swing: native look and feel and native performance.
%
Due to the difficulty of obtaining source, compiling and running
test suites with the SWT project, we only examined some test cases
manually, after a bug report indicated test dependence. The
numbers reported in Figure~\ref{fig:example-summary} are the number
of tests we manually examined, and the number of dependencies we found
among those respectively.

As is common practice in GUI toolkits, SWT permits only one
\texttt{Display} object per thread. Attempting to create multiple
\texttt{Display}s in a single thread causes an \ite{}. 
%In other words each thread is responsible of disposing its \texttt{Display} after it is done with it. 
To permit the reuse of \texttt{Display}s, SWT provides two 
methods: \texttt{Display.getDefault} and \texttt{new Shell}. These
methods return the existing \texttt{Display} or create a new one if none exists.


In the test suite of SWT, all tests except those in the class \texttt{Test\_org\_eclipse\_swt\_widgets\_Display}
(\texttt{TestDisplay} for short) retrieve the current \texttt{Display} by using
one of the latter methods. On the other hand, all tests in
\texttt{TestDisplay} create their \texttt{Display} at the beginning of the test
and dispose of it at the end. 

%\texttt{TestDisplay.setup} contains the following
%comment:
%\begin{quote}
%There can only be one Display object per thread. If a second Display is created
%on the same thread, an InvalidThreadAccessException is thrown. 
% \\ Each test will create its own Display and must dispose of it before
% completing.
%\end{quote}


In September 2003, a user reported a
bug,\footnote{\url{https://bugs.eclipse.org/bugs/show_bug.cgi?id=43500}}
stating that tests throw an \ite{}
if she runs any other test before \texttt{DisplayTest}. 
The cause of this is simple: any other test creates, but does not
dispose of a \code{Display} object. Then the tests in
\code{TestDisplay} attempt to create a new object, which fails, as one
is already associated with the current thread.
Since this is the expected and desired behavior, the bug report is
spurious (except maybe it points to a problem in the test suite,
rather than the code).


%Let us examine the bug
%report: running any other test would create a \texttt{Display} (through one of
%the latter two methods) and would not dispose it. Thus, when these test
%complete, the main thread owns a \texttt{Display}. At this moment, when the same
%thread tries to run \texttt{DisplayTest} and thus tries to create another
%\texttt{Display}, an \ite{} is thrown. However, note that this is really the
%intended case when a thread attempts to create multiple \texttt{Display}s. In
%other words, this dependency leads to a spurious bug: there is a change in the
%test outcome when the order of tests are changed, however this does not
%correspond to a bug in the program. Nevertheless, understanding this dependency
%--- even though the comment on \texttt{DisplayTest.setup} existed --- takes
%about a month for the developers. One of the developers closes the bug with the
%following comment:
%\begin{quote}
%Turns out that the tests really are order-dependent - the Display tests must 
%be run first. It's not an SWT bug or anything, it's just the way the tests are 
%written, and I think it would be weird to code around it. \\
%\ldots I'm not going to make any code changes, but for the `fix' I have added a big 
%comment in the AllTests method saying that the Display tests must go first.
%\end{quote}
%We believe that the way this bug is handled shows that the
%dependencies between tests can lead to confusion even when there is no real bug. 


% This led to a spurious bug report. This is actually a good example,
% because it shows how hard it is to tell the difference between a bug
% and dependent test.
% So what do we fix? The "bug" or the tests?


\todo{need one more example for poor test construction}

\subsubsection{Implications}

\vspace{1mm}
\noindent \textbf{Most dependent tests can be revealed by running two tests.}
In some cases, they are intentional, developers are aware
of them and document them, but in other cases they are
inadvertent. Test dependence can cause problems, not only
when test suites are reordered, but even when they are
executed in the intended order.

\vspace{1mm}
\noindent \textbf{Most dependent tests can be revealed by running two tests.}

\vspace{1mm}
\noindent \textbf{Most dependent tests are not easy to identify unless explicitly searched for.}

\subsection{Caveats}

Our findings need to be taken with the methodology in mind. The
bug repositories in our study cover representative and important
software applications with comprehensive test suites; and
all studied dependent tests are from well-tested
applications. Nevertheless, most applications we studied
are written in Java, and the tests are in the form of JUnit.
Thus, we do not claim our findings can be extended to
arbitrary programs.

The dependent tests in our study are collected from bug
repositories. We have followed the decisions made by
developers to classify the category, severity, and
\todo{other characteristics} of a dependent test,
and did not intentionally ignore
any test dependence in bug repository. However,
it is entirely possible that some dependent tests may never
be identified, or reported to the developers. Unfortunately,
there is no conceivable way to study these unreported
dependent tests. We believe the dependent tests in our study
provide a representative sample in these software applications.

In our study, we do not emphasize any quantitative characteristic
results, and most of our findings are consistent across
the examined dependent tests.
